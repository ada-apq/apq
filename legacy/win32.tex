% win32.tex - $Id: win32.tex,v 1.2 2004/10/25 02:06:08 wwg Exp $
% Warren W. Gay
%
% Needs:
%	gnat-3.15p-nt.exe
%	gnatwin-3.15p.exe
%
\documentclass[english]{report}
\usepackage[T1]{fontenc}
\usepackage[latin1]{inputenc}
\setcounter{secnumdepth}{3}
\setcounter{tocdepth}{3}
\usepackage{graphicx}

\makeatletter
%%%%%%%%%%%%%%%%%%%%%%%%%%%%%% Textclass specific LaTeX commands.
 \newenvironment{lyxcode}
   {\begin{list}{}{
     \setlength{\rightmargin}{\leftmargin}
     \setlength{\listparindent}{0pt}% needed for AMS classes
     \raggedright
     \setlength{\itemsep}{0pt}
     \setlength{\parsep}{0pt}
     \normalfont\ttfamily}%
    \item[]}
   {\end{list}}

\usepackage{babel}
\makeatother
\begin{document}

\title{APQ-2.1 Win32 Build Instructions}


\author{Warren W. Gay}


\date{September 24, 2003}

\maketitle

\chapter{Getting Started}

This document provides a guide to those people who prefer to compile
products they install, from the source code. This guide specifically
deals with compiling and installing APQ in a win32 environment.

What do you get from APQ in the win32 environment?

The following are the products of this compile and install procedure:

\begin{itemize}
\item apq\_myadapter.dll for MySQL support
\item apq-mysql.ads generated spec for MySQL support
\item libapq.a static library for APQ client programs
\item APQ static spec and body sources
\item APQ-2.1.EXE Installer program
\end{itemize}
You may also choose which databases you wish to support. For example,
you may only wish to include MySQL support for the win32 environment.
Or instead, you may choose to support all products%
\footnote{At the present time, this is PostgreSQL and MySQL.%
}, since PostgreSQL databases may reside on hosts other than the win32
environment that you are using. The choice is yours.


\section{Does APQ Need CYGWIN?}

No. When APQ is built and installed using these instructions, is completely
independent of any CYGWIN tools and libraries. You may run your APQ
application in or out of a CYGWIN environment. It only requires the
win32 environment, and those libraries that come installed with GNAT.

APQ does however, require the use of CYGWIN environment tools, to
perform the build process. The compiler used is only GNAT and GNAT's
underlying gcc.%
\footnote{Microsoft's tool set is also required to compile some components.%
} However, the win32 environment is not rich in the way of scripted
build processes, so the CYGWIN POSIX like environment is used instead.


\section{Tools Needed}

In order to build the APQ library from sources, you will need a number
of tools installed:

\begin{itemize}
\item CYGWIN (see below for list)
\item Microsoft Visual Studio (Version 6.0)%
\footnote{It is possible that an earlier version such as 5.0 might be sufficient,
but this is untested.%
}
\item makensis (Nullsoft Scriptable Install System)%
\footnote{Dowload it from nsis.sourceforge.net/site/index.php%
}
\item GNAT Compiler (3.14p was tested)
\end{itemize}
The makensis is actual somewhat optional. The actual build does not
require this tool. If you are prepared to move the various components
into the correct places (or custom locations), you do not need it.
However, the built APQ-2.1.EXE install program is highly recommended,
because it provides the following benefits:

\begin{itemize}
\item makes installation easy and foolproof
\item allows you to choose what components to install
\item registers its version in the registry
\item is {}``GNAT aware''
\item provides an uninstall capability%
\footnote{From the Control Panel, select Add/Remove Programs.%
}
\end{itemize}

\section{CYGWIN Tools}

In order to provide a POSIX like environment, the CYGWIN tools are
necessary to facilitate the complex process of building the APQ library
from sources. If binary versions of the library are available, you
may want to use them instead.

The following list are the CYGWIN tools needed:

\begin{itemize}
\item bash (shell)
\item sed
\item sort
\item ar
\item tar
\item cp
\item rm
\item chmod
\item mkdir
\item make
\item cygpath
\item uname
\end{itemize}

\chapter{PostgreSQL Preparation}

This section shows you what you need to do to prepare the PostgreSQL
components of APQ. If you do not plan to provide win32 support for
PostgreSQL, then you can skip to the next chapter. Note that this
chapter is only about preparing a PostgreSQL client facility. Installing
and using a PostgreSQL database under Windows is well beyond the scope
of this document.


\section{Shell Sessions}

This chapter will assume that you are using the cmd.exe native shell
sessions (ie. not CYGWIN).


\section{Source Code}

This document will assume that your 3rd party source code will be
downloaded and unpacked in the directory:

\begin{itemize}
\item c:\textbackslash{}work
\end{itemize}
If you don't have this directory, then create one now.


\subsection{Download Sources}

Download the PostgreSQL sources that you plan to use. In this example
the download file:

\begin{lyxcode}
postgresql-base-snapshot.tar.gz
\end{lyxcode}
was used. You may want to choose a stable production quality release
instead. Once you have downloaded your source file, unpack it to your
work directory. Here we will use the file shown above.


\subsection{Unpack Sources}

Unpack your PostgreSQL sources to the work subdirectory. Using the
file downloaded above, the sources unpacked into the directory named:

\begin{itemize}
\item c:\textbackslash{}work\textbackslash{}postgresql-snapshot
\end{itemize}
If you downloaded a production level release, then {}``snapshot''
was likely replaced by a version number. Change to the src subdirectory,
and then list the files there. You should find a file named win32.mak
(using cmd.exe shell):

\begin{lyxcode}


{\footnotesize C:\textbackslash{}work>~cd~postgresql-snapshot\textbackslash{}src\textbackslash{}interfaces\textbackslash{}libpq}{\footnotesize \par}

{\footnotesize C:\textbackslash{}work\textbackslash{}postgresql-snapshot\textbackslash{}src\textbackslash{}interfaces\textbackslash{}libpq>~dir~/w}{\footnotesize \par}


\end{lyxcode}

\section{Compile PostgreSQL Components}

In that directory, you should see a file named win32.mak (the prompt
here will be abbreviated):

\begin{lyxcode}
{\footnotesize C:>~nmake~/f~win32.mak}{\footnotesize \par}
\end{lyxcode}
Note: you should always check with the documentation that comes with
your source code. These directions should work on all recent releases
up to and including 7.4beta2. Build and install instructions often
change, so check for them, with each new release.


\subsection{Check for libpq.dll}

This should now build the PostgreSQL client library libpq.dll. This
process might fail at some point.%
\footnote{Release 7.4beta2 failed building a component after libpq.dll.%
} This can be ignored, if the important part (libpq.dll) was built.
Change to the dll directory to check:

\begin{lyxcode}
{\footnotesize C:>~cd~interfaces\textbackslash{}libpq\textbackslash{}Release}{\footnotesize \par}

{\footnotesize C:>~dir~libpq.dll}{\footnotesize \par}

~{\footnotesize Volume~in~drive~C~has~no~label.}{\footnotesize \par}

~{\footnotesize Volume~Serial~Number~is~587F-CB45}{\footnotesize \par}

~{\footnotesize Directory~of~C:\textbackslash{}work\textbackslash{}postgresql-snapshot\textbackslash{}src\textbackslash{}interfaces\textbackslash{}libpq\textbackslash{}Release}{\footnotesize \par}

{\footnotesize 09/19/2003~~10:42p~~~~~~~~~~~~~~90,112~libpq.dll}{\footnotesize \par}

~{\footnotesize ~~~~~~~~~~~~~~1~File(s)~~~~~~~~~90,112~bytes}{\footnotesize \par}

~{\footnotesize ~~~~~~~~~~~~~~0~Dir(s)~~32,435,912,704~bytes~free}{\footnotesize \par}

{\footnotesize C:>}{\footnotesize \par}
\end{lyxcode}
The example session above, shows that the libpq.dll file was successfully
created.


\section{Create Staging Area}

Now create a directory to place the important PostgreSQL files into.
Create the following directories:

\begin{description}
\item [c:\textbackslash{}postgresql]top level staging directory
\item [c:\textbackslash{}postgresql\textbackslash{}include]directory of
C header files for compiling
\item [c:\textbackslash{}postgresql\textbackslash{}lib]optional directory
for DLL
\end{description}
These directories create a place for the PostgreSQL DLL file and the
C header files.%
\footnote{It is possible to install everything from its original location, provided
that you answer the configuration prompts correctly. But this install
procedure has been designed to keep things simple.%
}


\subsection{Install the DLL File}

Copy the PostgreSQL DLL file into the staging area. You want to copy
the file:

\begin{itemize}
\item c:\textbackslash{}work\textbackslash{}postgresql-snapshot\textbackslash{}src\textbackslash{}interfaces\textbackslash{}libpq\textbackslash{}release\textbackslash{}libpq.dll
\end{itemize}
to the location:

\begin{itemize}
\item c:\textbackslash{}postgresql\textbackslash{}lib\textbackslash{}libpq.dll
\end{itemize}
If this directory is not on your windows PATH, you will need to either
put it on your path, or place it somewhere else that is on the path.
One suggestion is GNAT's bin directory. On the author's system, this
would be:

\begin{itemize}
\item c:\textbackslash{}opt\textbackslash{}gnat\textbackslash{}bin\textbackslash{}libpq.dll
\end{itemize}
\begin{lyxcode}

\end{lyxcode}

\subsection{Install C Header Files}

Now it is necessary to copy over the C header files that APQ will
need to examine. Copy all the files from:

\begin{itemize}
\item c:\textbackslash{}work\textbackslash{}postgresql-snapshot\textbackslash{}src\textbackslash{}interfaces\textbackslash{}libpq\textbackslash{}{*}.h
\end{itemize}
to the directory:

\begin{itemize}
\item c:\textbackslash{}work\textbackslash{}postgresql\textbackslash{}include
\end{itemize}
\begin{lyxcode}

\end{lyxcode}
There are a few other C header files that APQ will need. Copy the
header files from:

\begin{itemize}
\item c:\textbackslash{}work\textbackslash{}postgresql-snapshot\textbackslash{}src\textbackslash{}include\textbackslash{}{*}.h
\end{itemize}
additionally to the directory:

\begin{itemize}
\item c:\textbackslash{}work\textbackslash{}postgresql\textbackslash{}include
\end{itemize}
\begin{lyxcode}

\end{lyxcode}
Now the PostgreSQL preparation is ready for APQ.


\chapter{MySQL Preparation}

Like the PostgreSQL chapter, this chapter is optional. If you do not
plan to include MySQL client support, then skip to the next chapter.
Just be sure that you include support for at least one of PostgreSQL
or MySQL!


\section{Shell Sessions}

Where shell sessions are required, this chapter is assuming that you
will be using cmd.exe sessions (ie. not CYGWIN).


\section{Create Work Directory}

If you don't already have a C:\textbackslash{}WORK directory, then
create one now.

Note: numerous shortcuts are possible in this procedure if you know
what you are doing. We will be taking a cautious approach to the procedure,
so that the steps will be easy to understand, and to avoid confusion.


\section{Unpack MySQL Sources}

Unpack (or unzip) the tar/zip file into the C:\textbackslash{}WORK
directory. Here we used WinZip to unpack the sources into a subdirectory.

\includegraphics[scale=0.5]{winzip.jpg}

Note carefully that the {}``Use folder names'' checkbox is checked.


\section{Build MySQL DLL}

You will need to start the Microsoft Visual Studio 6.0 at this point.%
\footnote{Check with the MySQL documentation for the necessary build instructions.
This procedure was effective for MySQL version 4.0.14b. If their instructions
vary from these, follow theirs.%
} Select File->Open Workspace, and then browse in the unpacked subdirectory
for a file named mysql.dsw :

\includegraphics[scale=0.5]{open_workspace.jpg}

Next, from the Build->Set Active Configuration Menu, choose libmsql
- Win32 Release (or Win32 Debug if you are debugging changes).

\includegraphics[scale=0.5]{set_active.jpg}

Then compile by pressing F7. If all went well, you should see a successful
completion of the compile:

\includegraphics[scale=0.5]{comp1.jpg}


\section{Create MySQL Staging Area}

\marginpar{You may find that you already have a MySQL lib\textbackslash{}debug
and lib\textbackslash{}opt library directory. The former has components
for debugging while the later contains optimized libraries.%
}This APQ build procedure will assume that you have the following directories
in place for the client MySQL components:

\begin{description}
\item [c:\textbackslash{}mysql]top level MySQL directory
\item [c:\textbackslash{}mysql\textbackslash{}include]containing all client
MySQL C header files
\item [c:\textbackslash{}mysql\textbackslash{}bin]containing the libmySQL.dll
DLL file
\item [c:\textbackslash{}mysql\textbackslash{}lib]containing the libmySQL.lib
import file
\end{description}
Your directory structure can vary from this theme, but we'll refer
to the directories in this document as noted above.

You may have installed MySQL from binary distributions. If it included
the install of a libmySQL.dll client library (as it should), you might
already have include files, that were installed with the dll. Note
however, the author has found that you need more than the include
files that were provided, for the compile of the APQ component. So
it is recommended that you follow this procedure, including the staging
area.


\subsection{Staging Directories}

Create the c:\textbackslash{}mysql staging area directories and subdirectories
as shown in the prior section, if they don't already exist. If you
are using other directories, you may want to write your notes on this
document page.


\subsection{Copy Include Files}

Copy the include files from:%
\footnote{The directory name obviously reflects MySQL version 4.0.14b, which
may be different in your case.%
}

\begin{itemize}
\item c:\textbackslash{}work\textbackslash{}mysql-4.0.14b\textbackslash{}include\textbackslash{}{*}.h
\end{itemize}
to the directory:

\begin{itemize}
\item c:\textbackslash{}mysql\textbackslash{}include
\end{itemize}

\subsection{Copy the libmySQL.dll File}

If you built the libmySQL.dll file from sources, you will want to
install that. Copy the file:

\begin{itemize}
\item c:\textbackslash{}work\textbackslash{}mysql-4.0.14b\textbackslash{}lib\_release\textbackslash{}libmysql.dll
\end{itemize}
to your directory:

\begin{itemize}
\item c:\textbackslash{}mysql\textbackslash{}bin\textbackslash{}libmysql.dll
\end{itemize}
You will need to make certain that this DLL is on the PATH, in order
to be executable. You may prefer to put the DLL with the GNAT binary
directory such as:

\begin{itemize}
\item c:\textbackslash{}opt\textbackslash{}gnat\textbackslash{}bin\textbackslash{}libmysql.dll
\end{itemize}
instead. The choice is up to you.


\subsection{Copy the Import Library}

You will need the import library to assist with the linking of the
APQ DLL file. Copy the file:

\begin{itemize}
\item c:\textbackslash{}work\textbackslash{}mysql-4.0.14b\textbackslash{}lib\_release\textbackslash{}libmysql.lib
\end{itemize}
to your directory:

\begin{itemize}
\item c:\textbackslash{}mysql\textbackslash{}lib\textbackslash{}libmysql.lib
\end{itemize}
Now your staging area for MySQL is ready for APQ.


\chapter{Compiling APQ}

Now we are ready to start into building the APQ components. The first
step is one of configuration.

APQ has been designed to allow PostgreSQL and/or MySQL support to
be included. The preceeding chapters documented preparations for both,
but either of those database products can been omitted.%
\footnote{As long as at least one is chosen.%
} We will assume in this document, that you want all the support you
can build, and so will build both. However, you can easily bypass
one or the other, with the appropriate answers to the scripted prompts.


\section{Shell Sessions}

The APQ work will primarily use a CYGWIN shell session using the bash
shell. Open a CYGWIN bash session now, and make a work subdirectory
there, and then unpack APQ and finally change to that directory. In
this section you may need to switch between this CYGWIN bash session,
and a cmd.exe session window.

If you downloaded APQ-2.1, then your unpacked sources should probably
be unpacked to a directory named:

\begin{itemize}
\item c:\textbackslash{}work\textbackslash{}apq-2.1 (ie. /cygdrive/c/work/apq-2.1
for CYGWIN)
\end{itemize}

\section{Configuration}

Before you can build the APQ sources, you must configure them. Run
the shell script ./configure in the CYGWIN bash session:

\begin{lyxcode}
\$~./configure



THIS~IS~A~WIN32~INSTALL~OF~APQ-2.1~:

-

-{}-Building~gnat\_info~command,~to~obtain~some~registry

-{}-information~about~your~installed~GNAT~compiler..

-

You~have~GNAT~3.14p~installed~:


\end{lyxcode}
Initially the script should identify this as a WIN32 configure step%
\footnote{The script eventually execs the script win32\_config.%
}. Then the program gnat\_info.adb is compiled to allow the configuration
script to access the registry. It locates where your GNAT compiler
is installed, and later tests to see if APQ has already been installed.
If APQ is found to be installed, you will need to uninstall it to
compile APQ successfully.%
\footnote{If you installed APQ manually, this test will likely fail to detect
that APQ is already installed.%
}


\subsection{Configure Edit Mode}

The script will ask for a number of pathnames for input. This is more
easily done, if editing is permitted at the input prompts. But to
do this, the script must know which edit mode the user prefers:

\begin{lyxcode}


~~Please~choose~your~preference~for~input~editing

~~bindings:

-

~~e~-~emacs~mode

~~v~-~vi~mode

~~n~-~neither

-

Choose~e/v/n~?
\end{lyxcode}
So the script prompts you for the editing mode you want to use. Choose:

\begin{description}
\item [e]for emacs editor bindings
\item [v]for vi editor bindings
\item [n]for no special editing
\end{description}
If you are not in the habit of using emacs or vi UNIX/Linux editors,
then you might prefer to use {}``n''. This document will assume
that you have chosen {}``e'' for emacs mode, because this permits
some special features to be highlighted, for the convenience of power
users.


\subsection{Microsoft Toolset Check}

After answering the edit mode prompt, you'll get an assessment made
of the Microsoft Toolset. If you have Microsoft Visual Studio 6.0
installed, you should see:

\begin{lyxcode}
Choose~e/v/n~?~e

-

You~seem~to~have~the~Microsoft~Toolset~available..
\end{lyxcode}
Here the script is confirming that you have the necessary tools to
build the APQ\_MYADAPTER.DLL file. Certainly if you built the MySQL
or PostgreSQL DLL files, you should receive a favourable report here.
However, it is possible that those components be downloaded in binary
form instead and then placed into the work directories indicated.

If you have the Visual Studio installed, and you don't receive confirmation
of this from the script, then check your PATH variable within the
CYGWIN environment. The PATH setting probably needs adjustment.

The Microsoft Toolset required by APQ includes:

\begin{description}
\item [CL]the C/C++ Compiler
\item [LIB]the library utility for creating export and import files
\item [LINK]the Microsoft C/C++ linker utility
\end{description}
Before you despair of not owning the Microsoft toolset, consider:

\begin{itemize}
\item If you are not building support for MySQL, you do not need them. 
\item If you have downloaded a binary form of APQ\_MYADAPTER.DLL you do
not need them.
\end{itemize}
If on the other hand, you are planning to compile APQ, with MySQL
support, and lack the Microsoft toolset above, then you must now locate
and download a binary copy of apq\_myadapter.dll. Otherwise, you will
be required to use the Microsoft toolset to create that dll file.
This file may be available in your package, or perhaps made available
separately, to reduce download file sizes.


\subsection{Choosing MySQL Support}

Now you must decide whether or not you want your installed copy of
APQ to include MySQL support. The only real reason not to include
it is to save the installer time and effort. Otherwise, I suggest
you include it for flexibility.

\begin{lyxcode}
Questions~for~MySQL~:

-

Installing~MySQL~database~support~?~(Yes/No)?~Yes
\end{lyxcode}
I'm going to assume that you are answering yes, to get your money's
worth from APQ.


\subsection{Specifying MySQL Pathnames}

Once you answer \emph{Yes} to the MySQL support prompt, you are prompted
for two more pathnames:

\begin{lyxcode}
{\footnotesize Questions~for~MySQL~:}{\footnotesize \par}

-

{\footnotesize Installing~MySQL~database~support~?~(Yes/No)?~yes}{\footnotesize \par}

{\footnotesize Windows~pathname~to~MySQL~include~directory~:~c:\textbackslash{}mysql\textbackslash{}include}{\footnotesize \par}

{\footnotesize Windows~path~of~MySQL~DLL~file~is~~~~~~~~~~~:~c:\textbackslash{}MySQL\textbackslash{}BIN\textbackslash{}libmysql.dll}{\footnotesize \par}

{\footnotesize Windows~path~to~libmysql.lib~file~~~~~~~~~~~:~c:\textbackslash{}mysql\textbackslash{}lib\textbackslash{}libmysql.lib}{\footnotesize \par}
\end{lyxcode}
While the session appears to show three prompts, the pathname of the
MySQL DLL is determined by the script. It does this by looking for
libmySQL.dll on your PATH. If your PATH is not correctly set, or the
DLL not correctly installed, then this step will fail. Correct the
PATH and try again, if necessary.

The two important inputs are:

\begin{enumerate}
\item Directory pathname to MySQL include files 
\item Pathname of the MySQL DLL import file (libmysql.lib)
\end{enumerate}
Your pathnames may vary from the example, but these are the ones I
am using for illustration purposes.


\subsection{Choosing PostgreSQL Support}

Like MySQL, you can choose whether or not you want to build support
for it:

\begin{lyxcode}
Questions~for~PostgreSQL~:

-

Installing~PostgreSQL~database~support~?~(Yes/No)?~yes
\end{lyxcode}
Here, once again, we are assuming you want to get your money's worth.


\subsection{Specifying PostgreSQL Pathnames}

Currently, PostgreSQL only needs to know the location of the C header
(include) files:

\begin{lyxcode}
{\footnotesize Questions~for~PostgreSQL~:}{\footnotesize \par}

-

{\footnotesize Installing~PostgreSQL~database~support~?~(Yes/No)?~yes}{\footnotesize \par}

{\footnotesize Windows~pathname~to~PostgreSQL~include~directory~:~c:\textbackslash{}postgresql\textbackslash{}include}{\footnotesize \par}

{\footnotesize Windows~path~of~PostgreSQL~DLL~file~is~~~~~~~~~~~:~c:\textbackslash{}OPT\textbackslash{}GNAT\textbackslash{}bin\textbackslash{}libpq.dll}{\footnotesize \par}
\end{lyxcode}
The script locates the PostgreSQL DLL file libpq.dll, by searching
the PATH variable. If the script fails to locate the DLL file, then
you may need to change the PATH (under CYGWIN) and/or make the DLL
file in a directory that is searched by the PATH variable setting.
Then try the configure step again.

In this example, I will just hit return and accept the default. But
you may specify a different directory from this.


\subsection{Concluding the Configuration}

The configuration session should look something like the following: 

\begin{lyxcode}
{\footnotesize You~seem~to~have~the~Microsoft~Toolset~available..}{\footnotesize \par}

-

{\footnotesize Questions~for~MySQL~:}{\footnotesize \par}

-

{\footnotesize Installing~MySQL~database~support~?~(Yes/No)?~yes}{\footnotesize \par}

{\footnotesize Windows~pathname~to~MySQL~include~directory~:~c:\textbackslash{}mysql\textbackslash{}include}{\footnotesize \par}

{\footnotesize Windows~path~of~MySQL~DLL~file~is~~~~~~~~~~~:~c:\textbackslash{}MySQL\textbackslash{}BIN\textbackslash{}libmysql.dll}{\footnotesize \par}

{\footnotesize Windows~path~to~libmysql.lib~file~~~~~~~~~~~:~c:\textbackslash{}mysql\textbackslash{}lib\textbackslash{}opt\textbackslash{}libmysql.lib}{\footnotesize \par}

-

{\footnotesize Questions~for~PostgreSQL~:}{\footnotesize \par}

-

{\footnotesize Installing~PostgreSQL~database~support~?~(Yes/No)?~yes}{\footnotesize \par}

{\footnotesize Windows~pathname~to~PostgreSQL~include~directory~:~c:\textbackslash{}postgresql\textbackslash{}include}{\footnotesize \par}

{\footnotesize Windows~path~of~PostgreSQL~DLL~file~is~~~~~~~~~~~:~c:\textbackslash{}OPT\textbackslash{}GNAT\textbackslash{}bin\textbackslash{}libpq.dll}{\footnotesize \par}

-

~{\footnotesize ~~~You~are~now~ready~to~'make'}{\footnotesize \par}

-

To~proceed~with~the~build,~perform~the~following:

-

\$~make-{}-{}-{}-{}-{}-{}-{}-{}-{}-{}-{}-{}-{}-{}-{}-(do~'make~clean'~prior~to~rebuilds)

\$~make~installer-{}-{}-{}-(optional)

\$~make~install


\end{lyxcode}
You may of course, choose different directories and preferences. This
document is just a guideline. 

At this point, you should receive a message indicating that you are
ready to perform a {}``make''. If you are making changes and need
to rebuild the sources, perform a {}``make clean'' first. If you
are making changes to the installer programm, then you can repeat
builds of it by simply doing a {}``make installer''.

A {}``make install'' will make the installer program, and then execute
it.


\section{Building APQ}

If you have all of the CYGWIN components required, then this should
be the easiest part of the WIN32 build process.


\subsection{Win32 Makefile}

The APQ components are compiled using a Makefile that is tailored
to the Windows environment. The actual make file used is named:

\begin{itemize}
\item Makefile.win32
\end{itemize}
When the configure script ran%
\footnote{The configure script invokes script win32\_config under Windows.%
}, it also created a symlink to the Makefile.win32 file. The symlink
is named:

\begin{itemize}
\item GNUmakefile
\end{itemize}
This symlink permits the make command to reference Makefile.win32
by default. If you must customize the make file, be sure to edit Makefile.win32.

The configuration process also created another file:

\begin{itemize}
\item config.win32
\end{itemize}
The configured values for the build are included in the config.win32
file. This is the file that is included from Makefile.win32. If you
need edit configuration values, edit that file. Just be aware that
those changes will be lost if the configure script is successfully
rerun.

To compile the APQ components, in a CYGWIN bash session, using the
APQ source directory%
\footnote{Using our example this would be /cygdrive/c/work/apq-2.1.%
} as the current directory, you just type make:

\begin{lyxcode}
\$~make
\end{lyxcode}
This should successfully build everything that is required, depending
upon the options you selected in the configuration process. The following
products should come out of the build:

\begin{description}
\item [apq-mysql.ads]This is generated for MySQL support only
\item [apq\_myadapter.dll]This is for MySQL support only
\item [libapq.a]A static GNAT library for APQ
\end{description}
Additionally, all static APQ spec and body sources must also be installed.


\section{Rebuilding/Configuring}

Sometimes things don't go right on the first try. To allow for that
possibility, the following tools are at your disposal:

\begin{enumerate}
\item make clean
\item make clobber
\end{enumerate}

\subsection{Make Clean}

Use this when you want to delete all compiled objects, binary components
and generated sources (like apq-mysql.ads). This allows you to tweak
as necessary, but rebuild again without redoing the configuration
process.


\subsection{Make Clobber}

This goes one step further than {}``make clean'', in that it destroys
the configuration information that was gathered. Use this procedure
when you want to start over from scratch.


\section{Installing APQ}

Once you have successfully built APQ, you need to install its components.
If it were only a few binary files, you could do this by hand. However,
in addition to the library components, the Ada specification and body
source files must be installed in the correct place.%
\footnote{Some of the Ada bodies are mandatory because of the generics that
must be compiled by GNAT. So APQ just installs them all.%
}

If you perform {}``make installer'', this will build the installer
program for you. If you perform a {}``make install'', it will build
the installer program and invoke it.

\begin{lyxcode}
\$~make~install
\end{lyxcode}
The installer used is a nice GUI installer, provided complements of
the Nullsoft Scriptable Installer System folks. Visit their website
at:

\begin{lyxcode}
nsis.sourceforge.net/site/index.php
\end{lyxcode}
When the installer is run, simply follow the GUI prompts.


\section{Uninstalling APQ}

If you are making changes, you'll frequently need to uninstall and
re-install. This is easily provided for in the Control Panel. Choose
{}``Add/Remove Programs'' and look for the entry named {}``APQ-2.1
(remove only)''. Click on the button {}``Change/Remove'' to proceed.
The APQ\_Uninstall.exe program will then be launched to guide you
through the process.


\section{Conclusion}

If you reached this point, you have successfully installed database
support, which is now at your Ada95 finger tips! Congratulations!
The next chapter will show you how to test your APQ library.


\chapter{Testing APQ}

Naturally, you want to test out your shiny new APQ library, now that
you can connect to databases both on your win32 platform, and remotely
over the network. The example programs in the ./eg and the ./eg2 are
UNIX flavoured tests and are not suitable.

Starting with APQ-2.1, there is now a template program named win32\_test.adb.
This chapter will describe how you can use that source file to perform
some simple tests.


\section{Creating the Test Program}

First you need to create a file named win\_test.adb from the win32\_test.adb
source file provided. Do not modify the original win32\_test.adb file.
You may need it for testing another database, later.

Create the win\_test.adb by copying the win32\_test.adb file (bash):

\begin{lyxcode}
\$~cp~win32\_test.adb~win\_test.adb
\end{lyxcode}
But don't compile that file yet! You need to supply some changes.

\begin{quote}
Note: if you are testing a binary installed release, look for the
win32\_test.adb program in the APQ bindings directory. If GNAT is
installed as C:\textbackslash{}OPT\textbackslash{}GNAT, then look
for the file named:

C:\textbackslash{}OPT\textbackslash{}GNAT\textbackslash{}Bindings\textbackslash{}APQ\textbackslash{}win32\_test.adb

Copy this file to your work area, and name it win\_test.adb to match
the procedure name within it.
\end{quote}

\subsection{Editing the win\_test.adb Source}

The test file win\_test.adb is nearly complete. But it still needs
you to define:

\begin{enumerate}
\item Which database product to use
\item What host name (if not local) to use
\item What account name (userid) to use
\item What password to use (if any)
\end{enumerate}
Once these are defined, the program is ready to compile and test.

The unmodified start of the win\_test.adb program should look like
this:

\begin{lyxcode}
~{\footnotesize ~~~~1~~-{}-~Very~Simple~Connect~to~Database~Test}{\footnotesize \par}

~{\footnotesize ~~~~2}{\footnotesize \par}

~{\footnotesize ~~~~3~~-{}-~Uncomment~one~of~the~following:}{\footnotesize \par}

~{\footnotesize ~~~~4~~-{}-~with~APQ.PostgreSQL.Client;~use~APQ,~APQ.PostgreSQL,~...}{\footnotesize \par}

~{\footnotesize ~~~~5~~-{}-~with~APQ.MySQL.Client;~use~APQ,~APQ.MySQL,~APQ.MySQL.Client;}{\footnotesize \par}

~{\footnotesize ~~~~6}{\footnotesize \par}

~{\footnotesize ~~~~7~~with~Ada.Text\_IO;~use~Ada.Text\_IO;}{\footnotesize \par}

~{\footnotesize ~~~~8}{\footnotesize \par}

~{\footnotesize ~~~~9~~procedure~Win\_Test~is}{\footnotesize \par}

~{\footnotesize ~~~10~~~~~C~:~Connection\_Type;}{\footnotesize \par}

~{\footnotesize ~~~11~~~~~Q~:~Query\_Type;}{\footnotesize \par}

~{\footnotesize ~~~12~~begin}{\footnotesize \par}

~{\footnotesize ~~~13}{\footnotesize \par}

~{\footnotesize ~~~14~~~~~Put\_Line(\char`\"{}Win32\_Test~Started:\char`\"{});}{\footnotesize \par}

~{\footnotesize ~~~15}{\footnotesize \par}

~{\footnotesize ~~~16~~-{}-~Set\_Host\_Name(C,\char`\"{}<hostname>\char`\"{});}{\footnotesize \par}

~{\footnotesize ~~~17~~~~~Set\_User\_Password(C,\char`\"{}<userid>\char`\"{},\char`\"{}<password>\char`\"{});}{\footnotesize \par}

~{\footnotesize ~~~18~~~~~Set\_DB\_Name(C,\char`\"{}<database\_name>\char`\"{});}{\footnotesize \par}
\end{lyxcode}
The line numbers are not part of the source code however, since they
are here only for our ease of reference. Now perform the following
edits:

\begin{enumerate}
\item Uncomment line 4 if you are testing PostgreSQL. Else uncomment line
5 for MySQL.
\item If your database is a remote database, uncomment line 16. Then replace
<hostname> with the host name or IP number of the database server.
\item Edit your userid value in place of <userid> on line 17.
\item Edit your password value in place of <password> on line 17. If there
is no password, supply a null string like {}``''.
\item Edit the database name in place of <database\_name> on line 18.
\end{enumerate}
Save the file to win\_test.adb and compile it as follows:

\begin{lyxcode}
\$~gnatmake~win\_test
\end{lyxcode}
You will not need to specify any linking arguments, since the GNAT
feature:

\begin{lyxcode}
pragma~Linker\_Options({}``...'');
\end{lyxcode}
was used in the APQ package specs to avoid this inconvenience.


\section{Running the Test}

Before the test run can be successful, there are some obvious prerequisites
on the database side:

\begin{itemize}
\item The database server must accept connections from you
\item The database server userid and password must be acceptable
\item The database named must already exist (the test program does not create
it)
\end{itemize}
The test program win\_test.exe, first attempts to drop a table TEST\_TBL,
and then creates a new one. This is done so that you can run the test
multiple times. This has the implication that:

\begin{itemize}
\item The userid must have table create privileges
\item The userid must have table drop privileges
\item The userid must have sequence drop privileges, for PostgreSQL only
\end{itemize}
Configuring the security and the privileges of the user accounts,
is outside the scope of this document. Note however, that you can
test the connectivity and rights with the mysql command for MySQL,
and psql for PostgreSQL. Once those work, you should have success
with win\_test.exe as well.

The test results look something like the following for MySQL:

\begin{lyxcode}
\$~./win\_test

{\footnotesize Win32\_Test~Started:}{\footnotesize \par}

{\footnotesize My~Userid~is~'wwg'}{\footnotesize \par}

{\footnotesize My~Password~is~''}{\footnotesize \par}

{\footnotesize My~Database~is~'wwg'}{\footnotesize \par}

{\footnotesize My~Host~Name~is~''}{\footnotesize \par}

{\footnotesize My~Engine~is~ENGINE\_MYSQL}{\footnotesize \par}

{\footnotesize Connecting..}{\footnotesize \par}

{\footnotesize Connected!}{\footnotesize \par}

-

{\footnotesize DROP~TABLE~TEST\_TBL}{\footnotesize \par}

-

{\footnotesize Table~was~dropped.}{\footnotesize \par}

-

{\footnotesize CREATE~TABLE~TEST\_TBL~(}{\footnotesize \par}

~{\footnotesize ~NAME~~~VARCHAR(32)~NOT~NULL,}{\footnotesize \par}

~{\footnotesize ~ID~~~~~INTEGER~NOT~NULL~AUTO\_INCREMENT~PRIMARY~KEY}{\footnotesize \par}

{\footnotesize )}{\footnotesize \par}

-

{\footnotesize Table~created.}{\footnotesize \par}

-

{\footnotesize INSERT~INTO~TEST\_TBL~(~NAME~)}{\footnotesize \par}

~{\footnotesize ~VALUES(~'ONE'~)}{\footnotesize \par}

-

{\footnotesize OID=~1}{\footnotesize \par}



{\footnotesize INSERT~INTO~TEST\_TBL~(~NAME~)}{\footnotesize \par}

~{\footnotesize ~VALUES(~'TWO'~)}{\footnotesize \par}

-

{\footnotesize OID=~2}{\footnotesize \par}

-

{\footnotesize INSERT~INTO~TEST\_TBL~(~NAME~)}{\footnotesize \par}

~{\footnotesize ~VALUES(~'THREE'~)}{\footnotesize \par}

-

{\footnotesize OID=~3}{\footnotesize \par}

-

{\footnotesize SELECT~NAME,ID}{\footnotesize \par}

{\footnotesize FROM~TEST\_TBL}{\footnotesize \par}

-

{\footnotesize Got~3~Tuples..}{\footnotesize \par}

{\footnotesize NAME='ONE',~ID='1'}{\footnotesize \par}

{\footnotesize NAME='TWO',~ID='2'}{\footnotesize \par}

{\footnotesize NAME='THREE',~ID='3'}{\footnotesize \par}

{\footnotesize <END>}{\footnotesize \par}

{\footnotesize Test~Completed.}{\footnotesize \par}
\end{lyxcode}
Note that this win\_test.exe can run independently of CYGWIN as well,
in a cmd.exe session. APQ is not dependent upon the CYGWIN infrastructure.
CYGWIN is only used as a tool to build APQ from sources.


\section{Congratulations!}

If you successfully built and installed APQ on your win32 environment,
then you are ready to branch out into entire new horizons with APQ.
This software has been developed and provided to you for free. One
way you can help with the development is to provide support in the
way of:

\begin{itemize}
\item bug reports
\item user experiences with APQ (good and bad)
\item suggestions for enhancements
\item testimonials about the APQ library to encourage others to try it
\item financial support
\end{itemize}
APQ is Open Sourced, and so financial support is not required. However,
financial support is always welcome, and may help with future releases.
An example of where financial support may be necessary is to fund
the purchase of newer Microsoft development tools. The success of
win32 ports of any package, depend upon access to these tools. So
if you use APQ in a business setting, please consider providing some
level of support to the APQ project.

Testimonials and bug reports are another major area of support that
don't cost anything. Testimonials help others realize that they need
to take the time to test drive it. If you want to provide a testimonial,
just let me know if you want it kept anonymous, or the email address
removed. I understand the pain of spam. Bug reports help eliminate
problems for everyone, including yourself in future revisions of APQ. 

User experiences and suggestions are always welcome. Just drop a line
or three, and let me know how APQ can be improved, or how it has made
your projects successful.


\subsection*{Thanks for using APQ!}

\begin{quote}
Warren W. Gay VE3WWG

ve3wwg@cogeco.ca
\end{quote}
\begin{lyxcode}
\end{lyxcode}

\end{document}
