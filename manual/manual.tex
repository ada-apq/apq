% APQ-2.2 Manual

\documentclass[english,letterpaper]{book}
\usepackage{times}
\usepackage[T1]{fontenc}
\usepackage[latin1]{inputenc}
\usepackage{longtable}
\usepackage{amsmath}
\usepackage{amssymb}

\usepackage{floatflt}
\usepackage{fancyhdr}

\pagestyle{fancy}

%\lhead{}
%\chead{}
\rhead{}

\lfoot{}
\cfoot{\thepage}
\rfoot{APQ 2.2}


\newcommand\Ref[1]{\textsection\ref{#1} (page~\pageref{#1})}

\usepackage{fancyvrb}
\usepackage{listings}

\usepackage{makeidx}
\makeindex

\IfFileExists{url.sty}{\usepackage{url}}
                      {\newcommand{\url}{\texttt}}

\makeatletter

\usepackage{babel}
\makeatother



%==========%
% HYPERREF %
%==========%
\usepackage[dvipdfm, bookmarks, colorlinks, breaklinks, pdftitle={Glare},
    pdfauthor={Philipp Bunge}]{hyperref}
\hypersetup{
	linkcolor=DarkSkyBlue,
	citecolor= DarkSkyBlue,
	filecolor= DarkSkyBlue,
	urlcolor= DarkSkyBlue
}



%========================%
% Listings Package Setup %
%========================%
\usepackage{xcolor}
\usepackage{listings}


% COLORS (Tango)
\definecolor{LightButter}{rgb}{0.98,0.91,0.31}
\definecolor{LightOrange}{rgb}{0.98,0.68,0.24}
\definecolor{LightChocolate}{rgb}{0.91,0.72,0.43}
\definecolor{LightChameleon}{rgb}{0.54,0.88,0.20}
\definecolor{LightSkyBlue}{rgb}{0.45,0.62,0.81}
\definecolor{LightPlum}{rgb}{0.68,0.50,0.66}
\definecolor{LightScarletRed}{rgb}{0.93,0.16,0.16}
\definecolor{Butter}{rgb}{0.93,0.86,0.25}
\definecolor{Orange}{rgb}{0.96,0.47,0.00}
\definecolor{Chocolate}{rgb}{0.75,0.49,0.07}
\definecolor{Chameleon}{rgb}{0.45,0.82,0.09}
\definecolor{SkyBlue}{rgb}{0.20,0.39,0.64}
\definecolor{Plum}{rgb}{0.46,0.31,0.48}
\definecolor{ScarletRed}{rgb}{0.80,0.00,0.00}
\definecolor{DarkButter}{rgb}{0.77,0.62,0.00}
\definecolor{DarkOrange}{rgb}{0.80,0.36,0.00}
\definecolor{DarkChocolate}{rgb}{0.56,0.35,0.01}
\definecolor{DarkChameleon}{rgb}{0.30,0.60,0.02}
\definecolor{DarkSkyBlue}{rgb}{0.12,0.29,0.53}
\definecolor{DarkPlum}{rgb}{0.36,0.21,0.40}
\definecolor{DarkScarletRed}{rgb}{0.64,0.00,0.00}
\definecolor{Aluminium1}{rgb}{0.93,0.93,0.92}
\definecolor{Aluminium2}{rgb}{0.82,0.84,0.81}
\definecolor{Aluminium3}{rgb}{0.73,0.74,0.71}
\definecolor{Aluminium4}{rgb}{0.53,0.54,0.52}
\definecolor{Aluminium5}{rgb}{0.33,0.34,0.32}
\definecolor{Aluminium6}{rgb}{0.18,0.20,0.21}




\lstset{
	keywordstyle=[1]{\color{DarkSkyBlue}},
	keywordstyle=[2]{\color{DarkScarletRed}},
	keywordstyle=[3]{\bfseries},
	keywordstyle=[4]{\color{DarkPlum}},
	keywordstyle=[5]{\color{SkyBlue}},
	commentstyle={\color{Aluminium4}\small},
	stringstyle={\color{Chocolate}},
	tabsize=4,
	breaklines=true,
	basicstyle={\ttfamily\small},
	xleftmargin=21pt,
	xrightmargin=11pt,
	frame=single,
	rulecolor=\color{black!30},
	captionpos=b,
	framesep=10pt,
	framexleftmargin=18pt,
	numbers=none,
	numberstyle={\tiny},
	stepnumber=1,
	numbersep=15pt
}



\lstdefinelanguage{Ada}{%
	morekeywords={alfa,and,array,begin,boolean,byte,case,char,const,div,%
	do,downto,else,end,false,file,for,function,get,goto,if,in,%
	integer,label,maxint,mod,new,not,of,or,pack,packed,page,program,%
	procedure,put,read,readln,real,record,repeat,reset,rewrite,set,%
	text,then,to,true,type,unpack,until,var,while,with,write,writeln},%
	sensitive=false,%
	morecomment=[s]{(*}{*)},%
	morecomment=[s]{\{}{\}},%
	morestring=[d]{'}%
}
\lstdefinelanguage{SQL}{%
	morekeywords={select,insert,table,from,into,create,delete,alfa,and,array,begin,boolean,byte,case,char,const,div,%
	do,downto,else,end,false,file,for,function,get,goto,if,in,%
	integer,label,maxint,mod,new,not,of,or,pack,packed,page,program,%
	procedure,put,read,readln,real,record,repeat,reset,rewrite,set,%
	text,then,to,true,type,unpack,until,var,while,with,write,writeln},%
	sensitive=false,%
	morecomment=[s]{(*}{*)},%
	morecomment=[s]{\{}{\}},%
	morestring=[d]{'},%
	caption=SQL%
}




\lstnewenvironment{SQL}{\lstset{language=SQL}}{}

\lstnewenvironment{Code}{\lstset{language=Ada}}{}

\lstnewenvironment{Example}{\lstset{language=Ada,label=Example}}{}




%\DefineVerbatimEnvironment{SQL}{Verbatim}%
%   {frame=single,fontsize=\small,label={SQL},labelposition=topline}
%\DefineVerbatimEnvironment{Code}{Verbatim}%
%   {frame=single,fontsize=\small}
%\DefineVerbatimEnvironment{Example}{Verbatim}%
%   {frame=single,fontsize=\small,label={Example},labelposition=topline}
\DefineVerbatimEnvironment{NumberedExample}{Verbatim}%
   {frame=single,numbers=left,fontsize=\small,label={Example},labelposition=topline,commandchars=\\\{\}}


%==============%
% The document %
%==============%


\begin{document}

\title{APQ Ada95 Database Binding}
\author{Copyright (c) 2002-2004, Warren W. Gay VE3WWG}
\date{\today}
\maketitle

\tableofcontents{}
\listoftables
%\listoffigures

\chapter{Introduction}


\section{APQ Version 2.2}

This manual documents APQ Version 2.2, which is released under a dual
ACL\index{ACL} and GPL\index{GPL} (GNU Public License)\index{GNU Public
License} arrangement. The dual license arrangement is designed to give
both the \index{distributor} and user the necessary freedoms to enjoy
the fair use and distribution of the sources \index{distribution of
sources} contained in this project. See file COPYING\index{copying} for
more details.


\section{Supported Databases}

The APQ binding\index{binding} was initially created to satisfy the simple need to
allow Ada\index{Ada} programs to use a PostgreSQL\index{PostgreSQL} database. However, as open
sourced database technologies continue to advance, the need to allow
other databases to be used, becomes greater. Rather than write a unique
Ada binding for each one, it was conceptualized that a common API\index{API}
could emerge from the APQ framework\index{framework}. To this end, the APQ binding
has been reworked rather extensively for version 2.1, to permit increasing
levels of general support of other database technologies, including
MySQL\index{MySQL}.

In the fall of 2003 the foundation was laid for Sybase\index{Sybase} support in
the unreleased APQ 2.2 software using Sybase on a trial download\index{download} basis.
Sybase highlighted some new APQ design \index{design, APQ} wrinkles, which required a
few new adjustments and API additions. However, this development ground
to a halt for a spell, until September 9, 2004 when it was announced
on www.slashdot.org\index{www.slashdot.org} that ``Sybase Releases Free Enterprise Database
on Linux''\index{Linux}.

\begin{quote}
``Sybase announced today that they are releasing a free (as in
beer) version of their flagship database for Linux. The free version
is limited to 1 CPU, 2 GB of RAM, and 5 GB of data, which is more
than adequate for all but the most demanding applications. This release
provides a very attractive alternative to Microsoft SQL Server\index{Microsoft SQL Server}, and
gives developers and DBAs an extremely powerful argument to use against
the adoption of Microsoft-based solutions.''%
\footnote{Posted by ``Tassach'' (tassach@rapiertech.com)}
\end{quote}

Now there was greater reason to put enterprise class database support
into an Ada thick binding\index{thick binding} for databases. Sybase support in APQ would
allow developers to develop real applications in using Ada, without
trial software expiring. Once developed, these same Ada applications
could be deployed within the enterprise and on a large scale.

\begin{floatingtable}{
   \begin{tabular}{lccc}
   Database       &  As of APQ Version &  SQL   &  Blob\\
   \hline
   PostgreSQL     &  1.x               &  Yes   &  Yes\\
   MySQL          &  2.x               &  Yes   &  No\\
   Sybase 12.52   &  2.2               &  Yes   &  No\\
   \end{tabular}}
   \caption{Database Product Support}\label{t:DBSupport}
\end{floatingtable}

Thanks in part to the generosity of Sybase to developers, APQ 2.2
now supports the following list of database\index{list of database technologies} technologies:
Table \ref{t:DBSupport} is further described as follows:

\begin{description}
   \item [As of APQ Version]is the version where the database was first supported.
   \item [SQL]indicates whether the common SQL functions are supported (sans blob).
   \item [Blob]indicates whether blob support is present.
\end{description}

As the reader can observe in the table above, the support for MySQL
is incomplete in APQ 2.1. The blob support\index{blob support} is lacking in APQ for MySQL,
because MySQL's blob interface is not as complete as provided by PostgreSQL.
Where PostgreSQL provides the facility for virtually limitless sized
blobs, a MySQL blob must fit within a ``column'', very much like
a text field. For this reason, the facility to perform stream oriented\index{stream oriented}
I/O is lacking on a blob in APQ for MySQL.


\subsection{The Future of Blob Support for MySQL}

Much investigation and research\index{research, APQ} is required to adequately resolve
the blob issue in APQ. Rather than hold back the binding from general
use, where blob functionality may have limited use anyway, it was
decided to release APQ 2.0 with the common API for the two databases,
and leaving the resolution of the blob API for a future release.

If you are a developer, who hopes to write portable database\index{portable database code} code,
then please be aware that the PostgreSQL\index{PostgreSQL} blob\index{blob} API is subject to future
revision. Potentially, this could be fairly extensively revised, but
every attempt will be made to leave a migration path open to the developer.


\section{Generic Database Support}

One of the main goals of the APQ version 2.0 release, was to develop
a common API\index{API, common}, that does not discriminate based upon the database technology
being selected. The ideal was to allow a developer to write a procedure
that would accept a classwide\index{classwide} database objects, and perform database
operations without needing to be concerned whether the database being
used was PostgreSQL\index{PostgreSQL}, MySQL\index{MySQL} or Sybase\index{Sybase}.
To a large extent, the author believes that this goal has been achieved.


\subsection{Generic Limitations}

It must be admited however, there are some areas where the database
technologies were very different. Consequently, some exceptions and
work-arounds will be required by the programmer. An example of this
is that MySQL\index{MySQL} requires that all rows be fetched from a SELECT\index{SELECT} query.
A failure to do this, corrupts the communication between the server\index{communication, server}
and the client. Consequently, APQ works around this by defaulting
to use the C library call mysql\_store\_result()\index{mysql\_store\_result()} instead of the alternative,
which is mysql\_use\_result()\index{mysql\_use\_result()}. However, if the result set is large,\index{large result sets}
then receiving all of the rows into the clients memory\index{memory, client} is not a suitable
choice. Consequently, APQ does provide some MySQL specific ways to
manage this setting. 

The MySQL database software also provides the special ``LIMIT n''\index{LIMIT n}
extension, if the client program is only interested in the first n
rows of the result. If for example, you have a price file containing
stock price history, you may want to query the most recent price for
it. The simplest way to do this would be to perform a SELECT\index{SELECT} on the
table with a descending price date sort sequence (or index). But if
you only want the first (most recent) row \index{most recent row} returned, you do not want
to retrieve the entire price history into the memory of your client!
This is what mysql\_store\_result()\index{mysql\_store\_result()} implies (APQ default). So the
application programmer will need to plan for this, when MySQL is used.
He will need to do one of the following:

\begin{itemize}
   \item Cause mysql\_use\_result() to be used instead (change the APQ default),
         and then fetch all of the rows, one by one.
   \item Use the MySQL ``LIMIT 1''\index{LIMIT n} SQL extension to limit the results to
         1 row.
\end{itemize}

The problem of course, is that this type of handling must only be
done for MySQL\index{MySQL} databases. Consequently, APQ also provides an API so
that the application may query which database is being used.


\section{The APQ Database Binding}

This software represents a binding\index{binding to objects} to objects and procedures that
enable the Ada95\index{Ada95}%
\footnote{Hereafter, we'll just refer to the language as Ada, even though the
version of the language implied is Ada95.%
} programmer to manipulate or query a relational database.\index{relational database} This document
describes the design principles and goals\index{goals of APQ} of this APQ binding. It
also supplies reference documentation\index{reference documentation} to the programmer, enabling
the reader to write applications using the PostgreSQL\index{PostgreSQL}, MySQL\index{MySQL} 
or Sybase\index{Sybase}  databases,
in the Ada programming language.

The APQ binding\index{binding} was initially developed using GNAT
3.13p\index{GNAT} under FreeBSD 4.4\index{FreeBSD} release.
APQ version 2.0 was developed using Debian\index{Debian Linux} Linux\index{Linux} and GNAT 3.14p. The
examples presented will be tested under the same development environments.

The source code avoids any use of GNAT\index{GNAT} specific language extensions\index{language extensions}.
The possible exception to this rule is that the gnatprep\index{gnatprep} tool may
be used to precompile\index{precompile} optional support of optional databases\index{optional databases}. There
is some C\index{C language} language source\index{source code} code used, to facilitate Ada\index{Ada} and database
C language library linkages.

\begin{table}
   \begin{center}
   \begin{tabular}{ll}
   Library           &  Database\\
   \hline 
   libpq             &  PostgreSQL\\
   libmysqlclient    &  MySQL\\
   OCS-12\_5         &  Sybase\\
   \end{tabular}
   \end{center}
\caption{Client Libraries}\label{t:ClientLib}
\end{table}

Table \ref{t:ClientLib} 
lists the C language libraries\index{libraries, C} that are
used in addition to the APQ client\index{library, APQ client} library, while linking\index{linking} your
application:

GNAT specific features are avoided where possible. The exception
however, is the use of the GNAT\index{GNAT} specific pragma\index{pragma} statements of the form:

\index{Linker\_Options}
\begin{Code}

   pragma Linker_Options("-lapq");

\end{Code}

is used for example, to save the programmer from having to specify
linking\index{linking} arguments. Therefore those using non-ACT\index{non-ACT} vendor
supplied Ada compilers\index{Ada compilers} might be able to compile and use this binding\index{binding}
without a huge investment.

A 32-bit\index{32-bit Windows} Windows library for APQ can be built for use with the PostgreSQL\index{PostgreSQL},
MySQL\index{MySQL} or Sybase\index{Sybase} DLL\index{DLL} client libraries. APQ release 2.1 included the
win32 build instructions necessary, but was omitted in the 2.0 release.


\subsection{General Features}

This binding\index{binding} supports all of the normal database functions that a
programmer would want to use. Additionally blob\index{blob} support
is included%
\footnote{For PostgreSQL only, at release 2.0.%
}, and implemented using the Ada streams\index{streams, Ada} interface\index{interface}.
This provides the programmer with the Ada convenience and safety of
the streams interface\index{streams interface} when working with blobs.

This binding includes the following general features:

\begin{enumerate}
   \item Open and Close one or more concurrent database connections\index{connections}
   \item Create and Execute one or more concurrent SQL queries\index{queries, concurrent} on a selected
         database connection\index{connection, database}
   \item Begin work\index{begin work}, Commit work\index{commit work} or Rollback work\index{rollback work}
   \item Access error message\index{error messages} text
   \item Generic functions and procedures to support specialized application
         types
   \item The NULL indicator\index{NULL indicator} is supported
   \item Blob\index{blob} support using the Ada streams\index{streams, Ada} facility
   \item A wide range of native\index{native types} and builtin\index{builtin data types} data types are supported
   \item Database neutral API\index{neutral API, database} is now supported for most functions
\end{enumerate}

\subsection{Binding Type}

This library represents a thick Ada binding\index{thick binding} to the PostgreSQL C\index{C programmer}  programmer's
libpq\index{libpq} library %
\footnote{C++ programs can also make use of this library but there exists the
library libpq++ for C++ native support.%
}, and with version 2.0, MySQL's\index{MySQL} C programming library. As a thick
binding, there are consequently Ada objects\index{objects, Ada} and data types that are
tailored specifically to the Ada programmer. Some data types and objects
exist to mirror those used in the C language, while others are provided
to make the binding easier or safer to apply.

A thin binding\index{thin binding} would have required the Ada programmer to be continually
dealing with C language\index{C language data types} data type issues. Conversions to and from
various types and pointers\index{pointers} would be necessary making the use of the
binding rather tedious. Furthermore, the resulting Ada\index{Ada} program would
be much harder to read and understand.

A thick binding\index{thick binding} introduces new objects and types in order to provide
an API to the programmer. This approach however, fully insulates the
Ada\index{Ada} programmer from interfacing with C programs\index{C programs}, pointers\index{pointers} and strings\index{strings}.
The design goal\index{design goal} has additionally been to keep the number of new objects
and types to a minimum. This has been done without sacrificing convenience
and safety\index{safety}. Readability\index{readability} of the resulting Ada\index{Ada} program was also considered
to be important.

The objects and data types involved in the use of this binding can
be classified into the following main groups:

\begin{enumerate}
   \item Native\index{native data types} data types and objects
   \item Database \index{database objects} manipulation objects
   \item New database related objects and types for holding data
\end{enumerate}

Native data types need no explanation in this document. The database
manipulation objects will be described in 
\Ref{Database_Objects:Section}.
The following section will introduce the Ada types\index{Ada types} that are used to
hold data.


\section{Binding Data Types}

The PostgreSQL\index{PostgreSQL} database supports many standard SQL\index{SQL data types} data types as well
as a few exotic ones. This section documents the database base types\index{data types, database}
that are supported by the Ada\index{Ada binding} binding to the database. This list is
expected to grow with time as the Ada binding continues to mature
in its own software development.

The ``Data Type Name'' column in the following table refers to
a binding type\index{binding type} if the type name is prefixed with ``APQ\_''%
\footnote{Formerly, the PostgreSQL specific types had used a PG\_ prefix.%
}. These data types were designed to mimic common database data types
in use. They can be used as they are provided, or you may subtype
from them or even derive new types from them in typical Ada\index{Ada} fashion.
All other data types are references to native Ada data types (for
some of these, the package where they are defined are shown in the
``Notes'' column).

The column labelled ``Root Type''\index{root types} documents the data type that
the APQ\_ data type was derived\index{derived} from. Where they represent an Ada\index{subtype, Ada}
subtype, the column ``Subtype'' indicates a ``Y''. For type
derivations a ``N'' is shown in this column, indicating that the
APQ\_ type \index{APQ\_ types} listed is made ``unique''.


\subsection{PostgreSQL Data Types\label{PostgreSQL SQL Data Types}}

The ``Notes'' column of Table \ref{t:pqtypes} provides PostgreSQL notes, package names\index{package names} and
PostgreSQL\index{data types, PostgreSQL} data type names where the name is given in all capitals.

\begin{longtable}{|l|c|c|l|}
\hline 
Data Type Name    &  Root Type   &  Subtype  &  PostgreSQL Notes\\
\hline \hline 
Row\_ID\_Type     &  -           &  N        &  Used for blobs and rows\\
\hline 
String(<>)        &  -           &  -        &  Native Strings\\
\hline 
String(a..b)      &  -           &  -        &  Fixed length strings\\
\hline 
Unbounded\_String &  -           &  -        &  Ada.Strings.Unbounded\\
\hline 
Bounded\_String   &  -           &  -        &  Ada.Strings.Bounded\\
\hline 
APQ\_Smallint     &  -           &  N        &  SMALLINT\\
\hline 
APQ\_Integer      &  -           &  N        &  INTEGER\\
\hline 
APQ\_Bigint       &  -           &  N        &  BIGINT\\
\hline 
APQ\_Real         &  -           &  N        &  REAL\\
\hline 
APQ\_Double       &  -           &  N        &  DOUBLE PRECISION\\
\hline 
APQ\_Serial       &  -           &  N        &  SERIAL\\
\hline 
APQ\_Bigserial    &  -           &  N        &  BIGSERIAL\\
\hline 
APQ\_Boolean      &  Boolean     &  Y        &  BOOLEAN\\
\hline 
APQ\_Date         &  Ada.Calendar.Time & Y   &  DATE\\
\hline 
APQ\_Time         &  Ada.Calendar.Day\_Duration & Y & TIME\footnote{No timezone information}\\
\hline 
APQ\_Timestamp    &  Ada.Calendar.Time & N   & TIMESTAMP\footnote{No timezone information}\\
\hline 
APQ\_Timezone     &  Integer     &  N        &  range -23..23\\
\hline 
APQ\_Bitstring    &  -           &  N        &  BIT or BIT VARYING\\
\hline 
Decimal\_Type     &  -           &  -        &  PostgreSQL.Decimal\\
\hline 
range <>          &  -           &  -        &  Native Integers\\
\hline 
delta <>          &  -           &  -        &  Native Fixed Point\\
\hline 
digits <>         &  -           &  -        &  Native Floating Point\\
\hline 
delta <> digits <> & -           &  -        &  Native Decimal\\
\hline
\caption{PostgreSQL Data Types}\label{t:pqtypes}
\end{longtable}

The data type shown as ``Decimal\_Type''\index{Decimal\_Type}
is special, in that it is supported from a child package APQ\-.PostgreSQL\-.Decimal\index{APQ.PostgreSQL.Decimal}.
It represents a tagged\index{tagged type} type that provides an interface to the C routines\index{C routines}
used by the PostgreSQL database server, for arbitrary precision decimal\index{precision, arbitrary}
values.


\subsection{MySQL Data Types\label{MySQL Data Types}}

Table \ref{t:mytypes} summarizes the MySQL\index{data types, MySQL} specific data types and the
corresponding APQ data types.

\begin{longtable}{|l|c|c|l|}
\hline 
APQ Data Type        &  Ada Spec                   &  Subtype     &  Comments\\
\hline
\hline 
Row\_ID\_Type        &  unsigned 64 bits           &  N           &  For all databases\\
\hline 
APQ\_Smallint        &  signed 16 bits             &  N           &  SMALLINT\\
\hline 
APQ\_Integer         &  signed 32 bits             &  N           &  INTEGER\\
\hline 
APQ\_Bigint          &  signed 64 bits             &  N           &  BIGINT\\
\hline 
APQ\_Real            &  digits 6                   &  N           &  REAL\\
\hline 
APQ\_Double          &  digits 15                  &  N           &  DOUBLE PRECISION\\
\hline 
APQ\_Serial          &  range 1..2147483647        &  N           &  \emph{INTEGER}\\
\hline 
APQ\_Bigserial       &  range 1..2{*}{*}63         &  N           &  \emph{BIGINT}\\
\hline 
APQ\_Boolean         &  Boolean                    &  Y           &  BOOLEAN\\
\hline 
APQ\_Date            &  Ada.Calendar.Time          &  Y           &  DATE\\
\hline 
APQ\_Time            &  Ada.Calendar.Day\_Duration &  Y           &  TIME\\
\hline 
APQ\_Timestamp       &  Ada.Calendar.Time          &  N           &  TIMESTAMP\\
\hline 
APQ\_Timezone        &  range -23..23              &  N           &  \emph{Not in MySQL}\\
\hline 
APQ\_Bitstring       &  array(Positive) of APQ\_Boolean & N       &  \emph{Not in MySQL}\\
\hline
\caption{MySQL Data Types}\label{t:mytypes}
\end{longtable}

Notice the italicized SQL keywords\index{keywords, SQL} in the table. They identify the
SQL keywords that differ from PostgreSQL\index{PostgreSQL}. However, the programmer
only needs to be concerned with these SQL\index{keywords, SQL} keywords when creating new
tables or temporary tables. For example a column of type SERIAL\index{SERIAL} in
a PostgreSQL table, should be declared as a INTEGER\index{INTEGER} type in MySQL.


\section{Sybase Data Types}

The chart below outlines the mappings between APQ data types and Sybase
server data types\index{data types, Sybase}. The italicized SQL keywords
\index{keywords, SQL} in the table identify
the keywords that differ from PostgreSQL\index{PostgreSQL}. The following notes are
particular to Sybase\index{Sybase}:

\begin{itemize}	
   \item APQ\_Bigint is not completely range compatible with Sybase's \emph{NUMERIC}
         \index{APQ\_Bigint}\index{NUMERIC}\index{DECIMAL}
         (or \emph{DECIMAL}) server types. APQ\_Bigint will always accept Sybase server
         values, but the server will not be able to accept the full APQ\_Bigint range of values.
         To comply with the database server's range,
         the application designer should use:
\end{itemize}

\index{Sybase\_Bigint}
\begin{Code}
subtype Sybase_Bigint is APQ_Bigint
   range -10**38..10**38-1;
\end{Code}

\begin{itemize}
   \item APQ\_Bigserial\index{APQ\_Bigserial} is not supported.
   \item APQ\_Boolean\index{APQ\_Boolean}  maps to Sybase's data type \emph{BIT.}
   \item APQ\_Timezone\index{APQ\_Timezone} is not supported by any Sybase data type.
   \item APQ\_Bitstring\index{APQ\_Bitstring} is not supported.
\end{itemize}

Table \ref{t:sytypes} lists the APQ types supported for Sybase.

\begin{longtable}{|l|c|c|l|}
\hline 
APQ Data Type        &  Ada Spec          &  Subtype           &  Comments\\
\hline
\hline 
Row\_ID\_Type        &  unsigned 64 bits  &  N                 &  For all databases\\
\hline 
APQ\_Smallint        &  signed 16 bits    &  N                 &  SMALLINT\\
\hline 
APQ\_Integer         &  signed 32 bits    &  N                 &  INTEGER/INT\\
\hline 
APQ\_Bigint          &  signed 64 bits ($-10^{38}..$$10^{38}\textrm{-1}$) & N & \emph{NUMERIC/DECIMAL}\\
\hline 
APQ\_Real            &  digits 6          &  N                 &  REAL\\
\hline 
APQ\_Double          &  digits 15         &  N                 &  DOUBLE PRECISION\\
\hline 
APQ\_Serial          &  range 1..2147483647 & N                &  \emph{INTEGER}\\
\hline 
APQ\_Bigserial       &  range 1..$2^{63}$ &  N                 &  ???\\
\hline 
APQ\_Boolean         &  Boolean           &  Y                 &  \emph{BIT}\\
\hline 
APQ\_Date            &  Ada.Calendar.Time &  Y                 &  DATE\\
\hline 
APQ\_Time            &  Ada.Calendar.Day\_Duration & Y         &  TIME\\
\hline 
APQ\_Timestamp       &  Ada.Calendar.Time &  N                 &  \emph{DATETIME}\\
\hline 
APQ\_Timezone        &  range -23..23     &  N                 &  \emph{Not in Sybase}\\
\hline 
APQ\_Bitstring       &  array(Positive) of APQ\_Boolean & N    &  \emph{Not in Sybase}\\
\hline
\caption{Sybase Data Types}\label{t:sytypes}
\end{longtable}


\section{Database Objects\label{Database_Objects:Section}}

Much of the binding between Ada\index{Ada} and the database server is provided
through the use of tagged\index{tagged record} record types. Presently the APQ binding
operates through three object types\index{object types, APQ} listed in Table \ref{t:APQObj}.

\begin{table}
   \begin{center}
   \begin{tabular}{lll}
   Derived From            &  Object Type       &  Purpose\\
   \hline
   Root\_Connection\_Type  &  Connection\_Type  & Database connection\\
   Root\_Query\_Type       &  Query\_Type       & SQL interface\\
   N/A                     &  Blob\_Type        & Blob operations\\
   \end{tabular}
   \end{center}
   \caption{APQ Object Types}\label{t:APQObj}
\end{table}

The APQ objects from Table \ref{t:APQObj} are described more fully as follows:

\begin{description}
   \item[Connection\_Type] object is used with Query\_Type and Blob\_Type objects
         to perform SQL queries and blob operations respectively. This object
         maintains the connection to the database server.
   \item[Query\_Type] objects are used with one Connection\_Type object
         to perform SQL query operations. This object holds the text of the
         SQL query, some query state and result information. To execute a 
         query, it must be used in conjunction with a Connection\_Type object.
   \item[Blob\_Type] objects are used with one Connection\_Type object to 
         perform blob operations. This is currently only supported for PostgreSQL
         by APQ. All blob operations must occur within a transaction.
\end{description}

\begin{floatingtable}{
      \begin{tabular}{lc}
      Object            & Finalized \\
      \hline 
      Connection\_Type  & Yes \\
      Query\_Type       & Yes \\
      Blob\_Type        & No \\
      \end{tabular}}
\caption{Object Finalization Behavior}\label{t:Finalization}
\end{floatingtable}

Table \ref{t:Finalization} lists the three APQ objects and their finalization
behaviour.
While the Connection\_Type\index{Connection\_Type}
and Query\_Type\index{Query\_Type} objects are subject to
finalization, the Blob\_Type\index{Blob\_Type} is not.
This is because it represents an access type to
a Blob\_Object. This is similar in concept to an open a file, using
the File\_Type data type. This design approach was necessary
to support Ada\index{Ada} Streams\index{streams, Ada} oriented access for blobs.

\subsection{Object Hierarchy}

Before multiple database products were supported, the APQ object hierarchy\index{hierarchy}
was simple. To provide generic level support however, there are now
root\index{root objects} objects and derived objects\index{derived objects}. In most programming
contexts, the application writer does not need to be concerned with this fact.
However, if you frequently inspect the spec files\index{spec files} instead of the documentation\index{documentation},
you must be aware that primitives\index{primitives} for a given object may be declared
in multiple places. Examine Table \ref{t:pkghier} to review the APQ package hierarchy.

\begin{table}
   \begin{center}
      \begin{tabular}{ll}
         Package Name            & Description \\
         \hline 
         APQ                     & Root objects and primitives \\
         APQ.PostgreSQL          & Declarations and constants unique to PostgreSQL \\
         APQ.PostgreSQL.Client   & Derived objects and added primitives \\
         APQ.MySQL               & Declarations and constants unique to MySQL \\
         APQ.MySQL.Client        & Derived objects and added primitives \\
         APQ.Sybase              & Declarations and constants unique to Sybase \\
         APQ.Sybase.Client       & Derived objects and added primitves for Sybase \\
      \end{tabular}
   \end{center}
   \caption{APQ Package Hierarchy}\label{t:pkghier}
\end{table}

Table \ref{t:pkghier} illustrates that support for a given database is derived\index{derived}
from the APQ top level\index{top level package} package. Root objects\index{root level} are declared in APQ,
with common functionality. Some primitives must be overridden\index{overridden} by the
derived object in database specific packages. For example, APQ.Root\_Query\_Type
declares a primitive named Value (\emph{APQ.Value)} to return a string
column result. If this particular method is called, the exception
\emph{Is\_Abstract}\index{Is\_Abstract} will be raised to indicate that it must be overriden
with code to handle the specific database being used. Normally the
user would invoke APQ\-.MySQL\-.Client\-.Value when using the MySQL\index{MySQL} database,
for example. The programmer normally need not be aware of these details,
since object dispatching\index{dispatching} takes care of these details.

For this reason, the APQ\-.MySQL\-.Query\_Type object for example, is
derived from the APQ\-.Root\_Query\_Type\index{Root\_Query\_Type} object. This Query\_Type\index{Query\_Type} object
will provide its own implementation of the Value\index{Value} function to return
a column result, and will work as required.

Consequently, when looking for primitives\index{primitives} available to the Query\_Type
object, don't forget that many common primitives\index{common primitives} will be inherited\index{inherited}
from the APQ\-.Root\-\_Query\-\_Type object. The same is true for Connection\-\_Type\index{Connection\_Type}
objects. A number of common primitives are inherited from the APQ\-.Root\_Connection\_Type
object.


\chapter{Connecting to the Database}

Before any useful work can be accomplished by a database client program,
a connection must be established between the Ada program and the database
server. This chapter will demonstrate how to use the APQ binding\index{binding} to
enable a program to connect\index{connect} and disconnect\index{disconnect}
from the database server.

\section{The Connection\_Type}

The Connection\_Type\index{Connection\_Type} object holds everything that is needed to maintain
a connection\index{connection} to the database server. There are seven groups of primitive\index{primitive}
operations for this object:

\begin{enumerate}
   \item Context\index{Context} setting operations
   \item Connection\index{Connection} operations
   \item Connection Information functions\index{information functions}
   \item General Information operations
   \item Implicit operations (Finalization\index{Finalization})
   \item Trace Facilities\index{trace facilities}
   \item Generic Database Operations
\end{enumerate}

\section{Context Setting Operations}\label{Context_Setting_Operations}

These primitives ``configure''\index{configure} the connection\index{connection} that is
to be made later. When the object is initially created, it is in the
disconnected\index{disconnected state} state. While disconnected, configuration changes can be
made in to affect the next connection attempt. With the exception of the
database name\index{database name}, the application should not make configuration\index{configuration} changes
while the object is in the connected state\index{connected state}.
\footnote{This is not yet enforced by APQ.}

The configuration primitives\index{primitives} are listed in Table \ref{t:ctxops}.
\footnote{The items marked ``Root'' are primitives from APQ.Root\_Query\_Type.
The items marked {}``Derived'' are those overrides that are declared
on the APQ.{*}.Query\_Type object.}

\begin{table}
   \begin{center}
      \begin{tabular}{ccll}
      Type & Derivation & Name                  & Purpose \\
      \hline 
      proc & Root       & Set\_Host\_Name       & Set server host name \\
      proc & Root       & Set\_Host\_Address    & Set server host IP address \\
      proc & Root       & Set\_Port             & Set server port number \\
      proc & Root       & Set\_DB\_Name         & Set database name \\
      proc & Root       & Set\_User\_Password   & Set userid and password \\
      proc & Root       & Set\_Instance         & Set instance name to use \\
      proc & Root       & Set\_Options          & Set userid and password \\
      \end{tabular}
   \end{center}
\caption{Context Setting Primitives}\label{t:ctxops}
\end{table}

It would be nice if all database software products worked the same
way but the reality is very different. Table \ref{t:ctxdiffs} indicates how
the primitives apply by database product.\label{Set_Instance Support Chart}
In the table you can see that PostgreSQL\index{PostgreSQL} and MySQL\index{MySQL} establish their
connection by specifying the host, port and database name. Sybase
on the other hand, specifies this information in their \emph{interface}
file. To access the appropriate Sybase\index{Sybase interface entry} interface entry requires only
the \emph{instance}\index{instance} information supplied by the Set\_Instance\index{Set\_Instance} call.

\begin{table}
   \begin{center}
      \begin{tabular}{lccc}
      Primitive            & PostgreSQL   & MySQL  & Sybase \\
      \hline 
      Set\_Host\_Name      & Yes          & Yes    & Ignored \\
      Set\_Host\_Address   & Yes          & Yes    & Ignored \\
      Set\_Port            & Yes          & Yes    & Ignored \\
      Set\_DB\_Name        & Yes          & Yes    & See Note %
      \footnote{Set\_DB\_Name is useful after the connection is made.}\\
      Set\_User\_Password  & Yes          & Yes    & Yes \\
      Set\_Instance        & No           & No     & Yes \\
      Set\_Options         & Yes          & Yes    & Yes \\
      \end{tabular}
   \end{center}
\caption{Context Setting Differences}\label{t:ctxdiffs}
\end{table}

\subsection{PostgreSQL Defaults}

The PostgreSQL database defines certain environment\index{environment, PostgreSQL} variables that
can specify defaults. These and the fallback\index{fallback values} values are documented
in Table \ref{t:pqdef}.

\begin{table}
   \begin{center}
      \begin{tabular}{cclll}
      Type & Derivation    & Name               & Default      & Fallback \\
      \hline 
      proc & Root          & Set\_Host\_Name    & PGHOST       & localhost\\
      proc & Root          & Set\_Host\_Address & PGHOST       & localhost\\
      proc & Root          & Set\_Port          & PGPORT       & 5432\\
      proc & Root          & Set\_DB\_Name      & PGDATABASE   & LOGNAME\\
      proc & Root          & Set\_User\_Password &
         \begin{tabular}{l}
            PGUSER\\
            PGPASSWORD\\
         \end{tabular}
        & LOGNAME\\
      proc & Root          & Set\_Options       & PGOPTIONS    & ""\\
      \end{tabular}
   \end{center}
   \caption{PostgreSQL Context Defaults}\label{t:pqdef}
\end{table}

The capitalized names in the table for the ``Default'' and ``Fallback''
columns represent environment\index{environment} variable names.
When any of the environment\index{environment variables} variables are undefined in the ``Default''
column, the value used is determined by the ``Fallback'' value
listed. The fallback\index{fallback} variable name LOGNAME\index{LOGNAME}
is simply used to represent the current user's userid\index{userid}.%
\footnote{The PostgreSQL libpq library may in fact, completely ignore the LOGNAME
environment variable, and simply look up the userid in the /etc/password
file.%
} When no password\index{password} value is provided and no PGPASSWORD\index{PGPASSWORD}
environment variable exists, then no password is assumed.


\subsection{Procedure Set\_Host\_Name}

The Set\_Host\_Name\index{Set\_Host\_Name} procedure accepts the following arguments

\begin{Code}
procedure Set_Host_Name(
   C :         in out Connection_Type;
   Host_Name : in     String
);
\end{Code}

The following example configures the Connection\_Type object to connect
to host\index{host name} ``witherspoon'':

\begin{Example}
declare
   C : Connection_Type;
begin
   Set_Host_Name(C,"witherspoon");
\end{Example}

\begin{description}
\item [Note:]Sybase ignores this API call.
\end{description}

\subsection{Procedure Set\_Host\_Address}

The procedure takes two arguments, in the same fashion as Set\_Host\_Name\index{Set\_Host\_Address}:

\begin{Code}
procedure Set_Host_Address(
   C :            in out Connection_Type;
   Host_Address : in     String
);
\end{Code}

The following example configures the Connection\_Type object to connect
to IP address\index{IP address} 10.0.0.7:

\begin{Example}
declare
   C : Connection_Type;
begin
   Set_Host_Address(C,"10.0.0.7");
\end{Example}

\begin{description}
\item [Note:]Sybase ignores this API call.
\end{description}

\subsection{Procedure Set\_Port (IP)}

This procedure configures the port\index{port} where the database
server\index{server} is listening\index{listening} (when using TCP/IP\index{TCP/IP}
as the transport\index{transport}):

\begin{Code}
procedure Set_Port(
   C :           in out Connection_Type;
   Port_Number : in     Integer
);
\end{Code}

\begin{floatingtable}{
   \begin{tabular}{ccc}
   Database                &  Default  &  Port\\
   \hline 
   PostgreSQL              &  IP       &  5432\\
   MySQL                   &  UNIX     &  ?\\
   Sybase                  &  N/A      &  Ignored\\
   \end{tabular}}
   \caption{Default Database Connections}\label{t:defdb}
\end{floatingtable}

Table \ref{t:defdb} shows the connection defaults by database product. Some of these
are influenced by APQ. For example, APQ defaults to using a TCP/IP connection for the
PostgreSQL standard port of 5432.

The next example
shows how the port\index{port} number is configured.
The example configures APQ to use TCP/IP 
port number 5432\index{port 5432} to connect to
a PostgreSQL database.

\begin{Example}
declare
   C : Connection_Type;
begin
   Set_Port(C,5432);
\end{Example}

\begin{description}
\item [Note:]Sybase ignores this API call.
\end{description}

\subsection{Procedure Set\_Port (UNIX)}

To create a local UNIX socket connection to the database, there is
an overloaded version of Set\_Port\index{Set\_Port} that accepts a string\index{string} parameter
instead of an integer\index{port, integer} port number.

\begin{Code}
procedure Set_Port(
   C :           in out Connection_Type;
   Port_Number : in     String
);
\end{Code}

Specify a null string\index{null string} for the connection\index{connection} 
to use the local\index{local socket} UNIX\index{UNIX socket} socket default.

This parameter has changed somewhat as PostgreSQL\index{PostgreSQL} evolves. If you
are experiencing trouble getting a UNIX\index{UNIX socket} socket connection to work,
check the database documentation for PostgreSQL carefully. The following
example has been tested to work on PostgreSQL version 7.3.5.

\begin{Example}
declare
   C : Connection_Type;
begin
   Set_Port(C,"5432");
\end{Example}

See the troubleshooting chapter for tips.

\subsection{Procedure Set\_DB\_Name}

This procedure call configures the name of the database that the server
is to use \emph{prior} to the connection being established. Once
the connection has been established, calling Set\_DB\_Name\index{Set\_DB\_Name} switches the 
connection\index{connection} to use the named database.
The calling signature for this primitive is as follows:

\begin{Code}
procedure Set_DB_Name(
   C :       in out Connection_Type;
   DB_Name : in     String
);
\end{Code}

The following code fragment shows how the database name\index{database name} is configured
to be ``production'':

\begin{Example}
declare
   C : Connection_Type;
begin
   Set_DB_Name(C,"production");
\end{Example}

The Set\_DB\_Name \emph{always} acts as a configuration setting \emph{prior}
to connecting to the database server. Calling Set\_DB\_Name on a Connection\_Type\index{Connection\_Type}
object that has an open server connection to the database, can result
in hidden SQL commands for some databases.%
\footnote{For PostgreSQL and Sybase, a USE <database> command must be executed.%
} Table \ref{t:setdb} summarizes the behaviour according to database product.

\begin{table}
   \begin{center}
      \begin{tabular}{llll}
      Database Vendor   &  Before Connecting &  While Connecting           &  After Connected\\
      \hline 
      PostgreSQL        &  Configuration     &  uses config     &  ``USE'' Executed\\
      MySQL             &  Configuration     &  uses config.    &  MySQL Client Call\\
      Sybase            &  Pending SQL       &  ``USE'' executed   &  ``USE'' Executed\\
      \end{tabular}
   \end{center}
   \caption{Set\_DB\_Name Behaviour}\label{t:setdb}
\end{table}

From the table you can see that PostgreSQL\index{PostgreSQL} and MySQL\index{MySQL} configure the
database prior to connecting\index{connecting} (connecting also establishes the database
to be used). When Set\_\-DB\_\-Name is called after a connection has been
established, the database server will switch\index{switch database} to the requested database.
For some databases, this happens through the native\index{native client API} client API library\index{library}
(MySQL), but for others, a ``USE~<database>''\index{USE database} SQL\index{SQL} command must
be executed (internally within APQ).

\begin{floatingtable}{
   \begin{tabular}{ll}
   Exception Name    &  Reason\\
   \hline 
   Use\_Error        &  Unsuccessful doing a ``USE <database>''\\
   \end{tabular}}
   \caption{Set\_DB\_Name Exceptions}\label{t:sdbx}
\end{floatingtable}

Sybase\index{Sybase} is different again, because does not establish the database
at connect time\index{connect time} (Sybase will use the configured default database).
So a call to Set\_DB\_Name will queue a ``USE <database>''\index{USE database} SQL\index{SQL}
statement, to be executed by APQ, once the connection succeeds. If
another Set\_DB\_Name is performed while connected, new ``USE <database>''
SQL statements are executed on the Sybase server\index{Sybase server} connection provided.

The exceptions listed in Table \ref{t:sdbx}\index{exception} are possible when
using a new database name.

The case of the database name used is affected according to the case policy\index{case policy}
that is in effect for the connection\index{connection}. This is summarized in Table \ref{t:cpol}
for each database vendor product.

\begin{table}
   \begin{center}
      \begin{tabular}{llll}
      Database Product  &  Case Policy          &  Effect on DB\_Name      &  DB Name Sensitive\\
      \hline 
      PostgreSQL        &  Ignored              &  Case preserved          &  Yes\\
      MySQL             &  Ignored              &  Case preserved          &  Yes\\
      Sybase            &  Ignored              &  Case preserved          &  Yes\\
      \end{tabular}
   \end{center}
   \caption{Case Policy by Product}\label{t:cpol}
\end{table}

\subsection{Procedure Set\_User\_Password}

This procedure call configures both the userid\index{userid} and the password\index{password} together.
If there is no password, then supply the null string\index{null string}:

\begin{Code}
procedure Set_User_Password(
   C :             in out Connection_Type;
   User_Name :     in     String;
   User_Password : in     String
);
\end{Code}

The following code fragment illustrates how the userid and
password is configured:

\begin{Example}
declare
   C :   Connection_Type;
begin
   Set_User_Password(C,"myuserid","xyzzy");
\end{Example}

\subsection{Procedure Set\_Case}

One of the goals of APQ was to make writing of portable database\index{portable database code} code
possible. While this isn't completely possible, the goal remains to
reduce database differences. Until the introduction of Sybase\index{Sybase} within
APQ, there was little concern for the case of SQL query\index{SQL query code} code used. 

This APQ manual has always used examples where SQL code is uppercased\index{uppercased SQL code},
with Ada\index{Ada} code using normal Ada95\index{Ada95} conventions. This helps to make the
SQ\index{SQL}L code standout, and separate it from the other code. Sybase introduces
a problem however, since the Sybase server is case sensitive\index{case sensitive} when
referring to tables\index{tables} and column\index{column names} names. This feature cannot be disabled
for a Sybase server%
\footnote{MySQL databases can be configured to be caseless or case sensitive.%
}. Since users of Sybase\index{Sybase} tend to use lowercase\index{lowercase names} names, this creates
a bit of a problem for portable\index{portable SQL} SQL text within APQ.

The approach that APQ 2.2 uses for Sybase\index{Sybase} (and future databases where
this matters), all SQL\index{SQL} code is changed to lowercase\index{lowercase} before being
sent to the server. Before the reader panics thinking that this won't
work, it should be made clear that APQ does not change the case\index{case of SQL} of
any string\index{string} that is quoted\index{quoted} (using the Append\_Quoted\index{Append\_Quoted} functions for
example). A WHERE\index{WHERE clause} clause such as the following will preserve the text
within quotes\index{quotes}:

\begin{SQL}

   where part_no = "XZ98-307"

\end{SQL}

APQ keeps track of what parts of the SQL\index{SQL query} query were quoted\index{quoted} and which
parts were not. This is \emph{not} done by parsing\index{parsing} the SQL text. That
would be prone to error and would require intimate knowledge of every
SQL dialect supported in APQ. Instead, the string\index{string} fragments are marked
for ``preserve''\index{preserve case} case or ``not preserve''\index{preserved, not} as the Query\_Type\index{Query\_Type}
object collects text. When the full SQL text\index{SQL text} is require by a routine
like To\_String\index{To\_String}, then the standing case policy\index{case policy} is applied to each
string\index{string fragment} fragment that has ``not preserve''\index{preserved, not}.

You can change this APQ case policy\index{case policy} for Sybase (or any other database).
The Set\_Case\index{Set\_Case} function gives you complete control over what the policy
to use for SQL case\index{SQL case}. You can choose one of the SQL\_Case\_Type\label{SQL_Case_Type Choices}
\index{SQL\_Case\_Type} options, which are listed in Table~\ref{t:cpolicies}.

\begin{table}
   \begin{center}
      \begin{tabular}{ll}
         Case Policy    &  Description\\
         \hline 
         Preserve\_Case &  Do not make any case changes to the SQL text\\
         Lower\_Case    &  Force unquoted SQL text to lowercase\\
         Upper\_Case    &  Force unquoted SQL text to uppercase\\
      \end{tabular}
   \end{center}
   \caption{Case Policies for SQL Text}\label{t:cpolicies}
\end{table}

\begin{floatingtable}{
   \begin{tabular}{ll}
   Database       &  Default SQL Case Policy\\
   \hline 
   PostgreSQL     &  Upper\_Case\\
   MySQL          &  Lower\_Case\\
   Sybase         &  Lower\_Case\\
   \end{tabular}}
\caption{Default Case Policies}\label{t:dfcpol}
\end{floatingtable}

Table~\ref{t:dfcpol} summarizes the default case policies\index{case policies} used in APQ 2.2
for the different database vendor products supported.

MySQL\index{MySQL} joins Sybase\index{Sybase} in this case policy because by default, MySQL is
case sensitive. If you have configured your MySQL server to be case
insenstive, then any setting will do (Upper\_Case\index{Upper\_Case} is suggested for
the benefit of the trace log displays). Most users of Sybase\index{Sybase} will
likely prefer either the Lower\_Case\index{Lower\_Case} or Preserve\_Case\index{Preserve\_Case}
policies instead.

The Set\_Case\index{Set\_Case} API procedure for the Connection\_Type object is defined
as follows:

\begin{Code}
procedure Set_Case(
   C :        in out Connection_Type;
   SQL_Case : in     SQL_Case_Type
);
\end{Code}

You may also use the Get\_Case\index{Get\_Case} function primitive to find out what
the current policy in effect is. Please note that a change in policy
only affects the outcome of the next call to the following
pimitives of the Query\_Type\index{Query\_Type} object:

\begin{itemize}
   \item To\_String
   \item Execute
   \item Execute\_Checked
\end{itemize}

The Set\_Case\index{Set\_Case} call will not actually change the SQL text\index{SQL text} stored within
the Query\_Type object. It only sets the policy\index{case policy} mode.

The following example shows how to undo the Sybase default of lowercasing
the SQL text:

\begin{Example}
declare
   C : Connection_Type;
   Q : Query_Type;
begin
   ...
   Set_Case(C,Preserve_Case);
   ...
   Append(Q,"from table Mixed_Case");
   Execute(Q,C); -- SQL case is preserved here
\end{Example}

\subsubsection{The Viral Nature of Connection\_Type}

Note that APQ does permit altering the case policy\index{case policy} for Query\_Type\index{Query\_Type}
objects in addition to the Connection\_Type\index{Connection\_Type} object being described
here. It should be noted however, that the setting held by the Connection\_Type
is viral\index{viral nature} in nature. Whenever a Query\_Type object is used in conjunction
with a Connection\_Type object, in a \emph{Execute}\index{Execute} 
or \emph{Execute\_Checked}\index{Execute\_Checked}
call for example, the Query\_Type object will inherit the setting
held by the corresponding Connection\_Type object. For this reason,
the best application practice is to establish your application case
policy\index{case policy} in all Connection\_Type objects used by your application. It
is also best practice to use one Connection\_Type object wherever
possible, since many queries can share one connection\index{shared connection}.

The only reason you should consider applying a case policy\index{case policy} directly
to a Query\_Object, is perhaps for application specific uses of the
SQL text\index{SQL text}. For example, you could build a query\index{query, build a} in a Query\_Type object,
set the case policy to Upper\_Case\index{Upper\_Case}, and then use the To\_String\index{To\_String} function
primitive to extract out the SQL text for logging\index{logging} or user display\index{display}
purposes. Once however the Query\_Type is used with a Connection\_Type
however, (in \emph{Execute}) the Connection\_Type's setting will not
only prevail for the primitive, but will also force the Query\_Type's
policy to agree upon return from the call.


\subsection{Procedure Set\_Options\label{Procedure Set_Options}}

This procedure call permits the caller to specify any specialized
database server options. The options are specified in string form
with this API call. The specific options, and the format\index{format} of those
options\index{options} will vary according to the database being used. See the following
subsections for additional information about the database engine\index{engine, database} specifics.

The procedure Set\_Options\index{Set\_Options} is documented as follows:

\begin{Code}
procedure Set_Options(
   C :       in out Connection_Type;
   Options : in     String
);
\end{Code}

Exceptions\index{exceptions} can be raised if the options are being set on a connection
that is already connected. Table \ref{t:soopts} lists the exceptions that may be
raised by Set\_Options.

\begin{table}
   \begin{center}
      \begin{tabular}{ll}
      Exception Name    &  Reason\\
      \hline 
      Failed            &  The option is either unsupported or setting was rejected\\
      \end{tabular}
   \end{center}
   \caption{Set\_Options Exceptions}\label{t:soopts}
\end{table}

If the options are configured for an unconnected\index{unconnected} Connection\_Type
object, the exceptions if any, will be raised at the time of connection
(See the Connect primitive).

The following PostgreSQL\index{PostgreSQL} code fragment illustrates how two options\index{options}
may be configured:

\begin{Example}
declare
   C : Connection_Type;
begin
   Set_Options(C,"requiressl=1 dbname=test");
\end{Example}

Note that in this example, the option string has been used to declare
the database name to be used. Standard values should be set through
the primitive functions provided (ie. use Set\_DB\_Name\index{Set\_DB\_Name} instead).
Otherwise, when information primitives are added, you may not get
correct results. Any non-standard options like the ``requiressl''\index{requiressl}
option, should be configured as shown in this procedure call.

MySQL\index{MySQL} and Sybase\index{Sybase}, use a comma delimited list\index{delimited list} of options\index{options}.
The following example shows how to get Sybase I/O\index{statistics, Sybase I/O} 
and time statistics\index{statistics, time} recorded in your APQ trace\index{trace log} log:

\begin{Example}
declare
  C : Connection_Type;
begin
   Set_Options(C,"STATS_IO=TRUE,STATS_TIME=TRUE");
\end{Example}

Option parsing\index{parsing} is rather limited. Everything between the '=' sign
and the next comma is the value for the parameter. Different database
products will use different parameter types\index{paramater types}. For example, MySQL\index{MySQL} will
use 0 to represent False\index{false}, and 1 to represent True\index{true}. For Sybase\index{Sybase}, use
F or False, and T or True for Boolean\index{Boolean} values.


\subsubsection{PostgreSQL Options}

The documentation is not very clear about the format\index{format} of these options,
but it appears that keyword=value\index{keyword value pairs}  pairs\index{keyword=value pairs}
separated by \emph{spaces} for multiple options\index{options} are
accepted. If you must include spaces or other special characters within
the value component, then you must follow PostgreSQL\index{PostgreSQL} escaping rules\index{escaping rules}.
Refer to the database server documentation for these details.


\subsubsection{MySQL Options}

MySQL's\index{MySQL} C interface\index{C interface} is much different than PostgreSQL's C interface
for options. MySQL uses an enumerated value\index{enumerated value} and argument pair\index{argument pair} when
setting an option\index{option}.%
\footnote{Although, some options do not use the argument.%
} To keep the APQ interface friendly and consistent, APQ will accept
all options and arguments in a string\index{string} form as documented in
\Ref{Procedure Set_Options}. However, these string options\index{string options} must be
processed by APQ and digested into arguments usable by the MySQL C
client interface. Consequently, APQ must anticipate these options
and the option format in advance. For these reasons, the MySQL options
and their arguments will be partially documented here.

The format\index{format of option string} of the option string should be one or more option names
and arguments, separated by commas. Option names\index{option names} are treated as caseless\index{caseless}
(internally upcased\index{upcased}).

\begin{Example}

   Set\_Options(C,"CONNECT_TIMEOUT=3,COMPRESS,LOCAL_INFIL=1");

\end{Example}

Each option should be separated by a comma. APQ processes each option
in left to right fashion, making multiple MySQL C API calls for each
one. Table~\ref{t:myopts} lists the APQ supported options for MySQL.

\begin{table}
   \begin{center}
      \begin{tabular}{lll}
         Option Name          &  Argument Type           &  Comments\\
         \hline 
         CONNECT\_TIMEOUT     &  Unsigned                &  Seconds\\
         COMPRESS             &  None                    &  Compressed comm link\\
         NAMED\_PIPE          &  None                    &  Windows: use a named pipe\\
         INIT\_COMMAND        &  String                  &  Initialization command\\
         READ\_DEFAULT\_FILE  &  String                  &  See MySQL\\
         READ\_DEFAULT\_GROUP &  String                  &  See MySQL\\
         SET\_CHARSET\_DIR    &  String                  &  See MySQL\\
         SET\_CHARSET\_NAME   &  String                  &  See MySQL\\
         LOCAL\_INFIL         &  Boolean                 &  See MySQL\\
      \end{tabular}
   \end{center}
   \caption{MySQL Options}\label{t:myopts}
\end{table}

It is important to observe that any option that requires an argument\index{argument, option},
must have one. Any argument that requires an unsigned integer\index{unsigned}, must
have an unsigned integer (otherwise an exception\index{exception} is raised). A MySQL\index{MySQL}
Boolean\index{Boolean} argument should be the value 0 or 1. At the present time,
APQ gathers string\index{string} data up until the next comma or the end of the
string. Currently an option argument string cannot contain a comma
character.
\footnote{This needs to be corrected in a future release of APQ.}


\subsubsection{Sybase Options}

Sybase documents several options\index{options, Sybase} in their ``Open Client C Programmer's
Reference'' manual. All documented options are made available to
the APQ developer, although some options are not recommended. 

\begin{floatingtable}{
   \begin{tabular}{ll}
   Option Value   &  Meaning\\
   \hline 
   TRUE           &  True\\
   T              &  True\\
   FALSE          &  False\\
   F              &  False\\
   \end{tabular}}
   \caption{Boolean Option Values}\label{t:boolarg}
\end{floatingtable}

Each Sybase option value must conform to certain data type format\index{format, option}
and/or restrictions. Table~\ref{t:boolarg} lists
acceptable argument values for Boolean\index{Boolean} options, within APQ. The
case is not significant, but uppercase is recommended to make it stand out from
the surrounding Ada code.

The following is an example of a Boolean option setting for the Sybase option STATS\_IO:

\begin{Example}

   "STATS_IO=TRUE"

\end{Example}

Some Sybase options require an integer\index{unsigned integer} (actually unsigned). Those
options listed in the option table\index{option table} as requiring an Unsigned value,
should be simply an unsigned value. The following is an example:

\begin{Example}

   "ROWCOUNT=1"

\end{Example}

String value arguments\index{string value arguments} are simply textual values that are placed between
the equal sign and the next comma (or end of string). The following
example illustrates:

\begin{Example}

   "AUTHOFF=sa"

\end{Example}

Some options are identified as taking ``String/NULL''. These options
will take normal string values or simply the word NULL\index{NULL options}. The value
NULL allows the system default to be used. The following is an example:

\begin{Example}

   "CHARSET=NULL"

\end{Example}

\begin{table}
   \begin{center}
      \begin{tabular}{ll}
         Value             &  Sybase Option\\
         \hline 
         SUNDAY            &  CS\_OPT\_SUNDAY\\
         MONDAY            &  CS\_OPT\_MONDAY\\
         TUESDAY           &  CS\_OPT\_TUESDAY\\
         WEDNESDAY         &  CS\_OPT\_WEDNESDAY\\
         THURSDAY          &  CS\_OPT\_THURSDAY\\
         FRIDAY            &  CS\_OPT\_FRIDAY\\
         SATURDAY          &  CS\_OPT\_SATURDAY\\
      \end{tabular}
   \end{center}
   \caption{Weekday Option Argument Values}\label{t:wkday}
\end{table}

The DATEFIRST option\index{DATEFIRST option} accepts an argument Weekday argument. The valid
range of values are listed in Table~\ref{t:wkday}.

The following example shows how the DATEFIRST option is used:

\begin{Example}

   "DATEFIRST=SUNDAY"

\end{Example}

\begin{table}
   \begin{center}
      \begin{tabular}{ll}
      Value    &  Sybase Option\\
      \hline 
      MDY      &  CS\_OPT\_FMTMDY\\
      DMY      &  CS\_OPT\_FMTDMY\\
      YMD      &  CS\_OPT\_FMTYMD\\
      YDM      &  CS\_OPT\_FMTYDM\\
      MYD      &  CS\_OPT\_FMTMYD\\
      DYM      &  CS\_OPT\_FMTDYM\\
      \end{tabular}
   \end{center}
   \caption{Date Format Argument Values}\label{t:dtfmt}
\end{table}

One last category of Sybase option arguments is the DATEFORMAT argument\index{DATEFORMAT option}
type. The valid list of values for this type of argument are given in Table~\ref{t:dtfmt}.

The following is an example of such an option:

\begin{Example}

   "DATEFORMAT=YMD"

\end{Example}

Table~\ref{t:syonames} lists all of the options that APQ is able to recognize for Sybase.
Each of the option names\index{option names} listed is equivalent to the 
Sybase C macro\index{C macro, Sybase}
constant if the prefix CS\_OPT\_\index{CS\_OPT\_{*}} is added to the option name. Consult
the Sybase documentation for descriptions of the purpose of these
options and when to apply them.

\begin{longtable}{lll}
Option Name       &  Data Type            &  Notes\\
\hline 
ANSINULL          &  Boolean              &\\
ANSIPERM          &  Boolean              &\\
ARITHABORT        &  Boolean              &\\
ARITHIGNORE       &  Boolean              &\\
AUTHOFF           &  String               &\\
AUTHON            &  String               &\\
CHAINXACTS        &  Boolean              &\\
CHARSET           &  String/NULL          &  Use NULL for default\\
CURCLOSEONXACT    &  Boolean              &\\
DATEFIRST         &  Weekday              &\\
DATEFORMAT        &  YMD/MDY/etc.         &  May interfere with APQ\\
FIPSFLAG          &  Boolean              &\\
FORCEPLAN         &  Boolean              &\\
FORMATONLY        &  Boolean              &\\
GETDATA           &  Boolean              &\\
IDENTITYOFF       &  String               &\\
IDENTITYON        &  String               &\\
IDENTITYUPD\_OFF  &  String               &\\
IDENTITYUPD\_ON   &  String               &\\
ISOLATION         &  Unsigned             &  0, 1 or 3\\
NATLANG           &  String/NULL          &  Use NULL for default\\
NOCOUNT           &  Boolean              &\\
NOEXEC            &  Boolean              &\\
PARSEONLY         &  Boolean              &\\
QUOTED\_IDENT     &  Boolean              &\\
RESTREES          &  Boolean              &\\
ROWCOUNT          &  Unsigned             &\\
SHOWPLAN          &  Boolean              &  Use at your own risk\\
STATS\_IO         &  Boolean              &\\
STATS\_TIME       &  Boolean              &\\
STR\_RTRUNC       &  Boolean              &\\
TEXTSIZE          &  Unsigned             &\\
TRUNCIGNORE       &  Boolean              &\\
\caption{Sybase Option Names}\label{t:syonames}
\end{longtable}

Keep in mind that not all option settings will be compatible for use
with APQ. Changing server date formats\index{date formats} to anything other than year-month-day
format, is likely to cause problems within APQ, for example.


\subsection{Procedure Set\_Notice\_Proc}

The PostgreSQL\index{PostgreSQL} database
\footnote{The Set\_Notice\_Proc procedure is not available with MySQL.}
server sends notice\index{notice messages} messages back to the libpq\index{libpq} C library\index{library}, that the
APQ binding uses. These are received by a callback\index{callback}, after certain
database operations have been completed. While the messages are saved
in the Connection\_Type object (see also \Ref{Function Notice_Message}),
they overwrite each other as each new message comes in. For this reason,
it may be desireable for some applications to also receive a callback,
so that they can process the messages without losing them. The most
common reason to do this is to simply display them on standard error\index{standard error}.

The callback\index{callback}\index{Set\_Notice\_Proc} procedure must be defined as follows:
\marginpar{The default setting for any new Connection\_Type object is No\_Notify.}

\begin{Code}
procedure Notice_Callback(
   C :       in out Connection_Type;
   Message : in     String
);
\end{Code}

The Set\_Notice\_Proc takes an argument named Notify\_Proc\index{Notify\_Proc} that is
of the following type:

\begin{Code}
type Notify_Proc_Type is access
   procedure(
      C :       in out Connection_Type;
      Message : in     String
   );
\end{Code}

Finally, the Set\_Notice\_Proc procedure has the following calling signature:
\marginpar{Note that the Reset or Disconnect call will clear any
registered Notify procedure.}

\begin{Code}
procedure Set_Notice_Proc(
   C :           in out Connection_Type;
   Notify_Proc : in     Notify_Proc_Type
);
\end{Code}

This call can be made at any time to change the Notify\index{notify procedure} procedure.
The object may or may not be connected\index{connected}. The new procedure takes effect
immediately upon return, and will be used when the object is connected.
The present implementation only maintains one such procedure.%
\footnote{Note that the replaced procedure is not returned. A future implementation
of APQ may address this.%
}


\subsubsection{Disabling Notify}

The APQ\-.PostgreSQL\-.Client\index{APQ\-.PostgreSQL\-.Client} package 
provides the special constant No\_Notify\index{No\_Notify}
for the application programmer to use. An example of disabling notification\index{disabling notification}
follows:

\begin{Example}
declare
   C : Connection_Type;
begin
   ...
   -- Enable notify processing
   Set_Notify_Proc(C,My_Notify'Access);
   ...
   -- Disable notification
   Set_Notify_Proc(C,No_Notify);
\end{Example}

\subsubsection{Using Standard\_Error\_Notify}

During the debugging\index{debugging} phase of a database application, it may be useful
to simply have the notice messages\index{notice messages} printed on Standard\_Error\index{Standard\_Error}. To
do this, simply provide the access constant Standard\_Error\_Notify\index{Standard\_Error\_Notify}
as the second argument:

\begin{Example}
declare
   C : Connection_Type;
begin
   ...
   -- Send notices to stderr
   Set_Notify_Proc(C,Standard_Error_Notify);
   ...
\end{Example}

\section{Connection Operations}

The APQ binding provides three primitives for connecting\index{connecting} and disconnecting\index{disconnecting}
from the database server. They are summarized in Table~\ref{t:conprims}.

\begin{table}
   \begin{center}
      \begin{tabular}{lll}
         Type     &  Name              &  Purpose\\
         \hline
         proc     &  Connect           &  Connect to the database server\\
         proc     &  Disconnect        &  Disconnect from the database server\\
         proc     &  Reset             &  Disconnect if connected\\
      \end{tabular}
   \end{center}
   \caption{APQ Connection Primitives}\label{t:conprims}
\end{table}

\subsection{Procedure Connect}

This primitive initiates a connection attempt with the database server
as configured by the \Ref{Context_Setting_Operations} primitives.
If the connection succeeds, the procedure call returns\index{Connect}
successfully, leaving the Connection\_Type object in a connected state.


\begin{Code}
procedure Connect(
   C : in out Connection_Type
);
\end{Code}

\begin{table}
   \begin{center}
      \begin{tabular}{ll}
         Exception Name       &  Reason\\
         \hline 
         Not\_Connected       &  The connection attempt failed\\
         Already\_Connected   &  There is already a connection\\
         Failed               &  One or more configured options failed\\
         Use\_Error           &  Connection OK, but USE database failed\\
      \end{tabular}
   \end{center}
   \caption{Connect Exceptions}\label{t:conx}
\end{table}

Table~\ref{t:conx} summarizes the exceptions that Connect may raise.

The Already\_Connected\index{Already\_Connected} exception indicates that you need to disconnect
first, or use another Connection\_Type object if you are maintaining
multiple connections\index{connections, multiple}. Failed will be raised if a pending option setting
is unsupported or its setting has been rejected. The exception Use\_Error\index{Use\_Error}
indicates that the connection itself succeeded, but the selection
of the database failed. The connection will be returned in an unconnected\index{unconnected}
state when Use\_Error is raised.

\begin{description}
   \item[PostgreSQL Note:] \index{PostgreSQL}The Connect primitive as of APQ 1.91 automatically executes a 'SET
      DATESTYLE TO ISO' \index{ISO}\index{DATESTYLE} command to guarantee that the APQ date routines
      will function correctly, even when the PGDATESTYLE\index{PGDATESTYLE} environment variable
      may choose something other than ISO. This implies however, that APQ
      applications should always format date information in the ISO format.
   \item[MySQL Note:]When MySQL\index{MySQL} returns dates in YYYYMMDD\index{YYYYMMDD} format, APQ
      will automatically make the necessary adjustment, based upon the length
      of the result.
\end{description}

The following is an example call:

\begin{Example}
declare
   C : Connection_Type;
begin
   ...
   begin
      Connect(C);
   exception
      when No_Connection =>
         -- Handle connection failure
      when Already_Connected =>
         ...; -- Indicates program logic problem
      when others =>
         raise;
   end;
\end{Example}

\subsection{Connection Cloning\label{Connection Cloning}}

Application writers may want additional connections cloned\index{cloning connections} from a
given connection. A web server may want to do this for example. This
could be performed by obtaining all of the connection information
from the given connection and then proceed to configure a new connection,
but this is tedious and error prone. To clone a new connection from
an existing connection, simply use the Connect\index{Connect} primitive with the
following calling signature (argument Same\_As is added):

\begin{Code}
procedure Connect(
   C :       in out Connection_Type;
   Same_As : in     Connection_Type
);
\end{Code}

This primitive configures object C to use the same parameters as object
Same\_As\index{Same\_As}. It then creates a connection to the database using these cloned\index{cloned}
parameters. The exceptions that may be raised are listed in Table~\ref{t:cclonx}.

\begin{table}
   \begin{center}
      \begin{tabular}{ll}
         Exception Name       &  Reason\\
         \hline 
         Not\_Connected       &  The connection attempt failed\\
         Already\_Connected   &  There is already a connection\\
      \end{tabular}
   \end{center}
   \caption{Connection Cloning Exceptions}\label{t:cclonx}
\end{table}

The Not\_Connected\index{Not\_Connected} exception can be raised if the Same\_As connection
is not connected (it must be connected). This same exception can be
raised if the new connection fails (this should rarely happen unless
your database is suddenly taken down or a network failure occurs).
The Already\_Connected\index{Already\_Connected} exception is raised if \emph{C} is already
connected.

\begin{description}
   \item[Note:] Note that the trace settings of the Same\_As object are not
      carried to the new object C. You must manually configure any trace
      settings you require in the newly connected object C.
\end{description}

The following example shows how a procedure My\_Subr can clone
a new connection:

\begin{Example}
procedure My_Subr(C : Connection_Type) is
   C2 : Connection_Type;
begin
   Connect(C2,C); -- Clone a connection
\end{Example}

\subsection{Procedure Disconnect}

The Disconnect\index{Disconnect} primitive closes the connection that was previously
established in the Connection\_Type\index{Connection\_Type} object. The Disconnect primitive
uses the following arguments:

\begin{Code}
procedure Disconnect(
   C : in out Connection_Type
);
\end{Code}

The list of possible exceptions for Disconnect are illustrated in Table~\ref{t:dconx}.

\begin{table}
   \begin{center}
      \begin{tabular}{ll}
         Exception Name       &  Reason\\
         \hline 
         No\_Connection       &  There is no connection to disconnect\index{No\_Connection}\\
      \end{tabular}
   \end{center}
   \caption{Disconnect Exceptions}\label{t:dconx}
\end{table}

The following code fragment shows the procedure call in action:

\begin{Example}
declare
   C : Connection_Type;
begin
   ...
   begin
      Disconnect(C);
   exception
      when No_Connection =>
         ...; -- Indicates program logic problem
      when others =>
         raise;
   end;
\end{Example}

\subsection{Procedure Reset}

The Reset\index{Reset} primitive is provided so that the programmer can recycle
the Connection\_Type object for use in a subsequent connection. Without
this primitive, the user would need to destroy the original and create
a new Connection\_Type. The Reset primitive accepts the following
arguments:

\begin{Code}
procedure Reset(
   C : in out Connection_Type
);
\end{Code}

In addition to closing the current connection, if it is open, the
notification procedure is also deregistered (if there was a Set\_Notify\_Proc\index{Set\_Notify\_Proc}
performed).

No exceptions should occur. If there is a connection pending\index{pending connection}, it is
disconnected\index{disconnected}. If there is no connection pending, the call is ignored.
The following shows an example of its use:

\begin{Example}
declare
   C : Connection_Type;
begin
   ...
   Reset(C);  -- C is now ready for re-use
\end{Example}

\section{Connection Information Operations}

A modular\index{modular software} piece of software may get handed a Connection\_Type object
as a parameter, and have a need to inquire about the details of the
provided connection. The function primitives that return information
about the connection are listed in Table~\ref{t:cinfp}.

\begin{table}
   \begin{center}
      \begin{tabular}{ll}
         Function Name     &  Information Returned\\
         \hline 
         Host\_Name        &  Host name of the connection\\
         Port              &  Port Number or Port Pathname\\
         DB\_Name          &  Database name\\
         User              &  User name for the database\\
         Password          &  Password for the database\\
         Instance          &  Instance name for the database\\
         Options           &  Database option parameters\\
      \end{tabular}
   \end{center}
   \caption{Connection Information Primitives}\label{t:cinfp}
\end{table}

These function primitive specifications are listed below:

\begin{Code}
function Host_Name(
   C : Connection_Type;
) return String;
\end{Code}

\begin{Code}
function Port(
   C : Connection_Type;
) return String;  -- UNIX path
\end{Code}

\begin{Code}
function Port(
   C : Connection_Type;
) return Integer; -- TCP/IP port
\end{Code}

\begin{Code}
function DB_Name(
   C : Connection_Type;
) return String;
\end{Code}

\begin{Code}
function User(
   C : Connection_Type;
) return String;
\end{Code}

\begin{Code}
function Password(
   C : Connection_Type;
) return String;
\end{Code}

\begin{Code}
function Instance(
   C : Connection_Type;
) return String;
\end{Code}

\begin{Code}
function Options(
   C : Connection_Type;
) return String;
\end{Code}

The Port\index{Port} primitive that returns a String is for use with database
connections using a UNIX socket\index{UNIX socket}\index{local socket}. The socket
pathname\index{pathname} is returned in this case.
\footnote{or at least a fragment of the pathname.}

When called on Connection\_Type objects without a current connection,
an empty string\index{string, empty} is returned for any value that has not been configured
(for example if Set\-\_Host\-\_Name\index{Set\_Host\_Name} has not been called, Host\-\_Name\index{Host\_Name} will
return ""). If the value has been set, then that value is returned
as expected. Once the Connection\-\_Type object is connected to the
database however, the values will be values fetched from the library\index{library}
libpq\index{libpq} instead.
\footnote{Normally, these values should agree with what was configured.}

The following code sample shows how to extract the host name\index{host name} and database
name for the current connection.

\begin{Example}
procedure My_Code(C : in out Connection_Type) is
   Host_Name :     String := Host_Name(C);
   Database_Name : String := DB_Name(C);
begin
   ...
\end{Example}

\begin{description}
\item [Sybase Note:]The Instance function primitive was added to APQ support Sybase.
   See the support level chart for Set\_Instance\index{Set\_Instance} on page \pageref{Set_Instance Support Chart}
   and other information primitives. Note also that setting and retrieving
   other parameters is possible in some cases, even though the information
   is ignored by Sybase.
\end{description}

\section{General Information Operations}

Due to the modular construction of software, it is sometimes necessary
to query an object for its present state. The primitives listed in Table~\ref{t:gifc}
for the Connection\_Type object are available for querying the state\index{query the state}
of the object. These are discussed in the next subsections.

\begin{table}
   \begin{center}
      \begin{tabular}{lll}
         Type  &  Name              &  Purpose\\
         \hline 
         func  &  Is\_Connected     &  Indicates connected state\\
         func  &  Error\_Message    &  Returns a error message text\\
      \end{tabular}
   \end{center}
   \caption{General Information for Connections}\label{t:gifc}
\end{table}

\subsection{Function Is\_Connected}

The Is\_Connected\index{Is\_Connected} function returns a Boolean result that indicates
the present state of the Connection\_Type object. The arguments are
as follows:

\begin{Code}
function Is_Connected(
   C : Connection_Type
) return Boolean;
\end{Code}

There are no exceptions raised by this primitive.

The following example shows how to test if the object C is currently
supporting a connection. The example disconnects from the server,
if it determines that C is connected.

\begin{Example}
declare
   C : Connection_Type;
begin
   ...
   if Is_Connected(C) then
      Disconnect(C);
      ...
\end{Example}

\subsection{Function Error\_Message}

The Error\_Message\index{Error\_Message} function makes it possible for the application
to report why the connection failed. This information is often crucial
to the user of a failed application. The arguments accepted are as
follows:

\begin{Code}
function Error_Message(
   C : Connection_Type
) return String;
\end{Code}

There are no exceptions raised by this function. If there is no present
connection or no present error to report, the null string is returned.
The following example shows how the connection failure is reported:

\begin{Example}
with Ada.Text_IO;
...
declare
   use Ada.Text_IO;

   C : Connection_Type;
begin
   ...
   begin
      Connect(C);
   exception
      when No_Connection =>
         Put_Line(Standard_Error,"Connection Failed!");
         Put_Line(Standard_Error,Error_Message(C));
         ...
      when Already_Connected =>
         ...;  -- Program logic error here
      when others =>
         raise;
   end;
\end{Example}

\subsection{Function Notice\_Message\label{Function Notice_Message}}

The C libpq\index{libpq} interface library\index{library}\footnote{The Notice\_Message function is
not available for MySQL.} provides the APQ binding with certain
notification messages\index{notification messages} during some calls, by means of a callback\index{callback}. Each
time one of these notifications is received from the database server,
the notification message is saved in the Connection\_Type object
(replacing any former notice message). The last notification message
received can be retreived using the Notice\_Message\index{Notice\_Message} function:

\begin{Code}
function Notice_Message(
   C : Connection_Type
) return String;
\end{Code}

No exception is raised, and the null string\index{null string} is returned if no notice
message has been registered.

The following example illustrates one example of the Notice\_Message
function:

\begin{Example}
with Ada.Text_IO;
...
declare
   use Ada.Text_IO;

   C : Connection_Type;
begin
   ...
   declare
      Msg : String := Notice_Message(C);
   begin
      if Msg'Length > 0 then
         Put_Line(Standard_Error,Msg);
         ...
\end{Example}

\subsection{In\_Abort\_State Function\label{In_Abort_State Function}}

\Ref{Abort_State exception} documents the Abort\_State\index{Abort\_State}
exception, which is unique to PostgreSQL\index{PostgreSQL}. This exception is raised in
response to a status flag stored in the
APQ\-.PostgreSQL\-.Client\-.Connection\_Type object. When a transaction is
started, any SQL error\index{SQL error} will put the PostgreSQL database server into an
``\emph{abort state}'', where all current and future commands will be
ignored, for the connection \footnote{MySQL does not support this
concept, and so it does not go into an abort state.}. To permit the
application programmer to query this status, the In\_Abort\_State\index{In\_Abort\_State}
function can be used. It returns True, if an error has occurred within a
transaction, which requires a Rollback\_Work (\Ref{Begin, Commit and Rollback Work functions})
call to clear this state. The calling
requirements are summarized in the following table:

\begin{Code}
function In_Abort_State(
   C : Connection_Type
) return Boolean;
\end{Code}

Exceptions for In\_Abort\_State are summarized in Table~\ref{t:iasx}.

\begin{table}
   \begin{center}
      \begin{tabular}{ll}
         Exception Name    &  Reason\\
         \hline 
         Not\_Connected    &  There is no connection to query\\
      \end{tabular}
   \end{center}
   \caption{In\_Abort\_State Exceptions}\label{t:iasx}
\end{table}

The following example shows how this function might be used:

\begin{Example}
declare
   C : Connection_Type;
   Q : Query_Type;
begin
   ...
   Begin_Work(Q,C);
   ...
   Execute(Q,C);
   ...
   if In_Abort_State(C) then
      Rollback(Q,C);
      ...
   end if;
\end{Example}

\section{Implicit Operations}

There are a few implicit operations\index{implicit operations} that are performed that the programmer
should be aware of. They are:

\begin{itemize}
   \item The Connection\_Type is subject to Finalization
   \item A default Commit/Rollback operation can occur at Finalization
\end{itemize}

The programmer is encouraged to call Commit\_Work\index{Commit\_Work} or Rollback\_Work\index{Rollback\_Work}
explicitly, whenever possible. This way, the programmer is in complete
control of the transaction outcome.

If a transaction has not been committed or rolled back, and the connected
Connection\_Type object is finalized\index{finalization}%
\footnote{Usually because the Connection\_Type object has fallen out of scope.%
}, then the default action for commit or rollback occurs. The default
for the APQ binding is to rollback the transaction, when the connection
is still active. If the programmer has disconnected\index{disconnected} from the database
prior to finalization, then no further action occurs. To change or
control the default action, use the Set\_Rollback\_On\_Finalize\index{Set\_Rollback\_On\_Finalize} procedure
described in the next section.


\subsection{Set\_Rollback\_On\_Finalize Procedure\label{Set_Rollback_On_Finalize Procedure}}

The Set\_Rollback\_On\_Finalize primitive allows the programmer to
change the default action for the Connection\_Type object. The calling
requirements are summarized in the following table:%

\begin{Code}
procedure Set_Rollback_On_Finalize(
   C :        in out Connection_Type;
   Rollback : in     Boolean
);
\end{Code}

To change the default to COMMIT WORK\index{COMMIT WORK} when the Connection\_Type object
finalizes, peform the following call:

\begin{Example}
declare
   C : Connection_Type;
begin
   Set_Rollback_On_Finalize(C,False); -- Commit on Finalize
\end{Example}

\subsection{Will\_Rollback\_On\_Finalize Function\label{Will_Rollback_On_Finalize Function}}

Programs sometimes need to inquire about the state of the Connection\_Type
object that they may have been passed. To inquire about the commit
or rollback default, the Will\_Rollback\_On\_Finalize\index{Will\_Rollback\_On\_Finalize} function can
be called. The following table summarizes the calling requirements:

\begin{Code}
function Will_Rollback_On_Finalize(
   C : Connection_Type
) return Boolean;
\end{Code}

\section{Trace Facilities\label{Trace Facilities}}

No matter how carefully a programmer writes a new program, problems
develop that are often difficult to understand. Good tracing facilities\index{trace facilities}
allow the problem to be quickly understood and corrected.

To gain trace support using APQ, it is only necessary to perform the
following steps:

\begin{enumerate}
   \item Open a trace capture file with Open\_DB\_Trace
   \item Optionally enable/disable tracing at various points in the program
      with Set\_Trace%
      \footnote{Tracing is enabled by default after a Open\_DB\_Trace call.}
   \item Perform your SQL operations
   \item Close the trace capture file with Close\_DB\_Trace%
      \footnote{Or allow it to be closed when the Connection\_Type object is finalized.}
\end{enumerate}

The Open\_DB\_Trace\index{Open\_DB\_Trace} procedure takes a Trace\_Mode\_Type\index{Trace\_Mode\_Type} parameter
that decides what trace content is being collected. The valid enumerated
choices are listed in Table~\ref{t:tmchoic}.

\begin{table}
   \begin{center}
      \begin{tabular}{ll}
         Value           & Description\\
         \hline
         APQ.Trace\_None & Collect no trace information (no file is written/created)\\
         APQ.Trace\_DB   & Collect only C library trace information%
            \footnote{Prior to APQ 2.0, this was Trace\_libpq.}\\
         APQ.Trace\_APQ  & Collect only APQ SQL trace information\\
         APQ.Trace\_Full & Collect both database library and Trace\_APQ information\\
      \end{tabular}
   \end{center}
   \caption{Trace\_Mode\_Type Choices}\label{t:tmchoic}
\end{table}

The Trace\_None\index{Trace\_None} value is provided so that the Open\_DB\_Trace procedure
does not need to be coded around if a trace variable is supplied,
which may or may not request tracing. Close\_DB\_Trace\index{Close\_DB\_Trace} can be called
on a Connection\_Type\index{Connection\_Type} for which Trace\_None is in effect, without
any exception being thrown (the call is ignored).

Trace\_DB\index{Trace\_DB} provides only what the C library (libpq\index{libpq} for PostgreSQL\index{PostgreSQL})
provides. This may be useful to the database software maintainers,
if they want a trace of the activity that you are reporting problems
with.

Trace\_APQ\index{Trace\_APQ} is what the author considers to be the most useful output
format to an APQ developer. The trace output in this mode is such
that the extra trace information is provided in SQL comment\index{SQL comment} form.
The actual queries that are executed are in their natural SQL form.
The captured Trace\_APQ trace then, is in a format that can be played
back, reproducing exactly what the application performed.%
\footnote{There are limitations however, since the blob functions are not traced
at the present release.%
} The full trace or portions of it then can be used to help debug SQL
related problems. 

The following shows a sample of what the Trace\_APQ output looks like:

\begin{SQL}
-- Start of Trace, Mode=TRACE_APQ
-- SQL QUERY:
BEGIN~WORK}
;
-- Result: 'BEGIN'

-- SQL QUERY:
INSERT INTO DOCUMENT
       (NAME,DOCDATE,BLOBID,CREATED,MODIFIED,ACCESSED)
VALUES ('compile.adb','2002-08-12~21:09:25',3339004,
        '2002-08-12~21:59:48','2002-08-12~21:09:25',
        '2002-08-19~22:11:36')
;
-- Result: 'INSERT 3339005 1'

-- SQL QUERY:
SELECT DOCID
FROM DOCUMENT
WHERE OID = 3339005
;
-- Result: 'SELECT'
...
-- SQL QUERY:
COMMIT WORK
;
-- Result: 'COMMIT'
-- End of Trace.
\end{SQL}

The following subsections describe the primitives that provide support
for trace facilities.


\subsection{Procedure Open\_DB\_Trace}

To start any capture of trace\index{trace capture} information, you must specify the name
of the text file to be written to. The file must be writable to the
current process. The Connection\_Type object must be connected prior
to calling Open\_DB\_Trace\index{Open\_DB\_Trace}:

\begin{Code}
procedure Open_DB_Trace(
   C :        in out Connection_Type;
   Filename : in     String;
   Mode :     in     Trace_Mode_Type := Trace_APQ
);
\end{Code}

Table \ref{t:odbtx} lists the possible exceptions for Open\_DB\_Trace.

\begin{table}
   \begin{center}
      \begin{tabular}{ll}
         Exception Name & Reason\\
         \hline 
         Not\_Connected & There is no connection\\
         Tracing\_State & Trace is already enabled\\
      \end{tabular}
   \end{center}
   \caption{Open\_DB\_Trace Exceptions}\label{t:odbtx}
\end{table}

Upon return from the Open\_DB\_Trace procedure, a text file will be
created and ready to have trace entries written to it.%
\footnote{Note that trace entries are buffered by C standard I/O routines, so
trace information may be held in memory buffers before it is flushed
out or closed.%
}

The following example shows how a call might be coded:

\begin{Example}
declare
   C : Connection_Type;
begin
   ...
   Open_DB_Trace(C,"./bugs.sql",Trace_APQ);
\end{Example}

\subsection{Procedure Close\_DB\_Trace}

Closing the tracing facility for a connection, suspends all further
trace writes. Once this has been done, the effect of Set\_Trace\index{Set\_Trace} is
superceeded, preventing any further trace information being written.
The calling requirements are outlined in the following table:

\begin{Code}
procedure Close_DB_Trace(
   C : in out Connection_Type
);
\end{Code}

If the Open\_DB\_Trace call was made with the Mode parameter set to
Trace\_None, then the call to Close\_DB\_Trace\index{Close\_DB\_Trace} has no effect and is
ignored for programmer convenience.

No exceptions are raised.

An example call is shown below:

\begin{Example}
declare
   C : Connection_Type;
begin
   ...
   Open_DB_Trace(C,"./bugs.sql",Trace_APQ);
   ...
   Close_DB_Trace(C);
\end{Example}

\subsection{Procedure Set\_Trace}

In large applications where large numbers of SQL statements are executed,
it may be desirable to trace only certain parts of its execution in
a dynamic fashion. The Set\_Trace\index{Set\_Trace} primitive gives the programmer a
way to disable and re-enable tracing at strategic\index{strategic tracing} points within the
application. The calling requirements are summarized as follows:

\begin{Code}
procedure Set_Trace(
   C :        in out Connection_Type;
   Trace_On : in     Boolean := True
);
\end{Code}

Tracing is enabled by default, after a successful call to Open\_DB\_Trace
is made (unless Mode was Trace\_None\index{Trace\_None}).

There are no exceptions raised.

Note that it is considered safe to invoke Set\_Trace, even if a former
Open\_DB\_Trace call was not successfully performed, or the trace
mode was Trace\_None. This allows the application to retain strategic
Set\_Trace calls without having to remove them, when the Open\_DB\_Trace
call is disabled%
\footnote{Setting Mode to Trace\_None effectively disables the trace facility
without requiring any code changes.%
} or commented out.


\subsection{Function Is\_Trace}

It may be helpful to the developer that is tracking down a problem
to know when tracing is enabled or not. The Is\_Trace\index{Is\_Trace} function returns
true when the trace collection file is receiving trace information.
The calling arguments are listed below:%

Note that the returned value tracks the last value provided by
Set\_Trace, whether or not an open trace file has been created/opened.

\begin{Code}
function Is_Trace(
   C : Connection_Type;
) return Boolean;
\end{Code}

Note that the initial state of the Connection\_Type object is to have
Is\_Trace to return True. Also after a successful Open\_DB\_Trace,
Is\_Trace will return True.

An example showing its use is given below:

\begin{Example}
declare
   C : Connection_Type;
begin
   ...
   Open_DB_Trace(C,"./bugs.sql",Trace_APQ);
   ...
   if Is_Trace(C) then
      -- We are collecting trace info..
\end{Example}

\section{Generic Database Operations}

APQ 2.0 and later is designed so that all but the most specialized
database operations, can be performed, given only a Root\_Connection\_Type'Class\index{Root\_Connection\_Type'Class}
object (declared in top level package APQ). The following sections
describe some generic database related primitives that are necessary
for successful generic database support.


\subsection{Package APQ}

Root object support\index{root object support} is provided in the package APQ\index{APQ, package}. Generic database
code will normally only use this package:

\begin{Code}
   with APQ;

   use APQ;  -- Optional use clause
\end{Code}

The data types that will be used will be:

\begin{itemize}
   \item APQ.Root\_Connection\_Type'Class
   \item APQ.Root\_Query\_Type'Class\index{Root\_Query\_Type'Class}
\end{itemize}

The generic primitives that will be covered in the next section are:

\begin{itemize}
   \item APQ.Engine\_Of
   \item APQ.New\_Query
\end{itemize}

\subsection{Predicate Engine\_Of\label{Generic Database Engine_Of}}

Given a Root\_Connection\_Type'Class object, generic database code
sometimes needs to determine which specific database is being used.
This allows the code to make special SQL syntax changes, depending
upon the technology being used (for example, MySQL\index{MySQL} permits the use
of a \emph{LIMIT}\index{LIMIT} keyword in queries).

The Engine\_Of\index{Engine\_Of} primitive (dispatching) will identify the
database technology that is being used:

\begin{Code}
type Database_Type is (
   Engine_PostgreSQL,
   Engine_MySQL,
   Engine_Sybase
);

function Engine_Of(
   C : Connection_Type
) return Database_Type;
\end{Code}

The following example code shows how to test if a PostgreSQL\index{PostgreSQL} database
is being used:

\begin{Example}
with APQ; use APQ;
...
procedure App(C : Root_Connection_Type'Class) is
begin
   ...
   if Engine_Of(C) = Engine_PostgreSQL then
      ...
\end{Example}

\subsection{Primitive New\_Query\label{Query_Type Factories}}

Normally, an application database procedure will receive a connection
object as one of its input parameters. Generally, this connection
is established in the main program and then used by the program components
as required. However, to pass the parameter in a generic way (allowing
for polymorphism), you would declare the procedure's argument as receiving
data type Root\_Connection\_Type'Class.

Within the called procedure however, you will need a Query\_Type object.
This too could be passed in as an argument, but this is unnecessary.
At other times, you may need additional Query\_Type objects for temporary
use. What you need is a convenient way to create a Query\_Type object
that matches the connection that you have received as a parameter. This
is done by the New\_Query function\index{New\_Query}.

If your connection object is a:

\begin{Code}

   APQ.PostgreSQL.Client.Connection_Type

\end{Code}

object, then your application will want to create a:

\begin{Code}

   APQ.PostgreSQL.Client.Query_Type

\end{Code}

object. You want to avoid tests like:

\begin{Example}

if Connection is in APQ.PostgreSQL.Client.Connection_Type then
   ...
elsif Connection is in APQ.MySQL.Client.Connection_Type then
   ...
elsif Connection is in APQ.Sybase.Client.Connection_Type then
   ...

\end{Example}

The above example code would force your portable code to also ``with''\index{with}
the following packages:

\begin{Example}
with APQ.PostgreSQL.Client;
with APQ.MySQL.Client;
...
\end{Example}

which would be very inconvenient and unnecessary.

To make portable\index{portable} database programming easier, APQ provides a dispatching\index{dispatching}
Query\_Type object factory primitive that can be used for this purpose.
For example:

\begin{NumberedExample}
with APQ;
use APQ;

procedure My_Generic_App(C : Root_Connection_Type'Class) is
   Q : Root_Query_Type'Class := New_Query(C);\label{Ex:New_Query_Assgn}
begin
   Prepare(Q,"SELECT NAME,INCOME");
   Append_Line(Q,"FROM~SALARIES");
\end{NumberedExample}

The assignment to Q in line~\ref{Ex:New_Query_Assgn}, shows the
application of the primitive New\_Query. This dispatching
primitive returns the correct Query\_Type object that matches the
connection that was given. The primitive New\_Query is more formally
presented as follows:

\begin{Code}
function New_Query(
   C : Root_Connection_Type
) return Root_Query_Type'Class;
\end{Code}

\subsection{Query\_Type Assignment\label{Query_Type Cloning}}

Prior to APQ version 2.0, the Query\_Type object was a \emph{limited}\index{limited tagged type}
tagged type. This meant that the Query\_Type object was never able
to be assigned to another Query\_Type object. With the need for a
factory\index{factory} primitive like \textbf{New\_Query} it was necessary to lift
that restriction (otherwise the factory was unable to return the created
object). So the Root\_Query\_Type and derived forms, permit assignment
as of APQ version 2.0 and later.

When a Query\_Type is assigned in APQ, nothing spectacular happens.
In fact, the contents of the object on the right hand side are effectively
ignored, leaving a new object on the left side. The following example
shows how Q1 and Q2 are essentially the same:

\begin{NumberedExample}
declare
   Q0 : Query_Type;
   Q1 : Query_Type;
   Q2 : Query_Type;
begin
   ...
   Q1 := Q0;   -- Q1 becomes initialized (Q0 ignored)\label{Ex:Q1}
   Clear(Q2);  -- Initialize Q2\label{Ex:Q2}
\end{NumberedExample}

In this example, both Q1~(line~\ref{Ex:Q1}) and Q2~(line~\ref{Ex:Q2})
end up in the same state, and no state information is taken from Q0. You
might wonder why this would be implemented. The following example code
fragment illustrates why this is convenient and useful:

\begin{NumberedExample}
with APQ;
usse APQ;

procedure My_Generic_App(C : Root_Connection_Type'Class) is
   Q : Root_Query_Type'Class := New_Query(C);
begin
   Prepare(Q,"SELECT NAME, INCOME");
   Append(Q, "FROM~SALARIES");
   Execute(Q,C);
   ...
   declare
      Q2 : Root_Query_Type'Class := Q;\label{Ex:Q_Clone}
   begin
      ...
   end;
\end{NumberedExample}

The example illustrates that the assignment (line~\ref{Ex:Q_Clone}) is
simply a convenient factory of its own kind. It is also likely to be
slightly more efficient than the New\_Query primitive on the
connection. Think of assignment of Query\_Type objects as cloning\index{cloning}
operations. The assigned object becomes a fresh initialized clone of the
Query\_Type object on the right hand side of the assignment.

\chapter{SQL Query Support}

Once a database connection has been established, the application is
ready to invoke operations on the database server. To ease the programmer's
burden in keeping track of the various components involved in these
transactions, the Query\_Type object is provided. The Query\_Type\index{Query\_Type}
object and the Connection\_Type object are used together when it comes
time to execute the query. Some operations require the connection
object, while others do not.

There are a large number of primitives associated with the Query\_Type
object. Most of them are related to the large number of data types
that are supported. These Query\_Type primitives fall into the following
basic categories:

\begin{enumerate}
   \item Object initialization
   \item SQL Query building
   \item SQL Execution on the Database server
   \item Transaction operations
   \item Fetch operations
   \item Column information functions
   \item Value fetching functions
   \item Value and Indicator fetching procedures
   \item Information operations
\end{enumerate}

In addition to these, are a number of generic functions and procedures
that permit the APQ user to custom tailor the API to his own specialized
Ada data types. This allows applications to continue the follow the
healthy tradition of using strong data types. There is no need for
a database application to take a back seat to safety\index{safety}.

\section{Initialization \label{SQL Initialization}}

The Query\_Type object is initialized when the object is created.
However, the Query\-\_Type object can be re-used as various SQL\index{SQL}
operations are performed by the program. To re-use the Query\_Type
object, one of the primitives in Table~\ref{t:qinit} may be used to recycle it for
re-use.

\begin{table}
   \begin{center}
      \begin{tabular}{lll}
         Type  &  Name     &  Purpose\\
         \hline
         proc  &  Clear    &  Clear object and re-initialize\\
         proc  &  Prepare  &  Reinitialize with start of new SQL query\\
      \end{tabular}
   \end{center}
   \caption{Query Initialization}\label{t:qinit}
\end{table}

The Clear\index{Clear} procedure does the initialization of the Query\_Type object.
The Prepare primitive implicitly invokes Clear
\footnote{Consequently, your application need not invoke Clear() prior to calling
Prepare().} and additionally starts the building of an SQL query. Very short
SQL statements may comletely specified by the Prepare API call. Longer
queries can be completed with some of the other primitives to be discussed
in this section.

\subsection{Procedure Clear}

With one exception, the Clear\index{Clear} primitive completely resets the state
of the Query\_Type object. This function primitive serves to basic
purposes:

\begin{enumerate}
\item Resets the object to its initial state so that it can be reused for a new query.
\item Releases any results of the current query (close cursors etc.)
\end{enumerate}

When performed after a query has been executed, this primitive releases
all results from the query. For Sybase, this can include issuing a
cancel back to the server and flushing all pending row results that
have not been retrieved by the APQ client program.

Clear accepts the Query\_Type object as its only argument:

\begin{Code}
procedure Clear(
   Q : in out Query_Type
);
\end{Code}

The one exception to clearing state however, is that the SQL case\index{case policy}
policy remains unchanged. If prior to the clear, the SQL case policy
is set to Preserve\_Case\index{Preserve\_Case}, it will remain so after the Clear call.

There are no exceptions raised by this call.

The use of the Clear primitive is recommended after all SQL processing
related to the query has been completed. This permits any database
server results to be released. Think of it as ``closing''\index{closing a query} the
query. Note however, that object finalization\index{finalization} will take care of this,
if the job is left unfinished by the programmer.

The following example illustrates it's use:

\begin{Example}
declare
   C : Connection_Type;
   Q : Query_Type;
begin
   ...
   Clear(Q);
\end{Example}

\subsubsection{Performance Issue}

There is one particular case where calling Clear is vitally important
for some database products. Let's use an example where you are fetching
the most recent stock price from a price history file. Here is the
table declaration:

\begin{SQL}
CREATE TABLE PRICE_HIST (
   SECURITY    CHAR(10) NOT NULL,
   PRICE_DATE  DATE NOT NULL,
   PRICE       REAL,
   PRIMARY KEY (SECURITY,PRICE_DATE)
);
\end{SQL}

With this table holding price history, keyed by the security name
and date, we can lookup the most recent price with a subroutine something
like the following:

\begin{NumberedExample}
procedure Last_Price(
   C :        in out Root_Connection_Type'Class;
   Security : in     String;
   Price :       out APQ_Double
) is
   function Value is new Float_Value(APQ_Double);

   Q : Root_Query_Type'Class := New_Query(C);\label{Ex:TheQ}
begin

   Begin_Work(Q,C);

   Prepare(Q,    "SELECT SECURITY,PRICE_DATE,PRICE");
   Append_Line(Q,"FROM PRICE_HIST");
   Append(Q,     "WHERE SECURITY = ");
   Append_Quoted(Q,C,Security,Line_Feed);
   Append_Line(Q,"ORDER BY SECURITY,PRICE_DATE DESC");\label{Ex:OrderBy}

   if Engine_Of(C) = Engine_MySQL then
      Append_Line(Q,"LIMIT 1");\label{Ex:Limit1}
   end if;

   Execute(Q,C);

   begin
      Fetch(Q);
   exception
      when No_Tuple =>
         raise;   -- No price found!
   end;

   Price := Value(Q,3);
   Clear(Q);\label{Ex:Clear}

end Last_Price;
\end{NumberedExample}

In this example, you can see that MySQL\index{MySQL} is covered by adding the "LIMIT 1"\index{LIMIT}
clause in line~\ref{Ex:Limit1}.\footnote{MySQL will limit the results to 1 row, if any.}
PostgreSQL\index{PostgreSQL} does not have a problem with this type of query because each row is
fetched on demand. Sybase\index{Sybase} however, will start returning all the rows
that it has available, in the order specified by the "ORDER BY"\index{ORDER BY} clause
(line~\ref{Ex:OrderBy}).

Given that the routine Last\_Price only calls Fetch\index{Fetch}
once, there is no point in the Sybase server producing and sending
more row data down the connection to the APQ client program. For this
reason, a call to Clear\index{Clear} (line~\ref{Ex:Clear}) will send a cancel notice to the Sybase server
to stop producing additional rows. It will also clear out any row data on
the connection that has not been fetched by the client program.

In this particular example, the Query\_Type object Q (line~\ref{Ex:TheQ}) would have been
finalized\index{finalized} anyway upon return from Last\_Price. Finalized Query\_Type
objects always perform the equivalent of a call to Clear prior to
releasing the object storage. However, it is important to explicitly
call Clear\index{Clear} if you have many other operations that follow the query
performed in order to allow a timely cancel if necessary and to release
any pending query results.


\subsection{Procedure Prepare\label{Procedure Prepare}}

The Prepare\index{Prepare} primitive goes one step further than Clear
in that it readies the object for the start of an SQL\index{SQL} statement build.
If the query is short, this will be the only building step required.
As in the case of Clear, the SQL case policy\index{case policy} is unchanged however.

The Prepare procedure takes the following arguments:

\begin{Code}
procedure Prepare(
   Q :     in out Query_Type;
   SQL :   in     String;
   After : in     String := Line_Feed
);
\end{Code}

The SQL argument defines the start of your SQL\index{SQL} statement. The
After argument may supply either the default (line feed)\index{line feed} or
some other text to append to the SQL text\index{SQL text}.
It is provided as a programmer convenience, since many times the
programmer will need to append a comma for example.

There are no exceptions raised by this call.

The following code shows an example of building a query to drop a
table:

\begin{Example}
declare
   Q : Query_Type;
begin
   ...
   Prepare(Q,"DROP TABLE DEAD_WEIGHT");
\end{Example}

\section{SQL Query Building \label{SQL Query Building}}

The previous section primitives ``cleared'' the Query\_Type for
a new query. The primitives provided in this section help to build
a new SQL query or to continue (append to)\index{append} the one started by the
Prepare call in \Ref{Procedure Prepare}. The programmer may
start\index{start} with a Prepare call and follow it by a number of ``append''
calls, or call Clear and build upon an empty query\index{empty query} and skip the
Prepare. ``PREPARE''\index{PREPARE} however, is a traditional first step
in embedded SQL\index{embedded SQL} software, so this tradition comes recommended
by the author.

There are two broad categories of support for creating SQL queries\index{SQL queries}.
They are:

\begin{enumerate}
   \item Append\index{Append} a value to the SQL query
   \item Encode\index{Encode} a value or NULL\index{NULL}, to the SQL query.
\end{enumerate}

Both of these categories append to the current query. Primitives in
category 2 , are prefixed with Encode and will be described
later in the present chapter. 

The Append category of support is useful for values that are never
\emph{NULL} (in SQL terms these columns that are declared as ``NOT
NULL''\index{NOT NULL}). The Encode category of support is provided for values in
your application that may be in the NULL\index{NULL} state. It is not absolutely
required that the Encode support be used, since it is possible for
the application to test for a NULL value. However, the programmer
will find that the Encode support provides application coding convenience
and economy of expression. With compact code, better readability\index{readability} and
safety\index{safety} is normally obtained.

Within category 1, there are five groups of primitives%
\footnote{The generic procedures have been lumped in with the primitives.%
} that build on the present query. They are:

\begin{enumerate}
   \item Append a string
   \item Append a string and a \emph{{}``newline''}
   \item Append a quoted string%
      \footnote{Quoted values are always spared from any automatic case conversions,
      if any are applied.}
   \item Append non string types
   \item Append using generic procedures for custom types
\end{enumerate}

Encode\index{Encode} support on the other hand, only provides for the needs of variables
that must be communicated to the database server. As a result, the
encode procedures consist only of the following two groups:

\begin{enumerate}
   \item Encode non-string\index{non-string types} types
   \item Encode using generic procedures for custom types\index{custom types}
\end{enumerate}

Presently only the second group is provided for by the APQ binding.%
\footnote{The reasoning is that most of the time, the user will want to instantiate
the generic procedures anyway. This permits both the data type and
the null indicator\index{null indicator} type to be a custom application type.%
} A future release may expand on category 1 support.

The append procedures (category 1) will be described first and then
followed by the encode procedures (category 2).


\subsection{Append SQL String}

The Append\index{Append} procedure permits the programmer to append text to the
SQL query being saved in the Query\_Type object. Unlike Prepare\index{Prepare},
Append does not clear the object. Append continues to add
SQL text\index{SQL text} to the query already gathered by the object Q (below):

\begin{Code}
procedure Append(
   Q :     in out Query_Type;
   SQL :   in     String;
   After : in     String := ""
);
\end{Code}

\begin{Code}
procedure Append(
   Q :     in out Query_Type;
   SQL :   in     Ada.Strings.Unbounded.Unbounded_String;
   After : in     String := ""
);
\end{Code}

There are no exceptions raised by this call.

The following example shows how Append is used:

\begin{Example}
declare
   Q : Query_Type;
begin
   ...
   Prepare(Q,"SELECT CUSTNO,CUST_NAME");
   Append(Q, "FROM CUSTOMER");
\end{Example}

Note that the Prepare\index{Prepare} call uses a default argument After=New\_Line\index{After}\index{New\_Line},
while Append uses a null string\index{null string} default.
You can put a line break in the SQL query by supplying the
value New\_Line in the argument "After" if you like.
The following example illustrates the use of Prepare and Append:

\begin{Example}
declare
   Q : Query_Type;
   Col_Name_1 : constant String := "CUSTNO";
   Col_Name_2 : constant String := "CUST_NAME";
begin
   ...
   Prepare(Q,"SELECT ");
   Append(Q,Col_Name_1,",");
   Append(Q,Col_Name_2,Line_Feed);
   Append(Q,"FROM CUSTOMER");
\end{Example}

This example builds up the same query that the previous example did,
except that the column names were provided by string variables\index{string variables}.

\subsection{Append SQL Line\label{Append SQL Line}}

The Append\_Line\index{Append\_Line} procedure is provided for added convenience and program
readability\index{readability}. The same effect can be had with a string Append call,
using string APQ.Line\_Feed supplied as the After\index{After} argument.
The Append\_Line procedure has the following specification:

\begin{Code}
procedure Append_Line(
   Q :   in out Query_Type;
   SQL : in     String;
);
\end{Code}

The Append\_Line procedure is one of the few that does not sport an
\emph{After} argument.


\subsection{Append Quoted SQL String}

Don't quote\index{quoted strings} your own strings at home. There are several reasons why
supplying your own quotes to a call to the normal Append call is a
bad idea:

\begin{enumerate}
   \item Some characters must be encoded\index{encoded} or escaped\index{escaped}, if they occur in the string.
   \item SQL portability\index{portability} is enhanced, since encoding and escaping\index{escaping} varies from
      database vendor to vendor.
   \item Internationalization\index{Internationalization} chosen by the user may require different processing.
   \item Quoted\index{quoted strings} strings should not be changed to upper or lowercase\index{lowercase} (by APQ).
\end{enumerate}

The Append\_Quoted\index{Append\_Quoted} procedure call is designed to make it easier for
the programmer to supply a string value that may contain special characters
within it. Since a string value is already supplied by APQ with outer\index{outer quotes}
quotes, any quote\index{quote} appearing within the string must be quoted. The
Append\_Quoted procedure provides the necessary outer quotes\index{quotes}
for the sring value and escapes\index{escapes} any special characters\index{special characters}
occuring within it as well. 

Another very important reason for using this API call for quoted strings
is that this will prevent the APQ library from changing the case of\index{case of SQL}
any SQL text that is created. This is true no matter what the current
SQL case policy\index{case policy} is (see type APQ\-.SQL\-\_Case\-\_Type on page \pageref{SQL_Case_Type Choices}).
Any string segment\index{string segment} that was added to the query with this API call
will not have its case changed when the query is executed or the text
of the SQL is returned in a call to the To\_String\index{To\_String} function.

There are two Append\_Quoted procedures, which differ only in the data
type of the \emph{SQL} argument:

\begin{Code}
procedure Append_Quoted(
   Q :          in out Query_Type;
   Connection : in out Root_Connection_Type'Class;
   SQL :        in     String;
   After :      in     String := ""
);
\end{Code}

\begin{Code}
procedure Append_Quoted(
   Q :          in out Query_Type;
   Connection : in out Root_Connection_Type'Class;
   SQL :        in     Ada.Strings.Unbounded.Unbounded_String;
   After :      in     String := ""
);
\end{Code}

Notice that this particular API call requires a connection\index{connection}. The reason
for this is that some databases require the server to make the appropriate
choices dealing with internationalized\index{internationalized} character sets\index{character sets}. The quoting\index{quoting}
conventions also vary from database to database, so APQ relies upon
the database vendor software to perform the quoting for you.

The following example illustrates the use of this call (using the
String type):

\begin{Example}
declare
   C :               Connection_Type;
   Q :               Query_Type;
   Freds_Emporium :  String := "Fred's Emporium";
begin
   ...
   Prepare(Q,    "SELECT COMPNO,COMPANY_NAME");
   Append_Line(Q,"FROM SUPPLIER");
   Append(Q,     "WHERE COMPANY_NAME = ");
   Append_Quoted(Q,C,Freds_Emporium,New_Line);
\end{Example}

The effect of these calls is to build an SQL query\index{SQL} that looks as follows
(for PostgreSQL\index{PostgreSQL}):

\begin{SQL}
SELECT COMPNO,COMPANY_NAME
FROM SUPPLIER
WHERE COMPANY_NAME = 'Fred\'s Emporium'
\end{SQL}

Notice how the quote character was escaped for use by the PostgreSQL\index{PostgreSQL}
database server. Note also that even though the SQL case policy\index{case policy} was to
use Upper\_Case\index{Upper\_Case}, the case of the quoted text was preserved.

APQ will automatically adjust the quoting conventions\index{quoting conventions} to match the
database being used. The same example above when used for Sybase\index{Sybase},
would produce the following SQL text\index{SQL text} instead:

\begin{SQL}
SELECT COMPNO,COMPANY_NAME
FROM SUPPLIER
WHERE COMPANY_NAME = 'Fred''s Emporium'
\end{SQL}

Sybase uses a doubled-up quote\index{doubled-up quote} instead.

Using APQ's Append\_Quoted frees the programmer from worrying about
quoting conventions and guarantees that the case of the text will be
preserved.

\subsection{Append Non-String Types to SQL Query}

A fairly large set of builtin non-string\index{non-string types} data types are supported by varied
Append calls that differ in the second argument V. The following
is a list of their specifications:

\begin{Code}
procedure Append(
   Q :     in out Query_Type;
   V :     in     APQ_Boolean;
   After : in     String := ""
);
\end{Code}

\begin{Code}
procedure Append(
   Q :     in out Query_Type;
   V :     in     APQ_Date;
   After : in     String := ""
);
\end{Code}

\begin{Code}
procedure Append(
   Q :     in out Query_Type;
   V :     in     APQ_Time;
   After : in     String := ""
);
\end{Code}

\begin{Code}
procedure Append(
   Q :     in out Query_Type;
   V :     in     APQ_Timestamp;
   After : in     String := ""
);
\end{Code}

\begin{Code}
procedure Append(
   Q :     in out Query_Type;
   V :     in     APQ_Bitstring;
   After : in     String := ""
);
\end{Code}

\begin{Code}
procedure Append(
   Q :     in out Query_Type;
   V :     in     Row_ID_Type;
   After : in     String := ""
);
\end{Code}

These Append\index{Append} procedure calls automatically convert\index{convert} the supplied data
type in argument V, internally into a string\index{string} using the To\_String\index{To\_String} function
appropriate to the data type. Internally, the string Append procedure
is then utilized to perform the remaining work. The following example
illustrates their use:

\begin{Example}
declare
   Q :         Query_Type;
   Ship_Date : APQ_Date;
begin
   ...
   Prepare(Q,    "SELECT COMPNO,COMPANY_NAME,SHIP_DATE");
   Append_Line(Q,"FROM SUPPLIER");
   Append(Q,     "WHERE SHIP_DATE = ");
   Append(Q,Ship_Date,New_Line);
\end{Example}

The example presented builds an SQL query that looks like this:

\begin{SQL}
SELECT COMPNO,COMPANY_NAME,SHIP_DATE
FROM SUPPLIER
WHERE SHIP_DATE = '2002-07-21'
\end{SQL}

Notice that the Append call for APQ\_Date\index{APQ\_Date} automatically supplies the
necessary quotes to the SQL query\index{SQL query} (if needed for the database being
used). All of the data types supported are molded into a format that
is acceptable in the native database SQL syntax\index{SQL syntax}. 

There is one additional Append procedure call that has a special set
of arguments in order to support dates\index{dates} with time zones\index{time zones}. The arguments
for this procedure call are as follows:

\begin{Code}
procedure Append(
   Q :     in out Query_Type;
   TS :    in     APQ_Timestamp;
   TZ :    in     APQ_Timezone;
   After : in     String := ""
);
\end{Code}

Apart from the different argument names TS and TZ, this
procedure works in the same fashion as the former Append\index{Append} procedure
call. The TZ argument simply supplies the additional time zone\index{time zone}
information to be added to the timestamp\index{timestamp}.

\begin{description}
\item [Note:] Not all databases support the use of timezone values.
\end{description}

\subsection{Generic Append SQL Procedures}

Ada programmers often take advantage of the strong typing \index{strong typing}that
is available in the language. To accomodate this programming aspect,
generic procedures are available so that type conversions\index{type conversions}
are unnecessary. The following table documents the generic procedures
that accept one generic argument named Val\_Type and the data
types that they support:

\begin{Code}
generic
   type Val_Type is new Boolean;
procedure Append_Boolean(
   Q :     in out Root_Query_Type'Class;
   V :     in     Val_Type;
   After : in     String := ""
);
\end{Code}

\begin{Code}
generic
   type Val_Type is range <>;
procedure Append_Integer(
   Q :     in out Root_Query_Type'Class;
   V :     in     Val_Type;
   After : in     String := ""
);
\end{Code}

\begin{Code}
generic
   type Val_Type is mod <>;
procedure Append_Modular(
   Q :     in out Root_Query_Type'Class;
   V :     in     Val_Type;
   After : in     String := ""
);
\end{Code}

\begin{Code}
generic
   type Val_Type is digits <>;
procedure Append_Float(
   Q :     in out Root_Query_Type'Class;
   V :     in     Val_Type;
   After : in     String := ""
);
\end{Code}

\begin{Code}
generic
   type Val_Type is delta <>;
procedure Append_Fixed(
   Q :     in out Root_Query_Type'Class;
   V :     in     Val_Type;
   After : in     String := ""
);
\end{Code}

\begin{Code}
generic
   type Val_Type is delta <> digits <>;
procedure Append_Decimal(
   Q :     in out Root_Query_Type'Class;
   V :     in     Val_Type;
   After : in     String := ""
);
\end{Code}

\begin{Code}
generic
   type Val_Type is new Ada.Calendar.Time;
procedure Append_Date(
   Q :     in out Root_Query_Type'Class;
   V :     in     Val_Type;
   After : in     String := ""
);
\end{Code}

\begin{Code}
generic
   type Val_Type is new Ada.Calendar.Day_Duration;
procedure Append_Time(
   Q :     in out Root_Query_Type'Class;
   V :     in     Val_Type;
   After : in     String := ""
);
\end{Code}

\begin{Code}
generic
   type Val_Type is new APQ_Timestamp;
procedure Append_Timestamp(
   Q :     in out Root_Query_Type'Class;
   V :     in     Val_Type;
   After : in     String := ""
);
\end{Code}

\begin{Code}
generic
   type Val_Type is new APQ_Bitstring;
procedure Append_Bitstring(
   Q :     in out Root_Query_Type'Class;
   V :     in     Val_Type;
   After : in     String := ""
);
\end{Code}

Each of the resulting instantiated procedures provide the following
calling signature:

\begin{Code}
procedure Append(
   Q :     in out Query_Type;
   V :     in     Val_Type;
   After : in     String := ""
);
\end{Code}

The following code fragment illustrates how these are instantiated\index{instantiated} and used:

\begin{NumberedExample}
declare
   type Price_Type is delta 0.01 digits 12;\label{Ex:TypeDef}
   procedure Append is new Append_Decimal(Price_Type);\label{Ex:Inst}

   Q :             Query_Type;
   Selling_Price : Price_Type;
begin
   ...
   Prepare(Q,"UPDATE SUPPL_ORDER");
   Append(Q."SET SELLING_PRICE = ");
   Append(Q,Selling_Price,New_Line);\label{Ex:InstUse}
   Append_Line(Q,"WHERE ...");
\end{NumberedExample}

In this example, the application defines its own type Price\_Type
in line~\ref{Ex:TypeDef}. After instantiating the Append\_Decimal
generic procedure as Append\index{Append} (line~\ref{Ex:Inst}), the application is
free to neatly append a price value from Selling\_Price in
line~\ref{Ex:InstUse}, as if it were natively supported.


\subsection{Generic Append\_Timezone}

The Append\_Timezone\index{Append\_Timezone} has an additional generic paramter. The instantiated\index{instantiated}
procedure also has a slightly different set of calling arguments. The generic
parameters are specified as follows:

\begin{Code}
generic
   type Date_Type is new Ada.Claendar.Time;
   type Zone_Type is new APQ_Timezone;
procedure Append_Timezone(
   Q :     in out Root_Query_Type'Class;
   V :     in     Date_Type;
   Z :     in     Zone_Type;
   After : in     String := ""
);
\end{Code}

The instantiated procedure has the following calling signature:

\begin{Code}
procedure Append(
   Q :     in out Root_Query_Type'Class;
   V :     in     Date_Type;
   Z :     in     Zone_Type;
   After : in     String := ""
);
\end{Code}

The following shows an example of its use:

\begin{NumberedExample}
declare
   type Ship_Date_Type is new APQ_Timestamp;
   type Ship_Zone_Type is new APQ_Timezone;

   procedure Append is new Append_Timezone(\label{Ex:InstTZ}
      Ship_Date_Type,Ship_Zone_Type);

   Q :         Query_Type;
   Ship_Date : Ship_Date_Type;
   Ship_Zone : Ship_Zone_Type;
begin
   ...
   Prepare(Q,"SELECT COUNT(*)");
   Append_Line(Q,"FROM ORDER");
   Append(Q,"WHERE SHIP_DATE = ");
   Append(Q,Ship_Date,Ship_Zone,New_Line);\label{Ex:UseInstTZ}
   ...
\end{NumberedExample}

The example shows how the application's types Ship\_Date\_Type
and Ship\_Zone\_Type are accomodated by the Append\index{Append} instantiation\index{instantiation}
of the generic procedure (line~\ref{Ex:InstTZ}). The instantiated
Append routine is applied in line~\ref{Ex:UseInstTZ}.


\subsection{Generic Append of Bounded SQL Text}

To accomodate the use of the package Ada\-.Strings\-.Bounded\index{Ada.Strings.Bounded},
the generic procedure Append\_Bounded\index{Append\_Bounded} was provided. Its instantiation
requirements differ from the preceeding ones because the instantiation
of the Bounded\_String\index{Bounded\_String} type must be provided to the Append\_Bounded
generic procedure. The generic procedure is defined as follows:

\begin{Code}
generic
   with package P 
      is new Ada.Strings.Bounded.Generic_Bounded_Length(<>);
procedure Append_Bounded(
   Q :     in out Query_Type;
   SQL :   in     P.Bounded_String;
   After : in     String
);
\end{Code}

In other words, Append\_Bounded can be instantiated from any instantiation
of the Ada.Strings.Bounded.Generic\_Bounded\_Length package. The following
example makes this easier to understand:

\begin{NumberedExample}
with Ada.Strings.Bounded;
...
declare
   package B80 
      is new Ada.Strings.Bounded.Generic_Bounded_Length(80);
   package B20 
      is new Ada.Strings.Bounded.Generic_Bounded_Length(20);

   procedure Append is new Append_Bounded(B80);\label{Ex:B80}
   procedure Append is new Append_Bounded(B20);\label{Ex:B20}

   Q :         Query_Type;
   Item_Code : B20;
   Item_Name : B80;
begin
   ...
   Prepare(Q,    "SELECT COUNT(*)");
   Append_Line(Q,"FROM ORDER");
   Append(Q,     "WHERE ITEM_CODE = ","'");
   Append(Q,Item_Code,"' AND ITEM_NAME = '");\label{Ex:UseB20}
   Append(Q,Item_Name,"'" & New_Line);\label{Ex:UseB80}
   ...
\end{NumberedExample}

The example shows how two different generic procedures named Append
are instantiated from the Bounded\_String instantiations B80 (line~\ref{Ex:B80})
and B20 (line~\ref{Ex:B20}). Note that the Append\_Bounded procedure does not escape
special characters, nor provide the outer quotes (this is a poor example of its
use actually).

The Append\index{Append} call used in line~\ref{Ex:UseB20} is the one that was
instantiated on line~\ref{Ex:B20}. The other Append used on
line~\ref{Ex:UseB80} was instantiated on line~\ref{Ex:B80}. These
are matched according to the normal Ada95\index{Ada95} argument type matching
rules.

\subsection{Generic Append\_Bounded\_Quoted Procedure}

To accomodate the quoting needs of Bounded\_Strings, the Append\_Bounded\_Quoted\index{Append\_Bounded\_Quoted}
generic procedure may be used:

\begin{Code}
generic
   with package P
      is new Ada.Strings.Bounded.Generic_Bounded_Length(<>);
procedure Append_Bounded_Quoted(
   Q :     in out Query_Type;
   C :     in     Connection_Type;
   SQL :   in     P.Bounded_String;
   After : in     String
);
\end{Code}

It is otherwise very similar to the previous Append\_Unbounded\index{Append\_Unbounded} procedure.
The following example illustrates a safer\index{safer} version of the prior example:

\begin{Example}
with Ada.Strings.Bounded;
...
declare
   package B80 
      is new Ada.Strings.Bounded.Generic_Bounded_Length(80);
   package B20 
      is new Ada.Strings.Bounded.Generic_Bounded_Length(20);

   procedure Append_Quoted is new Append_Bounded_Quoted(B80);
   procedure Append_Quoted is new Append_Bounded_Quoted(B20);

   C :         Connection_Type;
   Q :         Query_Type;
   Item_Code : B20;
   Item_Name : B80;
begin
   ...
   Prepare(Q,    "SELECT COUNT(*)");
   Append_Line(Q,"FROM ORDER");
   Append(Q,     "WHERE ITEM_CODE = ");
   Append_Quoted(Q,C,Item_Code," AND ITEM_NAME = ");
   Append_Quoted(Q,C,Item_Name,New_Line);
   ...
\end{Example}

The instantiations of Append\_Quoted\index{Append\_Quoted}%
\footnote{It is not necessary to instantiate these procedures as
Append\_Quoted, but it is recommended for readability.} here will
properly escape\index{escape} any special\index{special characters} characters that may appear in the program's
string variables Item\_Code and Item\_Name. Additionally, note that the
outer quotes\index{quotes} are provided automatically, easing the programmer's burden
in building up the SQL query. Like other Append\_Quoted API calls within
APQ, the case\index{case within quoted strings} within quoted strings is always preserved.


\subsection{Encoding Quoted Strings}

While strings are well covered by the category 1 support, it is necessary
to encode a NULL\index{NULL} in place of a quoted string\index{quoted string},
when the value is null. The generic specification for 
Encode\-\_String\-\_Quoted\index{Encode\_String\_Quoted} is as follows:

\begin{Code}
generic
   type Ind_Type is new Boolean;
procedure Encode_String_Quoted(
   Q :          in out Root_Query_Type'Class;
   Connection : in     Root_Connection_Type'Class;
   SQL :        in     String;
   Indicator :  in     Ind_Type;
   After :      in     String := ""
);
\end{Code}

An example of its instantiation and use is shown below:

\begin{Example}
declare
   type Cust_Name_Ind_Type is new Boolean;

   procedure Encode_Quoted 
      is new Encode_String_Quoted(Cust_Name_Ind_Type);

   Q :             Query_Type;
   Cust_Name :     String(1..30);
   Cust_Name_Ind : Cust_Name_Ind_Type; -- Indicator
begin
   ...
   Prepare(Q,"UPDATE CUSTOMER");
   Append_Line(Q,"SET CUST_NAME = ");
   Encode_Quoted(Q,Cust_Name,Cust_Name_Ind);
\end{Example}

In this example, the String Cust\_Name is given outer quotes
and any special characters are escaped before the value is appended
to the current SQL query being collected in object Q. If however,
the indicator Cust\_Name\_Ind is True (indicating that the
value Cust\_Name should be interpreted as NULL\index{NULL}), then the string
``NULL'' is appended instead. When NULL is supplied, no outer
quotes are supplied. The following two SQL statements are possible,
depending upon Cust\_Name\_Ind. When the indicator is false,
a quoted\index{quoted value} value is supplied:

\begin{SQL}
UPDATE CUSTOMER
SET CUST_NAME = 'Fred Willard'
...
\end{SQL}

When the indicator is true (the value is null), the resulting query
becomes the following:

\begin{SQL}
UPDATE CUSTOMER
SET CUST_NAME = NULL
...
\end{SQL}

\subsection{Encoding Quoted Unbounded\_String}

To provide quoting\index{quoting support} support for Unbounded\-\_Strings\index{Ada.Strings.Unbounded},
the Encode\-\_Bounded\-\_Quoted gen\-eric procedure is provided by APQ.
The specifications for this procedure is given below:

\begin{Code}
generic
   type Ind_Type is new Boolean;
procedure Encode_Unbounded_Quoted(
   Q :          in out Root_Query_Type'Class;
   Connection : in     Root_Connection_Type'Class;
   SQL :        in     Ada.Strings.Unbounded.Unbounded_String;
   Indicator :  in     Ind_Type;
   After :      in     String := ""
);
\end{Code}

An example of its use is illustrated as follows:

\begin{Example}
declare
   use Ada.Strings.Bounded;

   type Cust_Name_Ind_Type is new Boolean;

   procedure Encode_Quoted 
      is new Encode_Unbounded_Quoted(Cust_Name_Ind_Type);

   C :             Connection_Type;
   Q :             Query_Type;
   Cust_Name :     Unbounded_String;
   Cust_Name_Ind : Cust_Name_Ind_Type;
begin
   ...
   Prepare(Q,"UPDATE CUSTOMER");
   Append_Line(Q,"SET CUST_NAME = ");
   Encode_Quoted(Q,C,Cust_Name,Cust_Name_Ind);
\end{Example}

In this example, the Unbounded\_String Cust\_Name is given
outer quotes and any special characters are escaped before the value
is appended to the current SQL query\index{SQL query} being collected in object Q.
If however, the indicator\index{indicator, null} Cust\_Name\_Ind is True (indicating
that the value \emph{Cust\_Name} should be interpreted as NULL), then
the string ``NULL''\index{NULL} is appended\index{appended} instead. When NULL is supplied,
no outer\index{outer quotes} quotes are supplied. The following two SQL statements are
possible, depending upon \emph{Cust\_Name\_Ind}. When the indicator
is false, a quoted value\index{quoted value} is supplied:

\begin{SQL}
UPDATE CUSTOMER
SET CUST_NAME = 'Fred Willard'
...
\end{SQL}

When the indicator is true, the resulting query becomes the following:

\begin{SQL}
UPDATE CUSTOMER
SET CUST_NAME = NULL
...
\end{SQL}

\subsection{Encoding Bounded Quoted Strings}

Bounded strings\index{bounded strings} require instantiations of
Ada\-.Strings\-.Bounded\-.Gen\-eric\-\_Bound\-ed\-\_Len\-gth
\index{Ada.Strings.Bounded}with a specific length. The instantiated
package\index{package} reference must be provided as an argment to the
Encode\-\_Bounded\-\_Quoted\index{Encode\_Bounded\_Quoted} instantiation:

\begin{Code}
generic
   type Ind_Type is new Boolean;
   with package P 
      is new Ada.Strings.Bounded.Generic_Bounded_Length(<>);
procedure Encode_Bounded_Quoted(
   Q :          in out Root_Query_Type'Class;
   Connection : in     Root_Connection_Type'Class;
   SQL :        in     P.Bounded_String;
   Indicator :  in     Ind_Type;
   After :      in     String := ""
);
\end{Code}

An example showing its use is given below:

\begin{Example}
with Ada.Strings.Bounded;

declare
   package B80 is new 
      Ada.Strings.Bounded.Generic_Bounded_Length(80);

   type Cust_Name_Ind_Type is new Boolean;

   procedure Encode_Quoted is new
      Encode_Bounded_Quoted(Cust_Name_Ind_Type,B80);

   C :             Connection_Type;
   Q :             Query_Type;
   Cust_Name :     B80.Bounded_String;
   Cust_Name_Ind : Cust_Name_Ind_Type; -- Indicator
begin
   ...
   Prepare(Q,"UPDATE CUSTOMER");
   Append_Line(Q,"SET CUST_NAME = ");
   Encode_Quoted(Q,C,Cust_Name,Cust_Name_Ind);
   ...
\end{Example}

In this example, the Bounded\_String Cust\_Name is given outer
quotes and any special characters are escaped before the value is
appended to the current SQL query being collected in object Q.
If however, the indicator\index{indicator} Cust\_Name\_Ind is True (indicating
that the value Cust\_Name should be interpreted as NULL), then
the string ``NULL''\index{NULL} is appended instead. When NULL is supplied,
no outer quotes are supplied. The following two SQL\index{SQL} statements are
possible, depending upon Cust\_Name\_Ind. When the indicator is false,
a quoted value\index{quoted value} is supplied:

\begin{SQL}
UPDATE CUSTOMER
SET CUST_NAME = 'Fred Willard'
...
\end{SQL}

When the indicator is true, the resulting query becomes the following:

\begin{SQL}
UPDATE CUSTOMER
SET CUST_NAME = NULL
...
\end{SQL}

\subsection{Encoding Non-String Values}

There are a large number of generic\index{generic} encoding procedures that 
allow encoding of non-string\index{non-string values} values into SQL queries. They are
listed below:

\begin{Code}
generic
   type Val_Type is new Boolean;
   type Ind_Type is new Boolean;
procedure Encode_Boolean(
   Q :         in out Root_Query_Type'Class;
   V :         in     Val_Type;
   Indicator : in     Ind_Type;
   After :     in     String := ""
);
\end{Code}

\begin{Code}
generic
   type Val_Type is range <>;
   type Ind_Type is new Boolean;
procedure Encode_Integer(
   Q :         in out Root_Query_Type'Class;
   V :         in     Val_Type;
   Indicator : in     Ind_Type;
   After :     in     String := ""
);
\end{Code}

\begin{Code}
generic
   type Val_Type is mod <>;
   type Ind_Type is new Boolean;
procedure Encode_Modular(
   Q :         in out Root_Query_Type'Class;
   V :         in     Val_Type;
   Indicator : in     Ind_Type;
   After :     in     String := ""
);
\end{Code}

\begin{Code}
generic
   type Val_Type is digits <>;
   type Ind_Type is new Boolean;
procedure Encode_Float(
   Q :         in out Root_Query_Type'Class;
   V :         in     Val_Type;
   Indicator : in     Ind_Type;
   After :     in     String := ""
);
\end{Code}

\begin{Code}
generic
   type Val_Type is delta <>;
   type Ind_Type is new Boolean;
procedure Encode_Fixed(
   Q :         in out Root_Query_Type'Class;
   V :         in     Val_Type;
   Indicator : in     Ind_Type;
   After :     in     String := ""
);
\end{Code}

\begin{Code}
generic
   type Val_Type is delta <> digits <>;
   type Ind_Type is new Boolean;
procedure Encode_Decimal(
   Q :         in out Root_Query_Type'Class;
   V :         in     Val_Type;
   Indicator : in     Ind_Type;
   After :     in     String := ""
);
\end{Code}

\begin{Code}
generic
   type Val_Type is new APQ_Date;
   type Ind_Type is new Boolean;
procedure Encode_Date(
   Q :         in out Root_Query_Type'Class;
   V :         in     Val_Type;
   Indicator : in     Ind_Type;
   After :     in     String := ""
);
\end{Code}

\begin{Code}
generic
   type Val_Type is new APQ_Time;
   type Ind_Type is new Boolean;
procedure Encode_Time(
   Q :         in out Root_Query_Type'Class;
   V :         in     Val_Type;
   Indicator : in     Ind_Type;
   After :     in     String := ""
);
\end{Code}

\begin{Code}
generic
   type Val_Type is new APQ_Timestamp;
   type Ind_Type is new Boolean;
procedure Encode_Timestamp(
   Q :         in out Root_Query_Type'Class;
   V :         in     Val_Type;
   Indicator : in     Ind_Type;
   After :     in     String := ""
);
\end{Code}

\begin{Code}
generic
   type Val_Type is new APQ_Bitstring;
   type Ind_Type is new Boolean;
procedure Encode_Bitstring(
   Q :         in out Root_Query_Type'Class;
   V:          in     Val_Type;
   Indicator : in     Ind_Type;
   After :     in     String := ""
);
\end{Code}

The following example shows the application of one of these generic
procedures:

\begin{Example}
declare
   type Cust_No_Type is new Integer range 1000..100_000;
   type Cust_Age_Type is new Integer range 0..200;

   procedure Append is new Append_Integer(Cust_No_Type);
   procedure Encode is new 
      Encode_Integer(Cust_No_Type,Boolean);

   Q :            Query_Type;
   Cust_No :      Cust_No_Type;  -- Customer # NOT NULL
   Cust_Age :     Cust_Age_Type; -- Can be NULL
   Cust_Age_Ind : Boolean;       -- Indicator
   Cust_Name :    String(1..30); -- Customer Name NOT NULL
begin
   ...
   Prepare(Q,"INSERT INTO CUSTOMER (CUST_NO,AGE,CUST_NAME)");
   Append(Q, "VALUES ( ");
   Append(Q,Cust_No,",");
   Encode(Q,Cust_Age,Cust_Age_Ind,",");
   Append_Quoted(Q,Cust_Name," )" & New_Line);
   ...
\end{Example}

From the example, if variable Cust\_Name holds the value ``Martin Mull'',
and Cust\_No holds 12345, two possible SQL queries are possible,
depending upon the value of Cust\_Age\_Ind, the null indicator\index{indicator, null}. When
Cust\_Age\_Ind is false (not null), then the SQL query would be formed as
follows:

\begin{SQL}
INSERT INTO CUSTOMER (CUST_NO,AGE,CUST_NAME)
VALUES (12345,52,'Martin Mull')
\end{SQL}

When the indicator\index{indicator} Cust\_Age\_Ind is true (representing null), then the
query would be constructed as follows:

\begin{SQL}
INSERT INTO CUSTOMER (CUST_NO,AGE,CUST_NAME)
VALUES (12345,NULL,'Martin Mull')
\end{SQL}

Notice how Append\index{Append} procedures are used for values that can never be
null (no null indicator is involved). Encode routines are only necessary
when a null indicator may need to be encoded into the result.


\subsection{Encoding Timezone}

Encoding\index{encoding} APQ\_Timezone\index{APQ\_Timezone} values requires a special generic procedure
named Encode\-\_Timezone\index{Encode\_Timezone}. Its generic parameters are described by the
following table:

\begin{Code}
generic
   type Date_Type is new APQ_Timestamp;
   type Zone_Type is new APQ_Timezone;
   type Ind_Type is new Boolean;
procedure Encode_Timezone(
   Q :         in out Root_Query_Type'Class;
   D :         in     Date_Type;
   Z :         in     Zone_Type;
   Indicator : in     Ind_Type;
   After :     in     String := ""
);
\end{Code}

The following example demonstrates the instantiation and use of the
procedure:\label{Birthday Timezone Example}

\begin{Example}
declare
   type Cust_No_Type is new Integer range 1000..100_000;
   type Birthday_Type is new APQ_Timestamp;

   procedure Append is new Append_Integer(Cust_No_Type);
   procedure Encode is new
      Encode_Integer(Cust_No_Type,Boolean);
   procedure Encode is new
      Encode_Timezone(Birthday_Type,APQ_Timezone,Boolean);

   Q :            Query_Type;
   Cust_No :      Cust_No_Type;  -- Customer # NOT NULL
   Birthday :     Birthday_Type; -- Customer birthday
   Birthday_TZ :  APQ_Timezone;  -- Timezone of the birthday
   Birthday_Ind : Boolean;       -- True when Birthday is NULL
begin
   ...
   Prepare(Q,"INSERT INTO BIRTHDAY (CUST_NO,BIRTHDAY)");
   Append(Q, "VALUES (");
   Append(Q,Cust_No,",");
   Encode(Q,Birthday,Birthday_TZ,Birthday_Ind,")" & New_Line);
   ...
\end{Example}

If the Birthday\_Ind indicator\index{indicator} is false (not null\index{null, not}),
then the resulting query would look something like this:

\begin{SQL}
INSERT INTO BIRTHDAY (CUST_NO,BIRTHDAY)
VALUES (12345,'1984-09-25 22:47:06+03')
\end{SQL}
The ``+03'' after the time represents the time zone \index{time zone}UTC+3
hours.


\section{Query Execution}

Once the SQL query has been constructed using all of the techniques
described in \Ref{SQL Initialization} and \Ref{SQL Query Building},
you are ready to send\index{send the query} the query to the database engine\index{engine, database} to have it
executed. This is done with the help of the Query\_Type's primitive
Execute\index{Execute}. The Execute call requires the following calling arguments:

\begin{Code}
procedure Execute(
   Query :      in out Root_Query_Type;
   Connection : in out Root_Connection_Type'Class
);
\end{Code}

The Execute primitive can raise the exceptions that are listed
in Table~\ref{t:exx}.

\begin{table}
   \begin{center}
      \begin{tabular}{ll}
         Exception Name &  Reason\\
         \hline 
         Not\_Connected &  There is no connection to use\\
         Abort\_State   &  Transaction in {}``abort state{}``\\
         SQL\_Error     &  The submitted SQL query failed\\
         Failed         &  Hidden APQ query failure\\
      \end{tabular}
   \end{center}
   \caption{Execute Exceptions}\label{t:exx}
\end{table}

The PostgreSQL\index{PostgreSQL} Abort\_State\index{Abort\_State}
exception is described in \Ref{Abort_State exception}. This exception indicates that
the current transaction has failed\index{failed transaction}. All other types of errors\index{errors} raise\index{raise}
the SQL\_Error\index{SQL\_Error} exception, unless the connection is bad. The Failed\index{Failed exception}
exception only occurs for some databases where a hidden\index{hidden SQL query} SQL query
must be performed and it unexpectedly failed.%
\footnote{For Sybase, APQ will do a SELECT @@identity after an INSERT query
is performed.%
}

The use of the Execute\index{Execute} primitive is illustrated by extending the example
from page \pageref{Birthday Timezone Example}:

\begin{Example}
declare
   type Cust_No_Type is new Integer range 1000..100_000;
   type Birthday_Type is new APQ_Timestamp;

   procedure Append is new Append_Integer(Cust_No_Type);

   procedure Encode is new
      Encode_Integer(Cust_No_Type,Boolean);
   procedure Encode is new
      Encode_Timezone(Birthday_Type,APQ_Timezone,Boolean);

   C :            Connection_Type;
   Q :            Query_Type;
   Cust_No :      Cust_No_Type;     -- NOT NULL
   Birthday :     Birthday_Type;
   Birthday_TZ :  APQ_Timezone;     -- Timezone
   Birthday_Ind : Boolean;          -- Indicator
begin
   ...
   Prepare(Q,"INSERT INTO BIRTHDAY (CUST_NO,BIRTHDAY)");
   Append(Q, "VALUES (");
   Append(Q,Cust_No,",");
   Encode(Q,Birthday,Birthday_TZ,Birthday_Ind,")" & New_Line);
   Execute(Q,C);
\end{Example}

The example shows the Execute primitive pairing the query object Q
with the database connection object C.

\subsection{Error Message Reporting\label{Error Message Reporting}}

It is useful to know that your SQL query failed, but more information
is usually necessary. The Error\_Message\index{Error\_Message} primitive can be invoked
on the Query\_Type object. This function has the following calling
signature:

\begin{Code}
function Error_Message(
   Query : Query_Type;
) return String;
\end{Code}

The following example shows how this function might be used:

\begin{Example}
begin
   ...
   Execute(Q,C);
exception
   when SQL_Error =>
      Put(Standard_Error,"SQL Error: ");
      Put_Line(Standard_Error,Error_Message(Q));
      raise;
   when others =>
      raise;
end
\end{Example}

\subsection{Is\_Duplicate\_Key Function}

Duplicate\index{duplicate key} key errors\index{errors} often occur while performing SQL INSERT\index{INSERT} operations
on a table. A duplicate key insertion\index{insertion error} error is a special case because
the insert operation may not be considered a failure for some applications.
For this reason, the Is\_Duplicate\_Key\index{Is\_Duplicate\_Key} predicate function is provided
for use after a SQL\_Error exception has been raised%
\footnote{At present, this test is implemented by calling Error\_Message and
looking at the message text. Future versions of the APQ binding may
use a more reliable indicator if the PostgreSQL libpq library provides
such a status indication.%
}. The specification is listed as follows:

\begin{Code}
function Is_Duplicate_Key(
   Query : Query_Type
) return Boolean;
\end{Code}

The following example reports an SQL error\index{SQL error} only if the error is not
a duplicate key insert problem:

\begin{Example}
...
begin
   Execute(Q,C);  -- Execute an INSERT SQL statement
exception
   when SQL_Error =>
      if not Is_Duplicate_Key(Q) then
         -- Report error if not a duplicate insert
         Put(Standard_Error,"SQL Error: ");
         Put_Line(Standard_Error,Error_Message(Q));
         raise;
      else
         null; -- Ignore duplicate insert
      end;
end;
\end{Example}

\subsection{Command\_Status Function (PostgreSQL)\label{Command_Status Function}}

The Command\_Status\index{Command\_Status} function provides a string of status information
after a PostgreSQL\index{PostgreSQL} query has been executed. The results returned depends
upon the type of execution that was last performed. Table~\ref{t:csrv}
summarizes the types of return status strings available:%
\marginpar{This is a PostgreSQL specific function, only. Avoid its use for portability.}

\begin{table}
   \begin{center}
      \begin{tabular}{|l|l|l|}
         \hline
         After Event    &  Result               &  Comments\\
         \hline
         CREATE ...     &  "CREATE"             &  \\
         BEGIN WORK     &  "BEGIN"              &  \\
         COMMIT WORK    &  "COMMIT"             &  \\
         ROLLBACK WORK  &  "ROLLBACK"           &  \\
         SELECT ...     &  "SELECT"             &  \\
         INSERT ...     &  "INSERT <OID> <\#>"  &  \# is normally 1\\
         \hline
      \end{tabular}
   \end{center}
   \caption{Command\_Status Return Values}\label{t:csrv}
\end{table}

Notice that after an INSERT\index{INSERT} is performed, the returned status string
includes the OID\index{OID} (row ID\index{row ID}) of the new row, and the number of rows inserted
(normally 1). If you need to extract the OID value, see the Command\_Oid
function in \Ref{Command_Oid Function}.

The calling signature of Command\_Status is:

\begin{Code}
function Command_Status(
   Query : Query_Type
) return String;
\end{Code}

The Command\_Status function can raise the exceptions listed in
Table~\ref{t:cstsx}.

\begin{table}
   \begin{center}
      \begin{tabular}{ll}
         Exception Name    &  Reason\\
         \hline 
         No\_Result        &  There is no result status (no execution)\\
      \end{tabular}
   \end{center}
   \caption{Command\_Status Exceptions}\label{t:cstsx}
\end{table}


\subsection{Command\_Oid Function \label{Command_Oid Function}}

After an INSERT\index{INSERT} operation, it is often required to know the OID\index{OID} (row ID\index{row ID})
value for the newly created row. When using PostgreSQL, this can be
extracted from the Command\_Status\index{Command\_Status} return string (see
\Ref{Command_Status Function}), but this approach is not
portable to other vendor databases.

The APQ user is encouraged to call the Command\_Oid function when
it is necessary to know the identity\index{identity} of the newly inserted row. This
function primitive hides the differences\index{differences} between the different database
products and makes your code more portable.

The calling signature for Command\_Oid is as follows:

\begin{Code}
function Command_Oid(
   Query : Query_Type
) return Row_ID_Type;
\end{Code}

Table \ref{t:coidx} lists the exceptions that are possible for Command\_Oid.

\begin{table}
   \begin{center}
      \begin{tabular}{ll}
         Exception Name    &  Reason\\
         \hline 
         No\_Result        &  There is no result status (no execution)\\
         SQL\_Error        &  An SQL error occurred obtaining the OID\\
      \end{tabular}
   \end{center}
   \caption{Command\_Oid Exceptions}\label{t:coidx}
\end{table}

Be aware that the exception No\_Result\index{No\_Result} can be raised for two different
reasons:

\begin{itemize}
   \item There was no prior ``execution'' (thus no result)
   \item The prior execution was not an INSERT operation (hence no OID value available)
\end{itemize}

APQ applications should \emph{not} be written to provoke these exceptions,
since different database products behave differently in this respect
(Sybase\index{Sybase} users will never see an exception from Command\_Oid for example).
The raising of either of these exceptions does however indicate that
there is a programming error\index{error, programming} that requires correction in the calling
application.


\subsubsection{Database OID (Row ID) Support}

Some databases do not support the concept of OID\index{OID} (row ID\index{row ID}) values.
This creates a huge portability\index{portability} problem for the application programmer.
However, all databases provide some technique for identifying\index{identifying rows} rows
uniquely in a table. Table~\ref{t:roids} summarizes the ways that
APQ adapts the database server to concept of OID values:%
\footnote{Applications should be written to avoid row ID concepts and use key values instead.}

\begin{table}
   \begin{center}
      \begin{tabular}{|l|l|l|l|}
         \hline 
         Database          &  OID      &  How               &  How APQ Supports OID\\
         \hline
         PostgreSQL        & Yes       &  OID = Row ID      &  PostgreSQL Object ID (OID)\\
         MySQL             & No        &  AUTO\_INCREMENT   &  Serial value from primary key\\
         Sybase            & No        &  Identity column   &  SELECT @@identity\\
         \hline
      \end{tabular}
   \end{center}
   \caption{Row ID Database Support}\label{t:roids}
\end{table}


\subsubsection{MySQL OID (Row ID) Support}

From the preceeding table, you can observe that MySQL\index{MySQL} does not support
any concept of a row ID\index{row ID} value. APQ can fake OID-like\index{OID} support, if the
table is defined using a primary key in the following form\index{AUTO\_INCREMENT}:

\begin{SQL}

   INTEGER AUTO_INCREMENT NOT NULL PRIMARY KEY

\end{SQL}

When a MySQL table has a primary key like the one shown, APQ is able
to return the created primary key value using Command\_OID\index{Command\_OID}. 

The following is an example table declaration:

\begin{SQL}
CREATE TABLE ITEM_DESC (
   ITEM_ID     INTEGER AUTO_INCREMENT NOT NULL PRIMARY KEY,
   DESCRIPTION VARCHAR(80)
);
\end{SQL}

\subsubsection{Sybase OID (Row ID) Support}

Sybase\index{Sybase} does not support the concept of row ID\index{row ID} values. APQ however,
can provide row ID-like support if you declare your table with a primary
key in the following form\index{IDENTITY}:

\begin{SQL}

   NUMERIC IDENTITY NOT NULL PRIMARY KEY

\end{SQL}

For Sybase, when the APQ procedure Execute is performed, APQ will
check to see if the SQL query peformed was an INSERT\index{INSERT} command. If it
was, a hidden SQL query is performed immediately afterwards on the
same connection to query the identity\index{identity of a row} of the row that was created\index{@@identity}:

\begin{SQL}

   SELECT @@identity

\end{SQL}

The value returned by this hidden query will be saved as the row ID\index{row ID}
in the Query\-\_Type object used to perform the INSERT\index{INSERT}. A later call
to Command\_OID on this same Query\_Type object, will then return
this row identity value as a Row\_ID\_Type\index{Row\_ID\_Type}.

The following is an example Sybase table declaration:

\begin{SQL}
CREATE TABLE ITEM_DESC (
   ITEM_ID     NUMERIC IDENTITY NOT NULL PRIMARY KEY,
   DESCRIPTION VARCHAR(80)
);
\end{SQL}

\subsubsection{Example of Command\_OID Use}

The following example will always work for PostgreSQL\index{PostgreSQL}. It will also
work for MySQL\index{MySQL} and Sybase\index{Sybase}, if the primary key\index{primary key} guidelines in the prior
subsections were followed. 

This example shows how the INSERTed\index{INSERT} row's OID value (or primary key\index{primary key})
is obtained:

\begin{Example}
declare
   C :      Connection_Type;
   Q :      Query_Type;
   Obj_Id : Row_ID_Type;
begin
   ...
   Prepare(Q,"INSERT INTO CUST_ORDER (CATALOG_NO,QUANTITY,...");
   ...
   Execute(Q,C);
   Obj_Id := Command_Oid(Q);  -- row id/identity of inserted row
\end{Example}

The variable named Obj\_ID will contain the PostgreSQL\index{PostgreSQL} OID\index{OID} value of
the inserted row, when PostgreSQL is used. For MySQL\index{MySQL} and Sybase\index{Sybase}, the
variable Obj\_ID will contain the primary key value of the newly created
row.


\subsubsection{Generic\_Command\_Oid Function}

To allow strong typing\index{strong typing} to be used in place of the supplied Row\_ID\_Type\index{Row\_ID\_Type}
type, the Generic\_Command\_Oid\index{Generic\_Command\_Oid} function can be instantiated for use
in the application. The instantiated function otherwise behaves identical
with the Command\_Oid function described on page \pageref{Command_Oid Function}.
The instantiation arguments for Generic\_Command\_Oid are as follows:

\begin{Code}
generic
   type Oid_Type is new Row_ID_Type;
function Generic_Command_Oid(
   Query : Query_Type'Class
) return Oid_Type;
\end{Code}

An instantiation example follows:

\begin{Example}
declare
   type My_Oid_Type is new Row_ID_@Type;

   function Command_Oid 
      is new Generic_Command_Oid(My_Oid_Type);
\end{Example}

\subsection{Error Status Reporting\label{Error Status Reporting}}

The Result\index{Result} function primitive is documented here for completeness.
Applications should avoid using this function, since the values that
it returns are very database technology specific.


\subsubsection{Result Codes}

The Result function primitive should be avoided in portable code. However,
there may be circumstances where a particular code is important to the
application developer.

\begin{Code}
function Result(
   Query : Query_Type
) return Natural;
  -- Returns Result_Type'Pos()
\end{Code}

The Result function returns a Natural\index{Natural} value. To obtain the Result\_Type\index{Result\_Type}
for the database being used, you should use;

\begin{Example}
declare
   Q : Query_Type;
   R : Result_Type;
begin
   R := Result_Type'Val(Result(Q));
\end{Example}

\subsubsection{Notes:}

\begin{itemize}
   \item The Execute\index{Execute} primitive will throw an exception if the execution failed
         (the PostgreSQL Nonfatal\_Error\index{Nonfatal\_Error} case).
   \item For SELECT\index{SELECT} queries, the fact that no rows are returned will be identifiable upon
         the first FETCH\index{FETCH} operation (the PostgreSQL Empty\_Query\index{Empty\_Query}
         case), or upon calling End\_of\_Query\index{End\_of\_Query}.
   \item When rows are returned (the PostgreSQL Tuples\_OK\index{Tuples\_OK} case), the application will
         successfully fetch at least one row.
   \item For other SQL commands, successful execution is determined by Execute
         not throwing an exception (the PostgreSQL Command\_OK\index{Command\_OK} case). 
\end{itemize}

\subsubsection{PostgreSQL Result Codes}

The PostgreSQL\index{PostgreSQL} result\index{result types} types are declared in the APQ\-.PostgreSQL package.
The Result\_Type values are highly PostgreSQL engine\index{engine} specific and are
enumerated in Table~\ref{t:pqresc}.

\begin{longtable}{|c|c|l|}
\hline 
Name              &  Value          &  Description\\
\hline
Empty\_Query      &  0              &  The query returned 0 rows of data\\
Command\_OK       &  1              &  The non-query statement executed successfully\\
Tuples\_OK        &  2              &  The SQL query returned at least 1 row of data\\
Copy\_Out         &  3              &  \\
Copy\_In          &  4              &  \\
Bad\_Response     &  5              &  Bad response from database server\\
Nonfatal\_Error   &  6              &  A non fatal error has occurred\\
Fata\_Error       &  7              &  A fatal error has occurred\\
\hline
\caption{PostgreSQL Result Codes}\label{t:pqresc}
\end{longtable}

\begin{quote}
Note that the numeric values in Table~\ref{t:pqresc} are subject to change if the
PostgreSQL\index{PostgreSQL} database server software designers choose to do so. Use
the enumerated names instead.
\end{quote}

\subsubsection{MySQL Result Codes}

Result types for MySQL\index{MySQL} are declared in package APQ\-.MySQL. These are
generated by the APQ install process from the MySQL database 
C macros provided. The following is the list from MySQL version 4.0.14 :

\begin{Code}
type Result_Type is (
   CR_NO_ERROR,
   ER_HASHCHK,
   ER_NISAMCHK,
   ER_NO,
   ER_YES,
   ER_CANT_CREATE_FILE,
   ER_CANT_CREATE_TABLE,
   ER_CANT_CREATE_DB,
   ER_DB_CREATE_EXISTS,
   ER_DB_DROP_EXISTS,
   ER_DB_DROP_DELETE,
   ER_DB_DROP_RMDIR,
   ER_CANT_DELETE_FILE,
   ER_CANT_FIND_SYSTEM_REC,
   ER_CANT_GET_STAT,
   ER_CANT_GET_WD,
   ER_CANT_LOCK,
   ER_CANT_OPEN_FILE,
   ER_FILE_NOT_FOUND,
   ER_CANT_READ_DIR,
   ER_CANT_SET_WD,
   ER_CHECKREAD,
   ER_DISK_FULL,
   ER_DUP_KEY,
   ER_ERROR_ON_CLOSE,
   ER_ERROR_ON_READ,
   ER_ERROR_ON_RENAME,
   ER_ERROR_ON_WRITE,
   ER_FILE_USED,
   ER_FILSORT_ABORT,
   ER_FORM_NOT_FOUND,
   ER_GET_ERRNO,
   ER_ILLEGAL_HA,
   ER_KEY_NOT_FOUND,
   ER_NOT_FORM_FILE,
   ER_NOT_KEYFILE,
   ER_OLD_KEYFILE,
   ER_OPEN_AS_READONLY,
   ER_OUTOFMEMORY,
   ER_OUT_OF_SORTMEMORY,
   ER_UNEXPECTED_EOF,
   ER_CON_COUNT_ERROR,
   ER_OUT_OF_RESOURCES,
   ER_BAD_HOST_ERROR,
   ER_HANDSHAKE_ERROR,
   ER_DBACCESS_DENIED_ERROR,
   ER_ACCESS_DENIED_ERROR,
   ER_NO_DB_ERROR,
   ER_UNKNOWN_COM_ERROR,
   ER_BAD_NULL_ERROR,
   ER_BAD_DB_ERROR,
   ER_TABLE_EXISTS_ERROR,
   ER_BAD_TABLE_ERROR,
   ER_NON_UNIQ_ERROR,
   ER_SERVER_SHUTDOWN,
   ER_BAD_FIELD_ERROR,
   ER_WRONG_FIELD_WITH_GROUP,
   ER_WRONG_GROUP_FIELD,
   ER_WRONG_SUM_SELECT,
   ER_WRONG_VALUE_COUNT,
   ER_TOO_LONG_IDENT,
   ER_DUP_FIELDNAME,
   ER_DUP_KEYNAME,
   ER_DUP_ENTRY,
   ER_WRONG_FIELD_SPEC,
   ER_PARSE_ERROR,
   ER_EMPTY_QUERY,
   ER_NONUNIQ_TABLE,
   ER_INVALID_DEFAULT,
   ER_MULTIPLE_PRI_KEY,
   ER_TOO_MANY_KEYS,
   ER_TOO_MANY_KEY_PARTS,
   ER_TOO_LONG_KEY,
   ER_KEY_COLUMN_DOES_NOT_EXITS,
   ER_BLOB_USED_AS_KEY,
   ER_TOO_BIG_FIELDLENGTH,
   ER_WRONG_AUTO_KEY,
   ER_READY,
   ER_NORMAL_SHUTDOWN,
   ER_GOT_SIGNAL,
   ER_SHUTDOWN_COMPLETE,
   ER_FORCING_CLOSE,
   ER_IPSOCK_ERROR,
   ER_NO_SUCH_INDEX,
   ER_WRONG_FIELD_TERMINATORS,
   ER_BLOBS_AND_NO_TERMINATED,
   ER_TEXTFILE_NOT_READABLE,
   ER_FILE_EXISTS_ERROR,
   ER_LOAD_INFO,
   ER_ALTER_INFO,
   ER_WRONG_SUB_KEY,
   ER_CANT_REMOVE_ALL_FIELDS,
   ER_CANT_DROP_FIELD_OR_KEY,
   ER_INSERT_INFO,
   ER_INSERT_TABLE_USED,
   ER_NO_SUCH_THREAD,
   ER_KILL_DENIED_ERROR,
   ER_NO_TABLES_USED,
   ER_TOO_BIG_SET,
   ER_NO_UNIQUE_LOGFILE,
   ER_TABLE_NOT_LOCKED_FOR_WRITE,
   ER_TABLE_NOT_LOCKED,
   ER_BLOB_CANT_HAVE_DEFAULT,
   ER_WRONG_DB_NAME,
   ER_WRONG_TABLE_NAME,
   ER_TOO_BIG_SELECT,
   ER_UNKNOWN_ERROR,
   ER_UNKNOWN_PROCEDURE,
   ER_WRONG_PARAMCOUNT_TO_PROCEDURE,
   ER_WRONG_PARAMETERS_TO_PROCEDURE,
   ER_UNKNOWN_TABLE,
   ER_FIELD_SPECIFIED_TWICE,
   ER_INVALID_GROUP_FUNC_USE,
   ER_UNSUPPORTED_EXTENSION,
   ER_TABLE_MUST_HAVE_COLUMNS,
   ER_RECORD_FILE_FULL,
   ER_UNKNOWN_CHARACTER_SET,
   ER_TOO_MANY_TABLES,
   ER_TOO_MANY_FIELDS,
   ER_TOO_BIG_ROWSIZE,
   ER_STACK_OVERRUN,
   ER_WRONG_OUTER_JOIN,
   ER_NULL_COLUMN_IN_INDEX,
   ER_CANT_FIND_UDF,
   ER_CANT_INITIALIZE_UDF,
   ER_UDF_NO_PATHS,
   ER_UDF_EXISTS,
   ER_CANT_OPEN_LIBRARY,
   ER_CANT_FIND_DL_ENTRY,
   ER_FUNCTION_NOT_DEFINED,
   ER_HOST_IS_BLOCKED,
   ER_HOST_NOT_PRIVILEGED,
   ER_PASSWORD_ANONYMOUS_USER,
   ER_PASSWORD_NOT_ALLOWED,
   ER_PASSWORD_NO_MATCH,
   ER_UPDATE_INFO,
   ER_CANT_CREATE_THREAD,
   ER_WRONG_VALUE_COUNT_ON_ROW,
   ER_CANT_REOPEN_TABLE,
   ER_INVALID_USE_OF_NULL,
   ER_REGEXP_ERROR,
   ER_MIX_OF_GROUP_FUNC_AND_FIELDS,
   ER_NONEXISTING_GRANT,
   ER_TABLEACCESS_DENIED_ERROR,
   ER_COLUMNACCESS_DENIED_ERROR,
   ER_ILLEGAL_GRANT_FOR_TABLE,
   ER_GRANT_WRONG_HOST_OR_USER,
   ER_NO_SUCH_TABLE,
   ER_NONEXISTING_TABLE_GRANT,
   ER_NOT_ALLOWED_COMMAND,
   ER_SYNTAX_ERROR,
   ER_DELAYED_CANT_CHANGE_LOCK,
   ER_TOO_MANY_DELAYED_THREADS,
   ER_ABORTING_CONNECTION,
   ER_NET_PACKET_TOO_LARGE,
   ER_NET_READ_ERROR_FROM_PIPE,
   ER_NET_FCNTL_ERROR,
   ER_NET_PACKETS_OUT_OF_ORDER,
   ER_NET_UNCOMPRESS_ERROR,
   ER_NET_READ_ERROR,
   ER_NET_READ_INTERRUPTED,
   ER_NET_ERROR_ON_WRITE,
   ER_NET_WRITE_INTERRUPTED,
   ER_TOO_LONG_STRING,
   ER_TABLE_CANT_HANDLE_BLOB,
   ER_TABLE_CANT_HANDLE_AUTO_INCREMENT,
   ER_DELAYED_INSERT_TABLE_LOCKED,
   ER_WRONG_COLUMN_NAME,
   ER_WRONG_KEY_COLUMN,
   ER_WRONG_MRG_TABLE,
   ER_DUP_UNIQUE,
   ER_BLOB_KEY_WITHOUT_LENGTH,
   ER_PRIMARY_CANT_HAVE_NULL,
   ER_TOO_MANY_ROWS,
   ER_REQUIRES_PRIMARY_KEY,
   ER_NO_RAID_COMPILED,
   ER_UPDATE_WITHOUT_KEY_IN_SAFE_MODE,
   ER_KEY_DOES_NOT_EXITS,
   ER_CHECK_NO_SUCH_TABLE,
   ER_CHECK_NOT_IMPLEMENTED,
   ER_CANT_DO_THIS_DURING_AN_TRANSACTION,
   ER_ERROR_DURING_COMMIT,
   ER_ERROR_DURING_ROLLBACK,
   ER_ERROR_DURING_FLUSH_LOGS,
   ER_ERROR_DURING_CHECKPOINT,
   ER_NEW_ABORTING_CONNECTION,
   ER_DUMP_NOT_IMPLEMENTED,
   ER_FLUSH_MASTER_BINLOG_CLOSED,
   ER_INDEX_REBUILD,
   ER_MASTER,
   ER_MASTER_NET_READ,
   ER_MASTER_NET_WRITE,
   ER_FT_MATCHING_KEY_NOT_FOUND,
   ER_LOCK_OR_ACTIVE_TRANSACTION,
   ER_UNKNOWN_SYSTEM_VARIABLE,
   ER_CRASHED_ON_USAGE,
   ER_CRASHED_ON_REPAIR,
   ER_WARNING_NOT_COMPLETE_ROLLBACK,
   ER_TRANS_CACHE_FULL,
   ER_SLAVE_MUST_STOP,
   ER_SLAVE_NOT_RUNNING,
   ER_BAD_SLAVE,
   ER_MASTER_INFO,
   ER_SLAVE_THREAD,
   ER_TOO_MANY_USER_CONNECTIONS,
   ER_SET_CONSTANTS_ONLY,
   ER_LOCK_WAIT_TIMEOUT,
   ER_LOCK_TABLE_FULL,
   ER_READ_ONLY_TRANSACTION,
   ER_DROP_DB_WITH_READ_LOCK,
   ER_CREATE_DB_WITH_READ_LOCK,
   ER_WRONG_ARGUMENTS,
   ER_NO_PERMISSION_TO_CREATE_USER,
   ER_UNION_TABLES_IN_DIFFERENT_DIR,
   ER_LOCK_DEADLOCK,
   ER_TABLE_CANT_HANDLE_FULLTEXT,
   ER_CANNOT_ADD_FOREIGN,
   ER_NO_REFERENCED_ROW,
   ER_ROW_IS_REFERENCED,
   ER_CONNECT_TO_MASTER,
   ER_QUERY_ON_MASTER,
   ER_ERROR_WHEN_EXECUTING_COMMAND,
   ER_WRONG_USAGE,
   ER_WRONG_NUMBER_OF_COLUMNS_IN_SELECT,
   ER_CANT_UPDATE_WITH_READLOCK,
   ER_MIXING_NOT_ALLOWED,
   ER_DUP_ARGUMENT,
   ER_USER_LIMIT_REACHED,
   ER_SPECIFIC_ACCESS_DENIED_ERROR,
   ER_LOCAL_VARIABLE,
   ER_GLOBAL_VARIABLE,
   ER_NO_DEFAULT,
   ER_WRONG_VALUE_FOR_VAR,
   ER_WRONG_TYPE_FOR_VAR,
   ER_VAR_CANT_BE_READ,
   ER_CANT_USE_OPTION_HERE,
   ER_NOT_SUPPORTED_YET,
   ER_MASTER_FATAL_ERROR_READING_BINLOG,
   ER_SLAVE_IGNORED_TABLE,
   CR_UNKNOWN_ERROR,
   CR_SOCKET_CREATE_ERROR,
   CR_CONNECTION_ERROR,
   CR_CONN_HOST_ERROR,
   CR_IPSOCK_ERROR,
   CR_UNKNOWN_HOST,
   CR_SERVER_GONE_ERROR,
   CR_VERSION_ERROR,
   CR_OUT_OF_MEMORY,
   CR_WRONG_HOST_INFO,
   CR_LOCALHOST_CONNECTION,
   CR_TCP_CONNECTION,
   CR_SERVER_HANDSHAKE_ERR,
   CR_SERVER_LOST,
   CR_COMMANDS_OUT_OF_SYNC,
   CR_NAMEDPIPE_CONNECTION,
   CR_NAMEDPIPEWAIT_ERROR,
   CR_NAMEDPIPEOPEN_ERROR,
   CR_NAMEDPIPESETSTATE_ERROR,
   CR_CANT_READ_CHARSET,
   CR_NET_PACKET_TOO_LARGE,
   CR_EMBEDDED_CONNECTION,
   CR_PROBE_SLAVE_STATUS,
   CR_PROBE_SLAVE_HOSTS,
   CR_PROBE_SLAVE_CONNECT,
   CR_PROBE_MASTER_CONNECT,
   CR_SSL_CONNECTION_ERROR,
   CR_MALFORMED_PACKET
);
\end{Code}

\subsubsection{Sybase Result Codes}

The Result\_Type values available from package APQ.Sybase are as follows;

\begin{Code}
type Result_Type is (
   Execution_Failed, -- ct_results() call failed
   No_Results,       -- Cmd processed, but no results
   Row_Results,      -- Cmd processed, and row results
   Cursor_Results,   -- Cmd processed, and cursor row result
   Info_Results,     -- Cmd processed, no row data, but info
   Compute_Results,  -- Computed results
   Param_Results,    -- Parameter results
   Status_Results    -- Status results
);
\end{Code}

\subsection{Generic APQ.Result\label{Generic APQ.Result}}

To enable generic database processing, APQ version 2.0 adds a new
API function which is declared at the APQ\-.Root\_Query\_Type\index{APQ.Root\_Query\_Type} object
level. This function returns a Natural result:

\begin{Code}
function Result(
   Query : Root_Query_Type;
) return Natural;
\end{Code}

The value returned, represents the Result\_Type'Pos(arg)\index{Result\_Type'Pos}. In generic
database code, you could use this generic function to retrieve the
value. Later it can be turned into the appropriate Result\_Type when
required by doing a conversion. The following example illustrates:

\begin{Example}
with APQ.MySQL.Client, APQ.PostgreSQL.Client;
...
procedure App(Q : Root_Query_Type'Class) is
   R :      Natural;
   PQ_R :   APQ.PostgreSQL.Result_Type;
   My_R :   APQ.MySQL.Result_Type;
begin
   ...
   R := APQ.Result(Q);
   if APQ.Engine_Of(Q) = Engine_MySQL then
      My_R := APQ.MySQL.Result_Type'Val(R);
      ...
   elsif APQ.Engine_Of(Q) = Engine_PostgreSQL then
      PQ_R := APQ.PostgreSQL.Result_Type'Val(R);
      ...
   ...
\end{Example}

The code above demonstrates how generic database code is able to test
for specific database error codes, when required.


\subsection{Generic APQ.Engine\_Of\label{Generic APQ.Engine_Of}}

As seen in the example of \Ref{Generic APQ.Result}, it is
sometimes necessary to determine in portable code, what database
technology is being used. Once this fact is known, the correct, more
specific action can be taken.  For example, the ``LIMIT n''\index{LIMIT} clause can
be added to MySQL\index{MySQL} queries to limit the number of returned rows for
greater efficiency.

The function primitive Engine\_Of\index{Engine\_Of} can be used to determine
the database technology being used:

\begin{Code}
type Database_Type is (
   Engine_PostgreSQL,
   Engine_MySQL,
   Engine_Sybase
);

function Engine_Of(
   Query : Root_Query_Type;
) return Database_Type;
\end{Code}

\subsection{Checked Execution}

For many utility programs \index{utility programs}where error reporting
and recovery have simple requirements, a more compact and convenient
way to execute queries can be applied. With checked execution, the
query is not only executed, but any SQL errors are intercepted and
reported\index{reported} to Standard\_Error\index{Standard\_Error} automatically. This saves the programmer
effort when writing simple utility programs. Once the SQL\_Error\index{SQL\_Error} exception
is intercepted and reported, the exception is re-raised to leave control
in the caller's hands. The important thing here is that the error\index{error}
is caught and reported.

The Execute\_Checked\index{Execute\_Checked} primitive has the following calling signature:

\begin{Code}
procedure Execute_Checked(
   Query :      in out Root_Query_Type;
   Connection : in out Root_Connection_Type'Class;
   Msg :        in     String := ""
);
\end{Code}

When the argument Msg is a non-empty\index{non-empty string} string like ``Dropping table
temp\_tbl'', the error message reported will be of the following
format:

\begin{Example}
*** SQL ERROR: Dropping table temp_tbl
FATAL_ERROR: ERROR: Relation "temp_tbl" does not exist
\end{Example}

The first line just identifies the fact that an SQL error occurred,
and reports the Msg text. The second line first reports Result\_Type'Image
of the error, and then reports the error message text as returned
by Error\_Message\index{Error\_Message}. In this case, the example shows that Result\_Type
Fatal\_Error was returned, and the error message returned from the
database server was ``ERROR: Relation ``temp\_tbl'' does not
exist''.

When the null string\index{null string} (or the default value for
the parameter) is given to argument Msg, then the SQL query is dumped\index{dumped}
out to Standard\_Error\index{Standard\_Error} instead. This is often useful for debugging\index{debugging}
purposes.

Changing the example found on page \pageref{Birthday Timezone Example}
slightly, we can apply the Execute\_Checked primitive in the place
of the Execute\index{Execute} call.

\begin{Example}
declare
   type Cust_No_Type is new Integer range 1000..100_000;
   type Birthday_Type is new APQ_Timestamp;

   procedure Append is new Append_Integer(Cust_No_Type);

   procedure Encode is new
      Encode_Integer(Cust_No_Type,Boolean);
   procedure Encode is new
      Encode_Timezone(Birthday_Type,APQ_Timezone,Boolean);

   C :            Connection_Type;
   Q :            Query_Type;
   Cust_No :      Cust_No_Type;     -- NOT NULL
   Birthday :     Birthday_Type;
   Birthday_TZ :  APQ_Timezone;     -- Timezone
   Birthday_Ind : Boolean;          -- Indicator
begin
   ...
   Prepare(Q,"INSERT INTO BIRTHDAY (CUST_NO,BIRTHDAY)");
   Append(Q, "VALUES (");
   Append(Q,Cust_No,",");
   Encode(Q,Birthday,Birthday_TZ,Birthday_Ind,")" & New_Line);
   Execute_Checked(Q,C);
\end{Example}


\subsection{Suppressing Checked Exceptions}

For utility work, it is sometimes convenient to have Execute\_Checked
report errors, but not raise SQL\_Error\index{SQL\_Error}. This is useful when you don't
care about the outcome but want the error to be reported when detected.
The raising or not raising of SQL\_Error can be controlled for the
Execute\_Checked\index{Execute\_Checked} primitive by calling
Raise\_Exceptions\index{Raise\_Exceptions}. It has the following calling
requirements:

\begin{Code}
procedure Raise_Exceptions(
   Query :    in out Query_Type;
   Raise_On : in     Boolean := True
);
\end{Code}

The following example shows how exceptions can be suppressed:

\begin{Example}
declare
   C : Connection_Type;
   Q : Query_Type;
begin
   ...
   Raise_Exceptions(Q,False); -- Suppress SQL_Error exception
   Execute_Checked(Q,C);      -- Report errors only
   Raise_Exceptions(Q,True);  -- Re-enable SQL_Error exceptions
\end{Example}

\subsection{Suppressing Checked Reports}

Occaisionally, it is useful to control whether or not reporting is
performed in the event of an SQL\_Error\index{SQL\_Error}. The reporting of errors can
be controlled by the Report\_Errors\index{Report\_Errors} primitive procedure:

\begin{Code}
procedure Report_Errors(
   Query :     in out Query_Type;
   Report_On : in     Boolean := True
);
\end{Code}

The default behaviour of a Query\_Type is to report errors and raise
SQL\_Error when Execute\_Checked experiences an SQL\_Error exception.
The reporting behaviour can be disabled\index{disable reporting} as follows:

\begin{Example}
declare
   C :   Connection_Type;
   Q :   Query_Type;
begin
   ...
   Report_Errors(Q,False); -- Suppress error reporting
   Execute_Checked(Q,C);
\end{Example}

Normally application programmers would not use Execute\_Checked with
error reporting disabled. However, it may be useful as a temporary
measure to cause error reporting while debugging\index{debugging} a program. Once the
debugging has been completed, a global Boolean\index{Boolean} value could be set
to false to prevent these errors from being reported.


\section{Transaction Operations}

Database transaction\index{Transaction} operations consist of:

\begin{itemize}
   \item BEGIN WORK\index{BEGIN WORK}
   \item COMMIT WORK\index{COMMIT WORK}
   \item ROLLBACK WORK\index{ROLLBACK WORK}
\end{itemize}

It is possible to build your own queries to accomplish these operations
but the programmer is encourage to use the primitive operations below
instead. One reason for using the APQ provided functions is to make
your application portable to different databases. There are slight
variations on the SQL syntax\index{syntax, SQL} required for these operations. It may
also be possible in the future to query the state of the transaction.%
\footnote{It is likely that a function like an In\_Transaction function will
be added in the future.%
}

The three primitives are named according to function \label{Begin, Commit and Rollback Work functions}
and are listed in Table~\ref{t:txp}.

\begin{table}
   \begin{center}
      \begin{tabular}{ll}
         Primitive Name    &  SQL Function\\
         \hline 
         Begin\_Work       &  BEGIN WORK\\
         Commit\_Work      &  COMMIT WORK\\
         Rollback\_Work    &  ROLLBACK WORK\\
      \end{tabular}
   \end{center}
   \caption{Transaction Primitives}\label{t:txp}
\end{table}

The specifications of these primitives are listed below:

\begin{Code}
procedure Begin_Work(
   Query :      in out Query_Type;
   Connection : in out Connection_Type'Class
);
\end{Code}

\begin{Code}
procedure Commit_Work(
   Query :      in out Query_Type;
   Connection : in out Connection_Type'Class
);
\end{Code}

\begin{Code}
procedure Rollback_Work(
   Query :      in out Query_Type;
   Connection : in out Connection_Type'Class
);
\end{Code}

These primitives will raise the exceptions listed in Table~\ref{t:tranpx}.

\begin{table}
   \begin{center}
      \begin{tabular}{ll}
         Exception Name    &  Reason\\
         \hline 
         Not\_Connected    &  There is no connection\\
         Abort\_State      &  A ROLLBACK is required\\
         SQL\_Error        &  This should not normally occur\\
      \end{tabular}
   \end{center}
   \caption{Transaction Primitives Exceptions}\label{t:tranpx}
\end{table}

The following simple example demonstrates the use of these primitives:

\begin{Example}
declare
   C :   Connection_Type;
   Q :   Query_Type;
begin
   ...
   Begin_Work(Q,C);
   ...
   Commit_Work(Q,C);
\end{Example}

\begin{description}
   \item [Note:] There is an implicit\index{implicit Clear} Clear operation before and after the
      execution of these operations for the Query\_Type\index{Query\_Type} object. The original
      fetch mode of the object is preserved after the call. 
\end{description}

\subsection{The PostgreSQL Abort\_State}

The Abort\_State\index{Abort\_State} is unique to the PostgreSQL\index{PostgreSQL} database. Other databases
will tolerate failed steps within a transaction, but PostgreSQL will not.
The Abort\_State \label{Abort_State exception} exception indicates that the database
was in a transaction%
\footnote{A {}``BEGIN WORK'' statement was executed.%
} when a processing error occurred (like a duplicate key on insert
error). Once an error\index{error in transaction} is encountered within a PostgreSQL transaction,
the only course to recovery is by executing a ROLLBACK\index{ROLLBACK} WORK statement
(this is done by the Rollback\_Work\index{Rollback\_Work} call shown above). If duplicate\index{duplicate inserts}
inserts\index{inserts} may occur, you must test for them in advance of the INSERT\index{INSERT},
to avoid placing the transaction into the ``abort state''\index{abort state}. Note
that this PostgreSQL\index{PostgreSQL} behavior is different from other database vendor
offerings.

The ``abort state''\index{abort state} itself is maintained in the Connection\_Type\index{Connection\_Type}
object, causing the state\index{state} to influence all Query\_Type objects using
the same connection. To clear the status\index{status}, you must perform a Rollback\_Work\index{Rollback\_Work}
call on any Query\_Type object, using the affected Connection\_Type
object where the status is saved.

The Query\_Type object is used to form the SQL\index{SQL} statement and to hold
the result\index{result status} status. In application programming, you may want to dedicate
one Query\_Type object for each transaction in progress.%
\footnote{This will be important later, if you want to query whether or not
you are in a transaction.%
}

\section{Fetch Operations}

Some database operations, particularly SELECT\index{SELECT}, return
results. There are two fetch\index{fetch} related primitives: 

\begin{enumerate}
   \item Sequential\index{Sequential access} row fetch\index{fetch, sequential}
   \item Random\index{Random access} access row fetch\index{fetch, random}
\end{enumerate}

The sequential fetch permits serial\index{serial access} access of the resulting rows (tuples\index{tuples}).
Random access fetching permits rapid\index{rapid access} access to particular row results.

\subsection{Fetch Limitations\label{Fetch Limitations}}

Some databases like PostgreSQL\index{PostgreSQL} have \emph{no} limitations\index{limitations} on how row
data is fetched. The fetch\index{fetch} may be sequential or random, as the application
requires. Some other databases however, require some planning by the
application programmer in this area. This distinction, and the API
to control this problem is new to APQ 2.0 and later, for database
engines\index{engines} that require it.

For example, MySQL\index{MySQL} retrieves row data into the client program's address\index{address space}
space in one of two ways:

\begin{itemize}
   \item one row at a time, but all rows must be fetched
   \item all rows are loaded into client memory\index{client memory}, for random access by the application
\end{itemize}

For large result sets\index{large result sets}, fetching one row%
\footnote{This is done using the mysql\_use\_result() function.%
} at a time is very practical. However, MySQL\index{MySQL} requires that the program
fetch \emph{all} row data. If the result set\index{result set} is large, and only an
initial number of rows are required, this can be a serious performance
issue. This is a greater problem if the application would like to
cancel\index{cancel} the operation.

When random access%
\footnote{This is done using the mysql\_store\_result() function.%
} of rows is required, MySQL\index{MySQL} requires that all row data be retrieved
and stored into the client program (behind the scenes). Fetching all
of this data into client memory\index{client memory} can be impractical for size\index{size} reasons
for large numbers of rows (there is an SQL work-around\index{SQL work-around} for this).

The default APQ query mode varies according to database as listed in Table~\ref{t:fchmd}.

\begin{table}
   \begin{center}
      \begin{tabular}{lll}
         Database       &  Default Mode      &  Comments\\
         \hline 
         PostgreSQL     &  Random\_Fetch     &  Random or Sequential supported\\
         MySQL          &  Random\_Fetch     &  Watch \# of rows returned\\
         Sybase         &  Sequential\_Fetch &  Random\_Fetch not supported\\
      \end{tabular}
   \end{center}
   \caption{APQ Fetch Modes}\label{t:fchmd}
\end{table}

From the table, you can see that PostgreSQL\index{PostgreSQL} has no difficulty in either
mode. MySQL\index{MySQL} supports both modes, but the programmer must be careful
about the result set size returned if Random\_Fetch mode is in use.
Finally, Sybase\index{Sybase} support does not support Random\_Fetch\index{Random\_Fetch} mode (Sybase
\emph{can} fetch rows randomly with a cursor, but only sequential
cursors\index{cursors} are currently supported by APQ).

All database engines\index{engines} support sequential\index{sequential} access \- even in random\index{random}
access mode. Even if you are using the default or configured Random\_Fetch\index{Random\_Fetch}
mode, APQ will return rows sequentially unless a specific row is requested.
If the application programmer is using a database that is limited
in this way (MySQL), and has determined that fetching all results
into client memory is not suitable, then the mode of the Query\_Type\index{Query\_Type}
needs to be changed by the program to use sequential access instead.
See the next few sections on how to control the fetch query mode.

If you are only planning to use PostgreSQL\index{PostgreSQL}, you can effectively ignore
the sections about Fetch Query modes. However, if you plan to write
your application in a database generic sort of way, or support MySQL\index{MySQL}
and/or Sybase\index{Sybase} code, then you need to plan for the fetch query modes
in your code.


\subsection{Fetch Query Modes\label{Fetch Query Modes}}

Due to the performance limitations of different database engines,
APQ provides the application programmer a way to control the fetch\index{fetch}
mode used. Package APQ defines the Fetch\_Mode\_Type\index{Fetch\_Mode\_Type} for this purpose:

\begin{Code}
type Fetch_Mode_Type is (
   Sequential_Fetch,    -- All databases
   Random_Fetch,        -- PostgreSQL, MySQL, not Sybase yet
   Cursor_For_Update,   -- Sybase
   Cursor_For_Read_Only -- Sybase
);
\end{Code}

The last two modes related to cursors will be discussed separately.

The application programmer can query the fetch mode that is in effect.
The Fetch\-\_Mode\index{Fetch\_Mode} function primitive returns the current state
of the Query\-\_Type object:

\begin{Code}
function Fetch_Mode(
   Q : Query_Type
) return Fetch_Mode_Type;
\end{Code}

To change the current mode in effect, use the function primitive Set\_Fetch\_Mode\index{Set\_Fetch\_Mode}:

\begin{Code}
procedure Set_Fetch_Mode(
   Q :    in out Query_Type;
   Mode : in     Fetch_Mode_Type
);
\end{Code}

The application should only change the query mode \emph{prior} to
the \emph{execution} of the query. When Execute\index{Execute} or Execute\_Checked\index{Execute\_Checked}
are called, APQ must commit to the fetch method being used. For this
reason, set the query mode when the Query\_Type is initially declared,
after a call to Reset\index{Reset} or Prepare\index{Prepare}. The mode must
be established prior to executing the query.

Table \ref{t:sfmdx} lists the exceptions that may be raised by Set\_Fetch\_Mode.

\begin{table}
   \begin{center}
      \begin{tabular}{ll}
         Exception Name    &  Reason\\
         \hline 
         Failed            &  Query results exist - cannot change mode\\
      \end{tabular}
   \end{center}
   \caption{Set\_Fetch\_Mode Exceptions}\label{t:sfmdx}
\end{table}


\subsection{Sequential Fetch}

The \emph{Query\_Type} object is always positioned at the first row\index{first row}
after the query has been executed. Sequential fetches\index{fetches} can then be
performed to retrieve the first row, through to the last resulting
row. The sequential \textbf{Fetch} primitive has the following calling
arguments:

\begin{Code}
procedure Fetch(
   Q : in out Query_Type
);
\end{Code}

A sequential fetch can always be made, whether the query object is
in sequential or random mode. However, be aware that some databases
(MySQL\index{MySQL}) require that all results be fetched when in Sequential\_Fetch\index{Sequential\_Fetch}
mode. The APQ default varies according to the database software being
used. See \Ref{Fetch Query Modes} to establish a mode explicitly.

Table \ref{t:fchx} lists the exceptions that can be raised by Fetch.

\begin{table}
   \begin{center}
      \begin{tabular}{ll}
         Exception Name    &  Reason\\
         \hline 
         No\_Result        &  There was no command executed\\
         No\_Tuple         &  There were no result rows returned\\
         Not\_Supported    &  The fetch mode in effect is not supported\\
      \end{tabular}
   \end{center}
   \caption{Fetch Exceptions}\label{t:fchx}
\end{table}

The No\_Result\index{No\_Result} exception is raised when the Query\_Type
object is in the wrong\index{wrong state} state. For example, if the Query\_Type
object is cleared, and/or an SQL query is built but not Executed,
then a No\_Result exception will be raised.

The No\_Tuple\index{No\_Tuple} exception is raised to indicate that no rows
were available, or that there are no more rows remaining. In Sequential\_Fetch
mode, this indicates that there are no more rows to be returned. In
Random\_Fetch\index{Random\_Fetch} mode, this indicates that no rows were returned.%
\footnote{Note that requesting a non existant row in random fetch mode will
not raise an exception until a value is extracted.%
}

The Not\_Supported\index{Not\_Supported} exception is also possible with APQ 2.2
and later. This exception is raised to indicate that the database
software being used does not support the current Fetch\_Mode\index{Fetch\_Mode} that
is in effect.

The following code shows a normal sequential fetch loop:\label{Sequential Fetch Example}

\begin{Example}
declare
   C : Connection_Type;
   Q : Query_Type;
begin
   Prepare(Q,"SELECT CUST_NO,CUST_NAME");
   Append(Q, "FROM CUSTOMER");
   Execute(Q,C);
   loop
      begin
         Fetch(Q);
      exception
         when APQ.No_Tuple =>
            exit;
      end;
      ...
   end loop;
   Clear(Q); -- Release any query results
\end{Example}

Clearing the query (or allowing it to fall out of scope) is recommended.
This releases\index{releases} resources\index{resources} that are
holding any prior query results.


\subsection{Random Fetch}

The random fetch\index{fetch} operation requires the use of the Tuple\_Index\_Type\index{Tuple\_Index\_Type}
defined in the package APQ:

\begin{Code}
type Tuple_Index_Type is mod 2 ** 64;
First_Tuple_Index : constant Tuple_Index_Type := 1;
\end{Code}

The random Fetch primitive has the following calling arguments:

\begin{Code}
procedure Fetch(
   Q :  in out Query_Type;
   TX : in     Tuple_Index_Type
);
\end{Code}

Table \ref{t:rfchx} lists the possible exceptions for
a random Fetch call.

\begin{table}
   \begin{center}
      \begin{tabular}{ll}
         Exception Name    &  Reason\\
         \hline 
         No\_Result        &  There was no command executed\\
         No\_Tuple         &  There were no result rows returned\\
         Not\_Supported    &  Random access is not supported\\
      \end{tabular}
   \end{center}
   \caption{Random Fetch Exceptions}\label{t:rfchx}
\end{table}

A random fetch example is provided below:\label{Random Fetch Example}

\begin{Example}
declare
   C : Connection_Type;
   Q : Query_Type;
begin
   Prepare(Q,"SELECT CUST_NO,CUST_NAME");
   Append(Q, "FROM CUSTOMER");
   Execute(Q,C);

   for TX in 1..Tuple_Index_Type(Tuples(Q)) loop
      Fetch(Q,TX);
      ...
   end loop;
   Clear(Q);
\end{Example}

The function Tuples(Q)\index{Tuples} that was used in the \emph{for} loop, returns
the number of result rows for the query.%
\footnote{Which can be zero.%
} A slight modification of this loop could permit processing the rows
in reverse order.


\subsubsection{Notes:}

\begin{enumerate}
   \item If the tuple\index{tuple} index \emph{TX} provided to Fetch is out
         of range for the result set, no exception will be raised. An exception
         \emph{will} be raised however, when the application attempts to fetch
         any value from that out of range row.
   \item Any subsequent sequential fetch operation will fetch the row following
         the last randomly accessed row.
\end{enumerate}

\subsection{Function End\_of\_Query\label{End_of_Query}}

To facilitate sequential
fetch operations, the End\_of\_Query primitive function was provided in
early versions of APQ. This API remains, but is considered
\emph{obsolete} and should be avoided. MySQL\index{MySQL} users should not use it at
all, due to the bug\index{bug} present in the MySQL client library. See the note
that follows the tables.

\begin{quote}
Note: This function is depreciated. Catch the No\_Tuple exception
instead for greater database portability.
\end{quote}

The calling requirements are summarized in the following specification:

\begin{Code}
function End_of_Query(
   Q : Query_Type
) return Boolean;
\end{Code}

Table \ref{t:eoqx} lists the exceptions for this function.

\begin{table}
   \begin{center}
      \begin{tabular}{ll}
         Exception Name    &  Reason\\
         \hline 
         No\_Result        &  There was no command executed\\
      \end{tabular}
   \end{center}
   \caption{End\_of\_Query Exceptions}\label{t:eoqx}
\end{table}

The End\_of\_Query\index{End\_of\_Query} function returns a Boolean\index{Boolean} result:

\begin{description}
   \item [False] there is at least one more result row available (not at end)
   \item [True] there are no more rows\index{rows} available (at end)
\end{description}

\begin{quote}
MySQL Note: 
The MySQL\index{MySQL} implementation of End\_Of\_Query is not a good
one. End\_Of\_Query\index{End\_of\_Query} should \emph{not} be used. The problem is located
in the MySQL C client library that comes with MySQL. The C mysql\_eof()\index{mysql\_eof()}
function returns false after reading the last row. It is only by fetching
one more row and discovering that there are no more rows, that mysql\_eof()
then starts to return true. In other words, it returns true, when
the end has already been reached. Since there is no way to work around
this problem in MySQL, a developer should avoid using End\_Of\_Query
completely.
\end{quote}

Catch the exception No\_Tuple\index{No\_Tuple} instead, when fetching rows.


\subsection{Function Tuple}

The Tuple\index{Tuple} function primitive is an information function that
returns the current tuple number\index{tuple number} that was last
fetched. If there has been no fetch yet, the No\_Tuple\index{No\_Tuple} exception
is raised. The calling signature is as follows:

\begin{Code}
function Tuple(
   Q : Query_Type
) return Tuple_Index_Type;
\end{Code}

Tuple can raise the exceptions listed in Table~\ref{t:tupx}.

\begin{table}
   \begin{center}
      \begin{tabular}{ll}
         Exception Name    &  Reason\\
         \hline 
         No\_Tuple         &  There was no fetch performed yet\\
      \end{tabular}
   \end{center}
   \caption{Tuple Exceptions}\label{t:tupx}
\end{table}

The following example shows the function being used:

\begin{Example}
declare
   C : Connection_Type;
   Q : Query_Type;
   X : Tuple_Index_Type;
begin
   Prepare(Q,"SELECT CUST_NO,CUST_NAME");
   Append(Q, "FROM CUSTOMER");
   Execute(Q,C);

   loop
      begin
         Fetch(Q);
      exception
         when APQ.No_Tuple =>
            exit;
      end;
      TX := Tuple(Q); -- Get Row #
      ...
   end loop;
   Clear(Q);
\end{Example}

\subsection{Rewind Procedure}

Sometimes it is desireable to reprocess results sequentially. This
is easily accomplished with the Rewind\index{Rewind} primitive. This
primitive merely alters the state of the Query\_Type object such that
the next fetch\index{fetch} operation will start with the first row. This only
works when the fetch mode is Random\_Fetch\index{Random\_Fetch}.

The calling requirements are listed as follows:

\begin{Code}
procedure Rewind(
   Q : in out Query_Type
);
\end{Code}

Table \ref{t:rwx} lists the possible exceptions for Rewind.

\begin{table}
   \begin{center}
      \begin{tabular}{ll}
         Exception Name    &  Reason\\
         \hline 
         SQL\_Error        &  The Query\_Type is not in Random\_Fetch mode\\
      \end{tabular}
   \end{center}
   \caption{Rewind Exceptions}\label{t:rwx}
\end{table}

The following example shows the Rewind procedure being used:

\begin{Example}
declare
   C : Connection_Type;
   Q : Query_Type;
begin
   Prepare(Q,"SELECT CUST_NO,CUST_NAME");
   Append(Q, "FROM CUSTOMER");
   Execute(Q,C);

   loop
      begin
         Fetch(Q);
      exception
         when APQ.No_Tuple =>
            exit;
      end;
      ...
   end loop;

   -- REPROCESS THE QUERY RESULTS :

   Rewind(Q);

   loop
      begin
         Fetch(Q);
      exception
         when APQ.No_Tuple =>
            exit;
      end;
      ...
   end loop;
   Clear(Q);
\end{Example}

\subsection{Tuples Function}

You have already seen this function used in the example on page
\pageref{Random Fetch Example}. This information function returns the
number of result rows that are available. It should only be called after
the \emph{Query\_Type} has been executed however. Otherwise the
No\_Result exception will be raised by Tuples\index{Tuples}.

The calling requirement for this function is summarized as follows:

\begin{Code}
function Tuples(
   Q : Query_Type
) return Tuple_Count_Type;
\end{Code}

Tuples can raise the exceptions listed in Table~\ref{t:tupsx}. For an
example of use, see \Ref{Random Fetch Example}.

\begin{table}
   \begin{center}
      \begin{tabular}{ll}
         Exception Name    &  Reason\\
         \hline 
         No\_Result        &  There was no execute performed yet\\
      \end{tabular}
   \end{center}
   \caption{Tuples Exceptions}\label{t:tupsx}
\end{table}

\subsection{Using Cursors}

For databases supporting cursor\index{cursor} operations in their client\index{client libraries} libraries
(Sybase), APQ will automatically generate cursor names for each Query\_Type
object. The function Cursor\_Name\index{Cursor\_Name} will retrieve this for you (see
\Ref{Cursor_Name}).

To perform cursor operations, there are the following general steps
to follow:

\begin{enumerate}
   \item Set the cursor mode of the Query\_Type object.
   \item Prepare the SQL query to fetch rows.
   \item Execute the SQL query.
   \item Fetch a row.
   \item Prepare the inner SQL query using ``WHERE CURRENT OF''\index{WHERE CURRENT OF}
   \item Execute the inner query
   \item Repeat steps 4-6 as required
\end{enumerate}

Describing cursor operation in APQ is probably best done with a concrete
example. Assume a simple SALARIES table like the one below:

\begin{SQL}
CREATE TABLE SALARIES (
   EMPNO    INT NOT NULL PRIMARY KEY,
   SALARY   REAL NOT NULL
);
\end{SQL}

Our example, will assume the following values in the table:

\begin{SQL}
select * from salaries
go
empno       salary
----------- --------------------
          2         56000.000000
          3         82800.000000
          4         82500.000000
          5         43600.000000

4 rows affected
1>
\end{SQL}

Our application is tasked with the job of rewarding all employees
with salaries greater than or equal to \$50,000 with an increase of
5\% (life is often unfair). It is possible to construct an SQL\index{SQL} statement
to do this without using cursors but we're going to build the shell
of a program that will be required to get approval from the user (using
a GUI\index{GUI} etc.) Getting user approval is not possible in SQL, so we need
to code a fetch loop and an inner query to do the update if the approval
is granted. This is one example of how a cursor can be useful.

The following is a Sybase\index{Sybase} APQ program listing for the
update:\label{Cursor Example Program}

\begin{Example}
with Ada.Text _IO;
with APQ.Sybase.Client;
use APQ, APQ.Sybase.Client, Ada.Text_IO;

procedure Salaries is
   function Value is new Integer_Value(Integer);
   function Value is new Float_Value(APQ_Double);
   procedure Append is new Append_Float(APQ_Double);

   C :      Connection_Type;
   Q :      Query_Type;
   Q2 :     Query_Type;
   Empno :  Integer;
   Salary : APQ_Double;
begin
   Set_Instance(C,"SYBIL");
   Set_DB_Name(C,"wwg");
   Set_User_Password(C,"wwg","somepw");
   Connect(C);

   Begin_Work(Q,C);
   Set_Fetch_Mode(Q,Cursor_For_Update);

   Prepare(Q,    "SELECT EMPNO,SALARY");
   Append_Line(Q,"FROM SALARIES");
   Execute(Q,C);

   loop
      begin
         Fetch(Q);
      exception
         when No_Tuple =>
            exit;
      end;

      Empno  := Value(Q,1);
      Salary := Value(Q,2);
      Put_Line("Empno=" & Empno'Img & " Salary=" 
         & Salary'Img);
      if Salary >= 50_000.0 then
         Salary := Salary * 1.05; -- 5% raise

         Prepare(Q2,    "UPDATE SALARIES");
         Append(Q2,     "SET SALARY = ");
         Append(Q2,Salary,Line_Feed);
         Append_Line(Q2,"WHERE CURRENT OF " & Cursor_Name(Q));
         Execute(Q2,C);
         Clear(Q2);
      end if;
   end loop;
   Clear(Q);

   Commit_Work(Q,C);
   Disconnect(C);
end Salaries;
\end{Example}

The fetch mode of query object Q is set to Cursor\_For\_Update\index{Cursor\_For\_Update}. This is
critical for the cursor\index{cursor} operation to work (you can also use
Cursor\_For\_Read\_Only for non-update operations). The outer query\index{outer query} is
formed, which is then executed. When the program decides it wants to
update a particular row (here we pretend the user has approved it from
the terminal), we prepare and execute the inner query\index{inner query} Q2. The special
WHERE clause ``WHERE CURRENT OF''\index{WHERE CURRENT OF} is used, naming the cursor being
used from the outer query object Q. The function Cursor\_Name(Q)\index{Cursor\_Name}
supplies the cursor name for it. Once the inner query is processed, more
outer query rows can be fetched\index{fetched} and processed in like manner.

Note that the inner query can also include DELETE\index{DELETE} ... WHERE CURRENT
OF\index{WHERE CURRENT OF} operations. Using a cursor\index{cursor} guarantees that when you go to delete
the row, that some other external process hasn't deleted it before
you have (this avoids experiencing an error). Similarly for updates\index{updates},
if the correct isolation\index{isolation level} level is used, you can be sure that the values
in the row have not changed by the time you perform the inner query.

\section{Column Information Functions}

After a query has been executed, which returns a set of rows, it is
sometimes necessary to obtain information\index{information, column} about the columns. Many
of the functions make use of the following data type: 

\begin{Code}

	type Column_Index_Type is new Postive;

\end{Code}

\begin{table}
   \begin{center}
      \begin{tabular}{ll}
         Function Name           &  Purpose\\
         \hline   
         Columns                 &  Return the \# of columns in each row\\
         Column\_Name            &  Return the column name for an index value\\
         Column\_Index           &  Return the index for a column name\\
         Column\_Type            &  Return type information for the column\\
         Is\_Null                &  Test if a column is null\\
      \end{tabular}
   \end{center}
   \caption{Column Information Functions}\label{t:cif}
\end{table}

Table \ref{t:cif}
lists four column information functions that are available to the
application programmer.
All of these functions will raise the exception No\_Result\index{No\_Result}
if an execute has not been performed successfully on the Query\_Type
object.

Note that the case of the column name returned will match the SQL
case policy for the connection. If the policy is Upper\_Case\index{Upper\_Case}, then
column names are returned in uppercase. Other possibilities follow
the established case policy\index{case policy}.


\subsection{Function Columns}

The Columns primitive function returns the number of columns\index{columns}
available in each row of the result set. The calling arguments are
summarized as follows:

\begin{Code}
function Columns(
   Q : Query_Type
) return Natural;
\end{Code}

The Columns\index{Columns} function may raise the exceptions found in
Table~\ref{t:cex}.

\begin{table}
   \begin{center}
      \begin{tabular}{ll}
         Exception Name    &  Reason\\
         \hline 
         No\_Result        &  There was no execute performed\\
      \end{tabular}
   \end{center}
   \caption{Columns Exceptions}\label{t:cex}
\end{table}

The following example shows the function at work:

\begin{Example}
declare
   C : Connection_Type;
   Q : Query_Type;
begin
   Prepare(Q,"SELECT CUST_NO,CUST_NAME");
   Append(Q, "FROM CUSTOMER");
   Execute(Q,C);

   loop
      begin
         Fetch(Q);
      exception
         when APQ.No_Tuple =>
            exit;
      end;

      for CX in 1..Column_Index_Type(Columns(Q)) loop
         ...process each column...
      end loop;
   end loop;
   Clear(Q);
\end{Example}

\subsection{Function Column\_Name}

The primitive Column\_Name\index{Column\_Name} returns the name of the column for a particular
column index\index{column index} value. The calling requirements
are summarized in the following table:

\begin{Code}
function Column_Name(
   Query : in Query_Type;
   Index : in Column_Index_Type
) return String;
\end{Code}

The Column\_Name function may raise the exceptions listed in Table~\ref{t:cnx}.

\begin{table}
   \begin{center}
      \begin{tabular}{ll}
         Exception Name    &  Reason\\
         \hline 
         No\_Result        &  There was no execute performed\\
         No\_Column        &  Bad Column\_Index\_Type value\\
      \end{tabular}
   \end{center}
   \caption{Column\_Name Exceptions}\label{t:cnx}
\end{table}

Note that the case of the column name returned will match the SQL
case policy for the connection. If the policy is Upper\_Case\index{Upper\_Case}, then
column names are returned in uppercase. Other possibilities follow
the established case policy\index{case policy}.

The following example shows the function in use:

\begin{Example}
declare
   C : Connection_Type;
   Q : Query_Type;
begin
   Prepare(Q, "SELECT CUST_NO,CUST_NAME");
   Append(Q,  "FROM CUSTOMER");
   Execute(Q,C);

   loop
      begin
         Fetch(Q);
      exception
         when No_Tuple =>
            exit;
      end;

      for CX in 1..Column_Index_Type(Columns(Q)) loop
         Put_Line("Column Name: " & Column_Name(Q,CX));
      end loop;
   end loop;
   Clear(Q);
\end{Example}

\subsection{Function Column\_Index}

If you have a column name, but want to know the column index value,
then the Column\_Index\index{Column\_Index} primitive function can be used. It's calling
requirements are as follows:

\begin{Code}
function Column_Index(
   Query : in Query_Type;
   Name :  in String
) return Column_Index_Type;
\end{Code}

The function may raise any of the exceptions listed in Table~\ref{t:cidxx}.

\begin{table}
   \begin{center}
      \begin{tabular}{ll}
         Exception Name    &  Reason\\
         \hline 
         No\_Result        &  There was no execute performed\\
         No\_Column        &  Unknown column name\\
      \end{tabular}
   \end{center}
   \caption{Column\_Index Exceptions}\label{t:cidxx}
\end{table}

The following rather contrived example shows the Column\_Index function
used in the pragma\index{pragma assert} assert statement:

\begin{NumberedExample}
declare
   C : Connection_Type;
   Q : Query_Type;
begin
   Prepare(Q,"SELECT CUST_NO,CUST_NAME");
   Append(Q, "FROM~CUSTOMER");
   Execute(Q,C);

   loop
      begin
         Fetch(Q);
      exception
         when No_Tuple =>
            exit;
      end;

      for CX in 1..Column_Index_Type(Columns(Q)) loop
         declare
            Col_Name : String := Column_Name(Q,CX);
         begin
            Put_Line("Column Name: " & Col_Name);
            pragma assert(CX = Column_Index(Col_Name));\label{Ex:pragma}
         end;
      end loop;
   end loop;
   Clear(Q);
\end{NumberedExample}

The pragma statement in line~\ref{Ex:pragma} is not necessary, but 
was included in this example to show how the Column\_Index and the
Column\_Name are related.

\subsection{Function Column\_Type}

\index{Column\_Type}The column type information varies according to the database product
being used. The following subsections are specific to the APQ supported
database products.

\subsubsection{PostgreSQL Type Information}

The Column\_Type\index{Column\_Type} primitive is the beginning of type information for
the column. This function returns the Row\_ID\_Type\index{Row\_ID\_Type}
value that describes the type in the pg\_type\index{pg\_type} PostgreSQL\index{PostgreSQL} table\index{table}. See
the PostgreSQL database documentation for more details.

The Column\_Type calling signature is as follows:

\begin{Code}
function Column_Type(
   Q :  in Query_Type;
   CX : in Column_Index_Type
) return Row_ID_Type;
\end{Code}

Table \ref{t:ctypx} lists the exceptions that
may be raised by Column\_Type.

\begin{table}
   \begin{center}
      \begin{tabular}{ll}
         Exception Name    &  Reason\\
         \hline 
         No\_Result        &  There was no execute performed\\
         No\_Column        &  Unknown column name\\
      \end{tabular}
   \end{center}
   \caption{Column\_Type Exceptions}\label{t:ctypx}
\end{table}

\subsubsection{MySQL Type Information}

The field\index{field types} types supported by MySQL\index{MySQL} are defined by APQ\-.MySQL\-.Field\_Type.
The programmer may use the Query\_Type primitive APQ\-.MySQL\-.Client\-.Column\_Type
to determine the column's type:

\begin{Code}
function Column_Type(
   Q :  in Query_Type;
   CX : in Column_Index_Type
) return Field_Type;
\end{Code}

At the writing of this manual, the values in Table~\ref{t:mysqlftyp}
are supported MySQL field (column) type names.

\begin{table}
   \begin{center}
      \begin{tabular}{ll}
         Field\_Type                &  MySQL Datatype\\
         \hline 
         FIELD\_TYPE\_DECIMAL       &  DECIMAL\\
         FIELD\_TYPE\_TINY          &  TINYINT | BOOLEAN\\
         FIELD\_TYPE\_SHORT         &  SMALLINT\\
         FIELD\_TYPE\_LONG          &  INTEGER\\
         FIELD\_TYPE\_FLOAT         &  FLOAT\\
         FIELD\_TYPE\_DOUBLE        &  DOUBLE\\
         FIELD\_TYPE\_NULL          &  BOOLEAN\\
         FIELD\_TYPE\_TIMESTAMP     &  TIMESTAMP\\
         FIELD\_TYPE\_LONGLONG      &  BIGINT\\
         FIELD\_TYPE\_INT24         &  MEDIUMINT\\
         FIELD\_TYPE\_DATE          &  DATE\\
         FIELD\_TYPE\_TIME          &  TIME\\
         FIELD\_TYPE\_DATETIME      &  DATETIME\\
         FIELD\_TYPE\_YEAR          &  YEAR\\
         FIELD\_TYPE\_NEWDATE       &  ?\\
         FIELD\_TYPE\_ENUM          &  ENUM\\
         FIELD\_TYPE\_SET           &  SET\\
         FIELD\_TYPE\_TINY\_BLOB    &  TINYTEXT | TINYBLOB\\
         FIELD\_TYPE\_MEDIUM\_BLOB  &  MEDIUMTEXT | MEDIUMBLOB\\
         FIELD\_TYPE\_LONG\_BLOB    &  LONGTEXT | LONGBLOB\\
         FIELD\_TYPE\_BLOB          &  TEXT | BLOB\\
         FIELD\_TYPE\_VAR\_STRING   &  VARCHAR(N)\\
         FIELD\_TYPE\_STRING        &  CHAR(N)\\
      \end{tabular}
   \end{center}
   \caption{MySQL Field Types}\label{t:mysqlftyp}
\end{table}

APQ does not yet fully support all of MySQL data types.


\subsection*{Sybase Type Information}

Sybase\index{Sybase} also provides type information:

\begin{Code}
function Column_Type(
   Q :  in Query_Type;
   CX : in Column_Index_Type
) return Field_Type;
\end{Code}

Sybase is capable of returning types listed in the Column\_Type column
of Table~\ref{t:sycoltyp}.

\begin{table}
   \begin{center}
      \begin{tabular}{lll}
         Column\_Type               &  Sybase Database Type &  APQ Support\\
         \hline 
         Type\_CHAR                 &  CHAR/VARCHAR         &  Yes\\
         Type\_BINARY               &  BINARY/VARBINARY     &  No\\
         Type\_LONGCHAR             &  \emph{None}          &  No\\
         Type\_LONGBINARY           &  \emph{None}          &  No\\
         Type\_TEXT                 &  TEXT                 &  Yes\\
         Type\_IMAGE                &  IMAGE                &  No\\
         Type\_TINYINT              &  TINYINT              &  Yes\\
         Type\_SMALLINT             &  SMALLINT             &  Yes\\
         Type\_INT                  &  INT/INTEGER          &  Yes\\
         Type\_REAL                 &  REAL                 &  Yes\\
         Type\_FLOAT                &  FLOAT                &  Yes\\
         Type\_BIT                  &  BIT                  &  Yes\\
         Type\_DATETIME             &  DATETIME             &  Yes\\
         Type\_DATETIME4            &  SMALLDATETIME        &  Yes\\
         Type\_MONEY                &  MONEY                &  Yes\\
         Type\_MONEY4               &  SMALLMONEY           &  Yes\\
         Type\_NUMERIC              &  NUMERIC              &  Yes\\
         Type\_DECIMAL              &  DECIMAL              &  Yes\\
         Type\_VARCHAR              &  VARCHAR              &  Yes\\
         Type\_VARBINARY            &  VARBINARY            &  No\\
         Type\_LONG                 &  LONG                 &  Yes\\
         Type\_SENSITIVITY          &  \emph{None}          &  No\\
         Type\_BOUNDARY             &  BOUNDARY             &  No\\
         Type\_VOID                 &  \emph{None}          &  No\\
         Type\_USHORT               &  USHORT               &  Yes\\
         Type\_UNICHAR              &  UNICHAR/UNIVARCHAR   &  No\\
         Type\_BLOB                 &  BLOB                 &  No\\
         Type\_DATE                 &  DATE                 &  Yes\\
      \end{tabular}
   \end{center}
   \caption{Sybase Column Types}\label{t:sycoltyp}
\end{table}


\subsection{Is\_Null Function}

If a column is capable of returning a NULL\index{NULL} value, it
becomes necessary to test for this. The Is\_Null\index{Is\_Null} function calling
specification is as follows:

\begin{Code}
function Is_Null(
   Q :  in Query_Type;
   CX : in Column_Index_Type
) return Boolean;
\end{Code}

The Is\_Null function may raise the exceptions listed in Table~\ref{t:isnullx}.

\begin{table}
   \begin{center}
      \begin{tabular}{ll}
         Exception Name    &  Reason\\
         \hline 
         No\_Result        &  There was no execute performed\\
         No\_Column        &  Unknown column name\\
      \end{tabular}
   \end{center}
   \caption{Is\_Null Exceptions}\label{t:isnullx}
\end{table}

The following example shows how to test if the CUST\_NAME column is
null\index{null} or not:

\begin{Example}
declare
   C : Connection_Type;
   Q : Query_Type;
begin
   Prepare(Q,"SELECT CUST_NO,CUST_NAME,BIRTH_DATE");
   Append(Q, "FROM CUSTOMER");
   Execute(Q,C);

   loop
      begin
         Fetch(Q);
      exception
         when No_Tuple =>
            exit;
      end;
      ...

      if Is_Null(Q,2) then
         -- CUST_NAME value is null
      end if;
      ...
   end loop;
   Clear(Q);
\end{Example}

\subsection{Column\_Is\_Null Generic Function}

If you need to test for null\index{null} using strongly typed
indicators\index{indicators}, you may want to instantiate the
Column\_Is\_Null\index{Column\_Is\_Null} generic function. The generic parameters are:

\begin{Code}
generic
   type Ind_Type is new Boolean;
function Column_Is_Null(
   Q :  in Query_Type'Class;
   CX : in Column_Index_Type
) return Ind_Type;
\end{Code}

Table~\ref{t:isnullx} lists exceptions that also apply to the Column\_Is\_Null function.

%\begin{tabular}{ll}
%Exception Name    &  Reason\\
%\hline 
%No\_Result        &  There was no execute performed\\
%No\_Column        &  No column at index\\
%\end{tabular}

The following example illustrates its use:

\begin{Example}
declare
   type Cust_Name_Ind_Type is new Boolean;
   function Is_Null
       is new Column_Is_Null(Cust_Name_Ind_Type);

   C :             Connection_Type;
   Q :             Query_Type;
   Cust_Name_Ind : Cust_Name_Ind_Type;
 begin
   Prepare(Q,"SELECT CUST_NO,CUST_NAME");
   Append(Q, "FROM CUSTOMER");
   Execute(Q,C);
   loop
      begin
         Fetch(Q);
      exception
         when No_Tuple =>
            exit;
      end;
      Cust_Name_Ind := Is_Null(Q,2);
      if not Cust_Name_Ind then
         -- Get Cust_Name since it is not null
      end if;
   end loop;
   Clear(Q);
\end{Example}

\section{Value Fetching Functions\label{Value Fetching Functions}}

Once a fetch\index{fetch} operation has been performed, the application needs to
retrieve the values for each column from the row. The function primitive
Value\index{Value} assumes that the column's value is \emph{not} going
to be NULL\index{NULL}. If it should be null however, the exception
Null\_Value\index{Null\_Value} is raised. A better set of primitives should be used for
columns that may return NULL. See \Ref{Value and Indicator Fetch Procedures}.

The following subsections will cover the function primitives for extracting
the values for builtin types\index{builtin types}.


\subsection{Function Value\label{Standard Value Functions}}

The value for a Row\_ID\_Type\index{OID}, string, bit string or
Unbounded\_String\index{Unbounded\_String} can be extracted for a column
using the Value function primitive. This function should only used
for columns that cannot return a NULL \index{NULL}value.%
\footnote{If the value is NULL, the exception Null\_Value will be raised.%
} The calling requirements for these primitives are the same with 
different return types:

\begin{Code}
function Value(
   Query : in Query_Type;
   CX :    in Column_Index_Type
) return Row_ID_Type;
\end{Code}

\begin{Code}
function Value(
   Query : in Query_Type;
   CX :    in Column_Index_Type
) return String;
\end{Code}

\begin{Code}
function Value(
   Query : in Query_Type;
   CX :    in Column_Index_Type
) return Ada.Strings.Unbounded.Unbounded_String;
\end{Code}

\begin{Code}
function Value(
   Query : in Query_Type;
   CX :    in Column_Index_Type
) return APQ_Bitstring;
\end{Code}

Table \ref{t:valx} lists the possible exceptions for these functions.

\begin{table}
   \begin{center}
      \begin{tabular}{ll}
         Exception Name    &  Reason\\
         \hline 
         No\_Result        &  There was no execute performed\\
         No\_Column        &  No column at index\\
         Null\_Value       &  The column's value is NULL\\
      \end{tabular}
   \end{center}
   \caption{Value Exceptions}\label{t:valx}
\end{table}

The following example shows how all the column values are fetched:

\begin{Example}
   declare
     C : Connection_Type;
     Q : Query_Type;
   begin
     Prepare(Q,"SELECT CUST_NO,CUST_NAME");
     Append(Q, "FROM CUSTOMER");
     Execute(Q,C);
     loop
        begin
           Fetch(Q);
        exception
           when No_Tuple =>
              exit;
        end;
        for CX in 1..Column_Index_Type(Columns(Q)) loop
           declare
               Col_Value : String := Value(Q,CX);
           begin
               Put("Column");
               Put(Column_Index_Type'Image(CX));
               Put(" = '");
               Put(Value(Q,CX));
               Put_Line("'");
           end;
        end loop;
     end loop;
     Clear(Q);
\end{Example}

\subsection{Null\_Oid Function\label{Null_Oid Function}}

Since different database engines have different approaches to row
ID values%
\footnote{MySQL for example, does not return row ID values.%
}, it is necessary to know how to represent a null row ID\index{row ID} value in
the current database context. Use the Null\_Oid\index{Null\_Oid} primitive
to obtain that value:

\begin{Code}
function Null_Oid(
   Query : Query_Type
) return Row_ID_Type;
\end{Code}

\begin{floatingtable}{
   \begin{tabular}{lc}
      Database       &  Value\\
      \hline 
      PostgreSQL     &  0\\
      MySQL          &  0\\
      Sybase         &  0\\
   \end{tabular}}
   \caption{Null Oid Values by Product}\label{t:nulloid}
\end{floatingtable}

Using the Null\_Oid function in generic database code allows helps
to eliminate the need to identify the database engine being used. 
Table~\ref{t:nulloid} lists the numeric values that represent a 
null OID (row ID) value. The application programmer should rely
of the function Null\_Oid to avoid hardcoding a numeric value
that might change in the future.

\subsection{Generic Value Functions}

The Value functions documented in \Ref{Standard Value Functions}
were suitable for the specific data types that they supported. However,
Ada programmers often derive distinct new types to prevent accidental
mixing of values in expressions. To accomodate all of these custom
data types, you need to use generic functions for the purpose:

\begin{Code}
generic
   type Val_Type is new Boolean;
function Boolean_Value(
   Query : in Query_Type'Class;
   CX :    in Column_Index_Type
) return Val_Type;
\end{Code}

\begin{Code}
generic
   type Val_Type is range <>;
function Integer_Value(
   Query : in Query_Type'Class;
   CX :    in Column_Index_Type
) return Val_Type;
\end{Code}

\begin{Code}
generic
   type Val_Type is mod <>;
function Modular_Value(
   Query : in Query_Type'Class;
   CX :    in Column_Index_Type
) return Val_Type;
\end{Code}

\begin{Code}
generic
   type Val_Type is digits <>;
function Float_Value(
   Query : in Query_Type'Class;
   CX :    in Column_Index_Type
) return Val_Type;
\end{Code}

\begin{Code}
generic
   type Val_Type is delta <>;
function Fixed_Value(
   Query : in Query_Type'Class;
   CX :    in Column_Index_Type
) return Val_Type;
\end{Code}

\begin{Code}
generic
   type Val_Type is delta <> digits <>;
function Decimal_Value(
   Query : in Query_Type'Class;
   CX :    in Column_Index_Type
) return Val_Type;
\end{Code}

\begin{Code}
generic
   type Val_Type is new APQ_Date;
function Date_Value(
   Query : in Query_Type'Class;
   CX :    in Column_Index_Type
) return Val_Type;
\end{Code}

\begin{Code}
generic
   type Val_Type is new APQ_Time;
function Time_Value(
   Query : in Query_Type'Class;
   CX :    in Column_Index_Type
) return Val_Type;
\end{Code}

\begin{Code}
generic
   type Val_Type is new Ada.Calendar.Time;
function Timestamp_Value(
   Query : in Query_Type'Class;
   CX :    in Column_Index_Type
) return Val_Type;
\end{Code}

Table \ref{t:gvalx} lists the exceptions possible for these
instantiated routines.

\begin{table}
   \begin{center}
      \begin{tabular}{ll}
         Exception Name    &  Reason\\
         \hline 
         No\_Result        &  There was no execute performed\\
         No\_Column        &  No column at index\\
         Null\_Value       &  The column's value is NULL\\
      \end{tabular}
   \end{center}
   \caption{Generic Value Exceptions}\label{t:gvalx}
\end{table}

The following example illustrates the use of the Integer\_Value and
Date\_Value generic functions:

\begin{Example}
declare
   type Cust_No_Type is new APQ_Integer;
   type Cust_Birthday_Type is new APQ_Date;
   function Value is new Integer_Value(Cust_No_Type);
   function Value is new Date_Value(Cust_Birthday_Type);
   C : Connection_Type;
   Q : Query_Type;
begin
   Prepare(Q,"SELECT CUST_NO,CUST_NAME,BIRTH_DATE");
   Append(Q, "FROM CUSTOMER");
   Execute(Q,C);
   loop
      begin
         Fetch(Q);
      exception
         when No_Tuple =>
            exit;
      end;
      declare
         Cust_Name : String := Value(Q,2);
         Cust_No :   Cust_No_Type;
         Birthday :  Cust_Birthday_Type;
      begin
         Cust_No  := Value(Q,1);  -- CUST_NO is col 1
         Birthday := Value(Q,3);  -- BIRTH_DATE is col 3
         ...
      end;
   end loop;
   Clear(Q);
\end{Example}

In the example shown, \emph{Cust\_Name} is returned by the builtin
function Value for String types. The variables \emph{Cust\_No} and
\emph{Birthday} are assigned through the generic instantiations of
the functions Integer\_Value\index{Integer\_Value} and Date\_Value\index{Date\_Value} respectively.


\subsection{Fixed Length String Value Procedure}

Sometimes in an application it is desireable to work with fixed\index{fixed length} length
string\index{string} values. The following Value\index{Value} procedure does just this:

\begin{Code}
procedure Value(
   Query: in     Query_Type;
   CX :   in     Column_Index_Type;
   V :       out String
);
\end{Code}

This function raises the exceptions listed in Table~\ref{t:fxflsx}.

\begin{table}
   \begin{center}
      \begin{tabular}{ll}
         Exception Name    &  Reason\\
         \hline 
         No\_Result        &  There was no execute performed\\
         No\_Column        &  No column at index\\
         Null\_Value       &  The column's value is null\\
      \end{tabular}
   \end{center}
   \caption{Fixed Length String Value Exceptions}\label{t:fxflsx}
\end{table}

The following example extracts the CUST\_NAME column result into variable
\emph{Cust\_Name} as a 30 byte string value:

\begin{Example}
declare
   C :         Connection_Type;
   Q :         Query_Type;
   Cust_Name : String(1..30);
begin
   Prepare(Q,"SELECT CUST_NO,CUST_NAME");
   Append(Q,"FROM CUSTOMER");
   Execute(Q,C);
   loop
      begin
         Fetch(Q);
      exception
         when No_Tuple =>
            exit;
      end;
      ...
      Value(Q,2,Cust_Name);
      ...
   end loop;
   Clear(Q);
\end{Example}

\subsection{APQ\_Timezone Value Procedure}

A TIMESTAMP\index{TIMESTAMP} value that carries with it a time zone\index{time zone}
value, requires a special procedure since two values must be extracted
for the same column:

\begin{itemize}
   \item The Ada.Calendar.Time%
      \footnote{From which APQ\_Timestamp is derived.} value holding the TIMESTAMP value
   \item The APQ\_Timezone\index{APQ\_Timezone} value holding the time zone
      offset.
\end{itemize}

The generic procedure requires the following type arguments:

\begin{Code}
generic
   type Date_Type is new Ada.Calendar.Time;
   type Zone_Type is new APQ_Timezone;
procedure Timezone_Value(
   Query : in     Query_Type'Class;
   CX :    in     Column_Index_Type;
   TS :       out Date_Type;
   TZ :       out Zone_Type
);
\end{Code}

Table \ref{t:tzvx} lists the exceptions for Timezone\_Value.

\begin{table}
   \begin{center}
      \begin{tabular}{ll}
         Exception Name    &  Reason\\
         \hline 
         No\_Result        &  There was no execute performed\\
         No\_Column        &  No column at index\\
         Null\_Value       &  The column's value is null\\
      \end{tabular}
   \end{center}
   \caption{Timezone\_Value Exceptions}\label{t:tzvx}
\end{table}

The following example shows how the procedure can be used:

\begin{Example}
declare
   procedure Value is new Timezone_Value(APQ_Timestamp,APQ_Timezone);
   C : Connection_Type;
   Q : Query_Type;
   Birth_Date : APQ_Timestamp;
   Birth_Zone : APQ_Timezone;
begin
   Prepare(Q,"SELECT CUST_NO,CUST_NAME,BIRTH_DATE");
   Append(Q,"FROM CUSTOMER");
   Execute(Q,C);
   loop
      begin
         Fetch(Q);
      exception
         when No_Tuple =>
            exit;
      end;
      ...
      Value(Q,3,Birth_Date,Birth_Zone); -- Column BIRTH_DATE
      ...
      end loop;
   end loop;
   Clear(Q);
\end{Example}

In this example code fragment, the Value procedure call expects the
column's value to be column 3 (argument CX). This works as long as
column BIRTH\_DATE can never be null. If a NULL value is encountered
in this program, the exception Null\_Value\index{Null\_Value} will be raised and not
caught by the code in this example.%
\footnote{Either an exception handler must be added or a different way of extracting
the value must be used. Fetch\_Timezone is recommended if NULL is
possible.}

\subsection{Bounded\_Value Function}

Bounded\index{bounded strings} strings require a separate instantiation of
Ada\-.Strings\-.Bounded\index{Ada.Strings.Bounded} for each string length.
For this reason, a function supporting bounded strings must be provided
in generic form. The Bounded\_Value\index{Bounded\_Value} generic function accepts
the following generic arguments:

\begin{Code}
generic
   with package P is new
      Ada.Strings.Bounded.Generic_Bounded_Length(<>);
function Bounded_Value(
   Query : in Query_Type'Class;
   CX :    in Column_Index_Type
) return P.Bounded_String;
\end{Code}

\begin{table}
   \begin{center}
      \begin{tabular}{ll}
         Exception Name    &  Reason\\
         \hline 
         No\_Result        &  There was no execute performed\\
         No\_Column        &  No column at index\\
         Null\_Value       &  The column's value is null\\
      \end{tabular}
   \end{center}
   \caption{Bounded\_Value Exceptions}\label{t:bvx}
\end{table}

Table \ref{t:bvx} lists the exceptions possible for Bounded\_Value.
The example below illustrates the use of Bounded\_Value:

\begin{Example}
with Ada.Strings.Bounded;
declare
   package B30 is new 
      Ada.Strings.Bounded.Generic_Bounded_Length(30);
   function Value is new Bounded_Value(B30); 
   C :         Connection_Type;
   Q :         Query_Type;
   Cust_Name : B30;
begin
   Prepare(Q,"SELECT CUST_NO,CUST_NAME,BIRTH_DATE");
   Append(Q, "FROM CUSTOMER");
   Execute(Q,C);
   loop
      begin
         Fetch(Q);
      exception
         when No_Tuple =>
      end;
      ...
      Cust_Name := Value(Q,2);
      ...
      end loop;
   end loop;
   Clear(Q);
\end{Example}

\section{Value and Indicator Fetch Procedures\label{Value and Indicator Fetch Procedures}}

The Value\index{Value} functions\index{value functions} presented in
\Ref{Value Fetching Functions} were useful when the returned value was
always going to be present. However, their use becomes clumsy and less
efficient if exception handlers must be used to handle the
NULL\index{NULL} value case. This section documents Fetch procedures to
return both a value and a null indicator together. With this convenience
comes the added responsibility of checking the null
indicator\index{indicator} values, that are returned.


\subsection{Char and Unbounded Fetch}

The Char\_Fetch and Unbounded\_Fetch generic procedures
fetch both a string value and an indicator value. In the Char\_Fetch
case, the returned value is blank filled to the full size of the receiving
String buffer. Each of these has the following instantiation parameters:

\begin{Code}
generic
   type Ind_Type is new Boolean;
procedure Char_Fetch(
   Query :     in     Query_Type'Class;
   CX :        in     Column_Index_Type;
   V :            out String;
   Indicator :    out Ind_Type
);
\end{Code}

\begin{Code}
generic
   type Ind_Type is new Boolean;
procedure Unbounded_Fetch(
   Query :     in     Query_Type'Class;
   CX :        in     Column_Index_Type;
   V :            out Ada.Strings.Unbounded.Unbounded_String;
   Indicator :    out Ind_Type
);
\end{Code}

The Unbounded\_Fetch routine may raise any of the exceptions listed in Table~\ref{t:ufx}.

\begin{table}
   \begin{center}
      \begin{tabular}{ll}
         Exception Name    &  Reason\\
         \hline 
         No\_Result        &  There was no execute performed\\
         No\_Column        &  No column at index\\
      \end{tabular}
   \end{center}
   \caption{Unbounded\_Fetch Exceptions}\label{t:ufx}
\end{table}

The following example illustrates two different column fetch applications:

\begin{Example}
with Ada.Strings.Unbounded;
declare
   type Cust_Name_Ind_Type is new Boolean;
   type Cust_City_Ind_Type is new Boolean;
   subtype Cust_Name_Type is String(1..30);
   subtype Cust_City_Type is
       Ada.Strings.Unbounded.Unbounded_String;

   procedure Value is new 
       Char_Fetch(Cust_Name_Ind_Type);
   procedure Value is new 
       Unbounded_Fetch(Cust_City_Ind_Type);

   C :             Connection_Type;
   Q :             Query_Type;
   Cust_Name :     Cust_Name_Type;
   Cust_Name_Ind : Cust_Name_Ind_Type;
   Cust_City :     Cust_City_Type;
   Cust_City_Ind : Cust_City_Ind_Type;
begin
   Prepare(Q,"SELECT CUST_NO,CUST_NAME,CITY");
   Append(Q, "FROM CUSTOMER");
   Execute(Q,C);
   loop
      begin
         Fetch(Q);
      exception
         when No_Tuple =>
      end;
      ...
      Value(Q,2,Cust_Name,Cust_Name_Ind);
      Value(Q,3,Cust_City,Cust_City_Ind);
      ...
   end loop;
   Clear(Q);
\end{Example}

\subsection{Varchar\_Fetch and Bitstring\_Fetch Procedures}

To return a varying length string\index{varying length string} requires
the use of a function. However, to return two values, we must resort
to a procedure call for the purpose. In order to return both a varying
length string and a null indicator, an additional return value is
returned that indicates the length of the string. To accomodate strongly
typed indicators\index{indicators}, these two procedures are provided in generic form.
The instantiation parameters are:

\begin{Code}
generic
   type Ind_Type is new Boolean;
procedure Varchar_Fetch(
   Query :     in     Query_Type'Class;
   CX :        in     Column_Index_Type;
   V :            out String;
   Last :         out Natural;
   Indicator :    out Ind_Type
);
\end{Code}

\begin{Code}
generic
   type Ind_Type is new Boolean;
procedure Bitstring_Fetch(
   Query :     in     Query_Type'Class;
   CX :        in     Column_Index_Type;
   V :            out APQ_Bitstring;
   Last :         out Natural;
   Indicator :    out Ind_Type
);
\end{Code}

\begin{table}
   \begin{center}
      \begin{tabular}{ll}
         Exception Name    &  Reason\\
         \hline 
         No\_Result        &  There was no execute performed\\
         No\_Column        &  No column at index\\
      \end{tabular}
   \end{center}
   \caption{Bitstring\_Fetch Exceptions}\label{t:bfx}
\end{table}

Table \ref{t:bfx} lists the exceptions for Bitstring\_Fetch.
The next example illustrates two different column fetch applications:

\begin{Example}
declare
   type Cust_Name_Ind_Type is new Boolean;
   type Cust_City_Ind_Type is new Boolean;

   procedure Value is new 
      Varchar_Fetch(Cust_Name_Ind_Type);
   procedure Value is new 
      Varchar_Fetch(Cust_City_Ind_Type);

   C :              Connection_Type;
   Q :              Query_Type;
   Cust_Name :      String(1..30);
   Cust_Name_Last : Natural;
   Cust_Name_Ind :  Cust_Name_Ind_Type;
   Cust_City :      String(1..40);
   Cust_City_Last : Natural;
   Cust_City_Ind :  Cust_City_Ind_Type;
begin
   Prepare(Q,"SELECT CUST_NO,CUST_NAME,CITY");
   Append(Q, "FROM CUSTOMER");
   Execute(Q,C);
   loop
      begin
         Fetch(Q);
      exception
         when No_Tuple =>
            exit;
      end;
      ...
      Value(Q,2,Cust_Name,Cust_Name_Last,Cust_Name_Ind);
      Value(Q,3,Cust_City,Cust_City_Last,Cust_City_Ind);
      ...
   end loop;
   Clear(Q);
\end{Example}

After the first Value call in the example, the customer name would
be represented by the expression:

\begin{Code}

   Cust_Name(1..Cust_Name_Last)

\end{Code}

provided that the value Cust\_Name\_Ind was false.


\subsection{Bounded\_Fetch Procedure}

To fetch both a Bounded\_String value and its associated null\index{null indicator} indicator,
you instantiate and call the Bounded\_Fetch\index{Bounded\_Fetch}. The instantiation
parameters are as follows:

\begin{Code}
generic
   type Ind is new Boolean;
   with package P is new 
      Ada.Strings.Bounded.Generic_Bounded_Length(<>);
procedure Bounded_Fetch(
   Query :     in     Query_Type'Class;
   CX :        in     Column_Index_Type;
   V :            out P.Bounded_String;
   Indicator :    out Ind);
\end{Code}

\begin{table}
   \begin{center}
      \begin{tabular}{ll}
         Exception Name    &  Reason\\
         \hline 
         No\_Result        &  There was no execute performed\\
         No\_Column        &  No column at index\\
      \end{tabular}
   \end{center}
   \caption{Bounded\_Fetch Exceptions}\label{t:bfx2}
\end{table}

The Bounded\_Fetch routine may raise any of the exceptions listed in Table~\ref{t:bfx2}.
The example below illustrates a fetch instantiation and call:

\begin{Example}
with Ada.Strings.Bounded;
declare
   package B32 is new 
      Ada.Strings.Bounded.Generic_Bounded_Length(32);
   type Cust_Name_Ind_Type is new Boolean;

   procedure Value is new 
      Bounded_Fetch(Cust_Name_Ind_Type,B32);

   C :              Connection_Type;
   Q :              Query_Type;
   Cust_Name :      B32;
   Cust_Name_Ind :  Cust_Name_Ind_Type;
begin
   Prepare(Q,"SELECT CUST_NO,CUST_NAME");
   Append(Q, "FROM CUSTOMER");
   Execute(Q,C);
   loop
      begin
         Fetch(Q);
      exception
         when No_Tuple =>
            exit;
      end;
      ...
      Value(Q,2,Cust_Name,Cust_Name_Ind); 
      ...
   end loop;
   Clear(Q);
\end{Example}

\subsection{Discrete Type Fetch Procedures}

Several of the discrete\index{discrete types} types can be grouped and documented in this
section. The following table indicates the generic procedure names
and their associated class of data type for which they are designed:

\begin{Code}
generic
   type Val_Type is new Boolean;
   type Ind_Type is new Boolean;
procedure Boolean_Fetch(
   Query :     in     Root_Query_Type'Class;
   CX :        in     Column_Index_Type;
   V :            out Val_Type;
   Indicator :    out Ind_Type
);
\end{Code}

\begin{Code}
generic
   type Val_Type is range <>;
   type Ind_Type is new Boolean;
procedure Integer_Fetch(
   Query :     in     Root_Query_Type'Class;
   CX :        in     Column_Index_Type;
   V :            out Val_Type;
   Indicator :    out Ind_Type
);
\end{Code}

\begin{Code}
generic
   type Val_Type is mod <>;
   type Ind_Type is new Boolean;
procedure Modular_Fetch(
   Query :     in     Root_Query_Type'Class;
   CX :        in     Column_Index_Type;
   V :            out Val_Type;
   Indicator :    out Ind_Type
);
\end{Code}

\begin{Code}
generic
   type Val_Type is digits <>;
   type Ind_Type is new Boolean;
procedure Float_Fetch(
   Query :     in     Root_Query_Type'Class;
   CX :        in     Column_Index_Type;
   V :            out Val_Type;
   Indicator :    out Ind_Type
);
\end{Code}

\begin{Code}
generic
   type Val_Type is delta <>;
   type Ind_Type is new Boolean;
procedure Fixed_Fetch(
   Query :     in     Root_Query_Type'Class;
   CX :        in     Column_Index_Type;
   V :            out Val_Type;
   Indicator :    out Ind_Type
);
\end{Code}

\begin{Code}
generic
   type Val_Type is delta <> digits <>;
   type Ind_Type is new Boolean;
procedure Decimal_Fetch(
   Query :     in     Root_Query_Type'Class;
   CX :        in     Column_Index_Type;
   V :            out Val_Type;
   Indicator :    out Ind_Type
);
\end{Code}

\begin{Code}
generic
   type Val_Type is new Ada.Calendar.Time;
   type Ind_Type is new Boolean;
procedure Date_Fetch(
   Query :     in     Root_Query_Type'Class;
   CX :        in     Column_Index_Type;
   V :            out Val_Type;
   Indicator :    out Ind_Type
);
\end{Code}

\begin{Code}
generic
   type Val_Type is new Ada.Calendar.Day_Duration;
   type Ind_Type is new Boolean;
procedure Time_Fetch(
   Query :     in     Root_Query_Type'Class;
   CX :        in     Column_Index_Type;
   V :            out Val_Type;
   Indicator :    out Ind_Type
);
\end{Code}

\begin{Code}
generic
   type Val_Type is new Ada.Calendar.Time;
   type Ind_Type is new Boolean;
procedure Timestamp_Fetch(
   Query :     in     Root_Query_Type'Class;
   CX :        in     Column_Index_Type;
   V :            out Val_Type;
   Indicator :    out Ind_Type
);
\end{Code}

\begin{table}
   \begin{center}
      \begin{tabular}{ll}
         Exception Name    &  Reason\\
         \hline      
         No\_Result        &  There was no execute performed\\
         No\_Column        &  No column at index\\
      \end{tabular}
   \end{center}
   \caption{Timestamp\_Fetch Exceptions}\label{t:tfx2}
\end{table}

The Timestamp\_Fetch exceptions are listed in Table~\ref{t:tfx2}.
The following example illustrates how to instantiate and call a
Integer\_Fetch procedure.

\begin{Example}
declare
   type Cust_No_Type is new Integer;
   type Cust_No_Ind_Type is new Boolean;
   procedure Value is new Integer_Fetch
      (Cust_No_Type,Cust_No_Ind_Type);
   C :              Connection_Type;
   Q :              Query_Type;
   Cust_No :        Cust_No_Type;
   Cust_No_Ind :    Cust_No_Ind_Type;
begin
   Prepare(Q,"SELECT CUST_NO,CUST_NAME");
   Append(Q, "FROM CUSTOMER");
   Execute(Q,C);
   loop
      begin
         Fetch(Q);
      exception
         when No_Tuple =>
            exit;
      end;
      ...
      Value(Q,1,Cust_No,Cust_No_Ind);
      ...
   end loop;
   Clear(Q);
\end{Example}

\subsection{Timezone\_Fetch Procedure}

The Timezone\_Fetch\index{Timezone\_Fetch} generic procedure is unique because it returns
an additional parameter: the timezone\index{timezone}. The procedure's
instantiation parameters are listed below:

\begin{Code}
generic
   type Date_Type is new Ada.Calendar.Time;
   type Zone_Type is new APQ_Timezone;
   type Ind_Type is new Boolean;
procedure Timezone_Fetch(
   Query :     in     Root_Query_Type'Class;
   CX :        in     Column_Index_Type;
   V :            out Date_Type;
   Z :            out Zone_Type;
   Indicator :    out Ind_Type
);
\end{Code}

\begin{table}
   \begin{center}
      \begin{tabular}{ll}
         Exception Name       &  Reason\\
         \hline 
         No\_Result           &  There was no execute performed\\
         No\_Column           &  No column at index\\
      \end{tabular}
   \end{center}
   \caption{Timezone\_Fetch Exceptions}\label{t:tzfx}
\end{table}

Table \ref{t:tzfx} lists the possible exceptions for Timezone\_Fetch.
The following example illustrates how to instantiate and call a Timezone\_Fetch
procedure.

\begin{Example}
declare
   type Bday_Type is new Integer;
   type Bday_Zone_Type is new APQ_Timezone;
   type Bday_Ind_Type is new Boolean;
   procedure Value is new Timezone_Fetch
      (Bday_Type,Bday_Zone_Type,Bday_Ind_Type);
   C :              Connection_Type;
   Q :              Query_Type;
   Bday :           Bday_Type;
   Bday_Zone :      Bday_Zone_Type;
   Bday_Ind :       Bday_Ind_Type;
begin
   Prepare(Q,"SELECT CUST_NO,CUST_NAME,BIRTH_DATE");
   Append(Q, "FROM CUSTOMER");
   Execute(Q,C);
   loop
      begin
         Fetch(Q);
      exception
         when No_Tuple =>
            exit;
      end;
      ...
      Value(Q,3,Bday,Bday_Zone,Bday_Ind);
      ...
   end loop;
   Clear(Q);
\end{Example}

\section{Information Functions}

You have already seen two information functions Result\index{Result} and Error\_Message\index{Error\_Message}
in \Ref{Error Message Reporting} and \Ref{Error Status Reporting}.
Another useful Query\_Type primitive is the To\_String\index{To\_String} function. It
is described in the next subsection.


\subsection{The To\_String Function}

The To\_String\index{To\_String} primitive allows the caller to retrieve the collected
SQL text from the Query\_Type object. The function's calling signature
is as follows:

\begin{Code}
function To_String(
   Query : Query_Type
) return String;
\end{Code}

The To\_String function returns the full text of the SQL query, including
newline characters.%
\footnote{One is included at the end of the string if it is missing.%
} If there has not been any SQL text\index{SQL text} collected, the
function returns an empty string.%
\footnote{In this case, no newline is provided.%
}

The following example shows how a programmer can dump out the SQL\index{SQL}
query, but only when it fails:

\begin{Example}
declare
   C :   Connection_Type;
   Q :   Query_Type;
begin
   Prepare(Q,"SELECT CUST_NO,CUST_NAME,BIRTH_DATE");
   Append(Q, "FROM CUSTOMER");
   begin
      Execute(Q,C);
   exception
      when SQL_Error =>
         Put_Line("The failed SQL Query was:"); 
         Put_Line(To_String(Q));
         raise;
      when others =>
         raise;
   end;
\end{Example}

\subsection{Cursor Names\label{Cursor_Name}}

While PostgreSQL\index{PostgreSQL} and MySQL\index{MySQL} do not yet support cursors\index{cursors} in their client
libraries \footnote{PostgreSQL supports server cursors, but cursor support is absent in libpq.},
other database products like Sybase\index{Sybase} do. Cursors allow the programmer
to work with the concept of a ``current row''\index{current row}. For those databases
that do support the use of cursors, you can retrieve the cursor name\index{cursor name}
with the use of the Cursor\_Name\index{Cursor\_Name} function primitive:

\begin{Code}
function Cursor_Name(
   Query : Query_Type
) return String;
\end{Code}

\begin{table}
   \begin{center}
      \begin{tabular}{ll}
         Exception Name    &  Reason\\
         \hline 
         No\_Result        &  No cursor results pending\\
      \end{tabular}
   \end{center}
   \caption{Cursor\_Name Exceptions}\label{t:cnamx}
\end{table}

The Cursor\_Name exceptions are documented in Table~\ref{t:cnamx}.
To use cursors within APQ, you must set the fetch mode of the Query\_Type
to one of the following:

\begin{itemize}
   \item Cursor\_For\_Update\index{Cursor\_For\_Update}
   \item Cursor\_For\_Read\_Only\index{Cursor\_For\_Read\_Only}
\end{itemize}

Once you have set the fetch mode to one of the cursor fetch modes
listed above, and you have executed the query to produce cursor results,
\emph{then} the call to Cursor\_Name is valid. For a description of
fetch query modes in \Ref{Fetch Query Modes}.


\chapter{Blob Support}

The MySQL\index{MySQL} database provides a very different type of blob support\index{blob support}.
For the moment, blobs are unsupported in MySQL. Blob support is also
absent for Sybase\index{Sybase}. These are two areas ripe for further work in APQ.

Blobs are supported for PostgreSQL\index{PostgreSQL} in APQ however. This provides the
application programmer with the ability to store large amounts of
information in a ``blob''\index{blob}. In many ways this resembles
a file\index{file}, with the exception that the contents are stored
in the database and is accessed by number (OID\index{OID}). The APQ
binding provides full PostgreSQL blob support for the Ada programmer.
In addition, the Ada\index{Ada} stream\index{stream} concept is employed to
provide reliable and Ada95\index{Ada95} convenient access to the blob.


\subsubsection{Endian Note:}

The application programmer must keep in mind that any binary data
written to a blob, by means of the Ada stream, is not endian\index{endian} neutral.
This becomes a concern when a client application accesses or writes
to blobs stored on a database over the network, on another host. If
the endianess of the server and client differ, endian problems will
emerge.


\section{Introduction}

Blob functions are managed primarily through the Blob\_Type\index{Blob\_Type} access
type.%
\footnote{The object itself is of type Blob\_Object. Blob\_Type is an access
to Blob\_Object type.%
} Stream I/O to and from the blob is performed using an Ada streams
access value.%
\footnote{Internally named Root\_Stream\_Access type, which is the type ''access
all Ada.Streams.Root\_Stream\_Type'Class''.%
} 

The blob support can generally be grouped into the following categories:

\begin{itemize}
   \item Create, Open and Close operations
   \item Index setting and querying operations
   \item Information operations for Size and OID
   \item Stream accessor\index{stream accessor} function
   \item Blob destruction\index{blob destruction}
   \item File and Blob operations
\end{itemize}

The following sections will document the blob support using these
groupings.


\section{Blob Memory Leak Prevention}

\index{memory leak prevention}It is extremely important that the
programmer realize that the Blob\_Type data type is an access type.%
\footnote{This design choice was necessary to accomodate Ada stream oriented
I/O.%
} Additionally this access\index{access value} value is a pointer\index{pointer} to a dynamically\index{dynamic} allocated
tagged\index{tagged} record (type Blob\_Object). For this reason, the programmer
must take great care to ``close'' the Blob\_Type before discarding
the Blob\_Type value, when it goes out of scope. Failure to close
a Blob\_Type value, will result in a memory leak\index{memory leak} and cause subsequent
database performance issues. Use the Blob\_Type value as if it were
an open file that needs closing.

The following example represents a ``Blob\_Type leak'':

\begin{Example}
declare
   C :   aliased Connection_Type;
   B :   Blob_Type;
begin
   ...
   B := Blob_Create(C'Access);
   ...
end;
\end{Example}

The example above is bad because a ``blob leak''\index{blob leak}
occurs when the ``end'' statement is reached.%
\footnote{It should also be noted that after creating a blob in the database,
the application must save the OID value for the blob somewhere. Otherwise,
you will have a blob in the database that will never be accessed!%
} The variable \emph{C} finalizes itself OK because it is a controled
object.%
\footnote{Both Connection\_Type and Query\_Type objects are controlled records
with finalization.%
} However, B is an access type, pointing to a Blob\_Object record.
When variable B falls out of scope, only the pointer value
in B is lost. The object it pointed to has not been released!

The following example code is better:

\begin{Example}
declare
   C :   aliased Connection_Type;
   B :   Blob_Type;
begin
   ...
   B := Blob_Create(C'Access);
   ...
   Blob_Close(B);
end;
\end{Example}

The call to Blob\_Close\index{Blob\_Close} insures that the memory associated with the
opened blob is released, before the value \emph{B} falls out of scope.%
\footnote{Note that Close does not destroy the blob in the database.%
} However, if there is a chance that an exception may be raised, you
may still be vulnerable to leaks. The following example covers all
of the bases:

\begin{Example}
declare
   C :   aliased Connection_Type;
   B :   Blob_Type;
begin
   ...
   B := Blob_Create(C'Access);
   ...
   Blob_Close(B);
exception
   when others =>
      if B /= null then
         Blob_Close(B);
      end if;
      ...recovery steps...
end;
\end{Example}

While the recovery steps have been left to the reader's imagination
in the example above, the exception is caught and the value for variable
\emph{B} is tested. Only if \emph{B} is not null, should the Blob\_Close
procedure call made. Only you can prevent blob memory leaks!


\section{Create, Open and Close of Blobs}

The most basic operations possible in an Ada program using blobs are:

\begin{itemize}
   \item Creating a new blob in the database (Blob\_Create)
   \item Opening an existing blob in the database (Blob\_Open)
   \item Flushing buffered writes to the database (Blob\_Flush)
   \item Closing a blob (Blob\_Close)
\end{itemize}

The following subsections will explain how to perform these operations
in detail.


\subsection{Blob\_Create Procedure}

Before the application can open an existing blob\index{opening a blob},
there must be some way to create a blob\index{creating a blob}. The
Blob\_Create\index{Blob\_Create} function does just this with the following calling signature:%
\marginpar{Note: Blob operations must be performed within the context of a transaction.%
}

\begin{Code}
function Blob_Create(
   DB :       access Connection_Type;
   Buf_Size : in     Natural := Buf_Size_Default
) return Blob_Type;
\end{Code}

The returned value is a Blob\_Type that is capable of being used to
read and/or write a blob. The blob is positioned at index position
1 (the beginning). See \Ref{Blob OID Function} for information
about how to determine the created blob's OID\index{OID}. 

Starting with APQ version 1.2, all blob I/O is buffered if the Buf\_Size
argument is supplied with a value greater than zero, or is not supplied
such that the default value applies. Table~\ref{t:bcbsgx} summarizes
the Buf\_Size\index{Buf\_Size} argument behaviour. The exceptions
are listed in Table~\ref{t:blbcx}.

\begin{table}
   \begin{center}
      \begin{tabular}{lll}
         Buf\_Size Value                        &  Description          &  Performance\\
         \hline 
         0                                      &  Unbuffered blob I/O  &  Very poor\\
         0 < Buf\_Size < 1024                   &  Buffered             &  Poor\\
         1024 <= Buf\_Size < Buf\_Size\_Default & Buffered blob I/O     &  Better\\
         Buf\_Size\_Default                     &  Buffered: 5120 bytes &  Very good\\
      \end{tabular}
   \end{center}
   \caption{Blob\_Create Buf\_Size Guidelines}\label{t:bcbsgx}
\end{table}

\begin{table}
   \begin{center}
      \begin{tabular}{ll}
         Exception Name    &  Reason\\
         \hline 
         Blob\_Error       &  There was no blob created\\
      \end{tabular}
   \end{center}
   \caption{Blob\_Create Exceptions}\label{t:blbcx}
\end{table}

Possible reasons for a Blob\_Error\index{Blob\_Error} exception to be raised would include:

\begin{itemize}
   \item bad database connection object%
      \footnote{Or the database connection object went out of scope.}
   \item a database error occurred (no more blob space?)
\end{itemize}

The following example shows how a new blob can be created:

\begin{Example}
declare
   C :   aliased Connection_Type;
   B :   Blob_Type;
begin
   ...
   B := Blob_Create(C'Access);
   ...
   Blob_Close(B);
end;
\end{Example}

\subsubsection{Note:}

The argument DB in the call to the Blob\_Create\index{Blob\_Create}, is an access
to Connection\_Type argument. You must guarantee that the Connection\_Type
object does not finalize before the created blob has been closed.


\subsection{Blob\_Open Function\label{Blob_Open Function}}

To open an existing blob\index{existing blob}, you must know the OID\index{OID} of the blob in the
database. This is normally a value that is stored in a database column
somewhere. See \Ref{Blob OID Function} for information on
how to determine the OID of a created blob.%
\marginpar{Note: Blob operations must be performed within the context of a transaction.%
}

Blobs can be opened for various types of access:

\begin{description}
   \item [Read\index{Read}] for readonly access to the blob contents
   \item [Write\index{Write}] for writing to the blob
   \item [Read\_Write\index{Write}] for both reading and writing of the blob
\end{description}

The Blob\_Open\index{Blob\_Open} function has the following specification:

\begin{Code}
type Mode_Type is (
   Write,
   Read,
   Read_Write
);

function Blob_Open(
   DB :       access Connection_Type;
   Oid :      in     Row_ID_Type;
   Mode :     in     Mode_Type;
   Buf_Size : in     Natural := Buf_Size_Default
) return Blob_Type;
\end{Code}

The returned value is a Blob\_Type that is accessable according to
the Mode selected. The blob is positioned\index{blob position} at index value 1
(the beginning).

Starting with APQ version 1.2, all blob I/O is buffered\index{buffered blob I/O} if the Buf\_Size
argument is supplied with a value greater than zero, or is not supplied
such that the default value applies. Refer to Table~\ref{t:bcbsgx} for guidance on
the value to use for Buf\_Size. The exceptions are documented
in Table~\ref{t:bopnx}.

\begin{table}
   \begin{center}
      \begin{tabular}{ll}
         Exception Name    &  Reason\\
         \hline 
         Blob\_Error       &  There was no blob opened\\
      \end{tabular}
   \end{center}
   \caption{Blob\_Open Exceptions}\label{t:bopnx}
\end{table}

Possible reasons for a Blob\_Error exception to be raised would include:

\begin{itemize}
   \item bad database connection object
   \item the Oid value supplied is not known by the database
   \item a database error occurred
\end{itemize}

The following example shows how blob 73763 can be opened for reading:

\label{Example-code-with-OID-73763}
\begin{Example}
declare
   C :   aliased Connection_Type;
   B :   Blob_Type;
   OID : Row_ID_Type := 73763;
begin
   ...
   B := Blob_Open(C'Access,OID,Read);
   ...
   Blob_Close(B);
exception
   when others =>
      if B /= null then
         Blob_Close(B);
      end if;
      raise;
end;
\end{Example}

\subsubsection{Note:}

The argument DB in the call to the Blob\_Open, is an access
to Connection\_Type argument. You must guarantee that the Connection\_Type
object does not finalize\index{finalize} before the opened blob has been closed.


\subsubsection{Generic\_Blob\_Open Function}

To allow the application programmer to use strong types in place of
the supplied Row\_ID\_Type\index{Row\_ID\_Type} type, a generic procedure for opening blobs
is also provided. The instantiated function behaves exactly as described
for Blob\_Open\index{Open\_Blob} on page \pageref{Blob_Open Function}. The instantiation
arguments for Generic\_Blob\_Open\index{Generic\_Blob\_Open} are:

\begin{Code}
type Mode_Type is (
   Write,
   Read,
   Read_Write
);

generic
   type Oid_Type is new Row_ID_Type;
function Generic_Blob_Open(
   DB :       access Connection_Type;
   Oid :      in     Oid_Type;
   Mode :     in     Mode_Type;
   Buf_Size : in     Natural := Buf_Size_Default
) return Blob_Type;
\end{Code}

The following example shows how to instantiate the function:

\begin{Example}
declare
type My_Oid_Type is new Row_ID_Type;
function Blob_Open is new
   Generic_Blob_Open(My_Oid_Type);
\end{Example}

\subsection{Blob\_Flush Procedure}

\index{Blob\_Flush}When you are using buffered\index{buffered blob I/O} blob I/O%
\footnote{Buffered blob I/O is the default for performance reasons.%
} and your application has performed one or more writes to the blob,
you may need to be certain that all of the buffered data is physically
written out to the database. For example, you may have a timing\index{timing} opportunity
to perform this expensive operation while the user is waiting for
something else to occur. Buffer flushes\index{flush} are automatically performed
when the blob is closed or due to changes made by the Blob\_Set\_Index\index{Blob\_Set\_Index}
operation. To give the application programmer control over the timing
of the physical write to the database, the Blob\_Flush\index{Blob\_Flush} procedure can
be used.%
\marginpar{Blob\_Flush calls are ignored when unbuffered blob I/O is being used.
This makes it easy for the application to choose buffered or unbuffered
operation without source code changes.}

The Blob\_Flush procedure has the specification below. Table~\ref{tblbflx} lists the exceptions.

\begin{Code}
procedure Blob_Flush(Blob : Blob_Type);
\end{Code}

\begin{table}
   \begin{center}
      \begin{tabular}{ll}
         Exception Name       &  Reason\\
         \hline 
         Blob\_Error          &  The blob is not open\\
      \end{tabular}
   \end{center}
   \caption{Blob\_Flush Exceptions}\label{t:blbflx}
\end{table}


\subsection{Blob\_Close Procedure}

When the programmer no longer requires access to a open/created blob,
the procedure Blob\_Close\index{Blob\_Close} should be called. Since an open blob depends
upon a hidden access value that points back to the Connection\_Type
object, the programmer should call Blob\_Close as soon as is practical.
This reduces the possibility of error that will occur if the Connection\_Type
object is finalized too soon.

The Blob\_Close procedure has the following specification. Exceptions are
listed in Table~\ref{t:bclsx}.

\begin{Code}
procedure Blob_Close(Blob : in out Blob_Type);
\end{Code}

\begin{table}
   \begin{center}
      \begin{tabular}{ll}
         Exception Name    &  Reason\\
         \hline 
         Blob\_Error       &  The blob is not open\\
      \end{tabular}
   \end{center}
   \caption{Blob\_Close Exceptions}\label{t:bclsx}
\end{table}

Normally the Blob\_Error\index{Blob\_Error} exception will indicate an attempt to close
a blob that is not open. However, it is possible that the database
engine may experience a problem that will raise the same exception.

The procedure Blob\_Close will also null out the the Blob\_Type value
that was passed in. This is done to eliminate any accidental access
to a Blob\_Object that no longer exists.


\section{Index Setting Operations}

Like a file, a blob's ``position''\index{position of a blob} can be changed and queried.
The index\index{index} operations require the use of two types defined for the
purpose. They are:

\begin{Code}

   type Blob_Count is new 
      Ada.Streams.Stream_Element_Offset
      range 0..Ada.Streams.Stream_Element_Offset'Last;

\end{Code}

\begin{Code}

   subtype Blob_Offset is 
      Blob_Count range 1..Blob_Count'Last;

\end{Code}

The type Blob\_Count\index{Blob\_Count} is used where there is a count involved
(which may require the value zero). The type Blob\_Offset\index{Blob\_Offset} is
used whenever a blob offset is used, since it starts at the value
1.

The next subsections describe facilities for performing blob indexing
operations.


\subsection{Blob\_Set\_Index Procedure}

The Blob\_Set\_Index\index{Blob\_Set\_Index} procedure is used when the caller needs to seek
to a new position within the opened blob. See \Ref{Blob Size Function}
if you need to know the size of the blob.

The calling requirements for Blob\_Set\_Index are summarized below
(exceptions in Table~\ref{t:blbsxx}).

\begin{Code}
procedure Blob_Set_Index (
   Blob : in Blob_Type;
   To :   in Blob_Offset
);
\end{Code}

\begin{table}
   \begin{center}
      \begin{tabular}{ll}
         Exception Name    &  Reason\\
         \hline 
         Blob\_Error       &  Not open or seek failed\\
      \end{tabular}
   \end{center}
   \caption{Blob\_Set\_Index Exceptions}\label{t:blbsxx}
\end{table}

The following example shows how to seek to the end of the blob:

\begin{Example}
declare
   C :        aliased Connection_Type;
   B :        Blob_Type;
   B_Size :   Blob_Count;
   End_Blob : Blob_Index;
begin
   ...
   B_Size := Blob_Size(B);
   if B_Size > 0 then
      End_Blob := B_Size;
      Blob_Set_Index(B,End_Blob);
   ...
\end{Example}

\section{Blob\_Index Function}

Applications sometimes need to query where they are positioned in
the blob. The Blob\_Index\index{Blob\_Index} function returns the current Blob\_Offset\index{Blob\_Offset}
position\index{position} information. The specification is shown below,
while the exceptions are listed in Table~\ref{t:blbxx}.

\begin{Code}
function Blob_Index(
   Blob : in Blob_Type
) return Blob_Offset;
\end{Code}

\begin{table}
   \begin{center}
      \begin{tabular}{ll}
         Exception Name    &  Reason\\
         \hline 
         Blob\_Error       &  Not open\\
      \end{tabular}
   \end{center}
   \caption{Blob\_Index Exceptions}\label{t:blbxx}
\end{table}

The following example code determines where in the presently opened
blob the blob position index is:

\begin{Example}
declare
   C :        aliased Connection_Type;
   B :        Blob_Type;
   Pos_Blob : Blob_Index;
begin
   ...
   Pos_Blob := Blob_Index(B);
\end{Example}

\section{Information Functions}

The following subsections describe information gathering functions.
They provide the programmer a way to obtain size\index{size} and identification\index{identification}
information.

\subsection{Blob Size Function\label{Blob Size Function}}

To determine the present size\index{blob size} of a blob, the Blob\_Size\index{Blob\_Size}
function can be used. Table~\ref{t:blbszx} documents the exceptions while the specification
is listed as follows:

\begin{Code}
function Blob_Size(
   Blob : in Blob_Type
) return Blob_Count;
\end{Code}

\begin{table}
   \begin{center}
      \begin{tabular}{ll}
         Exception Name    &  Reason\\
         \hline 
         Blob\_Error       &  Not open\\
      \end{tabular}
   \end{center}
   \caption{Blob\_Size Exceptions}\label{t:blbszx}
\end{table}

Notice that the return type Blob\_Count\index{Blob\_Count} does permit the value zero
to be returned (blob is empty).

The following example code determines the size of the presently opened
blob:

\begin{Example}
declare
   C :	       aliased Connection_Type;
   B :	       Blob_Type;
   Blob_Size : Blob_Count;
begin
   ...
   Blob_Size := Blob_Size(B);
\end{Example}

\subsection{Blob\_OID Function\label{Blob OID Function}}

After a blob is created, it is very necessary to determine the OID\index{OID}
for the blob. The Blob\_Oid function may be called after Blob\_Create\index{Blob\_Create}
or Blob\_Open\index{Blob\_Open} to obtain OID information.
Table~\ref{t:blboidx} lists the exceptions and the specification is
shown below:

\begin{Code}
function Blob_Oid(
   Blob : in Blob_Type
) return Row_ID_Type;
\end{Code}

\begin{table}
   \begin{center}
      \begin{tabular}{ll}
         Exception Name    &  Reason\\
         \hline 
         Blob\_Error       &  Not open\\
      \end{tabular}
   \end{center}
   \caption{Blob\_OID Exceptions}\label{t:blboidx}
\end{table}

The following example code determines the OID\index{OID} value for the newly
created blob\index{blob}:

\begin{Example}
declare
   C :         aliased Connection_Type;
   B :         Blob_Type;
   Blob_OID :  Row_ID_Type;
begin
   ...
   B        := Blob_Create(C'Access);
   Blob_OID := Blob_Oid(B);
\end{Example}

\subsubsection{Generic\_Blob\_Oid Function}

To use a strongly typed version of the Blob\_Oid\index{Blob\_Oid} function, the application
programmer can instantiate from Generic\_Blob\_Oid. The instantiated
function otherwise behaves exactly as the Blob\_Oid function on page
\pageref{Blob OID Function}. The instantiation parameters for Generic\_Blob\_Oid\index{Generic\_Blob\_Oid}
are:

\begin{Code}
generic
   type Oid_Type is new Row_ID_Type;
function Generic_Blob_Oid(
   Blob : in Blob_Type
) return Oid_Type;
\end{Code}

The following example shows how to instantiate the function:

\begin{Example}
declare
   type My_Oid_Type is new Row_ID_Type;
   function Blob_Oid is new Generic_Blob_Oid(My_Oid_Type);
\end{Example}

\subsection{End\_Of\_Blob Function}

The End\_Of\_Blob\index{End\_Of\_Blob} function can be used by programs that sequentially
read through a blob. The calling signature for this function is given
below:%
\marginpar{Note that this function results in poor performance if the buffer
size is set to zero (unbuffered) in the opening Blob\_Open/Blob\_Create
calls.%
}

\begin{Code}
function End_of_Blob(
   Blob : in Blob_Type
) return Boolean;
\end{Code}

The return value is True if the current position in the blob is at
the end of the blob. Otherwise the value False is returned. Table~\ref{teoblbx}
lists the possible exceptions.

\begin{table}
   \begin{center}
      \begin{tabular}{ll}
         Exception Name    &  Reason\\
         \hline 
         Blob\_Error       &  Not open\\
      \end{tabular}
   \end{center}
   \caption{End\_of\_Blob Exceptions}\label{t:eoblbx}
\end{table}

The following example code reads a series of strings from the blob,
using the End\_Of\_Blob function:

\begin{Example}
declare
   C : aliased Connection_Type;
   B : Blob_Type;
begin
   ...
   B := Blob_Open(...);
   declare
      S : Root_Stream_Access := Blob_Stream(B);
   begin
      while not End_Of_Blob(B) loop
         declare
            Line : String := String'Input(S);
         begin
            Put_Line(Line);
         end;
      end loop;
   end;
   Blob_Close(B);
\end{Example}

\section{Stream Access}

In order for an Ada program to perform stream I/O\index{stream I/O}
on a blob, you must obtain a useable stream pointer\index{stream pointer}.
The APQ binding defines a type named Root\_Stream\_Access\index{Root\_Stream\_Access} for
this purpose. It is defined as follows:

\begin{Code}

   type Root_Stream_Access is access all
      Ada.Streams.Root_Stream_Type'Class;

\end{Code}

The function Blob\_Stream\index{Blob\_Stream} returns this stream pointer to the caller.
Use of this returned pointer makes it possible to use the native Ada\index{Ada}
stream\index{stream I/O} I/O facilities. The function Blob\_Stream is 
specified below while Table~\ref{t:blbstrx} documents the exceptions
possible.

\begin{Code}
function Blob_Stream(
   Blob : in Blob_Type
) return Root_Stream_Access;
\end{Code}

\begin{table}
   \begin{center}
      \begin{tabular}{ll}
         Exception Name    &  Reason\\
         \hline 
         Blob\_Error       &  Not open\\
      \end{tabular}
   \end{center}
   \caption{Blob\_Stream Exceptions}\label{t:blbstrx}
\end{table}

The following example shows how a blob is created, a stream access
value is obtained, and a APQ\_Timestamp\index{APQ\_Timestamp} value is written to the new
blob:

\begin{Example}
declare
   C :         aliased Connection_Type;
   B :         Blob_Type;
   Str :       Root_Stream_Access;
   Some_Date : APQ_Timestamp;
begin
   ...
   B   := Blob_Create(C'Access);
   declare
      Str : Root_Stream_Access := Blob_Stream(B);
   begin
      APQ_Timestamp'Write(Str,Some_Date);
      ...
   end;
   Blob_Close(B);
end;
\end{Example}

Notice in this example, that the declaration and the existance of
the stream pointer Str was restricted as much as possible.
While not strictly necessary, there are good reasons for following
this practice. See the special following note for the details.


\subsubsection{Note:}

The programmer does not have to worry about ``closing'' or freeing\index{freeing}
the returned stream pointer (Str in the example). It can be
nulled or left to fall out of scope. Only the type Blob\_Type
must be ``closed'' by calling Blob\_Close\index{Blob\_Close}.

The programmer must however be careful to never use the stream\index{stream pointer} pointer
after the blob has been closed, or after the connection object has
been closed or finalized\index{finalized}. The stream pointer should be nulled\index{nulled} when
it has outlived its usefulness, or allowed to fall out of scope.


\section{Blob Destruction}

To release a blob\index{releasing a blob}, you must use the Blob\_Unlink\index{Blob\_Unlink}
procedure call. Table~\ref{t:blbunlkx} lists the exceptions while the specification
is shown below:

\begin{Code}
procedure Blob_Unlink(
   DB :  in Connection_Type;
   Oid : in Row_ID_Type
);
\end{Code}

\begin{table}
   \begin{center}
      \begin{tabular}{ll}
         Exception Name    &  Reason\\
         \hline 
         Blob\_Error       &  No such Oid\\
      \end{tabular}
   \end{center}
   \caption{Blob\_Unlink Exceptions}\label{t:blbunlkx}
\end{table}

The following code releases the blob referenced in the example on
page \pageref{Example-code-with-OID-73763}.

\begin{Example}
declare
   C :   aliased Connection_Type;
   Oid : Row_ID_Type := 73763;
begin
   ...
   Blob_Unlink(C,Oid);
\end{Example}

\subsubsection{Generic\_Blob\_Unlink Procedure}

To use a strongly typed version of the Blob\_Unlink procedure, the
application programmer can instantiate from Generic\_Blob\_Unlink\index{Generic\_Blob\_Unlink}.
The instantiated function otherwise behaves exactly as the Blob\_Unlink
function. The instantiation parameters for Generic\_Blob\_Unlink are:

\begin{Code}
generic
   type Oid_Type is new Row_ID_Type;
procedure Generic_Blob_Unlink(
   DB :  in Connection_Type;
   Oid : in Oid_Type
);
\end{Code}

The following example shows how to instantiate the function:

\begin{Example}
declare
   type My_Oid_Type is new Row_ID_Type;
   procedure Blob_Unlink is new 
      Generic_Blob_Unlink(My_Oid_Type);
\end{Example}

\section{File and Blob Operations}

Blobs are very similar to files\index{files}. It should be no surprise
then that sometimes a file is imported\index{imported} into a blob,
or exported\index{exported} from a blob.

A file is imported into a blob with the Blob\_Import\index{Blob\_Import} call:

\begin{Code}
procedure Blob_Import(
   DB :       in     Connection_Type;
   Pathname : in     String;
   Oid :         out Row_ID_Type
);
\end{Code}

Blob\_Import returns the Oid of the newly created blob, that now contains
a copy of the file specified by the Pathname argument.

A blob's contents can be written out (exported) to a file with a call
to Blob\_Export\index{Blob\_Export}:

\begin{Code}
procedure Blob_Export(
   DB :       in Connection_Type;
   Oid :      in Row_ID_Type;
   Pathname : in String
);
\end{Code}

After a successful return from Blob\_Export, the file named by the
Pathname argument, contains a copy of the specified blob.
If any error in these import/export operations occur, the
exceptions listed in Table~\ref{t:blbxpx} occur.

\begin{table}
   \begin{center}
      \begin{tabular}{ll}
         Exception Name       &  Reason\\
         \hline 
         Blob\_Error          &  Import/export failed\\
      \end{tabular}
   \end{center}
   \caption{Blob\_Export Exceptions}\label{t:blbxpx}
\end{table}

Note that Blob\_Import creates a new blob if necessary. Blob\_Export
creates a new file if necessary.


\subsubsection{Generic\_Blob\_Import and Generic\_Blob\_Export Procedures}

To use strongly typed versions of the Blob\_Import and Blob\_Export,
the application programmer can instantiate from Generic\_Blob\_Import\index{Generic\_Blob\_Import}
and Generic\_Blob\_Export\index{Generic\_Blob\_Export} respectively. The instantiated procedure
otherwise behaves exactly as the Blob\_Import or Blob\_Export function.
The instantiation parameters for Generic\_Blob\_Import or Generic\_Blob\_Export
are:

\begin{Code}
generic
   type Oid_Type is new Row_ID_Type;
procedure Generic_Blob_Import(
   DB :       in     Connection_Type;
   Pathname : in     String;
   Oid :         out Oid_Type
);
\end{Code}

\begin{Code}
generic
   type Oid_Type is new Row_ID_Type;
procedure Generic_Blob_Export(
   DB :       in     Connection_Type;
   Oid :      in     Oid_Type;
   Pathname : in     String
);
\end{Code}

The following example shows how to instantiate the function:

\begin{Example}
declare
   type My_Oid_Type is new Row_ID_Type;
   procedure Blob_Import is new
      Generic_Blob_Import(My_Oid_Type);
\end{Example}


\chapter{Utility Functions} % Chapter 5

\section{To\_String Support}

To ease the job for programming, a number of builtin To\_String\index{To\_String} functions
are provided to allow conversion from the data type to its string
representation. The following To\_String functions are available with
the following builtin types (only one function requires a second argument):

\begin{Code}
function To_String(
   V : APQ_Boolean
) return String;
\end{Code}

\begin{Code}
function To_String(
   V : APQ_Date
) return String;
\end{Code}

\begin{Code}
function To_String(
   V : APQ_Time
) return String;
\end{Code}

\begin{Code}
function To_String(
   V : APQ_Timestamp
) return String;
\end{Code}

\begin{Code}
function To_String(
   V :  APQ_Timestamp;
   TZ : APQ_Timezone
) return String;
\end{Code}

\begin{Code}
function To_String(
   V : APQ_Timezone
) return String;
\end{Code}

\begin{Code}
function To_String(
   V : APQ_Bitstring
) return String;
\end{Code}

The following illustrates one example:

\begin{Example}
declare
   Ship_Date : APQ_Date;
begin
   Put({
   Put_Line(To_String(Ship_Date));
\end{Example}

\section{Generic To\_String Support}

Programs that make use of distinct types will require the use of generic
functions to perform To\_String\index{generic To\_String} conversions. The following generic
functions are available for instantiation:

\begin{Code}
generic
   type Val_Type is new Boolean;
function Boolean_String(
   V : Val_Type
) return String;
\end{Code}

\begin{Code}
generic
   type Val_Type is range <>;
function Integer_String(
   V : Val_Type
) return String;
\end{Code}

\begin{Code}
generic
   type Val_Type is mod <>;
function Modular_String(
   V : Val_Type
) return String;
\end{Code}

\begin{Code}
generic
   type Val_Type is delta <>;
function Fixed_String(
   V : Val_Type
) return String;
\end{Code}

\begin{Code}
generic
   type Val_Type is digits <>;
function Float_String(
   V : Val_Type
) return String;
\end{Code}

\begin{Code}
generic
   type Val_Type is delta <> digits <>;
function Decimal_String(
   V : Val_Type
) return String;
\end{Code}

\begin{Code}
generic
   type Val_Type is new Ada.Calendar.Time;
function Date_String(
   V : Val_Type
) return String;
\end{Code}

\begin{Code}
generic
   type Val_Type is new Ada.Calendar.Day_Duration;
function Time_String(
   V : Val_Type
) return String;
\end{Code}

\begin{Code}
generic
   type Val_Type is new Ada.Calendar.Time;
function Timestamp_String(
   V : Val_Type
) return String;
\end{Code}

\begin{Code}
generic
   type Val_Type is new APQ_Timezone;
function Timezone_String(
   V : Val_Type
) return String;
\end{Code}

The following example illustrates their use:

\begin{Example}
declare
   type My_Date_Type is new APQ_Timestamp;

   function To_String is new 
      Timestamp_String(My_Date_Type);

   Execution_Date : My_Date_Type;
begin
   ...
   Put("Program Execution Date: ");
   Put_Line(To_String(Execution_Date));
\end{Example}

\section{Conversion Generic Functions}

Sometimes a programmer must convert from a text format string into
another data type for manipulation. Several generic conversion \index{conversion functions} functions are
provided for the purpose (exceptions listed in Table~\ref{t:cvtx}):

\begin{Code}
generic
   type Val_Type is new Boolean;
function Convert_To_Boolean(
   S : String
) return Val_Type;
\end{Code}

\begin{Code}
generic
   type Val_Type is new Ada.Calendar.Time;
function Convert_To_Date(
   S : String
) return Val_Type; -- YYYY-MM-DD format
\end{Code}

\begin{Code}
generic
   type Val_Type is new Ada.Calendar.Day_Duration;
function Convert_To_Time(
   S : String
) return Val_Type; -- HH:MM:SS format
\end{Code}

\begin{Code}
generic
   type Val_Type is new Ada.Calendar.Time;
function Convert_To_Timestamp(
   S : String  
) return Val_Type;  -- YYYY-MM-DD HH:MM:SS
\end{Code}

\begin{table}
   \begin{center}
      \begin{tabular}{ll}
         Exception Name    &  Reason\\
         \hline 
         Invalid\_Format   &  The input value was not a proper value for the type\\
      \end{tabular}
   \end{center}
   \caption{Conversion Exceptions}\label{t:cvtx}
\end{table}

The following example illustrates some converions:

\begin{Example}
declare
   type Bool is new APQ_Boolean;
   type Birth_Date_Type is new APQ_Date;
   function To_Boolean is new Convert_To_Boolean(Bool);
   function To_Date is new Convert_To_Date(Birth_Date_Type);
   My_Bool : Bool;
   Elvis :   Birth_Date_Type;
begin
   ...
   My_Bool := To_Boolean("F");
   Elvis   := To_Timestamp("1957-01-08");
\end{Example}

\section{The Convert\_Date\_and\_Time Generic Function}

Sometimes the programmer needs the convenience of putting separate
date and time values together into a returned timestamp\index{timestamp}
value. For example the date of birth may be stored in one database
column\index{column}, while the time of birth is stored in another.
It may be necessary to work with a timestamp value instead, that contains
both of these components. To permit the use of strongly typed values\index{strongly typed values},
a generic function is provided for this purpose.

The generic specification for Convert\_Date\_and\_Time\index{Convert\_Date\_and\_Time} is:

\begin{Code}
generic
   type Date_Type is new Ada.Calendar.Time;
   type Time_Type is new Ada.Calendar.Day_Duration;
   type Result_Type is new Ada.Calendar.Time;
function Convert_Date_and_Time(
   DT : Date_Type;
   TM : Time_Type
) return Result_Type;
\end{Code}

The following example shows how to apply this function:

\begin{Example}
declare
   type My_Date_Type is new APQ_Date;
   type My_Time_Type is new APQ_Time; 
   type My_Timestamp_Type is new APQ_Timestamp;
   function To_Timestamp is new 
      Convert_Date_and_Time(
         Date_Type => My_Date_Type,
         Time_Time => My_Time_Type,
         Result_Type => My_Timestamp_Type);
   Some_Date :      My_Date_Type;
   Some_Time :      My_Time_Type;
   Some_Timestamp : My_Timestamp_Type;
begin
   ...
   Some_Timestamp := To_Timestamp(Some_Date,Some_Time);
\end{Example}

\section{The Extract\_Timezone Generic Procedure}

When a database table or result column provides a timestamp and timezone
together, it is sometimes necessary to extract\index{extract} these
components so that they can be manipulated separately. To permit the
use of application defined types, a generic procedure is provided.
The Extract\_Timezone\index{Extract\_Timezone} generic procedure is defined below,
while Table~\ref{t:xtzx} documents the exceptions:

\begin{Code}
generic
   type Date_Type is new Ada.Calendar.Time;
   type Zone_Type is new APQ_Timezone;
procedure Extract_Timezone(
   S :  in     String;
   DT :    out Date_Type;
   TZ :    out Zone_Type
);
\end{Code}

\begin{table}
   \begin{center}
      \begin{tabular}{ll}
         Exception Name    &  Reason\\
         \hline 
         Invalid\_Format   &  The input value was not a proper value for the type\\
      \end{tabular}
   \end{center}
   \caption{Extract\_Timezone Exceptions}\label{t:xtzx}
\end{table}

The following example shows how to apply this procedure:

\begin{Example}
declare
   type My_Date_Type is new APQ_Timestamp;
   type My_Zone_type is new APQ_Timezone;
   procedure Extract is new 
      Extract_Timezone(My_Date_Type,My_Zone_Type);
   Ex_Date :      My_Date_Type;  -{
   Ex_Zone :      My_Zone_Type;  -{
begin
   ...
   Extract("1957-01-08 01:13:45+04",Ex_Date,Ex_Zone);
\end{Example}

\chapter{Calendar Functions}

There is frequently the need in applications to separate out the hour\index{hour},
minute\index{minute} and second\index{second} from a time\index{time}
value. To make this easier, and to permit the continued use of strong
typing, the following generic functions are available:%
\footnote{It could be argued that these generic functions do not go the full
generic distance, because the return value types are standard types
only (types Hour\_Number, Minute\_Number and Second\_Number).%
}

\begin{Code}
generic
   type Time_Type is new Ada.Calendar.Day_Duration;
function Generic_Hour(
   TM : Time_Type
) return Hour_Number;
\end{Code}

\begin{Code}
generic
   type Time_Type is new Ada.Calendar.Day_Duration;
function Generic_Minute(
   TM : Time_Type
) return Minute_Number;
\end{Code}

\begin{Code}
generic
   type Time_Type is new Ada.Calendar.Day_Duration;
function Generic_Second(
   TM : Time_Type
) return Second_Number;
\end{Code}

The following example shows how to apply this procedure:

\begin{Example}
declare
   type Evt_Time_Type is new Ada.Calendar.Day_Duration;
   function Hour is new Generic_Hour(Evt_Time_Type); 
   function Minute is new Generic_Minute(Evt_Time_Type);
   Evt_Time : Evt_Time_Type;
   HH :       Hour_Number;
   MM :       Minute_Number;
begin
   ...
   HH := Hour(Evt_Time);
   MM := Minute(Evt_Time);
\end{Example}

\chapter{Decimal Support}

In order to support accurate number calculations, particularly for
financial work, the package APQ\-.PostgreSQL\-.Decimal\index{APQ.PostgreSQL.Decimal}
is available for the programmer to use. This package is based upon
the C source\index{C source} code extracted out of the PostgreSQL server.%
\footnote{Portions copyright (c) 1996-2001, The PostgreSQL Global Development
Group, and portions copyright (c) 1994, The Regents of the University
of California.%
} The decimal support\index{decimal support} is \emph{not} a floating
point\index{floating point} package, but does support approximately
1,000 digits worth of accuracy\index{accuracy}. Note that this is currently PostgreSQL\index{PostgreSQL}
specific code, and as such, it has not been reworked for general use
in databases like MySQL\index{MySQL}.


\section{Introduction}

The APQ\-.PostgreSQL\-.Decimal package is a binding to the extracted server
decimal code. This gives the Ada programmer access to the same numeric
support\index{numeric support} as used by the database engine to
sum columns etc. Because it is decimal\index{decimal} based, you will not have to
worry about representation issues for values like 0.3,%
\footnote{In binary floating point, the value 0.3 must be represented as 0.29999
repeat.} allowing for accurate sums and hash total calculations.


\begin{quote}
Special Note: The APQ\-.PostgreSQL\-.Decimal package is still under development,
and is subject to change. One of the most serious limitations at present
is the fact that assignment clobbers any prior concept of precision\index{precision}
and scale\index{scale} for the variable being assigned to. To overcome this, it
is possible that a future implementation of this package may provide
task safe storage to preserve the variable's precision and scale.
This can be done by saving the variable's precision and scale in task
safe storage at finalization time. When the Adjust primitive is later
called, the saved precision and scale can be restored and followed
by a call to the Constrain()\index{Constrain} function. This will implicitly keep the
variable within its configured precision and scale parameters.
\end{quote}


\section{Decimal Exceptions}

The APQ\-.PostgreSQL\-.Decimal binding can raise any of the exceptions
listed in Table~\ref{t:decx}.

\begin{table}
   \begin{center}
      \begin{tabular}{ll}
         Exception Name    &  Reason\\
         \hline 
         Decimal\_NaN      &  The value is ``Not a Number{}`` (or NULL)\\
         Decimal\_Format   &  Input does not represent a decimal number\\
         Decimal\_Overflow &  Value over/under-flowed\\
         Undefined\_Result &  Computation does not have a defined result\\
         Divide\_By\_Zero  &  An attempt to divide by zero occurred\\
      \end{tabular}
   \end{center}
   \caption{Decimal Exceptions}\label{t:decx}
\end{table}

\section{``Not a Number'' Operations}

A new Decimal\_Type\index{Decimal\_Type} value is initialized to NaN\index{NaN} (Not a
Number). This is a special status for the value, which can be assigned
to other Decimal\_Type values. When this status is detected in an
expression where a computation is being performed, the exception
Decimal\_NaN\index{Decimal\_NaN exception} is raised to indicate that no
valid result can be determined.


\section{The Decimal\_Type Type}

Decimal values are manipulated in a type, which is defined in terms
of a tagged\index{tagged} controlled\index{controlled} record:

\begin{Code}

   type Decimal_Type is new 
      Ada.Finalization.Controlled with private;

\end{Code}

These values are further defined by the following additional two attributes:

\begin{description}
   \item [Precision] specifies the precision of the decimal variable
   \item [Scale] specifies the scale of the decimal variable
\end{description}

\section{Is\_NaN Function}

To test if a value is ``Not a Number'', the Is\_NaN\index{Is\_NaN} function can
be used:

\begin{Code}
function Is_Nan(
   DT : Decimal_Type
) return Boolean;
\end{Code}

The following example shows how to apply the function:

\begin{Example}
declare
   D : Decimal_Type;
begin
   if Is_NaN(D) then
   Put_Line("D is NaN!");
   ...
\end{Example}

\section{Convert Procedure}

To import\index{import} a large decimal value from a String, the programmer may
invoke the Convert\index{Convert} procedure:

\begin{Code}
procedure Convert(
   DT :        in out Decimal_Type;
   S :         in     String;
   Precision : in     Precision_Type := 0;
   Scale :     in     Scale_Type := 2
);
\end{Code}

The arguments \emph{Precision} and \emph{Scale} arguments can be omitted
if you can accept the default values of 0 and 2 for the precision
and scale respectively. When \emph{Precision} is given as zero, the
value has no defined precision, and may grow to whatever size is necessary
to carry the result.%
\footnote{The source code indicates that the maximum precision is approximately
1,000 decimal digits.}

The following example shows how to initialize a Decimal\_Type from
a string:

\begin{Example}
declare
   D : Decimal_Type;
begin
   Convert(D,"12345.6789",0,4);
\end{Example}

\section{To\_String Function}

To make a Decimal\_Type value printable, you can call upon the To\_String\index{To\_String}
function:

\begin{Code}
function To_String(
   DT : Decimal_Type
) return String;
\end{Code}

The To\_String will return the string ``NULL''\index{NULL} if the value is
in the ``Not a Number'' state.

The following example shows how to use it in a Put\_Line call:

\begin{Example}
declare
   D : Decimal_Type;
begin
   ...
   Put_Line("D := " & To_String(D));
\end{Example}

\section{Constrain Function}

Sometimes it is desireable to constrain\index{constraining values}
a result to a particular precision and scale, while watching for overflows.
The Constrain function takes an input value, and returns a new value
with the values constrained to the given precision and scale\index{Constrain}:

\begin{Code}
function Constrain(
   DT :        in Decimal_Type;
   Precision : in Precision_Type;
   Scale :     in Scale_Type
) return Decimal_Type;
\end{Code}

The returned value is rounded (if necessary) to accomodate the \emph{Scale}\index{scale}
argument. The result must fit within the precision\index{precision} given by the \emph{Precision}
argument. The following example illustrates how the function is used:

\begin{Example}
declare
   A : Decimal_Type;
   B : Decimal_Type;
begin
   A := ...some calculation...;
   B := Constrain(A,10,2);  -- Precision 10, Scale 2
\end{Example}

\section{Expression Operations}

The Decimal\_Type values can be both assigned and computed with the
normal set of operators\index{operators} as shown in Table~\ref{t:decopr}.

\begin{table}
   \begin{center}
      \begin{tabular}{|c|c|}
         \hline 
         Operator             &  Description\\
         \hline
         +                    &  Add\\
         -                    &  Subtract\\
         {*}                  &  Multiply\\
         /                    &  Divide\\
         unary -              &  Negate\\
         =                    &  Equal\\
         <                    &  Less than\\
         <=                   &  Less than or equal\\
         >                    &  Greater than\\
         >=                   &  Greater than or equal\\
         \hline
      \end{tabular}
   \end{center}
   \caption{Decimal Operators}\label{t:decopr}
\end{table}

The following code fragment shows some Decimal\_Type expressions\index{expressions}
at work:

\begin{Example}
declare
   Watts_per_Hp : Decimal_Type; 
   Watts :        Decimal_Type;
   Volts :        Decimal_Type;
   Amperes :      Decimal_Type;
   Hp :           Decimal_Type;
begin
   Convert(Watts_per_Hp,"745.577",0,3); -- Watts/HP
   Convert(Volts,"120.0",0,1);
   Convert(Amperes,"7.0",0,1);
   Watts := Volts * Amperes;       -- # of Watts
   Hp := Watts / Watts_per_Hp;     -- # of Horsepower
\end{Example}

\section{Minimum and Maximum Values}

The Min\_Value\index{Min\_Value} and Max\_Value\index{Max\_Value} functions are available to the programmer
to conveniently return the minimum\index{minimum} or maximum\index{maximum}
value of a pair, respectively. These functions share the following
common calling signature:

\begin{Code}
function Min_Value(
   Left, Right : Decimal_Type
) return Decimal_Type;
\end{Code}

\begin{Code}
function Max_Value(
   Left, Right : Decimal_Type
) return Decimal_Type;
\end{Code}

The following example illustrates its use:

\begin{Example}
declare
   A, B :     Decimal_Type;
   Smallest : Decimal_Type;
begin
   Smallest := Min_Value(A,B);
\end{Example}

\section{Abs\_Value, Sign, Ceil and Floor Functions}

The functions in the following table are documented in this section\index{Abs\_Value}\index{Sign}\index{Ceil}
\index{Floor}:

\begin{Code}
function Abs_Value(
   DT : Decimal_Type
) return Decimal_Type;
\end{Code}

\begin{Code}
function Sign(
   DT : Decimal_Type
) return Decimal_Type;
\end{Code}

\begin{Code}
function Ceil(
   DT : Decimal_Type
) return Decimal_Type;
\end{Code}

\begin{Code}
function Floor(
   DT : Decimal_Type
) return Decimal_Type;
\end{Code}

The following example illustrates its use:

\begin{Example}
declare
   A, B :    Decimal_Type;
   Pos_Val : Decimal_Type;
begin
   Pos_Val := Abs_Value(A,B);
\end{Example}

\section{Sqrt, Exp, Ln and Log10 Functions}

The functions in the following table are documented in this section%
\index{Sqrt}\index{Exp}\index{Ln}\index{Log10}:

\begin{Code}
function Sqrt(
   X : Decimal_Type
) return Decimal_Type;
\end{Code}

\begin{Code}
function Exp(
   X : Decimal_Type
) return Decimal_Type;
\end{Code}

\begin{Code}
function Ln(
   X : Decimal_Type
) return Decimal_Type;
\end{Code}

\begin{Code}
function Log10(
   X : Decimal_Type
) return Decimal_Type;
\end{Code}

The following example illustrates its use:

\begin{Example}
declare
   X :   Decimal_Type;
   L10 : Decimal_Type;
begin
   L10 := Log10(X);
\end{Example}

\section{The Log Function}

The Log\index{Log} function permits the caller to evaluate a logrithm $\log$
$_{\textrm{base}}$ x . It has the following calling requirements:

\begin{Code}
function Log(
   X, Base : Decimal_Type
) return Decimal_Type;
\end{Code}

The following example illustrates its use:

\begin{Example}
declare
   X :    Decimal_Type;
   Base : Decimal_Type;
   L :    Decimal_Type;
begin
   L := Log(X,Base);
\end{Example}

\section{The Power Function}

The Power\index{Power} function permits the caller to evaluate the expression x$^{\textrm{y}}$.
The function accepts the following calling arguments:

\begin{Code}
function Power(
   X, Y : Decimal_Type
) return Decimal_Type;
\end{Code}

The following example assigns to \emph{P}, the value x$^{\textrm{y}}$.

\begin{Example}
declare
   X : Decimal_Type;
   Y : Decimal_Type;
   P : Decimal_Type;
begin
   P := Power(X,Y);
\end{Example}

\section{The Round and Trunc Functions}

To round\index{round} or truncate\index{truncate} a Decimal\_Type
value, the application designer may call the Round\index{Round} and Trunc\index{Trunc} functions
respectively. They both accept the following calling arguments:

\begin{Code}
function Round(
   DT :    in Decimal_Type;
   Scale : in Scale_Type
) return Decimal_Type;
\end{Code}

\begin{Code}
function Trunc(
   DT :    in Decimal_Type;
   Scale : in Scale_Type
) return Decimal_Type;
\end{Code}

The following example assigns to \emph{R}, the rounded value of \emph{X},
to 2 decimal places:

\begin{Example}
declare
   X : Decimal_Type;
   R : Decimal_Type;
begin
   R := Round(X,2);
\end{Example}

\section{Builtin Decimal\_Type Constants}

The builtin decimal constants\index{decimal constants} are defined
as the following functions%
\index{Zero}\index{One}\index{Two}\index{Ten}\index{NaN}:

\begin{Code}

   function Zero return Decimal_Type;
   function One return Decimal_Type;
   function Two return Decimal_Type;
   function Ten return Decimal_Type;
   function NaN return Decimal_Type;

\end{Code}

The NaN function can be used to put a value into a ``Not a Number''
state. This doubles as setting the value to NULL\index{NULL}, for SQL query use.


\section{Using Decimal\_Types with Query\_Type}

Creating SQL\index{SQL} queries using the Decimal\_Type\index{Decimal\_Type} and retrieving values
from SQL queries has been made easy for the application programmer.
The NaN\index{NaN} state is used to represent a NULL\index{NULL} value. This eliminates the
need for the application programmer to define indicator values.

The Append procedure allows the programmer to build SQL queries with
Decimal\_Type values. The Value\index{Value} function, permits the programmer to
retrieve a column value into a Decimal\_Type variable (with or without
a NULL\index{NULL} value).


\subsection{Using Decimal\_Type with Append}

The Append procedure has the following specification:

\begin{Code}
procedure Append(
   Query : in out PostgreSQL.Client.Query_Type;
   DT :    in     Decimal_Type'Class;
   After : in     String := ""
);
\end{Code}

\subsection{Fetching Decimal\_Type Values}

A decimal value may be retrieved from a SQL query, using the Value
function:

\begin{Code}
function Value(
   Query : in PostgreSQL.Client.Query_Type;
   CX :    in Column_Index_Type
) return Decimal_Type;
\end{Code}

If the returned value for the column is NULL\index{NULL}, the value
returned will be in the NaN state. The following example illustrates
how to apply the Value\index{Value} function:

\begin{Example}
declare
   C : Connection_Type;
   Q : Query_Type;
   D : Decimal_Type := NaN;
begin
   ...
   Prepare(Q,  "SELECT QTY, ...");
   Append_Line("FROM ORDERS");
   Execute(Q,C);
   while not End_of_Query loop
      Fetch(Q);
      D := Value(Q,1);  -- Fetch Decimal_Type
   end loop;
   Clear(Q);
\end{Example}

\chapter{Generic Database Programming}

With APQ 2.0, the support of the MySQL database becomes available.
The future may allow APQ to support even more database technologies.
With this in mind, it becomes very desireable in some circumstances
to write applications in a database neutral way. Within this documentation,
we will use the term ``Generic Database Programming'' to describe
portable\index{portable} database code.

This chapter is about programming for databases generically. Given
that APQ is built using object oriented techniques (tagged\index{tagged} Ada95\index{Ada95} records),
it should be possible to leverage this in a way to write application
procedures once, and enjoy the flexibility of choosing or changing
the database technology used later. \Ref{Generic APQ.Result}
identified one aspect of generic database processing.


\section{Generic Connections}

For most routine database work, the only object that needs to be defined
up front, is the database connection. In APQ version 2.2 and later,
there are three concrete choices:

\begin{enumerate}
   \item APQ.PostgreSQL.Client.Connection\_Type
   \item APQ.MySQL.Client.Connection\_Type
   \item APQ.Sybase.Client.Connection\_Type
\end{enumerate}

Once the high level layer of the application chooses one of these
connection objects, and establishes a connection with the database,
the connection object may be passed around as parameters to procedures.
The recommended generic way to do this, is to pass the connection
as a APQ\-.Root\_Connection\_Type\-'Class\index{Root\_Connection\_Type'Class} parameter. See the following
example:

\begin{Example}
procedure MyApp(C : in out APQ.Root_Connection_Type'Class) is
begin
   ...
\end{Example}

The classwide parameter then permits any database connection object
to be passed as a parameter. The classwide attribute causes all operations
performed upon that object to be dispatching calls. By dispatching
on the object's primitives, you ensure that database specific operations
are carried out according to the type of database being used.


\section{Database Specific Code}

Due to the wide differences that sometimes exist between database
engines\index{engines}, it is sometimes necessary to take different course of action,
depending upon the database being used. For example, PostgreSQL\index{PostgreSQL} allows
a varchar(256)\index{varchar} column to be defined, where MySQL\index{MySQL} is limited to varchar(255)
instead.%
\footnote{MySQL requires you declare the type as TEXT if you need more than
255 characters.%
} 

To determine the database being used, use the Engine\_Of\index{Engine\_Of}
predicate\index{predicate} function. This primitive exists on both the Root\_Connection\_Type\index{Root\_Connection\_Type}
and Root\_Query\_Type objects. \Ref{Generic Database Engine_Of}
and \Ref{Generic APQ.Engine_Of} describe functions and examples for
this purpose.

Obviously, if you are only given a Root\_Connection\_Type'Class connection
argument to use, you cannot know in advance which Query\_Type\index{Query\_Type} object
to use. Make use of the New\_Query\index{New\_Query} factory\index{factory} function to create
one (See \Ref{Query_Type Factories}) or use cloning\index{cloning} 
(\Ref{Query_Type Cloning}) if you have an existing Query\_Type object
available.


\subsection{Row ID Values}\label{Portability Note for Row ID Values}

When designing a new system, it is important to plan for the use of
row ID\index{row ID} values. MySQL\index{MySQL} does not support them at all, while PostgreSQL\index{PostgreSQL}
encourages their use.%
\footnote{For example, a blob is identified by a Oid value, which is basically
a row ID.%
} MySQL encourages the use of serial values instead, which is a good
practice. For generic database programming you must plan for these
differences to reduce the amount of specialized code. For more information
about obtaining row ID values, see the \Ref{Command_Oid Function},
and \Ref{Portability Note for Row ID Values} for portability\index{portability}
notes.

Additionally, the assumptions about what constitutes a null row ID\index{null row ID}
must be scrutinized. Since different databases use different values
to represent ``no row'', you should make careful use of the APQ
Null\_Oid\index{Null\_Oid} function, rather than depend upon a particular
constant. See \Ref{Null_Oid Function} for information about
that.


\section{Data Types}

When writing generic database code it is important to choose your
datatypes\index{datatypes} very carefully. One example where this is important is when
using a PostgreSQL\index{PostgreSQL} timezone\index{timezone} type (APQ\_Timezone\index{APQ\_Timezone}).
While a timezone
variable can be used in MySQL\index{MySQL} specific code, it should be emphasized
that MySQL does not support timezone values within a TIMESTAMP\index{TIMESTAMP} database
column type. Sybase\index{Sybase} does not support timezones at all.

A similar problem exists with PostgreSQL bit string types (APQ\_Bitstring\index{APQ\_Bitstring}).
MySQL and Sybase do not support them. So if you want to write portable
code, stick to simple data types in your application. 

When you must make use of special database types, be prepared to specialize
the code somewhat, depending upon the database being used.


\subsection{Column Types}

Another area that is important to consider is the database column
\index{column types} types that are chosen for tables. Comparing the table in
\Ref{PostgreSQL SQL Data Types} and \Ref{MySQL Data Types}, the
reader can see that most columns can be declared in SQL using the
same column type declarations. However, there are some important differences.
For example, a SERIAL\index{SERIAL} column in PostgreSQL must be declared as an
INTEGER\index{INTEGER} type when using MySQL. For most applications, this is not
too much of an issue, because applications usually don't create and
drop tables. However, this can be an issue with temporary tables.

As noted in the prior section, MySQL also does not directly support
some data types such as the time zone value within a DATETIME\index{DATETIME} type.
If time zones must be supported, the writer may simply add a time
zone column declared in SQL as a SMALLINT\index{SMALLINT} value, and work with the
timezone separately.


\section{Pulling it All Together}

This section will examine a fairly trivial example of a generic database
procedure. It will make one exception for MySQL\index{MySQL}, so that the reader
will know how to work with database engine\index{engine} differences. Ideally, you
would want to avoid differences, where possible.

The example is a real world example. A procedure is required to fetch
the most recent stock price available on file, for a given security
(by ticker symbol). While there may be a price for the security on
the given day, if there is not one listed, the procedure is expected
to fall back to the most recent price available. The table being consulted,
is defined as follows:\label{PRICE_HIST Table Definition}

\begin{SQL}
CREATE TABLE PRICE_HIST (
    SECURITY   CHAR(10) NOT NULL,
    PRICE_DATE DATE NOT NULL,
    PRICE      REAL NOT NULL,
    PRIMARY KEY(SECURITY,PRICE_DATE)
);
\end{SQL}

Here is the package spec for the Prices module, which makes the procedure
Last\_Price available for use:

\begin{Example}
with APQ;
use APQ;
  
package Prices is
  
   procedure Last_Price(
      C :       in out Root_Connection_Type'Class;
      Security : in     String;
      Price :       out APQ_Double
   );
 
end Prices;
\end{Example}

Given any database connection, and a ticker symbol provided in argument
Security, the procedure Last\_Price is expected to lookup the most
recent stock price in the PRICE\_HIST table and return that price
in the Price argument. Here is the body of the package, written to
work with any database:

\begin{NumberedExample}
package body Prices is
   procedure Last_Price(
      C :        in out Root_Connection_Type'Class;\label{Ex:RootConn}
      Security : in     String;
      Price :       out APQ_Double
   ) is
      function Value is new Float_Value(APQ_Double);
  
      Q : Root_Query_Type'Class := New_Query(C);\label{Ex:QFact}
   begin
  
      Prepare(Q,     "SELECT SECURITY,PRICE_DATE,PRICE");
      Append_Line(Q, "FROM PRICE_HIST");
      Append(Q,      "WHERE SECURITY = ");
      Append_Quoted(Q,C,Security,Line_Feed);
      Append_Line(Q, "ORDER BY SECURITY,PRICE_DATE DESC");\label{Ex:OrBy}
 
      if Engine_Of(C) = Engine_MySQL then\label{Ex:EngOf}
         Append_Line(Q,"LIMIT 1");\label{Ex:Limit1_Again}
      end if;
 
      Execute(Q,C);
 
      begin
         Fetch(Q);
      exception
          when No_Tuple =>
             raise;    -- Indicates no price
      end;
 
      Price := Value(Q,3);
 
   end Last_Price;
 
end Prices;
\end{NumberedExample}

A few notes are in order here: The spec already with's and uses the
package APQ. So it is not repeated in the body of the package. The
Last\_Price procedure takes a Root\_Connection\_Type'Class
(line~\ref{Ex:RootConn} connection object type, so it can be supplied
with a MySQL or PostgreSQL database connection\index{connection} (Connection\_Type).

The New\_Query factory\index{factory} function (line~\ref{Ex:QFact}) is used to create
the correct Query\_Type object necessary to match the connection. It is
used to form and execute the query. The Prepare, Append\_line,
Append\_Quoted calls build up an SQL query, and then the type of the
database is queried by calling Engine\_Of (line \ref{Ex:EngOf}). If the
database being used is a MySQL database, the query is optimized
\footnote{One could also argue that it is {}``fixed'' here, since MySQL
insists that all row results of a query be fetched.} so that only one
row is returned by use of the extended SQL ``LIMIT 1''\index{LIMIT} clause
(line~\ref{Ex:Limit1_Again}).

Note that the ``ORDER BY''\index{ORDER BY} clause in line~\ref{Ex:OrBy} requires that
the sort order be descending (most recent dates first). Given that the
``ORDER BY'' clause specifies indexed columns, this retrieval should
be quick since a reverse index retrieval is possible (if the database
cannot do this, you should create a new index or fix the primary key to
be descending).

Since we are only interested in the most recent price (ie. one price),
only one Fetch call is made. This is not a problem for PostgreSQL,
but it is for MySQL if there there are more than one row (see 
\Ref{Fetch Limitations} to find out why). If the database is MySQL
the problem is addressed by adding to the query a MySQL extended clause
``LIMIT 1''\index{LIMIT}, to limit the results to one row (line~\ref{Ex:Limit1_Again}).

The price is then fetched from column 3 (PRICE) and the procedure
returns. If no rows are returned, we simply raise the APQ.No\_Tuple
exception here to keep the example simple. A finished application
would handle this in a more elegant way perhaps.

The important thing to recognize here is that there is nothing specific
to the type of database being used in this Prices package, except
for the MySQL work-around. The only APQ package being used is the
top level package APQ, and the root types for connection and query
types.

Here is a PostgreSQL main program:

\begin{Example}
with Ada.Text_IO;
with APQ.
with Prices;
  
use APQ, Prices, APQ.
  
procedure Price_PG is
   C : Connection_Type;
   P : APQ_Double;
begin
 
   Set_DB_Name(C,"investments");
   Connect(C);
 
   Last_Price(C,"RHAT",P);
   Put_Line("RHAT $" & APQ_Double'Image(P));
 
   Disconnect(C);
 
end Price_PG;
\end{Example}

Please notice the following points about the main program:

\begin{enumerate}
   \item The database specific package is with'ed as APQ.PostgreSQL.Client
         here to choose the database connection being used.
   \item The Connection\_Type object is declared in APQ.PostgreSQL.Client and
         derives from APQ.Root\_Connection\_Type.
   \item The database is chosen and a connection is established.
         \item The Last\_Price procedure is called, providing only the connection
         and the security's ticker symbol that the price is being sought for.
   \item The returned price P is then printed (albeit crudely)
   \item The application disconnects from the server and exits.
\end{enumerate}

The same main program using MySQL looks like this:

\begin{Example}
with Ada.Text_IO;
with APQ.
with Prices;
 
use APQ, Prices, APQ.
 
procedure Price_My is
   C : Connection_Type;
   P : APQ_Double;
begin
 
   Set_DB_Name(C,"investments");
   Connect(C);
 
   Last_Price(C,"RHAT",P);
   Put_Line("RHAT $" & APQ_Double'Image(P));
 
   Disconnect(C);
 
end Price_My;
\end{Example}

The only difference between this MySQL\index{MySQL} main program and the prior
PostgreSQL\index{PostgreSQL} main program, is the name of the package used (APQ\-.MySQL\-.Client).
APQ truly is the closest thing to database independance!


\section{Miscellaneous Portability Issues}

There are a number of other database portability issues that should
be born in mind. An incomplete list has begun in this document below:

\begin{itemize}
   \item temporary tables creation
   \item SELECT ... INTO TABLE ...
\end{itemize}

This list will grow with submissions and experience.
\footnote{Contributions are welcome.}


\subsection{Temporary Tables\label{Creating Temp Tables}}

Many databases allow the SQL programmer to create a temporary\index{temporary tables} table
prior to its use in the application. This temporary table is only
visible to the user of the established database connection. When the
database connection is closed or disconnected, the temporary table
is automatically discarded and its space recycled by the database
engine.

The difficulty that a generic database programmer needs to be aware
of is that some databases work differently. Normally, an application
would perform something like the following to create a temporary table
in the database session that required it:

\begin{SQL}
CREATE TEMP TABLE INTERMED_RESULTS (
   SECURITY  CHAR(10) NOT NULL,
   HOLDINGS  BIGINT NOT NULL
);
\end{SQL}

Subsequent to the successful creation\index{create temp table} of this temporary table, the
application would populate it and use it as necessary. The application
may even create indexes on the table after the table is populated,
to help performance in later stages of the table's use.

While this works for PostgreSQL\index{PostgreSQL} and MySQL\index{MySQL}, the generic programmer
should be aware that this will not work on an Oracle\index{Oracle} database (when
APQ gets there some day). Oracle permits the same syntax, but operates
differently: Oracle only permits one CREATE TEMP TABLE\index{CREATE TEMP TABLE} operation to
be performed, in the same way that a permanent table is created. Once
created, the user implicitly gets access to the table upon demand.
Upon the first reference to the temporary table in a particular session,
you get a temporary table created, which is empty. You then populate
and use that temporary table without ever having to create it within
that particular session. When the session is over, the table's contents
are discarded.

So how do you plan for this? If the Engine\_Of\index{Engine\_Of} function
indicates Engine\_Oracle\index{Engine\_Oracle}, you must \emph{not} create the
temporary table during your database session. This will be done when
the permanent tables are created.%
\footnote{In many respects, this is perhaps the best time to declare and plan
for a temporary table.%
} For other database engines, you should create the temporary table
when you are about to use them in your application.

Perhaps the most sensible thing for temporary tables is to isolate
much of that work to stored\index{stored procedures} procedures. Your program can then just
invoke the stored procedure in a portable fashion.


\subsubsection{Indexes on Temporary Tables}

There is an additional piece of advice to consider when creating indexes
for temporary tables. For performance reasons, it is often best to
populate the temp table with no indexes\index{indexes} created. After the table has
been populated, indexes can then be efficiently added and dropped
as the needs arise.

In the Oracle case, these indexes are likely to be predefined as is
the declaration of the temporary table itself. So when creating indexes
for temporary tables, you should also probably test for Oracle in
the generic code. When using Oracle, you probably do \emph{not} want
to create or drop the index\index{index, create/drop}, as it will probably affect all users
of that temporary table definition.%
\footnote{The author has not verified this point, but the reader is encouraged
to do so.%
}

\subsection{SELECT ... INTO TABLE}

A number of database engines support a SQL syntax along the lines
of:

\label{SelectIntoTempTable}
\begin{SQL}
SELECT *
FROM MY_TABLE
WHERE ...
INTO TEMP TABLE TEMP123
\end{SQL}

The above SQL code performs the usual SELECT\index{SELECT INTO TEMP}, but places its results
into a temporary table named TEMP123. \footnote{One application
framework that the author is familiar with used a suffix of ``123'' to
denote the name of a temporary table.} Alternatively, databases will
also often permit:

\begin{SQL}
SELECT *
FROM MY_TABLE
WHERE ...
INTO TABLE RESULTS
\end{SQL}

This query places the results into a permanent table named RESULTS
(note that the keyword TEMP was dropped).

Both of these operations are very convenient for processing intermediate
results. However, database engines vary in their support. Neither
of these formats are supported by PostgreSQL or MySQL, but they are
supported by INFORMIX.

The SQL-99\index{SQL-99} way to perform this operation is specified as follows:

\begin{SQL}
INSERT INTO RESULTS
SELECT *
FROM MY_TABLE
WHERE ...
\end{SQL}

This syntax is supported by both PostgreSQL\index{PostgreSQL} and MySQL\index{MySQL} for permanent
tables.

There is no provision currently to create a temporary table on the
fly with syntax shown on page \pageref{SelectIntoTempTable}
using PostgreSQL or MySQL. So even SQL-99 syntax won't help you there.
You can only create a temporary table, and then use the INSERT INTO\index{INSERT INTO}
... SELECT syntax after the temporary table is created. But don't
forget the limitation in \Ref{Creating Temp Tables}.


\chapter{Troubleshooting}

There are several problems\index{problems} that can crop up in applications using
the APQ Binding. These problems usually fall into one of the following
categories:

\begin{itemize}
   \item PostgreSQL database server ``personality''
   \item The PostgreSQL libpq C library interface
   \item The MySQL client library
   \item Sybase libraries and/or interfaces file
   \item The APQ binding itself
\end{itemize}

The APQ Binding attempts to insulate the user as much as is practical
from the client library issues and the database server. However, some
issues still manage to poke through. This chapter is an attempt to
provide some useful advice for those people that are encountering
unexpected behaviour, using the APQ Binding.


\section{General Problems}

The following subsections provide general troubleshooting help with
APQ binding issues.


\subsection{Missing Rows After Inserts (PostgreSQL)}

The first step in identifying whether this section applies or not,
is to ask:

\begin{itemize}
   \item is a transaction\index{transaction} processing being used?
   \item or, is a transaction pending when it shouldn't be?
\end{itemize}

If the second bullet applies to you, then you need to correct the
logic in your program (but read on to find out why).

On the other hand, if you are purposely using transactions (first
case) and you are losing\index{losing rows} inserted row information, then it is likely
that you are suffering from an aborted transaction in PostgreSQL\index{PostgreSQL} (this
is PostgreSQL specific). It might also represent an APQ binding bug\index{bug},
but check first to make sure that you have not put PostgreSQL into
an ``abort state''\index{abort state} (ie. in a snit). This is unusal database behavior,
but the PostgreSQL group seems to be proud of it.

The APQ Binding should prevent aborted\index{aborted transactions} transactions from being left
unnoticed.%
\footnote{If however, the Abort\_State\index{Abort\_State} exception is never being raised, then
it is possible you have a PostgreSQL porting issue. If the PostgreSQL
notice message format changes, the APQ binding code will fail to recognize
the notification of an ``abort state'' from the database server.%
} When the database server notificaties the APQ binding that an {}``abort
state'' has been entered%
\footnote{After a duplicate key on insert error, for example.%
}, any further attempts to execute SQL queries or COMMIT WORK\index{COMMIT WORK} on that
connection, will raise the Abort\_State exception for PostgreSQL (see
also \Ref{In_Abort_State Function} and \Ref{Abort_State exception}).

One common reason this happens is when an application inserts\index{insert of rows} rows
into a table, and intercepts the SQL\_Error\index{SQL\_Error} exception. This exception
is caught because the application writer wants to ignore the insert
on a duplicate\index{duplicate key insert} key error. The difficulty here is that the PostgreSQL
database engine will enter an ``abort state'' after the failed
INSERT\index{INSERT} operation, regardless of how the application handles the exception.
The only recourse to recovery at this point is to rollback the transaction
with a call to Rollback\_Work (\Ref{Begin, Commit and Rollback Work functions}).

This brings up a question about the APQ Binding. Why doesn't the APQ
binding raise Abort\_State immediately after the failed INSERT\index{INSERT} operation
within a transaction? There are two reasons:

\begin{enumerate}
   \item The ``Abort State'' notice is provided to the APQ binding in the
         form of a callback.
   \item Processing SQL\_Error exception within a transaction implies an aborted
         transaction (for PostgreSQL only).
\end{enumerate}

Notices are received by the APQ binding by registering a callback
with the database server. As a result, it is not always possible to
know about the ``abort state''\index{abort state} when it might be critical to the
application. Even if the information is available, this may not always
be so in future versions of PostgreSQL\index{PostgreSQL} (timing may change).

The very fact that you've started a transaction with Begin\_Work and
you have encountered an SQL\_Error exception should tell you that
you must Rollback\_Work and recover (remember this is only for PostgreSQL).
So as the developer, you should be thinking:

$Abort\, State=Begin\, Work+SQL\, Error$

The APQ binding has been designed to avoid several SQL statements
from being executed and being ignored because the database server
is in this ``Abort State''. This is why the Execute and Commit\_Work
calls check for this and raise the Abort\_State\index{Abort\_State} exception. However,
the best advice is to not rely on this mechanism when programming
the logic of your application.


\subsection{Missing Time Data (Or Time is 00:00:00)}

You build a query to insert or update a row with date and time\index{time} information,
but only the date\index{date} is getting stored in the database. The time component
always reads midnight\index{midnight} (00:00:00). Or perhaps you provide a date and
timestamp as part of a WHERE\index{WHERE clause} clause, but it fails because only the
date value is being put into the query (the time component always
shows as midnight). You print out the variables using To\_String and
they show the correct values, as follows:

\begin{Example}
declare
   type My_Date_Type is new APQ_Timestamp;
   My_Date : My_Date_Type;
begin
   ...
   Put_Line(
      "My_Date='"
      & To_String(APQ_Timestamp(My_Date))
      & "'"
   );
\end{Example}

The above symptoms are the result of a common problem. This human
error is easy to make and is due to choosing the incorrect generic
procedure. Look for a generic instantiation statement like the following:

\begin{Code}

   procedure Append
      is new Append_Date(My_Date_Type);

\end{Code}

If your data type My\_Date\_Type holds time information, then
it is likely that you meant to code the following instead:

\begin{Code}

   procedure~Append
      is new Append_Timestamp(My_Date_Type);

\end{Code}

The error was choosing generic procedure Append\_Date\index{Append\_Date} over the correct
Append\_Timestamp\index{Append\_Timestamp} routine.

A similar mistake can be made choosing between Encode\_Date\index{Encode\_Date} and Encode\_Timestamp\index{Encode\_Timestamp}
generic procedures. So watch for these subtle differences!


\subsection{Exception No\_Tuple}

If you are having the exception No\_Tuple\index{No\_Tuple} being raised when you don't
think it should be, then check to see if the following apply:

\begin{itemize}
   \item Are you using MySQL\index{MySQL}?
   \item Do you use the End\_of\_Query\index{End\_of\_Query} function calls?
\end{itemize}

If you do, then you need to look for any code calling End\_of\_Query,
particularly loops:

\begin{Example}
   while not End_of_Query(Q) loop
      Fetch(Q);
      ...
   end loop;
\end{Example}

This type of code works well for PostgreSQL\index{PostgreSQL}, but may be problematic
with some other databases (MySQL has a problem with this). Restructure
your code to eliminate the calls to End\_of\_Query (this call is now
considered \emph{obsolete} within APQ). \Ref{End_of_Query}
describes the problem in greater detail. Restructure the loop to something
like the following:

\begin{Example}
loop
   begin
      Fetch(Q);
   exception
      when No_Tuple =>
         exit;
   end;
   ...
end loop;
\end{Example}

\subsection{Null Values}

If you are using Sybase\index{Sybase} and experiencing a problem where you are receiving
a Null\_Value\index{Null\_Value} exception when there should be a value, then check to
make sure that you have called Fetch\index{Fetch}. MySQL and PostgreSQL will raise
No\_Result\index{No\_Result} when this happens. Sybase however, will appear to APQ as
holding row results (which it might be), but APQ will not realize
that Fetch has not yet been called. The Sybase library will then return
a NULL\index{NULL} value if an attempt to get a column value is made, without
a prior call to Fetch.

Forgetting to call Fetch is easy to do when you code a SELECT\index{SELECT} to return
one row with the help of a primary key reference in a WHERE\index{WHERE clause} clause.
Since you are expecting one row to be returned, you will not code
it as part of a traditional loop. So it is human nature to forget
to code the call to Fetch after the Execute call. The following example
might be typical of this:

\begin{Example}
   Prepare(Sy_Q,     "SELECT PUB_ID,PUB_NAME,CITY,STATE");
   Append_Line(Sy_Q, "FROM PUBLISHERS");
   Append(Sy_Q,      "WHERE PUB_ID = ");
   Append_Quoted(Sy_Q,Sy_C,Pub_Id);
   Execute(Sy_Q,Sy_C);
   Fetch(Sy_Q);      -- Easy to forget
\end{Example}

\subsection{Database Client Problems}

If problems with the database engine begin to occur after a particular
query, look for the following:

\begin{itemize}
   \item Are you using MySQL\index{MySQL}?
   \item Are your doing a SELECT\index{SELECT}, or otherwise returning row results?
   \item Is your Query\_Type object in Sequential\_Fetch mode?
   \item Is your code fetching all row data?
\end{itemize}

This problem may occur with MySQL\index{MySQL}, since the client library for MySQL
requires that all row result data be fetched. Failure to fetch all
rows may cause a backlog in communication with the database server
and cause strange behaviour or errors. Check the Query\_Type fetch
mode.


\subsection{Client Performance or Memory Problems}

Check to see if the following apply:

\begin{itemize}
   \item Are you using MySQL\index{MySQL}?
   \item Are you doing a SELECT\index{SELECT} or otherwise returning large row sets?
   \item Is your Query\_Type object in Random\_Fetch\index{Random\_Fetch} mode? 
\end{itemize}

If the above are true, then it is possible you have formed a query
that has generated a large row set. Since the default for the MySQL
Query\_Type object is for Random\_Fetch mode, the APQ library calls
upon mysql\_store\_result() to fetch all of the result set into the
client's memory for random access. For reasonable sized sets of rows,
this works well, but for larger results, this can be very expensive
and may run your application out of memory.

Consider the PRICE\_HIST table like the one discussed on page \pageref{PRICE_HIST Table Definition}.
For MySQL, if you fail to include the LIMIT 1 clause, and your Query\_Type
object uses the default Random\_Fetch mode, you could easily find
your application selecting the entire history of one security into
your application client memory! If you have several years of price
history for that security, your application may be destined to run
out of memory the first time that query is run.

When using the MySQL database, you must consider the following:

\begin{itemize}
   \item MySQL fetches in Random\_Fetch mode must guarantee a \emph{reasonably
         small result set} (use the LIMIT clause if necessary).
   \item MySQL fetches in Sequential\_Fetch mode must fetch \emph{all} row
         data
\end{itemize}

Unless otherwise noted, you probably do not need to be concerned about
this. PostgreSQL\index{PostgreSQL} for example, does not require all row data to be
fetched. However, if there are ways to restrict the row set, this
may improve database server performance.


\subsection{Can't Find Existing Table Names}

Some databases use caseless\index{caseless} references to database objects, while
others are case\index{case sensitive} sensitive (MySQL can be either). If you are using
MySQL and experiencing problems, one solution is to configure the MySQL parameter:

\begin{Code}

   [mysqld]
   set-variable = lower_case_table_names=1

\end{Code}

By default, MySQL distinguishes
between table names PRICE\_HIST, Price\_Hist, and price\_hist for
example. Setting the parameter true (1), causes all table names
to be lowercased (effectively caseless)\index{caseless table names}.

The another approach is to check the SQL case policy\index{case policy} being used. APQ
can work with MySQL in Lower\_Case, Upper\_Case or Preserve\_Case
mode. The last choice only makes sense of course, if the SQL text
in the program is correct.

The suggested approach is to configure lower\_case\_table\_names=1
(as shown above) and to Set\_Case(C,Upper\_Case). Uppercase\index{uppercase} is easier
to read in the trace logs, since the SQL stands out from the other trace
information.

If you are using Sybase\index{Sybase}, you cannot change the server. This means
that either a Upper\_Case or Lower\_Case policy has to work for you,
or you have to use exact case\index{exact case} for SQL code in the Ada code. It is
possible to use a Lower\_Case policy most of the time, and make exceptions
where they are required. The following shows how to make exceptions
for a table name only:

\begin{Example}
declare
   C : Connection_Type;
   Q : Query_Type;
begin
   Prepare(Q,"SELECT DESCRIPTION");
   Append(Q,"FROM ");
   Set_Case(Q,Preserve_Case);
   Append(Q,"Part_Info",Line_Feed);
   Set_Case(Q,Lower_Case);
   Append_Line(Q,"WHERE ...");
   ...
   Execute(Q,C);
\end{Example}

In the example presented, the case policy\index{case policy} was changed temporarily
while building the query. Here the table name Part\_Info will have
its case preserved, even though it is not a quoted string\index{quoted string}. The example
assumes here that the normal policy is Lower\_Case. 

Note however, that any case policy change in the Query\_Type object
is lost once Execute is called. An example will clarify this. Assume
from our example above, two things:

\begin{enumerate}
   \item The case policy in effect in the connection object C, is Lower\_Case
         (all SQL text is lowercased except where case must be preserved).
   \item Assume that the second Set\_Case(Q,Lower\_Case) call in the example
         code was commented out (the policy for Q is left at Preserve\_Case)
\end{enumerate}

If the above two assumptions are true, then upon return from the Execute(Q,C)
call, the Query\_Type object would now hold the case policy of Lower\_Case
(this came from Connection\_Type). Even after you call Clear, or Prepare
and start a new query, this is the SQL case policy that is now in
effect for Q.


\section{Blob Related Problems}

Blob operations on a database can be tricky to get right. The following
subsections provide some assistance with blob related issues.


\section{Blob\_Create and Blob\_Open Fails}

You have written what seems like a simple piece of code that creates
a blob, and then writes some data to it. It couldn't possibly fail
on paper, but it does when you run it. Or you are wondering why that
Blob\_Open\index{Blob\_Open failures} call keeps failing, because you are certain that the OID
of that blob surely does exist. These are both symptoms of the same
problem!

\begin{quote}
   \emph{``All blob operations in PostgreSQL must be performed within
      a transaction''}
\end{quote}

Repeat it to yourself again. It is too easy to forget this fact!

Unless you have started a transaction on the connection that you are
using, \emph{all blob operations will fail}. They will only succeed within
a transaction\index{transaction}. Furthermore, make certain your application commits\index{commit}
the changes to your blob, after they have been performed successfully.
Otherwise your application may fall prey to the default actions of
PostgreSQL, which may be to rollback\index{rollback} your changes.%
\footnote{Check your PostgreSQL documentation to determine what the default
transaction action is for your version of PostgreSQL database.}


\section{Blob I/O Buffering Bugs Suspected}

If you have good reason to believe that the APQ binding software%
\footnote{Only APQ versions 1.2 and later have buffered blob I/O.%
} has a bug\index{bug} in its buffered\index{buffered blob I/O} blob I/O, you can disable blob I/O buffering.
This is done by specifying a value of zero for the Buf\_Size argument
in the Blob\_Open and/or Blob\_Create calls that you are troubleshooting.
Be prepared to accept a large degradation in performance when specifying
unbuffered I/O this way. The performance\index{performance} is especially poor when array
I/O is performed.

Another possibility might be that you need to call upon Blob\_Flush\index{Blob\_Flush}
at strategic points in your application. While the buffering algorithms
used are such that you should not need to worry about this, it is
worth investigating when problems exist. 

Note also that multiple write access to the same blob is definitely
\emph{not} supported by APQ.


\section{Transaction Problems}

The following subsections deal with transaction problems that may
occur with application termination.


\subsection{Abnormal Termination of Transactions}

The APQ binding is designed to commit or rollback a transaction when the
Connection\_Type object finalizes. The default behavior of the
Connection\_Type is to ROLLBACK WORK\index{ROLLBACK WORK}, when the object finalizes\index{finalizes}%
\footnote{Provided that the Connection\_Type object is connected to the
database at the time of finalization.}. Consequently, if your program
raises an uncaught exception (perhaps Program\_Error or
Constraint\_Error), the Connection\_Type object will finalize\index{finalize} and
rollback\index{rollback} the transaction on you. \footnote{Unless you have changed the
default setting for the object. } If this is undesired behaviour, then
check out the Set\_Rollback\_On\_Finalize primitive in section
\Ref{Set_Rollback_On_Finalize Procedure}.


\subsection{Aborted Applications}

The APQ binding can only perform the default COMMIT/ROLLBACK action
(see \Ref{Set_Rollback_On_Finalize Procedure}) if the Connection\_Type
object is permitted to have its Finalize\index{finalize} primitive called. If the
process under UNIX for example, is terminated with a signal (by the
kill(1)\index{kill(1) command} command), the objects within your application may not experience
a Finalize call, because normal Ada shutdown procedures were not invoked.%
\footnote{There are Ada95 ways to deal with UNIX signals, which permit an orderly
Ada shutdown of your application. However, a kill -9 prevents any
action at all from being performed by the application!%
} If this is the reason for your problem, then you have two courses
of action:

\begin{itemize}
   \item Commit/Rollback explicitly in the program (prior to receiving a signal)
   \item Avoid signalling the application
   \item Add signal\index{signal handling} handling capability to your Ada application, to permit
         an orderly application shutdown
   \item Not use kill -9, which prevents any application recovery or notification
\end{itemize}

The choice is generally up to the application designer. Whenever possible
however, where it is important, the application should perform its
own explicit commit or rollback operation.

Note however, that if the Connection\_Type object is not permitted
to finalize\index{finalize} successfully on its own, the database software itself
has default procedures for dealing with disconnected sessions. Some
databases will rollback transactions if the socket/pipe connection
to the server is disconnected. Others will commit. You'll need to
check your server documentation and configuration to sort that one
out.


\section{SQL Problems\label{SQL Problems}}

If you are experiencing SQL problems that you don't understand, the
quickest way to inspect what is really going on is to use the APQ
trace\index{trace facility} facility. The documentation for the SQL trace facility is given
in \Ref{Trace Facilities}. So RTFM.


\subsection{Tracing SQL}

Where your Connection\_Type connects to the database, add a call to
Open\_DB\_Trace as follows:

\begin{Example}
declare
   C : Connection_Type;
begin
   ...
   Connect(C);
   Open_DB_Trace(C,"trace_file.txt",Trace_APQ);
\end{Example}

Without adding another line of code, every SQL interaction will be
captured to trace\_file.txt, which you can inspect when the application
completes.%
\footnote{Note however that the tracing facilities are not task (thread) safe.
If your application uses tasking, then disable tracing in the tasks
or avoid using APQ in more than one task.%
}


\subsection{Too Much Trace Output}

If your application performs so many SQL operations that the trace
file becomes too large, then disable\index{disable tracing} the tracing until you get to
a strategic point in the program:

\begin{Example}
declare
   C : Connection_Type;
begin
   ...
   Connect(C);
   Open_DB_Trace(C,"trace_file.txt",Trace_APQ);
   Set_Trace(C,False);  -- Disable trace for now..
   ...
   Set_Trace(C,True);   -- Start tracing now
\end{Example}

The overhead of the Set\_Trace primitive is light, unless you have
selected Trace\_libpq or Trace\_Full (these add the overhead of invoking
libpq functions PQtrace() and PQuntrace()). Light overhead permits
you to use Set\_Trace within a loop without severe penalty, to gather
only the information you need.


\subsection{Captured SQL Looks OK}

If the SQL code captured in \Ref{SQL Problems} looks OK,
but the database engine is still reporting a problem, then try the
following:

\begin{enumerate}
   \item Create a capture file using Trace\_APQ (or use the current one you
         have).
   \item Edit (cut) out the portion of the SQL queries in the capture file
         that you are having difficulty with.
   \item Use the database interactive SQL command (psql for PostgreSQL) and
         replay the extracted problem SQL queries.
   \item Edit SQL query and repeat step \#3 as necessary until you discover
         your error.
\end{enumerate}

This allows you to experiment with the SQL text as your applicaton
created it. Once you achieve success with psql, you can then go back
and correct your application to form the query correctly.


\subsection{You Want to Report a Problem to PostgreSQL}

If you want to report a trace file in terms that the PostgreSQL people
understand, simply choose the Trace\_libpq trace mode when creating
a trace file. Then send them the trace file with a description of
the problem.


\subsection{Missing Trace Information}

The trace information is collected at the Connection\_Type object
level. Check to see if you have more than one Connection\_Type object
involved. If so, make sure you set the appropriate trace settings\index{trace settings}
on the connections that you want to collect trace information for.
Different instances of a Connection\_Type object should write its
traces to different files.

Note also, that when a Connection\_Type object finalizes\index{finalizes}, its trace
file is closed.


\section{Cursor Related Problems}

To use a cursor\index{cursor} you generally need an SQL phrase of the form\index{WHERE CURRENT OF}:

\begin{SQL}

   WHERE CURRENT OF <Cursor_Name>

\end{SQL}

Obviously, the APQ function primitive Cursor\_Name\index{Cursor\_Name} is expected to
provide this name. But does this function primitive seem to raise
the exception \emph{No\_Results}\index{No\_Results} for you? If so, you've probably supplied
the wrong Query\_Type object to the function Cursor\_Name.

Review the test program on page \pageref{Cursor Example Program},
where you see the statement:

\begin{Example}

   Append_Line(Q2,"WHERE CURRENT OF " & Cursor_Name(Q));

\end{Example}

There are two Query\_Type objects being used:

\begin{enumerate}
   \item Q for the cursor fetch (outer query)
   \item Q2 for the inner SQL query (the UPDATE)
\end{enumerate}

When coding this statement, it is very easy to type Q2 for Cursor\_Name
argument, instead of Q.%
\footnote{The author made this very mistake the first time the example program
was written.%
} However, the cursor name is only available for the outer query\index{outer query} (the
one with the cursor and cursor results). So be certain to carefully
distinguish between the inner and outer query objects.


\section{Connection Related Problems}

If you are experiencing trouble establishing a connection to the database
itself, then there are a number of environment\index{environment} related issues.


\subsection{PostgreSQL Connections}

A number of environment variables affect a PostgreSQL database connection:%
\footnote{Check your PostgreSQL documentation for the final word on this subject.%
}

\begin{description}
   \item [PGHOST]Host name of the database server
   \item [PGPORT]IP port number or UNIX socket pathname of the database server
   \item [PGDATABASE]Database name within the database server
   \item [PGUSER]Database user name
   \item [PGPASSWORD]Database password
   \item [PGREALM]Kerberos realm for the database server
   \item [PGOPTIONS]Database server options
\end{description}

Any of these connection mode parameters that are not configured in
the application are defaulted to the ones defined by the above environment
variables. If you are experiencing trouble, make certain that your
variables are \emph{exported}\index{exported variables}. In many shells, like the Bourne\index{Bourne shell} and
Korn\index{Korn shell} shells,%
\footnote{I won't encourage anyone here to use the csh.%
} this is done as follows (for the PGHOST\index{PGHOST} variable):

\begin{Example}

   export PGHOST

\end{Example}

Some shells, like the Korn shell and the GNU bash\index{bash shell, GNU} shell, allow multiple
variable names to be listed at once:

\begin{Example}

   export PGHOST PGPORT PGDATABASE PGUSER

\end{Example}

Once you have the environment configured correctly, you should be
able to access the database with the PostgreSQL psql\index{psql} command. If the
psql%
\footnote{Do a ``man psql'' for details about the psql client command.%
} command still fails, then you may need to revisit your environment
variable settings or possibly even the database server configuration.%
\footnote{Check the security aspects of your connection first.%
}


\subsubsection{PostgreSQL UNIX (Local) Connections}

The way that the local socket\index{local socket} is specified for PostgreSQL seems to
have changed with its different releases. Make sure you choose the
correct Set\_Port\index{Set\_Port} procedure to specify the local UNIX socket. Use
the Set\_Port that accepts a string argument. The string indicates
that a local UNIX connection is required. PostgreSQL 7.3.5 and later
(unless it has changed again), requires only the ``number'' to
specify the local socket. For example, if you ``netstat -na'' command
shows the following:\footnote{The output lines have had the RefCnt
and Flags columns removed to allow the lines to fit the space
available.}

\begin{NumberedExample}
$ netstat -na --unix
Active UNIX domain sockets (servers and established)
Proto  Type    State         I-Node Path
unix   DGRAM                 2086   /dev/log
unix   STREAM  LISTENING     2937   /var/run/mysqld/mysqld.sock
unix   STREAM  LISTENING     2942   /tmp/.s.PGSQL.5432\label{Ex:5432}
unix   STREAM  CONNECTED     3363
unix   STREAM  CONNECTED     3362
unix   DGRAM                 3138
...
\end{NumberedExample}

From this display, you can see that a database connection is available
with UNIX\index{UNIX socket} socket /tmp/.s.PGSQL.5432 (line~\ref{Ex:5432}). For Set\_Port, you only need
to supply the last numerical part, namely ``5432'' (which happens
to match the default IP port number). So to specify a UNIX socket
connection for this, you would code:

\begin{Example}

   Set_Port(C,"5432");

\end{Example}

Note especially, that the number must be in a \emph{string} to cause
a UNIX\index{UNIX socket} local\index{local socket} socket to be used. If you specify an integer costant,
a TCP/IP\index{TCP/IP socket} socket will be used, which uses the IP port number you specified.


\subsection{MySQL Connections}

APQ does not currently look for environment variables for MySQL.\footnote{although changes to MySQL are possible}
If you suspect environmental influences, check the MySQL C client library
documentation for conditions where the environment may be consulted.

A successful MySQL connection must have:

\begin{enumerate}
   \item A designated host name or address for the database server\index{host name}
   \item A port number\index{port number}
   \item Userid\index{userid} and password\index{password} (or ``'')
   \item Options (only if necessary)\index{options}
\end{enumerate}

Any supplied instance information is ignored.


\subsection{Sybase Connections}

APQ itself does not currently look for environment variables for Sybase\index{Sybase}.
The Sybase client libraries \emph{probably} do, if the variables exist.
Check the Sybase documentation (\emph{todo: check documentation for
influencing environment variables that are believed to exist}).

A successful Sybase connection must have:

\begin{enumerate}
   \item A designated instance name (Set\_Instance)\index{Set\_Instance}
   \item A userid\index{userid} and password\index{password} (or {}``'')
   \item Options (only if necessary)\index{options}
\end{enumerate}

Sybase ignores the host name/address and port information. The database
name is only important once the connection has been established. See
the Sybase specific notes about Set\_DB\_Name in \Ref{Context_Setting_Operations}.

The instance\index{instance name} name must exist in the local hosts \emph{interfaces}
file. In most cases this file can be found in /opt/sybase/interfaces
(\$SYBASE/interfaces) on UNIX platforms or possibly C:\textbackslash{}opt\textbackslash{}sybase\textbackslash{}ini\textbackslash{}sql.ini
(\%SYBASE\%\textbackslash{}ini\textbackslash{}sql.ini) on windows
(depending upon where it was installed). There may be exceptions to
this rule in more advanced Sybase configurations.


\subsection{Sybase Disconnect Problems}

If you experience a APQ.Not\_Connected\index{Not\_Connected} exception
in one of the following scenarios:

\begin{itemize}
   \item When the Connection\_Type object is finalized\index{finalized}
   \item When Disconnect is called on the Connection\_Type object
\end{itemize}

then look for Query\_Type objects that are still hanging onto query
results. This most often happens in a program where the Connection\_Type
object and Query\_Type objects are declared together and the Connection\_Type
object finalizes\index{finalizes} first. The solution is to either arrange it so that
all Query\_Type objects finalize first, or clear the query results
explicitly by calling Clear, prior to disconnecting or finalizing
the connection. For example if Q is the Query\_Type object and C the
connection, then the following sequence will ensure that the disconnect
(or finalize) will always succeed\index{Clear}:

\begin{Example}

   Clear(Q);      -- Release query results

   Disconnect(C); -- Disconnect (or finalize)

\end{Example}

\subsection{Sybase Options}

APQ permits the application developer to set various Sybase\index{Options} server
options. Some of these are known to create problems for APQ. Rather
than dictate policy for the developer, APQ gives you full access to
the Sybase options. Clearly some options should be avoided when using
APQ.

One such option is the following:

\begin{Example}

   Set_Options(C,"SHOWPLAN=TRUE"); -- CS_OPT_SHOWPLAN

\end{Example}

This option setting seems to do something strange to the underlying
Sybase connection after it is set. APQ has been observed raising Program\_Error\index{Program\_Error}
when other exceptions were supposed to have been raised. Unless you
know what you are doing, you seem to be well advised to stay clear
of this option when using APQ for Sybase.

The developer should also avoid using any Sybase options that change
the way the date is formatted. APQ functions look for dates to be
in a specific format so that they can be converted into APQ data types.


\subsection{Connection Cloning Problems}

If the original Connection\_Type object connects OK, but the Connect
clone\index{clone} call fails, then it could be that the network has gone bad since
the original connection was made. If exceptions other than Not\_Connected
are being raised, then check that:

\begin{itemize}
   \item Make sure the parameters are in the correct order in the Connect call.
   \item Make certain that the connected object, is indeed connected.
   \item Make certain that the new object is not already connected.
   \item Make sure you haven't exceeded the number of allowed database server
         connections
\end{itemize}

\subsection{Connection Tracing}

Problem: Your first Connection\_Type object is tracing to a file successfully,
but the cloned\index{cloned} Connection\_Type object is not.

Reason: Cloned connections do not have the trace\index{cloned trace} file parameters cloned.
This was a compromise to make APQ more portable to other platforms
that may not share files well.

Solution: Configure the cloned connection to trace to a file separate
from the original connection, after the clone operation is complete.


\appendix
\chapter{PostgreSQL Credits}

\section*{PostgreSQL Decimal C Sources}

PostgreSQL Database Management System (formerly known as Postgres,
then as Postgres95)

\begin{itemize}
\item Portions Copyright (c) 1996-2001, The PostgreSQL Global Development
Group
\item Portions Copyright (c) 1994, The Regents of the University of California
\end{itemize}
Permission to use, copy, modify, and distribute this software and
its documentation for any purpose, without fee, and without a written
agreement is hereby granted, provided that the above copyright notice
and this paragraph and the following two paragraphs appear in all
copies.

IN NO EVENT SHALL THE UNIVERSITY OF CALIFORNIA BE LIABLE TO ANY PARTY
FOR DIRECT, INDIRECT, SPECIAL, INCIDENTAL, OR CONSEQUENTIAL DAMAGES,
INCLUDING LOST PROFITS, ARISING OUT OF THE USE OF THIS SOFTWARE AND
ITS DOCUMENTATION, EVEN IF THE UNIVERSITY OF CALIFORNIA HAS BEEN ADVISED
OF THE POSSIBILITY OF SUCH DAMAGE.

THE UNIVERSITY OF CALIFORNIA SPECIFICALLY DISCLAIMS ANY WARRANTIES,
INCLUDING, BUT NOT LIMITED TO, THE IMPLIED WARRANTIES OF MERCHANTABILITY
AND FITNESS FOR A PARTICULAR PURPOSE. THE SOFTWARE PROVIDED HEREUNDER
IS ON AN ``AS IS'' BASIS, AND THE UNIVERSITY OF CALIFORNIA
HAS NO OBLIGATIONS TO PROVIDE MAINTENANCE, SUPPORT, UPDATES, ENHANCEMENTS,
OR MODIFICATIONS.


\section*{Modification Notice}

The numeric C routines required extensive interface modifications
for use in this APQ Binding. These modified C sources were extracted
from the PostgreSQL server project. No guarantee is made with regard
to the quality of these modifications.


\section*{Contributor Notice}

In no event shall the author or contributors to the APQ Binding be
liable to any party for any cause that the modified PostgreSQL software
may cause or contribute to.


\chapter{APQ License}


\section*{Scope of the APQ Binding License}

The {}``APQ Binding'' license covers those software modules not
provided by or extracted from the PostgreSQL database software (such
as the Decimal C source modules). The APQ license does not cover database
products themselves that APQ binds to.


\section*{APQ Binding License}

The APQ Binding is covered under a dual-license arrangement. This
should be reflected in the file name COPYING. The following two licenses
are available:

\begin{enumerate}
   \item The Ada Community License (ACL).
   \item The GNU Public License 2 (GPL2)
\end{enumerate}

One or the other license must be chosen to cover the APQ sources being
used. For the ACL license details, see \url{http://www.adapower.com/booch/ACL/index.html},
file ACL.txt or Appendix C. For details about the GPL license, see
Appendix D or the file GPL.txt that was included with your software.


\section*{Patents}

The author is not aware of any patent infringements made by
the APQ software. However the author makes \emph{absolutely no warrante}
about any possible patent infringements.

You are encouraged to lobby against any kind of software patents, since
software patents today do not serve the public's best interest.
Companies and individuals are using software patents to
legally extort from other companies and individuals. This is not in
the spirit of the original patent legislation and works against
all reasonable public interests.

\chapter{Ada Community License}


\section*{The Ada Community License}

\marginpar{Copyright(C) 1997 David G. Weller} Permission to redistribute
in unmodified form is granted, all other rights reserved. This is a
modification of the Perl Artistic License, (c) 1989-1991, Larry Wall


\subsection*{Preamble}

The intent of this document is to state the conditions under which
the Ada library may be copied, such that the Copyright Holder maintains
some semblance of artistic control over its development, while giving
Ada users the right to use and distribute the Ada library in a more-or-less
customary fashion, plus the right to make reasonable modifications.


\subsection*{Definitions}

\begin{description}
\item [``Library'']refers to the collection of Ada source
files distributed by the Copyright Holder, and derivatives
of that collection of files created through textual modification.
\item [``Standard~Version'']refers to such a library if
it has ``Standard Version'' not been modified, or
has been modified as specified below. 
\item [``Copyright~Holder'']is whoever is named in the
copyright or copyrights for the Ada library.
\item [``You'']is you, if you're thinking about copying
or distributing this library. 
\item [``Reasonable~Copying~Fee'']is whatever you can
justify on the basis of media cost, duplication charges, time of people
involved, and so on. (You will not be required to justify it to the
Copyright Holder, but only to the computing community at large as
a market that must bear the fee.) 
\item [``Freely~Available'']means that no fee is charged
for the item itself, though there may be fees involved in handling
the item. It also means that recipients of the item may redistribute
it under the same conditions they received it.
\end{description}

\subsection*{Provisions}

\begin{enumerate}
\item You may make and give away verbatim copies of the source form of the
Standard Version of this Ada library without restriction, provided
that you duplicate all of the original copyright notices and associated
disclaimers.
\item You may apply bug fixes, portability fixes and other modifications
derived from the Public Domain or from the Copyright Holder. A library
modified in such a way shall still be considered the Standard Version.
\item You may otherwise modify your copy of this Ada library in any way,
provided that you insert a prominent notice in each changed file stating
how and when you changed that file, and provided that you do at least
ONE of the following:

\begin{enumerate}
\item Place your modifications in the Public Domain or otherwise make them
Freely Available, such as by posting said modifications to Usenet
or an equivalent medium, or placing the modifications on a major archive
site such as The Public Ada Library, or by allowing the Copyright
Holder to include your modifications in the Standard Version of the
Ada library.
\item Use the modified Ada library only within your corporation or organization.
\item Rename any non-standard executables so the names do not conflict with
standard executables, which must also be provided, and provide a separate
manual page for each non-standard executable that clearly documents
how it differs from the Standard Version.
\item Make other distribution arrangements with the Copyright Holder.
\end{enumerate}
\item You may distribute the programs of this Ada library in object code
or executable form, provided that you do at least ONE of the following:

\begin{enumerate}
\item Distribute a Standard Version of the executables and library files,
together with instructions (in the manual page or equivalent) on where
to get the Standard Version.
\item Accompany the distribution with the machine-readable source of the
Ada library with your modifications. Accompany any non-standard executables
with their
\item corresponding Standard Version executables, giving the non-standard
executables non-standard names, and clearly documenting the differences
in manual pages (or equivalent), together with instructions on where
to get the Standard Version.
\item Make other distribution arrangements with the Copyright Holder.
\end{enumerate}
\item You may charge a reasonable copying fee for any distribution of this
Ada library. You may charge any fee you choose for support of this
Ada library. You may not charge a fee for this Ada library itself.
However, you may distribute this Ada library in aggregate with other
(possibly commercial) programs as part of a larger (possibly commercial)
software distribution provided that you do not advertise this Ada
library as a product of your own.
\item The scripts and library files supplied as input to or produced as
output from the programs of this Ada library do not automatically
fall under the copyright of this Ada library, but belong to whomever
generated them, and may be sold commercially, and may be aggregated
with this Ada library.
\item System-level subroutines supplied by you and linked into this Ada
library in order to emulate the functionality defined by this Ada
library shall not be considered part of this Ada library, but are
the equivalent of input as in Paragraph 6, provided these subroutines
do not change the library in any way that would cause it to fail the
regression tests for the library.
\item The name of the Copyright Holder may not be used to endorse or promote
products derived from this software without specific prior written
permission.
\item THIS ADA LIBRARY IS PROVIDED ``AS IS'' AND WITHOUT
ANY EXPRESS OR IMPLIED WARRANTIES, INCLUDING, WITHOUT LIMITATION,
THE IMPLIED WARRANTIES OF MERCHANTIBILITY AND FITNESS FOR A PARTICULAR
PURPOSE.
\end{enumerate}

\subsubsection{How to Apply These Terms to Your New Libraries}

To apply these terms, attach the following notices to the library.
It is safest to attach them to the start of each source file to most
effectively convey the exclusion of warranty; and each file should
have at least the ``copyright'' line and a pointer
to where the full notice is found. Also, be sure to add information
on how to contact you by electronic and paper mail.


\section*{Copyright (C)}

This program is free software; you can redistribute it and/or modify
it under the terms of the ``Ada Community License''
which comes with this Library. This program is distributed in the
hope that it will be useful, but WITHOUT ANY WARRANTY; without even
the implied warranty of MERCHANTABILITY or FITNESS FOR A PARTICULAR
PURPOSE. See the Ada Community License for more details. You should
have received a copy of the Ada Community License with this library,
in the file named ``Ada Community License'' or ``ACL''.
If not, contact the author of this library for a copy.


\chapter{GNU Public License}


\section*{GNU GENERAL PUBLIC LICENSE Version 2, June 1991}

\begin{quote}
Copyright (C) 1989, 1991 Free Software Foundation,

Inc. 59 Temple Place, Suite 330, Boston, 

MA 02111-1307 USA

Everyone is permitted to copy and distribute verbatim copies of this
license document, but changing it is not allowed.
\end{quote}

\subsection*{Preamble}

The licenses for most software are designed to take away your freedom
to share and change it. By contrast, the GNU General Public License
is intended to guarantee your freedom to share and change free software--to
make sure the software is free for all its users. This General Public
License applies to most of the Free Software Foundation's software
and to any other program whose authors commit to using it. (Some other
Free Software Foundation software is covered by the GNU Library General
Public License instead.) You can apply it to your programs, too.

When we speak of free software, we are referring to freedom, not price.
Our General Public Licenses are designed to make sure that you have
the freedom to distribute copies of free software (and charge for
this service if you wish), that you receive source code or can get
it if you want it, that you can change the software or use pieces
of it in new free programs; and that you know you can do these things.

To protect your rights, we need to make restrictions that forbid anyone
to deny you these rights or to ask you to surrender the rights. These
restrictions translate to certain responsibilities for you if you
distribute copies of the software, or if you modify it.

For example, if you distribute copies of such a program, whether gratis
or for a fee, you must give the recipients all the rights that you
have. You must make sure that they, too, receive or can get the source
code. And you must show them these terms so they know their rights.

We protect your rights with two steps:

\begin{enumerate}
   \item copyright the software, and
   \item offer you this license which gives you legal permission to copy, distribute
      and/or modify the software.
\end{enumerate}

Also, for each author's protection and ours, we want to make certain
that everyone understands that there is no warranty for this free
software. If the software is modified by someone else and passed on,
we want its recipients to know that what they have is not the original,
so that any problems introduced by others will not reflect on the
original authors' reputations.

Finally, any free program is threatened constantly by software patents.
We wish to avoid the danger that redistributors of a free program
will individually obtain patent licenses, in effect making the program
proprietary. To prevent this, we have made it clear that any patent
must be licensed for everyone's free use or not licensed at all.

The precise terms and conditions for copying, distribution and modification
follow.


\section*{GNU GENERAL PUBLIC LICENSE TERMS AND CONDITIONS FOR COPYING, DISTRIBUTION
AND MODIFICATION}


\paragraph{0. This License applies to any program or other work which contains
a notice placed by the copyright holder saying it may be distributed
under the terms of this General Public License. The ``Program'',
below, refers to any such program or work, and a ``work based
on the Program'' means either the Program or any derivative
work under copyright law: that is to say, a work containing the Program
or a portion of it, either verbatim or with modifications and/or translated
into another language. (Hereinafter, translation is included without
limitation in the term ``modification''.) Each licensee
is addressed as ``you''.}


\paragraph{Activities other than copying, distribution and modification are
not covered by this License; they are outside its scope. The act of
running the Program is not restricted, and the output from the Program
is covered only if its contents constitute a work based on the Program
(independent of having been made by running the Program). Whether
that is true depends on what the Program does.}


\paragraph{1. You may copy and distribute verbatim copies of the Program's source
code as you receive it, in any medium, provided that you conspicuously
and appropriately publish on each copy an appropriate copyright notice
and disclaimer of warranty; keep intact all the notices that refer
to this License and to the absence of any warranty; and give any other
recipients of the Program a copy of this License along with the Program.}


\paragraph{You may charge a fee for the physical act of transferring a copy,
and you may at your option offer warranty protection in exchange for
a fee.}


\paragraph{2. You may modify your copy or copies of the Program or any portion
of it, thus forming a work based on the Program, and copy and distribute
such modifications or work under the terms of Section 1 above, provided
that you also meet all of these conditions:}


\subparagraph*{a) You must cause the modified files to carry prominent notices stating
that you changed the files and the date of any change.}


\subparagraph*{b) You must cause any work that you distribute or publish, that in
whole or in part contains or is derived from the Program or any part
thereof, to be licensed as a whole at no charge to all third parties
under the terms of this License.}


\subparagraph{c) If the modified program normally reads commands interactively
when run, you must cause it, when started running for such interactive
use in the most ordinary way, to print or display an announcement
including an appropriate copyright notice and a notice that there
is no warranty (or else, saying that you provide a warranty) and that
users may redistribute the program under these conditions, and telling
the user how to view a copy of this License. (Exception: if the Program
itself is interactive but does not normally print such an announcement,
your work based on the Program is not required to print an announcement.) }


\paragraph{These requirements apply to the modified work as a whole. If identifiable
sections of that work are not derived from the Program, and can be
reasonably considered independent and separate works in themselves,
then this License, and its terms, do not apply to those sections when
you distribute them as separate works. But when you distribute the
same sections as part of a whole which is a work based on the Program,
the distribution of the whole must be on the terms of this License,
whose permissions for other licensees extend to the entire whole,
and thus to each and every part regardless of who wrote it.}


\paragraph{Thus, it is not the intent of this section to claim rights or contest
your rights to work written entirely by you; rather, the intent is
to exercise the right to control the distribution of derivative or
collective works based on the Program.}


\paragraph{In addition, mere aggregation of another work not based on the Program
with the Program (or with a work based on the Program) on a volume
of a storage or distribution medium does not bring the other work
under the scope of this License.}


\paragraph*{3. You may copy and distribute the Program (or a work based on it,
under Section 2) in object code or executable form under the terms
of Sections 1 and 2 above provided that you also do one of the following:}


\subparagraph*{a) Accompany it with the complete corresponding machine-readable
source code, which must be distributed under the terms of Sections
1 and 2 above on a medium customarily used for software interchange;
or,}


\subparagraph{b) Accompany it with a written offer, valid for at least three years,
to give any third party, for a charge no more than your cost of physically
performing source distribution, a complete machine-readable copy of
the corresponding source code, to be distributed under the terms of
Sections 1 and 2 above on a medium customarily used for software interchange;
or,}


\subparagraph{c) Accompany it with the information you received as to the offer
to distribute corresponding source code. (This alternative is allowed
only for noncommercial distribution and only if you received the program
in object code or executable form with such an offer, in accord with
Subsection b above.)}


\paragraph{The source code for a work means the preferred form of the work for
making modifications to it. For an executable work, complete source
code means all the source code for all modules it contains, plus any
associated interface definition files, plus the scripts used to control
compilation and installation of the executable. However, as a special
exception, the source code distributed need not include anything that
is normally distributed (in either source or binary form) with the
major components (compiler, kernel, and so on) of the operating system
on which the executable runs, unless that component itself accompanies
the executable.}


\paragraph{If distribution of executable or object code is made by offering
access to copy from a designated place, then offering equivalent access
to copy the source code from the same place counts as distribution
of the source code, even though third parties are not compelled to
copy the source along with the object code. ~4. You may not copy,
modify, sublicense, or distribute the Program except as expressly
provided under this License. Any attempt otherwise to copy, modify,
sublicense or distribute the Program is void, and will automatically
terminate your rights under this License. However, parties who have
received copies, or rights, from you under this License will not have
their licenses terminated so long as such parties remain in full compliance.}


\paragraph*{5. You are not required to accept this License, since you have not
signed it. However, nothing else grants you permission to modify or
distribute the Program or its derivative works. These actions are
prohibited by law if you do not accept this License. Therefore, by
modifying or distributing the Program (or any work based on the Program),
you indicate your acceptance of this License to do so, and all its
terms and conditions for copying, distributing or modifying the Program
or works based on it.}


\paragraph{6. Each time you redistribute the Program (or any work based on the
Program), the recipient automatically receives a license from the
original licensor to copy, distribute or modify the Program subject
to these terms and conditions. You may not impose any further restrictions
on the recipients' exercise of the rights granted herein. You are
not responsible for enforcing compliance by third parties to this
License.}


\paragraph{7. If, as a consequence of a court judgment or allegation of patent
infringement or for any other reason (not limited to patent issues),
conditions are imposed on you (whether by court order, agreement or
otherwise) that contradict the conditions of this License, they do
not excuse you from the conditions of this License. If you cannot
distribute so as to satisfy simultaneously your obligations under
this License and any other pertinent obligations, then as a consequence
you may not distribute the Program at all. For example, if a patent
license would not permit royalty-free redistribution of the Program
by all those who receive copies directly or indirectly through you,
then the only way you could satisfy both it and this License would
be to refrain entirely from distribution of the Program.}


\paragraph{If any portion of this section is held invalid or unenforceable under
any particular circumstance, the balance of the section is intended
to apply and the section as a whole is intended to apply in other
circumstances.}


\paragraph{It is not the purpose of this section to induce you to infringe any
patents or other property right claims or to contest validity of any
such claims; this section has the sole purpose of protecting the integrity
of the free software distribution system, which is implemented by
public license practices. Many people have made generous contributions
to the wide range of software distributed through that system in reliance
on consistent application of that system; it is up to the author/donor
to decide if he or she is willing to distribute software through any
other system and a licensee cannot impose that choice.}


\paragraph{This section is intended to make thoroughly clear what is believed
to be a consequence of the rest of this License. ~8. If the distribution
and/or use of the Program is restricted in certain countries either
by patents or by copyrighted interfaces, the original copyright holder
who places the Program under this License may add an explicit geographical
distribution limitation excluding those countries, so that distribution
is permitted only in or among countries not thus excluded. In such
case, this License incorporates the limitation as if written in the
body of this License.}


\paragraph{9. The Free Software Foundation may publish revised and/or new versions
of the General Public License from time to time. Such new versions
will be similar in spirit to the present version, but may differ in
detail to address new problems or concerns.}


\paragraph{Each version is given a distinguishing version number. If the Program
specifies a version number of this License which applies to it and
``any later version'', you have the option of following
the terms and conditions either of that version or of any later version
published by the Free Software Foundation. If the Program does not
specify a version number of this License, you may choose any version
ever published by the Free Software Foundation.}


\paragraph{10. If you wish to incorporate parts of the Program into other free
programs whose distribution conditions are different, write to the
author to ask for permission. For software which is copyrighted by
the Free Software Foundation, write to the Free Software Foundation;
we sometimes make exceptions for this. Our decision will be guided
by the two goals of preserving the free status of all derivatives
of our free software and of promoting the sharing and reuse of software
generally.}


\subsection*{NO WARRANTY}


\paragraph{11. BECAUSE THE PROGRAM IS LICENSED FREE OF CHARGE, THERE IS NO WARRANTY
FOR THE PROGRAM, TO THE EXTENT PERMITTED BY APPLICABLE LAW. EXCEPT
WHEN OTHERWISE STATED IN WRITING THE COPYRIGHT HOLDERS AND/OR OTHER
PARTIES PROVIDE THE PROGRAM ``AS IS'' WITHOUT WARRANTY
OF ANY KIND, EITHER EXPRESSED OR IMPLIED, INCLUDING, BUT NOT LIMITED
TO, THE IMPLIED WARRANTIES OF MERCHANTABILITY AND FITNESS FOR A PARTICULAR
PURPOSE. THE ENTIRE RISK AS TO THE QUALITY AND PERFORMANCE OF THE
PROGRAM IS WITH YOU. SHOULD THE PROGRAM PROVE DEFECTIVE, YOU ASSUME
THE COST OF ALL NECESSARY SERVICING, REPAIR OR CORRECTION.}


\paragraph{12. IN NO EVENT UNLESS REQUIRED BY APPLICABLE LAW OR AGREED TO IN
WRITING WILL ANY COPYRIGHT HOLDER, OR ANY OTHER PARTY WHO MAY MODIFY
AND/OR REDISTRIBUTE THE PROGRAM AS PERMITTED ABOVE, BE LIABLE TO YOU
FOR DAMAGES, INCLUDING ANY GENERAL, SPECIAL, INCIDENTAL OR CONSEQUENTIAL
DAMAGES ARISING OUT OF THE USE OR INABILITY TO USE THE PROGRAM (INCLUDING
BUT NOT LIMITED TO LOSS OF DATA OR DATA BEING RENDERED INACCURATE
OR LOSSES SUSTAINED BY YOU OR THIRD PARTIES OR A FAILURE OF THE PROGRAM
TO OPERATE WITH ANY OTHER PROGRAMS), EVEN IF SUCH HOLDER OR OTHER
PARTY HAS BEEN ADVISED OF THE POSSIBILITY OF SUCH DAMAGES.}


\paragraph{END OF TERMS AND CONDITIONS}


\section*{How to Apply These Terms to Your New Programs}

If you develop a new program, and you want it to be of the greatest
possible use to the public, the best way to achieve this is to make
it free software which everyone can redistribute and change under
these terms.

To do so, attach the following notices to the program. It is safest
to attach them to the start of each source file to most effectively
convey the exclusion of warranty; and each file should have at least
the ``copyright'' line and a pointer to where the
full notice is found.

\begin{Code}
<one line to give the program's name 
   and a brief idea of what it does.>

Copyright (C) <year> <name of author>

This program is free software; you can redistribute
it and/or modify it under the terms of the GNU
General Public License as published by the Free
Software Foundation; either version 2 of the
License, or (at your option) any later version.

This program is distributed in the hope that it
will be useful, but WITHOUT ANY WARRANTY; without
even the implied warranty of MERCHANTABILITY or
FITNESS FOR A PARTICULAR PURPOSE. See the GNU
General Public License for more details.

You should have received a copy of the GNU General
Public License along with this program; if not,
write to the Free Software Foundation, Inc., 59
Temple Place, Suite 330, Boston, MA 02111-1307 USA
\end{Code}

Also add information on how to contact you by electronic and paper
mail.

If the program is interactive, make it output a short notice like
this when it starts in an interactive mode:

\begin{quote}
Gnomovision version 69, Copyright (C) year name of author Gnomovision 
comes with ABSOLUTELY NO WARRANTY; for details type `show w'. This is
free software, and you are welcome to redistribute it under certain
conditions; type `show c' for details.
\end{quote}

The hypothetical commands `show w' and `show c' should show the appropriate
parts of the General Public License. Of course, the commands you use
may be called something other than `show w' and `show c'; they could
even be mouse-clicks or menu items--whatever suits your program.

You should also get your employer (if you work as a programmer) or
your school, if any, to sign a ``copyright disclaimer''
for the program, if necessary. Here is a sample; alter the names:

\begin{quote}
Yoyodyne, Inc., hereby disclaims all copyright
interest in the program `Gnomovision' (which makes
passes at compilers) written by James Hacker.
<signature of Ty Coon>, 1 April 1989 Ty Coon,
President of Vice
\end{quote}

This General Public License does not permit incorporating your program
into proprietary programs. If your program is a subroutine library,
you may consider it more useful to permit linking proprietary applications
with the library. If this is what you want to do, use the GNU Library
General Public License instead of this License.


\chapter{Credits}

This appendix documents the contributors to the ``APQ Binding''
that are separate from the PostgreSQL project.


\section*{Author}

\begin{quote}
Warren W. Gay VE3WWG\par
29 Glen Park Road,\par
St. Catharines, Ontario\par
Canada L2N 3E3\par
\par
ve3wwg@cogeco.ca\par
\end{quote}

\section*{Contributions}

\subsection*{Source Code Modifications}

There are no other contributors at the present time.


\subsection*{Bug Reports}

\begin{itemize}
\item There appears to be a problem with win32 releases using PostgreSQL
7.2.1, where specifying the database server by IP \# creates a problem.
The problem is believed to be in libpq.dll. (September 22, 2004: this
problem is believed to be fixed in more current releases of PostgreSQL).
\end{itemize}


\subsection*{Bug Report/Fix Contributions}

\begin{itemize}
\item Charles Darcy <charlie@mullum.com.au>, November 23, 2002
\end{itemize}

\chapter{History}


\section*{APQ 1.0}

First released August 3, 2002, under the Ada Community License (ACL).


\section*{APQ 1.1}

Second release August 4, 2002, but under the dual ACL and GPL2%
\footnote{GNU Public License 2%
} license. This license change was suggested by Florian Weimer.


\section*{APQ 1.2}

\begin{itemize}
   \item Added function End\_Of\_Blob to streamline sequential processing.
   \item Added buffered blob I/O for higher performance.
   \item Added procedure Blob\_Flush to force unwritten blob data to the database
         server.
   \item Blob\_Create has new optional Buf\_Size argument.
   \item Blob\_Open has new optional Buf\_Size argument.
   \item Fixed Blob\_Create to release the created blob, if the blob cannot
         be opened. This most often happens when the caller is attempting to
         create a blob, outside of a transaction.
\end{itemize}

\section*{APQ 1.3}

\begin{itemize}
   \item Removed some debug Put\_Line statements that should have been removed
         in 1.2.
   \item Added a few pragma Inline statements to the spec PostgreSQL.Client
\end{itemize}

\section*{APQ 1.4}

\begin{itemize}
   \item Added Generic\_Command\_Oid for strong PG\_Oid type use
   \item Added Generic\_Blob\_Open for strong PG\_Oid type use
   \item Added Generic\_Blob\_Oid function for strong PG\_Oid type use
   \item Added Generic\_Blob\_Unlink for strong PG\_Oid type use
   \item Added Generic\_Blob\_Import and Generic\_Blob\_Export for strong PG\_Oid
         type use
\end{itemize}

\section*{APQ 1.5}

\begin{itemize}
   \item Bug fix: Append\_Time, Append\_Date and Append\_Timestamp now emit
         the surrounding single quote characters around the value to satisfy
         the SQL syntax required by the database server.
   \item Troubleshooting help added to this manual for {}``Missing Time Data
         (Or Time is 00:00:00)''.
\end{itemize}

\section*{APQ 1.6}

\begin{itemize}
   \item Added Set\_Rollback\_On\_Finalize controlling primitive for Connection\_Type.
   \item Added Will\_Rollback\_On\_Finalize function for Connection\_Type for
         inquiry.
   \item Expanded manual and added transaction problem help to the Troubleshooting
         chapter.
\end{itemize}

\section*{APQ 1.7}

\begin{itemize}
   \item Open\_DB\_Trace and Close\_DB\_Trace procedures added.
   \item Set\_Trace procedure added to enable and disable tracing.
   \item Is\_Trace function added to query the tracing state.
\end{itemize}

\section*{APQ 1.8}

\begin{itemize}
   \item Connection information functions like Host\_Name and Port were added.
   \item A connection cloning primitive was added.
\end{itemize}

\section*{APQ 1.9}

\begin{itemize}
   \item A compiler work-around was provided. Some versions of GNAT would not
         compile the APQ source code, because certain instantiations of Ada.Text\_IO.Integer\_IO
         were producing duplicate symbol errors in the assembler phase of the
         compile. This problem was absent in gnat 3.13p compiles, but have
         shown up in gcc-3.1.1 and probably in gnat 3.14p and later releases.
         The instantiation names INTIO were made unique within the source code
         to avoid this problem.
\end{itemize}

\section*{APQ 1.91}

\begin{itemize}
   \item The PostgreSQL Connect call now performs an automatic {}``SET DATESTYLE
         TO ISO'' command prior to returning from a successful connect. This
         is necessary to guarantee that ISO date format is returned from the
         database engine and recognized from the engine. This guarantees that
         APQ correctly handles dates, even when the user has specified a PGDATESTYLE
         environment variable value that is different than ISO.
         \item Fixed bug in Host\_Name function. It was returning a null string when
         a host name was set in the Connection\_Type object.
   \item Win32 apq-1.91 source release (subdirectory win32) and apq-1.91-win32-2.7.1
         binary release created.
\end{itemize}

\section*{APQ 1.92}

\begin{itemize}
   \item Fixed bug for floating point and fixed point types (was rounding the
         value to the nearest integer, due to the fact that the Ada.Text\_IO.Float\_IO.Put
         call was receiving the argument Aft => 0). Omitting the Aft parameter
         causes the value to be formatted as required for the SQL floating/fixed
         point type. The bug was reported by Charles Darcy <charlie@mullum.com.au>.
\end{itemize}

\section*{APQ 1.93}

\begin{itemize}
   \item Modified the Ada95 package hierarchy to insert a top level package
         named APQ. Hence package PostgreSQL now becomes APQ.PostgreSQL. This
         is an interim release, which will pave the way to future support of
         other database products such as MySQL.
\end{itemize}

\section*{APQ 2.0}

\begin{itemize}
   \item MySQL support added. This was made possible by the package restructuring
         done in the interim release APQ 1.93.
   \item Generic database programming support added. Special generic services
         like Engine\_Of, New\_Query etc. were added to make this possible.
         A heavy reliance is made upon object inheritance and polymorphism
         to make this work.
   \item A new example subdirectory eg2 was added. The programs in this subdirectory
         show the original example program, but done in a database generic
         way. The test program can be compiled and run for both PostgreSQL
         and MySQL databases.
   \item PG\_Oid is now named APQ\_Row\_ID and is 64 bits unsigned integer.
   \item Null\_Oid() function added for generic database support (and much
         more).
   \item Engine\_Of() function added for generic database support.
   \item Exception {}``Failed'' was added to handle some general failures.
   \item Fetch\_Mode() and Set\_Fetch\_Mode() were added to accommodate MySQL
         limitations.
   \item Documentation went through some restructuring to accomodate two databases,
         and their differences.
\end{itemize}

\section*{APQ 2.1}

\begin{itemize}
   \item This was the win32 port.
   \item See win32.pdf for instructions for building APQ on a win32 platform.
   \item win32\_test.adb program was added to the distribution to allow testing
         of APQ in the windows environment.
\end{itemize}

\section*{APQ 2.2}

\begin{itemize}
   \item PostgreSQL now defaults to Set\_Port(C,5432), which is an TCP/IP connection
         for port 5432.
   \item APQ now uses Ada.Exceptions.Raise\_Exception in many places to provide
         more information. For example if a Value() function raises a Constraint\_Error,
         it now indicates in the message which column \# the error occurred
         for (where Value takes a Column\_Index\_Type argument). The informative
         message makes it easier to debug a new APQ application.
   \item Documentation added for Set\_Port for UNIX socket connections.
   \item The primitives Begin\_Work, Commit\_Work and Rollback\_Work now include
         an implicit call to Clear before and after the Query\_Type's object's
         use. This is necessary in some cases to clean up results and to put
         the object into the right state (Sybase made this necessary). The
         fetch mode of the Query\_Type object is preserved, even though it
         may be changed to something else while being used.
   \item The Set\_DB\_Name now works differently for PostgreSQL, if the connection
         has already been established. In the past, APQ (for PostgreSQL) simply
         took note of the new database name, but did nothing with it. This
         was inconsitent behavior compared to the fact that APQ\-.MySQL\-.Client\-.Set\_DB\_Name()
         would make a database switch on the connected connection. Now this
         behaviour is harmonized, by having the PostgreSQL version perform
         a hidden {}``USE <database>'' SQL query to make it happen.
   \item When a APQ\-.MySQL\-.Set\_DB\_Name(C) fails on a connected C, the exception
         raised is now APQ.Use\_Error instead of APQ.Failed. This helps to
         distinguish the difference between a successful connection and failed
         database change.
   \item Sybase always connects to the server's configured default database
         for the user. To make APQ.Sybase consistent with other supported databases,
         any Set\_DB\_Name(C) on connection C prior to the connection, now
         queues up a ``USE <database>'' query to be performed, once the
         connection is successful (effectively making it appear the same in
         APQ). Any call to Set\_DB\_Name(C) while C is connected, will result
         in another ``USE <database>'' SQL query being performed behind
         the scenes.
   \item Fixed bug in APQ\-.MySQL\-.Client\-.Reset so that the Connection\_Type object
         was properly reset for re-use.
   \item APQ.Sybase.Client support was added
   \item Set\_Instance and Instance functions were added to permit specification
         of Sybase instances.
   \item Fetch modes Cursor\_For\_Update and Cursor\_For\_Read\_Only were added
         for Sybase cursor support.
   \item Function Cursor\_Name was added for Sybase cursor support.
   \item Functions Set\_Case and Case were added to deal with Sybase's case
         sensitivity for database agnostic code.
   \item Fixed MySQL encoding of APQ\_Boolean types. Now sends 1 or 0, for
         True or False respectively (MySQL uses bit or tinyint to represent
         boolean values).
   \item Fixed MySQL decoding of date/time types (MySQL provides YYYYMMDDHHMMSS
         instead of YYYY-MM-DD HH:MM:SS, which APQ expected).
   \item APQ.PostgreSQL.Client.Value now trims trailing blanks returned for
         string values. This helps the returned values to be consistent across
         all database vendor products (MySQL always trims the trailing blanks).
\end{itemize}

\printindex
\end{document}
